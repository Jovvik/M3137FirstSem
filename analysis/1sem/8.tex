\documentclass[12pt, a4paper]{article}

%<*preamble>
% Math symbols
\usepackage{amsmath, amsthm, amsfonts, amssymb}
\usepackage{accents}
\usepackage{esvect}
\usepackage{mathrsfs}
\usepackage{mathtools}
\mathtoolsset{showonlyrefs}
\usepackage{cmll}
\usepackage{stmaryrd}
\usepackage{physics}
\usepackage[normalem]{ulem}
\usepackage{ebproof}
\usepackage{extarrows}

% Page layout
\usepackage{geometry, a4wide, parskip, fancyhdr}

% Font, encoding, russian support
\usepackage[russian]{babel}
\usepackage[sb]{libertine}
\usepackage{xltxtra}

% Listings
\usepackage{listings}
\lstset{basicstyle=\ttfamily,breaklines=true}
\setmonofont{Inconsolata}

% Miscellaneous
\usepackage{array}
\usepackage{calc}
\usepackage{caption}
\usepackage{subcaption}
\captionsetup{justification=centering,margin=2cm}
\usepackage{catchfilebetweentags}
\usepackage{enumitem}
\usepackage{etoolbox}
\usepackage{float}
\usepackage{lastpage}
\usepackage{minted}
\usepackage{svg}
\usepackage{wrapfig}
\usepackage{xcolor}
\usepackage[makeroom]{cancel}

\newcolumntype{L}{>{$}l<{$}}
    \newcolumntype{C}{>{$}c<{$}}
\newcolumntype{R}{>{$}r<{$}}

% Footnotes
\usepackage[hang]{footmisc}
\setlength{\footnotemargin}{2mm}
\makeatletter
\def\blfootnote{\gdef\@thefnmark{}\@footnotetext}
\makeatother

% References
\usepackage{hyperref}
\hypersetup{
    colorlinks,
    linkcolor={blue!80!black},
    citecolor={blue!80!black},
    urlcolor={blue!80!black},
}

% tikz
\usepackage{tikz}
\usepackage{tikz-cd}
\usetikzlibrary{arrows.meta}
\usetikzlibrary{decorations.pathmorphing}
\usetikzlibrary{calc}
\usetikzlibrary{patterns}
\usepackage{pgfplots}
\pgfplotsset{width=10cm,compat=1.9}
\newcommand\irregularcircle[2]{% radius, irregularity
    \pgfextra {\pgfmathsetmacro\len{(#1)+rand*(#2)}}
    +(0:\len pt)
    \foreach \a in {10,20,...,350}{
            \pgfextra {\pgfmathsetmacro\len{(#1)+rand*(#2)}}
            -- +(\a:\len pt)
        } -- cycle
}

\providetoggle{useproofs}
\settoggle{useproofs}{false}

\pagestyle{fancy}
\lfoot{M3137y2019}
\cfoot{}
\rhead{стр. \thepage\ из \pageref*{LastPage}}

\newcommand{\R}{\mathbb{R}}
\newcommand{\Q}{\mathbb{Q}}
\newcommand{\Z}{\mathbb{Z}}
\newcommand{\B}{\mathbb{B}}
\newcommand{\N}{\mathbb{N}}
\renewcommand{\Re}{\mathfrak{R}}
\renewcommand{\Im}{\mathfrak{I}}

\newcommand{\const}{\text{const}}
\newcommand{\cond}{\text{cond}}

\newcommand{\teormin}{\textcolor{red}{!}\ }

\DeclareMathOperator*{\xor}{\oplus}
\DeclareMathOperator*{\equ}{\sim}
\DeclareMathOperator{\sign}{\text{sign}}
\DeclareMathOperator{\Sym}{\text{Sym}}
\DeclareMathOperator{\Asym}{\text{Asym}}

\DeclarePairedDelimiter{\ceil}{\lceil}{\rceil}

% godel
\newbox\gnBoxA
\newdimen\gnCornerHgt
\setbox\gnBoxA=\hbox{$\ulcorner$}
\global\gnCornerHgt=\ht\gnBoxA
\newdimen\gnArgHgt
\def\godel #1{%
    \setbox\gnBoxA=\hbox{$#1$}%
    \gnArgHgt=\ht\gnBoxA%
    \ifnum     \gnArgHgt<\gnCornerHgt \gnArgHgt=0pt%
    \else \advance \gnArgHgt by -\gnCornerHgt%
    \fi \raise\gnArgHgt\hbox{$\ulcorner$} \box\gnBoxA %
    \raise\gnArgHgt\hbox{$\urcorner$}}

% \theoremstyle{plain}

\theoremstyle{definition}
\newtheorem{theorem}{Теорема}
\newtheorem*{definition}{Определение}
\newtheorem{axiom}{Аксиома}
\newtheorem*{axiom*}{Аксиома}
\newtheorem{lemma}{Лемма}

\theoremstyle{remark}
\newtheorem*{remark}{Примечание}
\newtheorem*{exercise}{Упражнение}
\newtheorem{corollary}{Следствие}[theorem]
\newtheorem*{statement}{Утверждение}
\newtheorem*{corollary*}{Следствие}
\newtheorem*{example}{Пример}
\newtheorem{observation}{Наблюдение}
\newtheorem*{prop}{Свойства}
\newtheorem*{obozn}{Обозначение}

% subtheorem
\makeatletter
\newenvironment{subtheorem}[1]{%
    \def\subtheoremcounter{#1}%
    \refstepcounter{#1}%
    \protected@edef\theparentnumber{\csname the#1\endcsname}%
    \setcounter{parentnumber}{\value{#1}}%
    \setcounter{#1}{0}%
    \expandafter\def\csname the#1\endcsname{\theparentnumber.\Alph{#1}}%
    \ignorespaces
}{%
    \setcounter{\subtheoremcounter}{\value{parentnumber}}%
    \ignorespacesafterend
}
\makeatother
\newcounter{parentnumber}

\newtheorem{manualtheoreminner}{Теорема}
\newenvironment{manualtheorem}[1]{%
    \renewcommand\themanualtheoreminner{#1}%
    \manualtheoreminner
}{\endmanualtheoreminner}

\newcommand{\dbltilde}[1]{\accentset{\approx}{#1}}
\newcommand{\intt}{\int\!}

% magical thing that fixes paragraphs
\makeatletter
\patchcmd{\CatchFBT@Fin@l}{\endlinechar\m@ne}{}
{}{\typeout{Unsuccessful patch!}}
\makeatother

\newcommand{\get}[2]{
    \ExecuteMetaData[#1]{#2}
}

\newcommand{\getproof}[2]{
    \iftoggle{useproofs}{\ExecuteMetaData[#1]{#2proof}}{}
}

\newcommand{\getwithproof}[2]{
    \get{#1}{#2}
    \getproof{#1}{#2}
}

\newcommand{\import}[3]{
    \subsection{#1}
    \getwithproof{#2}{#3}
}

\newcommand{\given}[1]{
    Дано выше. (\ref{#1}, стр. \pageref{#1})
}

\renewcommand{\ker}{\text{Ker }}
\newcommand{\im}{\text{Im }}
\renewcommand{\grad}{\text{grad}}
\newcommand{\rg}{\text{rg}}
\newcommand{\defeq}{\stackrel{\text{def}}{=}}
\newcommand{\defeqfor}[1]{\stackrel{\text{def } #1}{=}}
\newcommand{\itemfix}{\leavevmode\makeatletter\makeatother}
\newcommand{\?}{\textcolor{red}{???}}
\renewcommand{\emptyset}{\varnothing}
\newcommand{\longarrow}[1]{\xRightarrow[#1]{\qquad}}
\DeclareMathOperator*{\esup}{\text{ess sup}}
\newcommand\smallO{
    \mathchoice
    {{\scriptstyle\mathcal{O}}}% \displaystyle
    {{\scriptstyle\mathcal{O}}}% \textstyle
    {{\scriptscriptstyle\mathcal{O}}}% \scriptstyle
    {\scalebox{.6}{$\scriptscriptstyle\mathcal{O}$}}%\scriptscriptstyle
}
\renewcommand{\div}{\text{div}\ }
\newcommand{\rot}{\text{rot}\ }
\newcommand{\cov}{\text{cov}}

\makeatletter
\newcommand{\oplabel}[1]{\refstepcounter{equation}(\theequation\ltx@label{#1})}
\makeatother

\newcommand{\symref}[2]{\stackrel{\oplabel{#1}}{#2}}
\newcommand{\symrefeq}[1]{\symref{#1}{=}}

% xrightrightarrows
\makeatletter
\newcommand*{\relrelbarsep}{.386ex}
\newcommand*{\relrelbar}{%
    \mathrel{%
        \mathpalette\@relrelbar\relrelbarsep
    }%
}
\newcommand*{\@relrelbar}[2]{%
    \raise#2\hbox to 0pt{$\m@th#1\relbar$\hss}%
    \lower#2\hbox{$\m@th#1\relbar$}%
}
\providecommand*{\rightrightarrowsfill@}{%
    \arrowfill@\relrelbar\relrelbar\rightrightarrows
}
\providecommand*{\leftleftarrowsfill@}{%
    \arrowfill@\leftleftarrows\relrelbar\relrelbar
}
\providecommand*{\xrightrightarrows}[2][]{%
    \ext@arrow 0359\rightrightarrowsfill@{#1}{#2}%
}
\providecommand*{\xleftleftarrows}[2][]{%
    \ext@arrow 3095\leftleftarrowsfill@{#1}{#2}%
}

\allowdisplaybreaks

\newcommand{\unfinished}{\textcolor{red}{Не дописано}}

% Reproducible pdf builds 
\special{pdf:trailerid [
<00112233445566778899aabbccddeeff>
<00112233445566778899aabbccddeeff>
]}
%</preamble>


\lhead{Конспект по матанализу}
\cfoot{}
\rfoot{November 11, 2019}

\begin{document}
%<*охарактеристикекомпактоввrmproof>
\begin{proof}
    Продолжим доказательство из прошлой лекции, докажем, $3\Rightarrow1$.

    Рассмотрим секвенциально компактное $K$ и пусть $K$ --- не ограничено. \textit{(случай ограниченного множества тривиален)}

    $\exists x_n : ||x_n||\to+\infty$

    Тогда в этой последовательности нет сходящейся последовательности, т.к. любая $x_{n_k}\to x_0\in\R$ ограничена. Противоречие $\Rightarrow$ $K$ --- не секвенциально компактно.

    Таким образом, если $K$ --- секвенциально компактно, то $K$ ограничено.

    Докажем замкнутость $K$.

    Пусть $\exists$ предельная точка $x_0\notin K$

    $\exists x_n\to x_0$

    По секвенциальности $\exists$ подпоследовательность $x_{n_k}\to a\in K$.

    \textcolor{red}{Не дописано.}
\end{proof}
%</охарактеристикекомпактоввrmproof>

\begin{corollary}
    Принцип выбора Больцано-Вейерштрасса.
    %<*принципвыборабольцановейерштрасса>
    Если в $\R^m$ $(x_n)$ --- ограниченная последовательность, то у неё существует сходящаяся подпоследовательность.
    %</принципвыборабольцановейерштрасса>
\end{corollary}
%<*принципвыборабольцановейерштрассаproof>
\begin{proof}
    $x_n$ --- огр. $\Rightarrow x_n$ содержится в замкнутом кубе. Так как куб секвенциально компактен, $x_{n_k}$ сходится.
\end{proof}
%</принципвыборабольцановейерштрассаproof>
\begin{remark}
    $(x_n)$ --- не огр. $\Rightarrow x_n\to\infty$, т.е. $||x_n||\to+\infty$
\end{remark}
\begin{definition}
    $X$ --- метрическое пространство, $(x_n)$ в $X$

    %<*сходящаясявсебе>
    $x_n$ --- \textbf{фундаментальная, последовательность Коши, сходящаяся в себе}, если:
    $$\forall \varepsilon>0 \ \ \exists N \ \ \forall m,n > N\ \ \rho(x_m, x_n)<\varepsilon$$
    %</сходящаясявсебе>
\end{definition}
\begin{lemma}
    %<*сходящаясявсебесвойства>
    \begin{enumerate}
        \item $x_n$ --- фунд. $\Rightarrow x_n$ --- ограничена
        \item $x_n$ --- фунд; $\exists x_{n_k}$ --- сходящ. Тогда $x_n$ сходится.
    \end{enumerate}
    %</сходящаясявсебесвойства>
\end{lemma}
%<*сходящаясявсебесвойстваproof>
\begin{proof}
    \begin{enumerate}
        \item $\varepsilon:=1 \ \ \exists N \ \ \forall m, n:=N+1 > N \ \ \rho(x_m, x_{N+1})<1$

              $R:=\max(1; \rho(x_1, x_{N+1}),\ldots,\rho(x_N, x_{N+1}))$

              $\forall n \ \ x_n\in B(x_{N+1}, R) \Rightarrow x_n$ ограничена.

        \item $\begin{cases}
                      \varepsilon > 0 \ \ \exists K \ \ \forall k>K \ \ \rho(x_{n_k}, a)<\varepsilon \\
                      \varepsilon > 0 \ \ \exists N \ \ \forall m,n>N \ \ \rho(x_m, x_n)<\varepsilon
                  \end{cases} \xRightarrow{?} x_n\to a$

              $\forall \varepsilon > 0 \ \ \exists \tilde N := \max(N, K)$ при $k>\tilde N$ выполняется $k>K$, значит $n_k\geq k > K \Rightarrow \rho(x_{n_k}, a)<\varepsilon$.

              При $n>\tilde N\geq N \ \  m:=n_k>\tilde N\geq N \Rightarrow \rho(x_n, x_{n_k})<\varepsilon$

              Итого $\forall n>\tilde N \ \ \rho(x_n, a)\leq \rho(x_{n_k}, a) + \rho(x_n, x_{n_k})<2\varepsilon$
    \end{enumerate}
\end{proof}
%</сходящаясявсебесвойстваproof>
\begin{theorem}
    %<*критерийкошидляпоследовательностей>
    \begin{enumerate}
        \item В любом метрическом пространстве $x_n$ --- сходящ. $\Rightarrow x_n$ --- фунд.
        \item В $\R^m$ $x_n$ --- фунд. $\Rightarrow x_n$ --- сходящ.
    \end{enumerate}
    %</критерийкошидляпоследовательностей>
\end{theorem}
%<*критерийкошидляпоследовательностейproof>
\begin{proof}
    \begin{enumerate}
        \item $x_n\to a \quad \forall \varepsilon>0 \ \ \exists N \ \ \forall n>N \ \ \rho(x_n, a)<\varepsilon$

              $x_n\to a \quad \forall \varepsilon>0 \ \ \exists N \ \ \forall n,m > N \rho(x_m, x_n)\leq \rho(x_n, a) + \rho(x_m, a) < 2\varepsilon$

        \item $x_n$ --- фунд. $\Rightarrow x_n$ --- огр. $\xRightarrow{\text{Б.-В.}} \exists x_{n_k}$ --- сходящ.

              $\begin{cases}
                      \exists x_{n_k}\text{ --- сходящ.} \\
                      x_n\text{ --- фунд.}
                  \end{cases} \Rightarrow x_n$ --- сходящ.
    \end{enumerate}
\end{proof}
%</критерийкошидляпоследовательностейproof>
\begin{definition}
    %<*полноепространство>
    $X$ --- метрическое пространство называется \textbf{полным}, если в нём любая фундаментальная последовательность --- сходящаяся.
    %</полноепространство>
\end{definition}

Верно: $x_n$ --- вещ. посл.

$$\forall \varepsilon>0 \ \ \exists N \ \ \forall n,m>N \ \ |x_n-x_m|<\varepsilon \Leftrightarrow \exists \text{ конечн. }\lim\limits_{n\to+\infty} x_n$$

Это критерий Больцано-Коши.

$f:D\subset X\to Y$, $x_0$ --- предельная точка $D$.

$\lim\limits_{x\to x_0} f(x)=L$

$D_1\subset D, x_0$ --- предельная точка $D_1$.

\begin{definition}
    %<*пределпомножеству>
    $f : D\subset X\to Y, D_1\subset D, x_0$ --- пред. точка $D_1$

    Тогда \textbf{предел по множеству} $D_1$ в точке $x_0$ --- это $\lim\limits_{x\to x_0} f|_{D_1}(x)$
    %</пределпомножеству>
\end{definition}

\begin{definition}
    %<*односторонниепределы>
    В $\R$ одностор. $=\{$ левостор., правостор. $\}$

    \textbf{Левосторонний предел} $\lim\limits_{x\to x_0-0}f(x)=L$ - это $\lim f|_{D\cap(-\infty, x_0)}$

    $$\forall \varepsilon > 0 \ \ \exists \delta > 0 \ \ \forall x\in(x_0-\delta, x_0)\cap D \ \ |f(x) - L|<\varepsilon$$

    Аналогично правосторонний.
    %</односторонниепределы>
\end{definition}

Если $\lim\limits_{x\to x_0-0} f = \lim\limits_{x\to x_0+0} f=L \Rightarrow \lim\limits_{x\to x_0} f = L$

$\lim\limits_{x\to x_0-0} f \stackrel{\text{обозн.}}{=} f(x_0-0)$

$\lim\limits_{x\to 0-0} f = \lim\limits_{x\to -0} f$

В $\R^2 \lim\limits_{(x_1,x_2)\to(a_1, a_2)} f$

Предел вдоль прямой: $\lim_{r\to 0} f(a_1+r\cos \alpha, a_2+r\sin\alpha)$

\begin{theorem}
    О пределе монотонной функции

    %<*определемонотоннойфункции>
    $f:D\subset \R \to \R$, монотонная, $a\in\overline\R$

    $D_1:=D\cap (-\infty, a), a$ --- пред. точка $D_1$. Тогда:

    \begin{enumerate}
        \item $f$ --- возрастает, огр. сверху на $D_1$. Тогда $\exists$ конечный предел $\lim\limits_{x\to a-0} f(x)$
        \item $f$ --- убывает, огр. снизу на $D_1$. Тогда $\exists$ конечный предел $\lim\limits_{x\to a-0} f(x)$
    \end{enumerate}
    %</определемонотоннойфункции>
\end{theorem}
%<*определемонотоннойфункцииproof>
\begin{proof}
    \begin{enumerate}
        \item $L:= \sup\limits_{D_1} f \quad L\stackrel{?}{=}\lim\limits_{x\to a-0} f(x)$

              $\forall \varepsilon > 0 \ \ L-\varepsilon$ --- не верхн. граница для $\{f(x) : x\in D_1\} \ \ \exists x_1: L-\varepsilon<f(x_1)$.

              Тогда при $x\in(x_1, a)\cap D_1 \ \ L-\varepsilon < f(x_1) \leq f(x) \leq L$

              $\exists \delta:=|x_1-a| \ \ \forall x : x\in(x_1, a) \ \ L-\varepsilon\leq f(x) < L+\varepsilon$

              Аналогично доказывается пункт 2.
    \end{enumerate}
\end{proof}
%</определемонотоннойфункцииproof>

Критерий Больцано-Коши для отображений.

\begin{theorem}
    %<*критерийбольцанокошидляотображений>
    $f:D\subset X\to Y$, $a$ --- пр. точка $D$, $Y$ --- полное метрическое пространство.

    Тогда $$\exists \lim\limits_{x\to a}f(x)\in Y \Leftrightarrow \forall \varepsilon > 0 \ \ \exists \delta>0 \ \ \forall x_1, x_2\in D \ \ \rho(x_1,a)<\delta; \rho(x_2, a) < \delta \ \ \rho(f(x_1), f(x_2))<\varepsilon$$
    %</критерийбольцанокошидляотображений>
\end{theorem}
%<*критерийбольцанокошидляотображенийproof>
\begin{proof}
    ``$\Rightarrow$'' как для последовательностей.

    Докажем ``$\Leftarrow$'' по Гейне.

    Заметим, что последовательность $f(x_n)$ --- фундаментальная, т.е.

    $$\forall \varepsilon>0 \ \ \exists N \ \ \forall m,n > N\ \ \rho(f(x_m), f(x_n))<\varepsilon$$

    $$x_n\to a \Rightarrow \exists N \ \ \forall n>N \ \ \rho(x_n, a)<\delta$$

    $$\forall m,n > N \ \ \rho(x_n, a)<\delta; \rho(x_m, a)<\delta \xRightarrow{\text{Фунд.}} \rho(f(x_n), f(x_m))<\varepsilon$$
\end{proof}
%</критерийбольцанокошидляотображенийproof>
\begin{remark}
    В $\R$ критерий Больцано-Коши для функций

    $f:D\subset \R\to\R, a$ --- пред. точка $D$

    $$\forall \varepsilon>0 \ \ \exists \delta>0 \ \ \forall x_1, x_2 \in D\setminus\{a\} \ \ |x_1-a|<\delta; |x_2-a|<\delta$$
\end{remark}

Для $\lim\limits_{x\to x_0} f(x)=+\infty$ критерий Больцано-Коши: $$\forall E \ \ \exists \delta>0 \ \ \forall x_1, x_2 \in D\setminus\{a\} \ \ |x_1-a|<\delta; |x_2-a|<\delta \ \ f(x_1)>E; f(x_2)>E$$ неинтересно.

Для $\lim\limits_{x\to+\infty}f(x)=L$:

$$\forall \varepsilon>0 \ \ \exists \Delta \ \ \forall x_1, x_2\in D \ \ x_1>\Delta; x_2>\Delta \ \ |f(x_1)-f(x_2)|<\varepsilon$$

\section{Непрерывные отображения}

\begin{definition}
    %<*непрерывноеотображение>
    $f:D\subset X\to Y \quad x_0\in D$

    $f$ --- \textbf{непрерывное} в точке $x_0$, если:
    \begin{enumerate}
        \item $\lim\limits_{x\to x_0} f(x)=f(x_0)$, либо $x_0$ --- изолированная точка $D$
        \item $\forall \varepsilon > 0 \ \ \exists \delta>0 \ \ \forall x\in D \ \ \rho(x, x_0)<\delta \ \ \rho(f(x), f(x_0))<\varepsilon$
        \item $\forall U(f(x_0)) \ \ \exists V(x_0) \ \ \forall x\in V(x_0)\cap D \ \ f(x)\in U(f(x_0))$
        \item По Гейне $\forall (x_n):x_n\to x_0; x_n\in D \ \ f(x_n)\xrightarrow[n\to+\infty]{} f(x_0)$
    \end{enumerate}
    %</непрерывноеотображение>
\end{definition}

\begin{definition}
    %<*точкаразрыва>
    Если $\nexists\lim\limits_{x\to x_0} f(x)$, либо $\exists\lim\limits_{x\to x_0}f(x)\not=f(x_0)$ --- \textbf{точка разрыва}.
    %</точкаразрыва>
\end{definition}

Для $\R \quad \forall \varepsilon>0 \ \ \exists \delta > 0 \ \ \forall x\in D \ \ |x-x_0|<\delta \ \ |f(x) - f(x_0)|<\varepsilon$

\begin{definition}
    Непр. слева и непр. справа
    %<*непрерывностьслева>
    $f$ --- непр. слева в $x_0$, если $f|_{(-\infty, x_0]\cap D}$ --- непрерывно в $x_0$
    %</непрерывностьслева>
\end{definition}

Если $f$ непрерывно слева и непрерывно справа в $x_0$, то $f$ непрерывно в $x_0$.

\begin{definition}
    %<*родточкиразрыва>
    Пусть $\exists f(x_0-0), f(x_0+0)$ и не все 3 числа равны: $f(x_0-0), f(x_0), f(x_0+0)$. Это \textbf{разрыв I рода \textit{(скачок)}}.

    Остальные точки разрыва --- \textbf{разрыв II рода}.

    \begin{remark}
        $$f(x_0-0)\Leftrightarrow \lim\limits_{x\to x_0-0} f(x)$$
    \end{remark}
    %</родточкиразрыва>
\end{definition}

\begin{example}
    \begin{enumerate}
        \item $f(x)=sign(x)=\begin{cases}
                      1, x>0 \\
                      0, x=0 \\
                      -1, x<0
                  \end{cases}$ $0$ --- разрыв I рода.

        \item $f(x)=sin(\frac{1}{x})$ $0$ --- разрыв II рода.
    \end{enumerate}
\end{example}

\begin{definition}
    Отображение \textbf{непрерывно} на множестве $D = $ непрерывно в каждой точке множества $D$.
\end{definition}

\begin{theorem}
    Арифметические свойства непрерывных отображений
    %<*арифметическиесвойстванепрерывныхотображений>
    \begin{enumerate}
        \item $f,g:D\subset X\to Y \ \ x_0\in D$ ($Y$ --- норм. пространство)

              $f, g$ --- непр. в $x_0; \lambda:D\to\R(\mathbb{C})$ --- непр. $x_0$

              Тогда $f\pm g, ||f||, \lambda f$ --- непр. $x_0$

        \item $f,g:D\subset X\to \R \ \ x_0\in D$

              $f, g$ --- непр. в $x_0$

              Тогда $f\pm g, |f|, fg$ --- непр. в $x_0$

              $g(x_0)\not=0$, тогда $\frac{f}{g}$ --- непр. $x_0$
    \end{enumerate}
    %</арифметическиесвойстванепрерывныхотображений>
\end{theorem}
%<*арифметическиесвойстванепрерывныхотображенийproof>
\textcolor{red}{Доказательство отсутствует}
%</арифметическиесвойстванепрерывныхотображенийproof>

\end{document}