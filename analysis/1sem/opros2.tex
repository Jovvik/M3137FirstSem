\documentclass[12pt, a4paper]{article}

%<*preamble>
% Math symbols
\usepackage{amsmath, amsthm, amsfonts, amssymb}
\usepackage{accents}
\usepackage{esvect}
\usepackage{mathrsfs}
\usepackage{mathtools}
\mathtoolsset{showonlyrefs}
\usepackage{cmll}
\usepackage{stmaryrd}
\usepackage{physics}
\usepackage[normalem]{ulem}
\usepackage{ebproof}
\usepackage{extarrows}

% Page layout
\usepackage{geometry, a4wide, parskip, fancyhdr}

% Font, encoding, russian support
\usepackage[russian]{babel}
\usepackage[sb]{libertine}
\usepackage{xltxtra}

% Listings
\usepackage{listings}
\lstset{basicstyle=\ttfamily,breaklines=true}
\setmonofont{Inconsolata}

% Miscellaneous
\usepackage{array}
\usepackage{calc}
\usepackage{caption}
\usepackage{subcaption}
\captionsetup{justification=centering,margin=2cm}
\usepackage{catchfilebetweentags}
\usepackage{enumitem}
\usepackage{etoolbox}
\usepackage{float}
\usepackage{lastpage}
\usepackage{minted}
\usepackage{svg}
\usepackage{wrapfig}
\usepackage{xcolor}
\usepackage[makeroom]{cancel}

\newcolumntype{L}{>{$}l<{$}}
    \newcolumntype{C}{>{$}c<{$}}
\newcolumntype{R}{>{$}r<{$}}

% Footnotes
\usepackage[hang]{footmisc}
\setlength{\footnotemargin}{2mm}
\makeatletter
\def\blfootnote{\gdef\@thefnmark{}\@footnotetext}
\makeatother

% References
\usepackage{hyperref}
\hypersetup{
    colorlinks,
    linkcolor={blue!80!black},
    citecolor={blue!80!black},
    urlcolor={blue!80!black},
}

% tikz
\usepackage{tikz}
\usepackage{tikz-cd}
\usetikzlibrary{arrows.meta}
\usetikzlibrary{decorations.pathmorphing}
\usetikzlibrary{calc}
\usetikzlibrary{patterns}
\usepackage{pgfplots}
\pgfplotsset{width=10cm,compat=1.9}
\newcommand\irregularcircle[2]{% radius, irregularity
    \pgfextra {\pgfmathsetmacro\len{(#1)+rand*(#2)}}
    +(0:\len pt)
    \foreach \a in {10,20,...,350}{
            \pgfextra {\pgfmathsetmacro\len{(#1)+rand*(#2)}}
            -- +(\a:\len pt)
        } -- cycle
}

\providetoggle{useproofs}
\settoggle{useproofs}{false}

\pagestyle{fancy}
\lfoot{M3137y2019}
\cfoot{}
\rhead{стр. \thepage\ из \pageref*{LastPage}}

\newcommand{\R}{\mathbb{R}}
\newcommand{\Q}{\mathbb{Q}}
\newcommand{\Z}{\mathbb{Z}}
\newcommand{\B}{\mathbb{B}}
\newcommand{\N}{\mathbb{N}}
\renewcommand{\Re}{\mathfrak{R}}
\renewcommand{\Im}{\mathfrak{I}}

\newcommand{\const}{\text{const}}
\newcommand{\cond}{\text{cond}}

\newcommand{\teormin}{\textcolor{red}{!}\ }

\DeclareMathOperator*{\xor}{\oplus}
\DeclareMathOperator*{\equ}{\sim}
\DeclareMathOperator{\sign}{\text{sign}}
\DeclareMathOperator{\Sym}{\text{Sym}}
\DeclareMathOperator{\Asym}{\text{Asym}}

\DeclarePairedDelimiter{\ceil}{\lceil}{\rceil}

% godel
\newbox\gnBoxA
\newdimen\gnCornerHgt
\setbox\gnBoxA=\hbox{$\ulcorner$}
\global\gnCornerHgt=\ht\gnBoxA
\newdimen\gnArgHgt
\def\godel #1{%
    \setbox\gnBoxA=\hbox{$#1$}%
    \gnArgHgt=\ht\gnBoxA%
    \ifnum     \gnArgHgt<\gnCornerHgt \gnArgHgt=0pt%
    \else \advance \gnArgHgt by -\gnCornerHgt%
    \fi \raise\gnArgHgt\hbox{$\ulcorner$} \box\gnBoxA %
    \raise\gnArgHgt\hbox{$\urcorner$}}

% \theoremstyle{plain}

\theoremstyle{definition}
\newtheorem{theorem}{Теорема}
\newtheorem*{definition}{Определение}
\newtheorem{axiom}{Аксиома}
\newtheorem*{axiom*}{Аксиома}
\newtheorem{lemma}{Лемма}

\theoremstyle{remark}
\newtheorem*{remark}{Примечание}
\newtheorem*{exercise}{Упражнение}
\newtheorem{corollary}{Следствие}[theorem]
\newtheorem*{statement}{Утверждение}
\newtheorem*{corollary*}{Следствие}
\newtheorem*{example}{Пример}
\newtheorem{observation}{Наблюдение}
\newtheorem*{prop}{Свойства}
\newtheorem*{obozn}{Обозначение}

% subtheorem
\makeatletter
\newenvironment{subtheorem}[1]{%
    \def\subtheoremcounter{#1}%
    \refstepcounter{#1}%
    \protected@edef\theparentnumber{\csname the#1\endcsname}%
    \setcounter{parentnumber}{\value{#1}}%
    \setcounter{#1}{0}%
    \expandafter\def\csname the#1\endcsname{\theparentnumber.\Alph{#1}}%
    \ignorespaces
}{%
    \setcounter{\subtheoremcounter}{\value{parentnumber}}%
    \ignorespacesafterend
}
\makeatother
\newcounter{parentnumber}

\newtheorem{manualtheoreminner}{Теорема}
\newenvironment{manualtheorem}[1]{%
    \renewcommand\themanualtheoreminner{#1}%
    \manualtheoreminner
}{\endmanualtheoreminner}

\newcommand{\dbltilde}[1]{\accentset{\approx}{#1}}
\newcommand{\intt}{\int\!}

% magical thing that fixes paragraphs
\makeatletter
\patchcmd{\CatchFBT@Fin@l}{\endlinechar\m@ne}{}
{}{\typeout{Unsuccessful patch!}}
\makeatother

\newcommand{\get}[2]{
    \ExecuteMetaData[#1]{#2}
}

\newcommand{\getproof}[2]{
    \iftoggle{useproofs}{\ExecuteMetaData[#1]{#2proof}}{}
}

\newcommand{\getwithproof}[2]{
    \get{#1}{#2}
    \getproof{#1}{#2}
}

\newcommand{\import}[3]{
    \subsection{#1}
    \getwithproof{#2}{#3}
}

\newcommand{\given}[1]{
    Дано выше. (\ref{#1}, стр. \pageref{#1})
}

\renewcommand{\ker}{\text{Ker }}
\newcommand{\im}{\text{Im }}
\renewcommand{\grad}{\text{grad}}
\newcommand{\rg}{\text{rg}}
\newcommand{\defeq}{\stackrel{\text{def}}{=}}
\newcommand{\defeqfor}[1]{\stackrel{\text{def } #1}{=}}
\newcommand{\itemfix}{\leavevmode\makeatletter\makeatother}
\newcommand{\?}{\textcolor{red}{???}}
\renewcommand{\emptyset}{\varnothing}
\newcommand{\longarrow}[1]{\xRightarrow[#1]{\qquad}}
\DeclareMathOperator*{\esup}{\text{ess sup}}
\newcommand\smallO{
    \mathchoice
    {{\scriptstyle\mathcal{O}}}% \displaystyle
    {{\scriptstyle\mathcal{O}}}% \textstyle
    {{\scriptscriptstyle\mathcal{O}}}% \scriptstyle
    {\scalebox{.6}{$\scriptscriptstyle\mathcal{O}$}}%\scriptscriptstyle
}
\renewcommand{\div}{\text{div}\ }
\newcommand{\rot}{\text{rot}\ }
\newcommand{\cov}{\text{cov}}

\makeatletter
\newcommand{\oplabel}[1]{\refstepcounter{equation}(\theequation\ltx@label{#1})}
\makeatother

\newcommand{\symref}[2]{\stackrel{\oplabel{#1}}{#2}}
\newcommand{\symrefeq}[1]{\symref{#1}{=}}

% xrightrightarrows
\makeatletter
\newcommand*{\relrelbarsep}{.386ex}
\newcommand*{\relrelbar}{%
    \mathrel{%
        \mathpalette\@relrelbar\relrelbarsep
    }%
}
\newcommand*{\@relrelbar}[2]{%
    \raise#2\hbox to 0pt{$\m@th#1\relbar$\hss}%
    \lower#2\hbox{$\m@th#1\relbar$}%
}
\providecommand*{\rightrightarrowsfill@}{%
    \arrowfill@\relrelbar\relrelbar\rightrightarrows
}
\providecommand*{\leftleftarrowsfill@}{%
    \arrowfill@\leftleftarrows\relrelbar\relrelbar
}
\providecommand*{\xrightrightarrows}[2][]{%
    \ext@arrow 0359\rightrightarrowsfill@{#1}{#2}%
}
\providecommand*{\xleftleftarrows}[2][]{%
    \ext@arrow 3095\leftleftarrowsfill@{#1}{#2}%
}

\allowdisplaybreaks

\newcommand{\unfinished}{\textcolor{red}{Не дописано}}

% Reproducible pdf builds 
\special{pdf:trailerid [
<00112233445566778899aabbccddeeff>
<00112233445566778899aabbccddeeff>
]}
%</preamble>


\usepackage{bm}
\usepackage{xcolor}
% \usepackage[raggedright]{titlesec}
\usepackage{sectsty}
\usepackage{catchfilebetweentags}

\allsectionsfont{\raggedright}
\subsectionfont{\fontsize{14}{15}\selectfont}

% magical thing that fixes paragraphs
\makeatletter
\def\CatchFBT@sanitize{%
   \@sanitize
   \@makeother\{%
   \@makeother\}%
}
\makeatother

\renewcommand{\import}[3]{
    \subsection{#1}
    \ExecuteMetaData[#2]{#3}
}

\lhead{Конспект к второму опросу}
\cfoot{}
\rfoot{}

\setlength\parindent{0pt}

\begin{document}
\section{Определения и формулировки}
%<*определения>
\import{\teormin Внутренняя точка множества, открытое множество, внутренность}{opros.tex}{внутренность}

\import{\teormin Предельная точка множества}{4.tex}{предельнаяточка}

\import{\teormin Замкнутое множество, замыкание, граница}{4.tex}{замкнутоемножество}

\ExecuteMetaData[5.tex]{замыкание}

\ExecuteMetaData[5.tex]{граница}

\import{\teormin Изолированная точка, граничная точка}{4.tex}{изолированнаяточка}

\ExecuteMetaData[5.tex]{граничнаяточка}

\import{Описание внутренности множества}{4.tex}{описаниевнутренности}

\import{Описание замыкания множества в терминах пересечений}{5.tex}{замыканиечерезпересечения}

\import{\teormin Верхняя, нижняя границы; супремум, инфимум}{5.tex}{верхняяграница}

\ExecuteMetaData[5.tex]{нижняяграница}

\ExecuteMetaData[5.tex]{супремуминфимум}

\import{Техническое описание супремума}{5.tex}{техническоеописаниесупремума}

\import{\teormin Последовательность, стремящаяся к бесконечности}{5.tex}{последовательностьстремящаясякбесконечности}

\import{\teormin Компактное множество}{6.tex}{компактноемножество}

\import{Секвенциальная компактность}{7.tex}{секвенциальнокомпактноемножество}

\import{\teormin Определения предела отображения (3 шт)}{6.tex}{пределотображения}

\import{Определения пределов в $\overline\R$}{opros.tex}{пределывrсчертой}

\import{Предел по множеству}{8.tex}{пределпомножеству}

\import{Односторонние пределы}{8.tex}{односторонниепределы}

\import{\teormin Непрерывное отображение}{8.tex}{непрерывноеотображение}

\import{Непрерывность слева}{8.tex}{непрерывностьслева}

\import{Разрыв, разрывы первого и второго рода}{8.tex}{точкаразрыва}

\ExecuteMetaData[8.tex]{родточкиразрыва}

\import{\teormin О большое, о маленькое}{10.tex}{осимволика}

\begin{remark}
    О большое и о малое --- разные вопросы в табличке.
\end{remark}

\import{\teormin Эквивалентные функции, таблица эквивалентных}{10.tex}{}

\label{equivtable}

Эквивалентные функции даны выше.

\ExecuteMetaData[10.tex]{таблицаэквивалентных}

\import{Асимптотически равные \textit{(сравнимые)} функции}{10.tex}{асимптотическиравныефункции}

\import{Асимптотическое разложение}{10.tex}{асимтотическоеразложение}

\import{Наклонная асимптота графика}{10.tex}{наклоннаяасимптота}

\import{Путь в метрическом пространстве}{11.tex}{путь}

\import{Линейно связное множество}{11.tex}{линейносвязноемножество}

\import{\teormin Функция, дифференцируемая в точке и производная}{12.tex}{дифференциируемая}

\begin{remark}
    Это два разных билета.
\end{remark}

%</определения>

\section{Теоремы}

\import{Теорема об открытых и замкнутых множествах в пространстве и в подпространстве}{6.tex}{открытыеизамкнутыемножествавпространствеиподпространстве}

\import{Теорема о компактности в пространстве и в подпространстве}{6.tex}{окомпактностивпространствеиподпространстве}

\import{Простейшие свойства компактных множеств}{7.tex}{опростейшихсвойствахкомпактныхмножеств}

\import{Лемма о вложенных параллелепипедах}{7.tex}{овложенныхпараллелепипедах}

\import{Компактность замкнутого параллелепипеда в $\R^m$}{7.tex}{компактностьпарллелепипеда}

\import{Теорема о характеристике компактов в $\R^m$}{7.tex}{охарактеристикекомпактоввrm}

\import{Эквивалентность определений Гейне и Коши}{7.tex}{эквивалентностьопределенийгейнеикоши}

\import{Единственность предела, локальная ограниченность отображения, имеющего предел, теорема о стабилизации знака}{7.tex}{оединственностипредела}

\ExecuteMetaData[7.tex]{олокальнойограниченностиотображенияимеющегопредел}

\ExecuteMetaData[7.tex]{остабилизациизнака}

\import{Арифметические свойства пределов отображений. Формулировка для $\overline\R$}{7.tex}{обарифметическихсвойствахпредела}

\ExecuteMetaData[7.tex]{арифметическиесвойствапределадляoverliner}

\import{Принцип выбора Больцано--Вейерштрасса}{8.tex}{принципвыборабольцановейерштрасса}

\import{Сходимость в себе и ее свойства}{8.tex}{сходящаясявсебе}

\ExecuteMetaData[9.tex]{свойствасходимостивсебе}

\import{Критерий Коши для последовательностей и отображений}{8.tex}{критерийбольцанокошидляотображений}

\subsubsection{Для последовательностей}

\ExecuteMetaData[8.tex]{критерийкошидляпоследовательностей}

\import{Теорема о пределе монотонной функции}{8.tex}{определемонотоннойфункции}

\import{Свойства непрерывных отображений: арифметические, стабилизация знака, композиция}{8.tex}{арифметическиесвойстванепрерывныхотображений}

\subsubsection{Стабилизация знака}

\ExecuteMetaData[9.tex]{стабилизациязнака}

\subsubsection{Непрерывность композиции непрерывных отображений}

\ExecuteMetaData[10.tex]{онепрерывностикомпозиции}

\import{Непрерывность композиции и соответствующая теорема для пределов}{10.tex}{}

Непрервыность композиции дана выше.

\subsubsection{Теорема о пределе композиции непрерывных отображений}

\ExecuteMetaData[10.tex]{определекомпозиции}

\import{Теорема о замене на эквивалентную при вычислении пределов. Таблица эквивалентных}{10.tex}{озамененаэквивалентную}

Таблица эквивалентных дана выше.

\import{Теорема единственности асимптотического разложения}{10.tex}{оединственностиасимтотическогоразложения}

\import{Теорема о топологическом определении непрерывности}{10.tex}{топологическоеопределениенепрерывности}

\import{Теорема Вейерштрасса о непрерывном образе компакта. Следствия}{10.tex}{вейерштрассаонепрерывномобразекомпакта}

\ExecuteMetaData[10.tex]{следствиявейерштрасса}

\import{Лемма о связности отрезка}{11.tex}{освязностиотрезка}

\import{Теорема Больцано-Коши о промежуточном значении}{11.tex}{больцанокошиопромежуточномзначении}

\import{Теорема о сохранении промежутка}{11.tex}{осохранениипромежутка}

%<*больцанокошиосохранениилинейнойсвязности>
\subsection{Теорема Больцано-Коши о сохранении линейной связности}

$X, Y$ --- метрические пространства, $f:X\to Y$ --- непрерывное и сюръекция

$X$ --- линейно связное множество. Тогда $Y$ --- линейно связное множество.
%</больцанокошиосохранениилинейнойсвязности>
%<*больцанокошиосохранениилинейнойсвязностиproof>
\begin{proof}
    Надо доказать, что $\exists$ путь $[a,b]\to[A,B]$

    $f(a)=A; f(b)=B$

    $X$ --- линейно связное $\Rightarrow \exists \gamma:[\alpha, \beta]\to X, \gamma(\alpha)=a, \gamma(\beta)=b, \gamma$ --- непрерывное

    $$f\circ\gamma[a, b]\to Y; f\circ\gamma(\alpha)=A, f\circ\gamma(\beta)=B$$

    Т.к. композиция непрерывных функций непрерывна, $f\circ\gamma$ --- непрерывна.
\end{proof}
%</больцанокошиосохранениилинейнойсвязностиproof>

\import{Описание линейно связных множеств в $\R$}{11.tex}{линейносвзяноевr}

\import{Теорема о бутерброде}{11.tex}{теоремаобутерброде}

\subsection{Теорема о вписанном $n$-угольнике максимальной площади}

\ExecuteMetaData[11.tex]{овписанномnугольникемаксимальнойплощади}

\import{Теорема о непрерывности монотонной функции. Следствие о множестве точек разрыва}{11.tex}{онепрерывностимонотонныхфункций}

\ExecuteMetaData[12.tex]{омножестветочекразрыва}

\import{Теорема о существовании и непрерывности обратной функции}{12.tex}{осуществованииинепрерывностиобратнойфункции}

\end{document}