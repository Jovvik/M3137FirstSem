\documentclass[12pt, a4paper]{article}

%<*preamble>
% Math symbols
\usepackage{amsmath, amsthm, amsfonts, amssymb}
\usepackage{accents}
\usepackage{esvect}
\usepackage{mathrsfs}
\usepackage{mathtools}
\mathtoolsset{showonlyrefs}
\usepackage{cmll}
\usepackage{stmaryrd}
\usepackage{physics}
\usepackage[normalem]{ulem}
\usepackage{ebproof}
\usepackage{extarrows}

% Page layout
\usepackage{geometry, a4wide, parskip, fancyhdr}

% Font, encoding, russian support
\usepackage[russian]{babel}
\usepackage[sb]{libertine}
\usepackage{xltxtra}

% Listings
\usepackage{listings}
\lstset{basicstyle=\ttfamily,breaklines=true}
\setmonofont{Inconsolata}

% Miscellaneous
\usepackage{array}
\usepackage{calc}
\usepackage{caption}
\usepackage{subcaption}
\captionsetup{justification=centering,margin=2cm}
\usepackage{catchfilebetweentags}
\usepackage{enumitem}
\usepackage{etoolbox}
\usepackage{float}
\usepackage{lastpage}
\usepackage{minted}
\usepackage{svg}
\usepackage{wrapfig}
\usepackage{xcolor}
\usepackage[makeroom]{cancel}

\newcolumntype{L}{>{$}l<{$}}
    \newcolumntype{C}{>{$}c<{$}}
\newcolumntype{R}{>{$}r<{$}}

% Footnotes
\usepackage[hang]{footmisc}
\setlength{\footnotemargin}{2mm}
\makeatletter
\def\blfootnote{\gdef\@thefnmark{}\@footnotetext}
\makeatother

% References
\usepackage{hyperref}
\hypersetup{
    colorlinks,
    linkcolor={blue!80!black},
    citecolor={blue!80!black},
    urlcolor={blue!80!black},
}

% tikz
\usepackage{tikz}
\usepackage{tikz-cd}
\usetikzlibrary{arrows.meta}
\usetikzlibrary{decorations.pathmorphing}
\usetikzlibrary{calc}
\usetikzlibrary{patterns}
\usepackage{pgfplots}
\pgfplotsset{width=10cm,compat=1.9}
\newcommand\irregularcircle[2]{% radius, irregularity
    \pgfextra {\pgfmathsetmacro\len{(#1)+rand*(#2)}}
    +(0:\len pt)
    \foreach \a in {10,20,...,350}{
            \pgfextra {\pgfmathsetmacro\len{(#1)+rand*(#2)}}
            -- +(\a:\len pt)
        } -- cycle
}

\providetoggle{useproofs}
\settoggle{useproofs}{false}

\pagestyle{fancy}
\lfoot{M3137y2019}
\cfoot{}
\rhead{стр. \thepage\ из \pageref*{LastPage}}

\newcommand{\R}{\mathbb{R}}
\newcommand{\Q}{\mathbb{Q}}
\newcommand{\Z}{\mathbb{Z}}
\newcommand{\B}{\mathbb{B}}
\newcommand{\N}{\mathbb{N}}
\renewcommand{\Re}{\mathfrak{R}}
\renewcommand{\Im}{\mathfrak{I}}

\newcommand{\const}{\text{const}}
\newcommand{\cond}{\text{cond}}

\newcommand{\teormin}{\textcolor{red}{!}\ }

\DeclareMathOperator*{\xor}{\oplus}
\DeclareMathOperator*{\equ}{\sim}
\DeclareMathOperator{\sign}{\text{sign}}
\DeclareMathOperator{\Sym}{\text{Sym}}
\DeclareMathOperator{\Asym}{\text{Asym}}

\DeclarePairedDelimiter{\ceil}{\lceil}{\rceil}

% godel
\newbox\gnBoxA
\newdimen\gnCornerHgt
\setbox\gnBoxA=\hbox{$\ulcorner$}
\global\gnCornerHgt=\ht\gnBoxA
\newdimen\gnArgHgt
\def\godel #1{%
    \setbox\gnBoxA=\hbox{$#1$}%
    \gnArgHgt=\ht\gnBoxA%
    \ifnum     \gnArgHgt<\gnCornerHgt \gnArgHgt=0pt%
    \else \advance \gnArgHgt by -\gnCornerHgt%
    \fi \raise\gnArgHgt\hbox{$\ulcorner$} \box\gnBoxA %
    \raise\gnArgHgt\hbox{$\urcorner$}}

% \theoremstyle{plain}

\theoremstyle{definition}
\newtheorem{theorem}{Теорема}
\newtheorem*{definition}{Определение}
\newtheorem{axiom}{Аксиома}
\newtheorem*{axiom*}{Аксиома}
\newtheorem{lemma}{Лемма}

\theoremstyle{remark}
\newtheorem*{remark}{Примечание}
\newtheorem*{exercise}{Упражнение}
\newtheorem{corollary}{Следствие}[theorem]
\newtheorem*{statement}{Утверждение}
\newtheorem*{corollary*}{Следствие}
\newtheorem*{example}{Пример}
\newtheorem{observation}{Наблюдение}
\newtheorem*{prop}{Свойства}
\newtheorem*{obozn}{Обозначение}

% subtheorem
\makeatletter
\newenvironment{subtheorem}[1]{%
    \def\subtheoremcounter{#1}%
    \refstepcounter{#1}%
    \protected@edef\theparentnumber{\csname the#1\endcsname}%
    \setcounter{parentnumber}{\value{#1}}%
    \setcounter{#1}{0}%
    \expandafter\def\csname the#1\endcsname{\theparentnumber.\Alph{#1}}%
    \ignorespaces
}{%
    \setcounter{\subtheoremcounter}{\value{parentnumber}}%
    \ignorespacesafterend
}
\makeatother
\newcounter{parentnumber}

\newtheorem{manualtheoreminner}{Теорема}
\newenvironment{manualtheorem}[1]{%
    \renewcommand\themanualtheoreminner{#1}%
    \manualtheoreminner
}{\endmanualtheoreminner}

\newcommand{\dbltilde}[1]{\accentset{\approx}{#1}}
\newcommand{\intt}{\int\!}

% magical thing that fixes paragraphs
\makeatletter
\patchcmd{\CatchFBT@Fin@l}{\endlinechar\m@ne}{}
{}{\typeout{Unsuccessful patch!}}
\makeatother

\newcommand{\get}[2]{
    \ExecuteMetaData[#1]{#2}
}

\newcommand{\getproof}[2]{
    \iftoggle{useproofs}{\ExecuteMetaData[#1]{#2proof}}{}
}

\newcommand{\getwithproof}[2]{
    \get{#1}{#2}
    \getproof{#1}{#2}
}

\newcommand{\import}[3]{
    \subsection{#1}
    \getwithproof{#2}{#3}
}

\newcommand{\given}[1]{
    Дано выше. (\ref{#1}, стр. \pageref{#1})
}

\renewcommand{\ker}{\text{Ker }}
\newcommand{\im}{\text{Im }}
\renewcommand{\grad}{\text{grad}}
\newcommand{\rg}{\text{rg}}
\newcommand{\defeq}{\stackrel{\text{def}}{=}}
\newcommand{\defeqfor}[1]{\stackrel{\text{def } #1}{=}}
\newcommand{\itemfix}{\leavevmode\makeatletter\makeatother}
\newcommand{\?}{\textcolor{red}{???}}
\renewcommand{\emptyset}{\varnothing}
\newcommand{\longarrow}[1]{\xRightarrow[#1]{\qquad}}
\DeclareMathOperator*{\esup}{\text{ess sup}}
\newcommand\smallO{
    \mathchoice
    {{\scriptstyle\mathcal{O}}}% \displaystyle
    {{\scriptstyle\mathcal{O}}}% \textstyle
    {{\scriptscriptstyle\mathcal{O}}}% \scriptstyle
    {\scalebox{.6}{$\scriptscriptstyle\mathcal{O}$}}%\scriptscriptstyle
}
\renewcommand{\div}{\text{div}\ }
\newcommand{\rot}{\text{rot}\ }
\newcommand{\cov}{\text{cov}}

\makeatletter
\newcommand{\oplabel}[1]{\refstepcounter{equation}(\theequation\ltx@label{#1})}
\makeatother

\newcommand{\symref}[2]{\stackrel{\oplabel{#1}}{#2}}
\newcommand{\symrefeq}[1]{\symref{#1}{=}}

% xrightrightarrows
\makeatletter
\newcommand*{\relrelbarsep}{.386ex}
\newcommand*{\relrelbar}{%
    \mathrel{%
        \mathpalette\@relrelbar\relrelbarsep
    }%
}
\newcommand*{\@relrelbar}[2]{%
    \raise#2\hbox to 0pt{$\m@th#1\relbar$\hss}%
    \lower#2\hbox{$\m@th#1\relbar$}%
}
\providecommand*{\rightrightarrowsfill@}{%
    \arrowfill@\relrelbar\relrelbar\rightrightarrows
}
\providecommand*{\leftleftarrowsfill@}{%
    \arrowfill@\leftleftarrows\relrelbar\relrelbar
}
\providecommand*{\xrightrightarrows}[2][]{%
    \ext@arrow 0359\rightrightarrowsfill@{#1}{#2}%
}
\providecommand*{\xleftleftarrows}[2][]{%
    \ext@arrow 3095\leftleftarrowsfill@{#1}{#2}%
}

\allowdisplaybreaks

\newcommand{\unfinished}{\textcolor{red}{Не дописано}}

% Reproducible pdf builds 
\special{pdf:trailerid [
<00112233445566778899aabbccddeeff>
<00112233445566778899aabbccddeeff>
]}
%</preamble>


\lhead{Математический анализ}
\cfoot{}
\rfoot{16.11.2020}

\begin{document}

\subsection*{Гомотопия}

Неформально гомотопия --- непрерывная деформация объектов. У нас рассматриваемые объекты --- пути.

\begin{definition}
    \textbf{Гопотопия} двух (\textit{непрерывных}) путей \(\gamma_0, \gamma_1 : [a, b] \to O\subset \R^m\) это непрерывное отображение \(\Gamma : \underbrace{[a, b]}_{t} \times \underbrace{[0, 1]}_{u} \to O\), такое что:
    \begin{itemize}
        \item \(\Gamma(\circ , 0) = \gamma_0\)
        \item \(\Gamma(\circ , 1) = \gamma_1\)
    \end{itemize}

    Гопотопия \textbf{связанная} (\textit{не связная}), если:
    \begin{itemize}
        \item \(\gamma_0(a) = \gamma_1(a)\)
        \item \(\gamma_0(b) = \gamma_1(b)\)
        \item \(\forall u\in[0, 1]\ \Gamma(a, u) = \gamma_0(a), \Gamma(b, u) = \gamma_1(b)\)
    \end{itemize}

    \begin{figure}[h]
        \centering
        \includesvg[scale=1]{images/гомотопия_связанная.svg}
        \caption{Связанная гопотопия.\\ Пунктиром --- \(\Gamma(\circ, u)\) для различных \(u\)}
    \end{figure}

    Гопотопия \textbf{петельная}, если:
    \begin{itemize}
        \item \(\gamma_0(a) = \gamma_0(b)\)
        \item \(\gamma_1(a) = \gamma_1(b)\)
        \item \(\forall u\in[0, 1]\ \Gamma(a, u) = \Gamma(b, u)\)
    \end{itemize}

    \begin{figure}[h]
        \centering
        \includesvg[scale=1]{images/гомотопия_петельная.svg}
        \caption{Петельная гопотопия.\\ Пунктиром --- \(\Gamma(\circ, u)\) для различных \(u\)}
    \end{figure}

    \pagebreak
\end{definition}

\begin{theorem}\itemfix
    %<*интегралпосвязанногомотопнымпутям>
    \begin{itemize}
        \item \(V\) --- локально потенциальное векторное поле в \(O\subset\R^m\)
        \item \(\gamma_0, \gamma_1\) --- связанно гомотопные пути
    \end{itemize}
    Тогда \(\int_{\gamma_0} \sum V_i dx_i = \int_{\gamma_1} \sum V_idx_i\)
    %</интегралпосвязанногомотопнымпутям>
\end{theorem}

\begin{remark}
    То же самое верно для петельных гомотопий.
\end{remark}

%<*интегралпосвязанногомотопнымпутямproof>
\begin{proof}
    \(\gamma_u(t) : = \Gamma(t, u), t\in[a, b], u\in[0, 1]\)

    \[\Phi(u) = \int_{\gamma_u} \sum V_i dx_i\]

    Мы хотим доказать, что \(\Phi(u) = \const\). Докажем более простой факт, что \(\Phi\) --- локально постоянна, тогда в силу компактности отрезка \(\Phi\) будет постоянна.

    Определение локально постоянной функции:
    \[\forall u_0\in[0, 1]\ \ \exists W(u_0) : \forall u\in W(u_0)\cap [0, 1]\quad \Phi(u) = \Phi(u_0)\]

    \(\Gamma\) --- непр. на \([a, b] \times [0, 1]\) --- комп. \( \Rightarrow \Gamma\) равномерно непрерывна:
    \[\forall \delta > 0\ \exists \sigma > 0\ \forall t, t' : |t - t'| < \sigma\ \forall u,u': |u - u'|< \sigma\quad |\Gamma(t, u) - \Gamma(t', u')| < \frac{\delta}{2} \]

    Возьмём \(\delta\) из леммы о похожести близких путей (\ref{лемма 3, лекция 9}) для пути \(\gamma_{u_0}\).

    Если \(|u - u_0| < \delta \ \ |\Gamma(t, u) - \Gamma(t, u_0)|< \frac{\delta}{2}\) при \(t\in[a, b]\), т.е. \(\gamma_u\) и \(\gamma_{u_0}\) похожи по лемме о похожести близких путей. Хочется сказать, что интегралы по \(\gamma_u\) и \(\gamma_{u_0}\) таким образом равны, однако это не обосновано, для этого необходимо, чтобы пути были кусочно-гладкими.

    Построим кусочно-гладкий путь \(\tilde \gamma_{u_0}\), \(\frac{\delta}{4}\)-близкий к \(\gamma_{u_0}\), т.е.
    \[\forall t\in[a, b] \ \ |\gamma_{u_0}(t) - \tilde \gamma_{u_0}(t)| < \frac{\delta}{4}\]
    и кусочно-гладкий путь \(\tilde \gamma_u\), \(\frac{\delta}{4}\)-близкий к \(\gamma_{u}\). Тогда \(\tilde \gamma_{u_0}\) и \(\tilde \gamma_u\) - \(\delta\)-близкие к \(\gamma_{u_0}\) \( \Rightarrow \) они \(V\)-похожи \( \Rightarrow \) \[\int_{\gamma_u} \sum V_i dx_i\defeq \int_{\tilde \gamma_u} \dots = \int_{\tilde \gamma_{u_0}} \dots \defeq \int_{\gamma_{u_0}} \dots \]
    Таким образом, \(\Phi(u) = \Phi(u_0)\) при \(|u - u_0|< \delta\), т.е. \(\Phi\) --- локально постоянна.
\end{proof}
%</интегралпосвязанногомотопнымпутямproof>

\begin{definition}
    Область \(O\subset \R^m\) --- \textbf{односвязная}, если любой замкнутый путь в ней гомотопен постоянному пути.

    Простыми словами --- в \(O\) нет дырок, иначе путь вокруг дырки нельзя было бы стянуть.

    \begin{figure}[h]
        \centering
        \includesvg[scale=1]{images/односвязно.svg}
        \caption{Стягивание замкнутого пути (сплошной линией) к постоянному пути (точке)}
    \end{figure}
\end{definition}

\begin{remark}\itemfix
    \begin{enumerate}
        \item Выпуклая область --- односвязная.

              \begin{figure}[h]
                  \centering
                  \includesvg[scale=0.9]{images/гомотетия.svg}
                  \caption{Применение гомотетии с центром \(A\)}
              \end{figure}

              Это доказывается тем, что для любого пути можно применить гомотетию в качестве гомотопии: \(\Gamma(t, u) = F_{1 - u}(\gamma(t))\), где \(F_{\alpha}\) --- гомотетия с центром \(A\) (лежит внутри области, огр. путём \(\gamma\)) и коэффициентом \(\alpha\)
        \item Гомеоморфный образ односвязного множества --- односвязен.

              \(\Phi : O \to O'\) --- гомеоморфизм, \(\gamma\) --- петля в \(O'\), \(\Phi^{ - 1}(\gamma)\) --- петля в \(O\).

              \(\Gamma : [a, b] \times [0, 1] \to O\) --- гомотопия \(\Phi^{ - 1}(\gamma)\) и постоянного пути \(\tilde \gamma\equiv A\)

              \(\Phi \circ \Gamma\) --- гомотопия \(\gamma\) с постоянным путём \(\dbltilde \gamma \equiv \Phi(A)\)
    \end{enumerate}
\end{remark}

\begin{theorem}\itemfix
    \begin{itemize}
        \item \(O\subset \R^m\) --- односвязная область
        \item \(V\) --- локально потенциальное векторное поле в \(O\)
    \end{itemize}
    Тогда \(V\) --- потенциальное в \(O\)
\end{theorem}

\begin{proof}
    \(V\) --- локально потенциально, \(\gamma_0\) --- кусочно-гладкая петля, тогда \(\gamma_0\) гомотопна постоянному пути \(\gamma_1\) \( \Rightarrow \)
    \[\int_{\gamma_0} = \int_{\gamma_1} = \int_a^b \langle V(\gamma_1(t)), \underbrace{\gamma'_1(t)}_{\equiv0} \rangle dt = 0\]
    Тогда по теореме о характеризации потенциальных векторных полей в терминах интегралов \(V\) потенциально.
\end{proof}

\begin{corollary}
    Теорема Пуанкаре верна в односвязной области.
\end{corollary}

Пусть даны две плоскости, соединенные гвоздём, между плоскостями есть зазор. На гвоздь надета веревочка в виде петли. Можно ли снять веревочку с гвоздя?

\begin{theorem}[о веревочке]\itemfix
    \begin{itemize}
        \item \(O = \R^2\{(0, 0)\}\)
        \item \(\gamma : [0, 2\pi] \to O, t\mapsto (\cos t, \sin t)\)
    \end{itemize}
    Тогда эта петля нестягиваема.
\end{theorem}

\begin{proof}
    \(V(x, y) = \left( \cfrac{ -y}{x^2 + y^2}, \cfrac{x}{x^2 + y^2}\right)\) --- векторное поле в \(\R^2\)
    \[\frac{\partial V_1}{\partial y} = \frac{\partial V_2}{\partial x} \quad \frac{\partial V_1}{\partial y} = \frac{ -(x^2 + y^2) + 2 y^2}{(x^2 + y^2)^2} \quad \frac{\partial V_2}{\partial x} = \frac{(x^2 + y^2) - 2x^2}{(x^2 + y^2)^2} \]
    Тогда \(V\) --- локально потенциально.

    При этом
    \begin{align*}
        \int_{\gamma}\sum V_idx_i & = \int_0^{2\pi} \frac{ -\sin t}{\cos^2 t + \sin^2 t} ( - \sin t) + \frac{\cos t}{1} \cos t dt \\
                                  & = \int_0^{2\pi} 1dt = 2\pi
    \end{align*}

    Таким образом, если бы существовал постоянный путь \(\tilde \gamma\), которому \(\gamma\) гомотопен, то \(\int_\gamma = \int_{\tilde \gamma} = 0\), но это не так.
\end{proof}

\subsection*{Степенные ряды}

\begin{example}
    \begin{enumerate}
        \item \(\sum_{n = 0}^{+\infty} z^n, R = \cfrac{1}{\overline \lim \sqrt[n]{1}} = 1 , |z| < 1\) --- сходится, \(|z|> 1\) --- расходится, \(|z|= 1\) --- расходится, т.к. слагаемые \(\not\to 0\)
        \item \(\sum \cfrac{z^n}{n}, R = \cfrac{1}{\overline \lim \sqrt[n]{\frac{1}{n}}} = 1\)
              \begin{enumerate}
                  \item \(z = 1, \sum \frac{1}{n}\) --- расходится
                  \item \(z = - 1, \sum \frac{( - 1)^n}{n} \) --- сходится
                  \item \(z = e^{i\varphi}, \varphi \neq 0, 2\pi \ \ \sum \cfrac{e^{in\varphi}}{n} = \sum \cfrac{\cos n\varphi + i \sin n\varphi}{n} \) --- сходится по признаку Дирихле.
              \end{enumerate}
        \item \(\sum \frac{z^n}{n^2}, R = 1, |z|= 1 \Rightarrow \left|\frac{z^n}{n^2}\right| \leq \frac{1}{n^2}\) сходится.
        \item \(\sum n!z^n, R = \cfrac{1}{\overline \lim \sqrt[n]{n!}} \approx \cfrac{1}{\overline \lim \sqrt[n]{n^ne^{ - n}\sqrt{2\pi n}}} = \cfrac{1}{\overline \lim \frac{n}{e}} = 0\), в \(0\) сходится, в остальных точках расходится.
        \item \(\sum \cfrac{z^n}{n!}, R = +\infty\) --- везде сходится.
    \end{enumerate}
\end{example}

\begin{theorem}[о равномерной сходимости и непрерывности степенного ряда]\itemfix
    \begin{itemize}
        \item \(\sum a_n(z - z_0)^n\)
        \item \(0 < R \leq +\infty\)
    \end{itemize}
    Тогда:
    \begin{enumerate}
        \item \(\forall r : 0 < r < R\) ряд сходится равномерно на \(\overline{B(z_0, r)}\)
        \item \(f(z) = \sum a_n(z - z_0)^n\) --- непрерывна в \(B(z_0, r)\)
    \end{enumerate}
\end{theorem}

\begin{proof}\itemfix
    \begin{enumerate}
        \item Если \(0 < r < R\), то при \(z = r\) ряд абсолютно сходится, т.е. \(\sum |a_n| r^n < +\infty\)

              Признак Вейерштрасса:
              \begin{enumerate}
                  \item При \(|z - z_0|\leq r\) \(|a_n(z - z_0)^n| \leq |a_n|r^n\)
                  \item \(\sum |a_n|r^n < +\infty\)
              \end{enumerate}
              \( \Rightarrow \) есть сходимость на \(\overline{B(z_0, r)}\)
        \item Следствие из пункта 1 и теоремы Стокса-Зайдля.

              Если \(z\) удовлетворяет \(|z - z_0| < R\), то \(\exists r_0 < R : z\in B(z_0, r_0)\)

              На \(B(z_0, r_0)\) есть равномерная сходимость \( \Rightarrow \) \(f\) непр. в точке \(z\).
    \end{enumerate}
\end{proof}

\begin{definition}
    \(f : \C \to \C\). Тогда производная \(f\) это:
    \[f'(z_0) = \lim_{z\to z_0} \frac{f(z) - f(z_0)}{z - z_0}\]
\end{definition}

\begin{remark}
    \(f(z_0 + h) = f(z_0) + f'(z_0)h + o(|h|), h\in\C\)
\end{remark}

\begin{lemma}\itemfix
    \begin{itemize}
        \item \(w, w_0\in\C\)
        \item \(|w|< r\)
        \item \(|w_0|< r\)
    \end{itemize}
    Тогда \(|w^n - w^n_0|\le n r^n |w - w_0|, n\in\N\).
\end{lemma}
\begin{proof}
    \[w^n - w^n_0 = (w - w_0)(w^{n - 1} + \underbrace{w^{n - 2}w_0}_{\text{по модулю} \le r^{n-1}} + \dots + w^{n - 1}_0)\]
\end{proof}

\begin{lemma}[о дифференцируемости степенного ряда]\itemfix
    \begin{itemize}
        \item [(A)] \(\sum_{n = 0}^{\infty} a_n(z - z_0)^n, 0 < R < +\infty\)
        \item [(A')] \(\sum_{n = 1}^{\infty} na_n (z - z_0)^{n - 1}\)
    \end{itemize}
    Тогда:
    \begin{enumerate}
        \item Радиус сходимости (A') равен \(R\)
        \item \(\forall r\in B(z_0,R) \ \ \exists f'(z)\) и \(f'(z) = \sum n a_n(z - z_0)^n\)
    \end{enumerate}
\end{lemma}

\begin{proof}\itemfix
    \begin{enumerate}
        \item По формуле Адамара.

              Ряд (A') сходится при каком-то \(z\) \( \Leftrightarrow \sum n a_n(z - z_0)^n\) --- сходится.

              \[\frac{1}{\overline \lim \sqrt[n]{n|a_n|}} = \frac{1}{1\cdot \overline \lim \sqrt[n]{|a_n|}} = R\]
        \item \(\sphericalangle a\in B(z_0, R) , \exists r < R : a\in B(z_0, r)\)

              \(a = z_0 + w_0, |w_0|< r\)

              \(z = z_0 + w\)

              \begin{figure}[h]
                  \centering
                  \includesvg[scale=0.7]{images/штука.svg}
              \end{figure}

              \begin{equation}
                  \frac{f(z) - f(a)}{z - a} = \sum_{n = 0}^{ +\infty} a_n\frac{(z - z_0)^n - (a - z_0)^n}{z - a} = \sum_{n = 1}^{ +\infty} \underbrace{a_n \frac{w^n - w^n_0}{w - w_0}}_{\substack{\text{модуль по лемме} \\ n r^{n-1}|a_n|}} \label{комплексная производная для a'}
              \end{equation}
              \(\sum nr^{n - 1}|a_n|\) сходится по пункту 1.

              То есть ряд \ref{комплексная производная для a'} в круге \(z\in B(z_0, r)\)

              \[\lim \frac{f(z) - f(a)}{z - a} = \sum_{n = 1}^{ +\infty} a_n\lim \frac{(z - z_0)^n - (a - z_0)^n}{z - a} = \sum n a_n(a - z_0)^{n - 1}\]
    \end{enumerate}
\end{proof}

\end{document}