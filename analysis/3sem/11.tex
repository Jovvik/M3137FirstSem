\documentclass[12pt, a4paper]{article}

%<*preamble>
% Math symbols
\usepackage{amsmath, amsthm, amsfonts, amssymb}
\usepackage{accents}
\usepackage{esvect}
\usepackage{mathrsfs}
\usepackage{mathtools}
\mathtoolsset{showonlyrefs}
\usepackage{cmll}
\usepackage{stmaryrd}
\usepackage{physics}
\usepackage[normalem]{ulem}
\usepackage{ebproof}
\usepackage{extarrows}

% Page layout
\usepackage{geometry, a4wide, parskip, fancyhdr}

% Font, encoding, russian support
\usepackage[russian]{babel}
\usepackage[sb]{libertine}
\usepackage{xltxtra}

% Listings
\usepackage{listings}
\lstset{basicstyle=\ttfamily,breaklines=true}
\setmonofont{Inconsolata}

% Miscellaneous
\usepackage{array}
\usepackage{calc}
\usepackage{caption}
\usepackage{subcaption}
\captionsetup{justification=centering,margin=2cm}
\usepackage{catchfilebetweentags}
\usepackage{enumitem}
\usepackage{etoolbox}
\usepackage{float}
\usepackage{lastpage}
\usepackage{minted}
\usepackage{svg}
\usepackage{wrapfig}
\usepackage{xcolor}
\usepackage[makeroom]{cancel}

\newcolumntype{L}{>{$}l<{$}}
    \newcolumntype{C}{>{$}c<{$}}
\newcolumntype{R}{>{$}r<{$}}

% Footnotes
\usepackage[hang]{footmisc}
\setlength{\footnotemargin}{2mm}
\makeatletter
\def\blfootnote{\gdef\@thefnmark{}\@footnotetext}
\makeatother

% References
\usepackage{hyperref}
\hypersetup{
    colorlinks,
    linkcolor={blue!80!black},
    citecolor={blue!80!black},
    urlcolor={blue!80!black},
}

% tikz
\usepackage{tikz}
\usepackage{tikz-cd}
\usetikzlibrary{arrows.meta}
\usetikzlibrary{decorations.pathmorphing}
\usetikzlibrary{calc}
\usetikzlibrary{patterns}
\usepackage{pgfplots}
\pgfplotsset{width=10cm,compat=1.9}
\newcommand\irregularcircle[2]{% radius, irregularity
    \pgfextra {\pgfmathsetmacro\len{(#1)+rand*(#2)}}
    +(0:\len pt)
    \foreach \a in {10,20,...,350}{
            \pgfextra {\pgfmathsetmacro\len{(#1)+rand*(#2)}}
            -- +(\a:\len pt)
        } -- cycle
}

\providetoggle{useproofs}
\settoggle{useproofs}{false}

\pagestyle{fancy}
\lfoot{M3137y2019}
\cfoot{}
\rhead{стр. \thepage\ из \pageref*{LastPage}}

\newcommand{\R}{\mathbb{R}}
\newcommand{\Q}{\mathbb{Q}}
\newcommand{\Z}{\mathbb{Z}}
\newcommand{\B}{\mathbb{B}}
\newcommand{\N}{\mathbb{N}}
\renewcommand{\Re}{\mathfrak{R}}
\renewcommand{\Im}{\mathfrak{I}}

\newcommand{\const}{\text{const}}
\newcommand{\cond}{\text{cond}}

\newcommand{\teormin}{\textcolor{red}{!}\ }

\DeclareMathOperator*{\xor}{\oplus}
\DeclareMathOperator*{\equ}{\sim}
\DeclareMathOperator{\sign}{\text{sign}}
\DeclareMathOperator{\Sym}{\text{Sym}}
\DeclareMathOperator{\Asym}{\text{Asym}}

\DeclarePairedDelimiter{\ceil}{\lceil}{\rceil}

% godel
\newbox\gnBoxA
\newdimen\gnCornerHgt
\setbox\gnBoxA=\hbox{$\ulcorner$}
\global\gnCornerHgt=\ht\gnBoxA
\newdimen\gnArgHgt
\def\godel #1{%
    \setbox\gnBoxA=\hbox{$#1$}%
    \gnArgHgt=\ht\gnBoxA%
    \ifnum     \gnArgHgt<\gnCornerHgt \gnArgHgt=0pt%
    \else \advance \gnArgHgt by -\gnCornerHgt%
    \fi \raise\gnArgHgt\hbox{$\ulcorner$} \box\gnBoxA %
    \raise\gnArgHgt\hbox{$\urcorner$}}

% \theoremstyle{plain}

\theoremstyle{definition}
\newtheorem{theorem}{Теорема}
\newtheorem*{definition}{Определение}
\newtheorem{axiom}{Аксиома}
\newtheorem*{axiom*}{Аксиома}
\newtheorem{lemma}{Лемма}

\theoremstyle{remark}
\newtheorem*{remark}{Примечание}
\newtheorem*{exercise}{Упражнение}
\newtheorem{corollary}{Следствие}[theorem]
\newtheorem*{statement}{Утверждение}
\newtheorem*{corollary*}{Следствие}
\newtheorem*{example}{Пример}
\newtheorem{observation}{Наблюдение}
\newtheorem*{prop}{Свойства}
\newtheorem*{obozn}{Обозначение}

% subtheorem
\makeatletter
\newenvironment{subtheorem}[1]{%
    \def\subtheoremcounter{#1}%
    \refstepcounter{#1}%
    \protected@edef\theparentnumber{\csname the#1\endcsname}%
    \setcounter{parentnumber}{\value{#1}}%
    \setcounter{#1}{0}%
    \expandafter\def\csname the#1\endcsname{\theparentnumber.\Alph{#1}}%
    \ignorespaces
}{%
    \setcounter{\subtheoremcounter}{\value{parentnumber}}%
    \ignorespacesafterend
}
\makeatother
\newcounter{parentnumber}

\newtheorem{manualtheoreminner}{Теорема}
\newenvironment{manualtheorem}[1]{%
    \renewcommand\themanualtheoreminner{#1}%
    \manualtheoreminner
}{\endmanualtheoreminner}

\newcommand{\dbltilde}[1]{\accentset{\approx}{#1}}
\newcommand{\intt}{\int\!}

% magical thing that fixes paragraphs
\makeatletter
\patchcmd{\CatchFBT@Fin@l}{\endlinechar\m@ne}{}
{}{\typeout{Unsuccessful patch!}}
\makeatother

\newcommand{\get}[2]{
    \ExecuteMetaData[#1]{#2}
}

\newcommand{\getproof}[2]{
    \iftoggle{useproofs}{\ExecuteMetaData[#1]{#2proof}}{}
}

\newcommand{\getwithproof}[2]{
    \get{#1}{#2}
    \getproof{#1}{#2}
}

\newcommand{\import}[3]{
    \subsection{#1}
    \getwithproof{#2}{#3}
}

\newcommand{\given}[1]{
    Дано выше. (\ref{#1}, стр. \pageref{#1})
}

\renewcommand{\ker}{\text{Ker }}
\newcommand{\im}{\text{Im }}
\renewcommand{\grad}{\text{grad}}
\newcommand{\rg}{\text{rg}}
\newcommand{\defeq}{\stackrel{\text{def}}{=}}
\newcommand{\defeqfor}[1]{\stackrel{\text{def } #1}{=}}
\newcommand{\itemfix}{\leavevmode\makeatletter\makeatother}
\newcommand{\?}{\textcolor{red}{???}}
\renewcommand{\emptyset}{\varnothing}
\newcommand{\longarrow}[1]{\xRightarrow[#1]{\qquad}}
\DeclareMathOperator*{\esup}{\text{ess sup}}
\newcommand\smallO{
    \mathchoice
    {{\scriptstyle\mathcal{O}}}% \displaystyle
    {{\scriptstyle\mathcal{O}}}% \textstyle
    {{\scriptscriptstyle\mathcal{O}}}% \scriptstyle
    {\scalebox{.6}{$\scriptscriptstyle\mathcal{O}$}}%\scriptscriptstyle
}
\renewcommand{\div}{\text{div}\ }
\newcommand{\rot}{\text{rot}\ }
\newcommand{\cov}{\text{cov}}

\makeatletter
\newcommand{\oplabel}[1]{\refstepcounter{equation}(\theequation\ltx@label{#1})}
\makeatother

\newcommand{\symref}[2]{\stackrel{\oplabel{#1}}{#2}}
\newcommand{\symrefeq}[1]{\symref{#1}{=}}

% xrightrightarrows
\makeatletter
\newcommand*{\relrelbarsep}{.386ex}
\newcommand*{\relrelbar}{%
    \mathrel{%
        \mathpalette\@relrelbar\relrelbarsep
    }%
}
\newcommand*{\@relrelbar}[2]{%
    \raise#2\hbox to 0pt{$\m@th#1\relbar$\hss}%
    \lower#2\hbox{$\m@th#1\relbar$}%
}
\providecommand*{\rightrightarrowsfill@}{%
    \arrowfill@\relrelbar\relrelbar\rightrightarrows
}
\providecommand*{\leftleftarrowsfill@}{%
    \arrowfill@\leftleftarrows\relrelbar\relrelbar
}
\providecommand*{\xrightrightarrows}[2][]{%
    \ext@arrow 0359\rightrightarrowsfill@{#1}{#2}%
}
\providecommand*{\xleftleftarrows}[2][]{%
    \ext@arrow 3095\leftleftarrowsfill@{#1}{#2}%
}

\allowdisplaybreaks

\newcommand{\unfinished}{\textcolor{red}{Не дописано}}

% Reproducible pdf builds 
\special{pdf:trailerid [
<00112233445566778899aabbccddeeff>
<00112233445566778899aabbccddeeff>
]}
%</preamble>


\lhead{Математический анализ}
\cfoot{}
\rfoot{23.11.2020}

\begin{document}

\textcolor{red}{Первые 40 минут пропущены}

\subsection*{Степенные ряды}

% \begin{theorem}\itemfix
%     \begin{itemize}
%         \item \(\sum\limits_{n = 0}^{ +\infty} c_n\) --- сходится
%         \item \(c_n\in \C\)
%         \item \(f(x) =\sum c_nx^n\)
%         \item \(R\ge 1\)
%         \item \( - 1 < x < 1\)
%     \end{itemize}
%     Тогда \(\lim\limits_{x\to 1-0} f(x) = \)
% \end{theorem}

\begin{example}
    \[\sum_{n = 1}^{ +\infty} \frac{1}{n(n + 1)} = \sum \frac{1}{n} - \frac{1}{n + 1} = 1 - \lim_{N\to +\infty} \frac{1}{N + 1} = 1\]
    \[f(x) = \sum_{n = 1}^{ +\infty} \frac{x^{n + 1}}{n(n + 1)}\]
    \[f'(x) = \sum_{n = 1}^{ +\infty} \frac{x^{n}}{n}\]
    \[f''(x) = \sum_{n = 1}^{ +\infty} x^{n - 1} = 1 + x + x^2 + \dots = \frac{1}{1 - x}\]
    Интегрируем:
    \[f'(x) = \sum \frac{x^n}{n} = -\ln(1 - x) + C\]
    При \(x = 0\) \(C = 0\)

    Ещё раз интегрируем:
    \[\sum \frac{x^{n + 1}}{n(n + 1)} = -\int \ln(1 - x) dx = (1 - x)\ln{(1 - x)} + x + C\]
    При \(x = 0\) \(C = 0\)
    \[\sum \frac{1}{n(n + 1)} = \lim_{x\to 1 - 0} (1 - x)\ln(1 - x) + x = 1\]
\end{example}

\begin{corollary}[т. Абеля]\itemfix
    \begin{itemize}
        \item \(\sum a_n = A\)
        \item \(\sum b_n = B\)
        \item \(c_n = a_0b_n + a_1 b_{n - 1} + \dots + a_n b_0\)
        \item \(\sum c_n = C\)
    \end{itemize}
    Тогда \(C = AB\)
\end{corollary}
\begin{proof}
    \(f(x) = \sum a_n x^n, g(x) = \sum b_n x^n, h(x) = \sum c_nx^n, x\in[0, 1]\)

    При \(x = 1\) есть абсолютная сходимость \(f(x)\) и \(g(x)\). Можно перемножать: \(f(x)g(x) = h(x)\), при \(x\to 1 - 0\) \(A\cdot B = C\)
\end{proof}

\subsection*{Экспонента (комплексной переменной)}

\begin{definition}
    \[\exp(z) : = \sum_{n = 0}^{ +\infty} \frac{z^n}{n!} \]
\end{definition}

Свойства:
\begin{enumerate}
    \item \(\exp(0) = 1\)
    \item \(\exp(z)' = \sum\limits_{n = 1}^{ +\infty} \frac{z^{n - 1}}{(n - 1)!} = \sum\limits_{k = 0}^{ +\infty} \frac{z^{k}}{k!} = \exp z\)
\end{enumerate}

Возвращаем кредит: в первом семестре говорилось, что \(\exists f_0\) --- показательная функция, такая что \(f(x + y) = f(x)\cdot f(y)\) и \(\lim\limits_{x\to 0} \frac{f_0(x) - 1}{x} = 1\)

\(f_0(x) : = \exp(x)\)
\[\lim_{x\to 0} \frac{\exp(x) - 1}{x} = \exp'(0) = 1\]

\begin{theorem}
    \[\forall z, w\in \C \ \ \exp(z + w) = \exp z \cdot \exp w\]
\end{theorem}
\begin{exercise}
    Доказать к следующей лекции
\end{exercise}

\section*{Теория меры}

\subsection*{Системы множеств}

Здесь и далее система \(\iff \) множество, так говорится, чтобы избежать ``множество множеств''

\begin{obozn}
    \(A_i\) --- множества. Тогда \(\bigsqcup\limits_i A_i\) --- дизъюнктное объединение.

    \(A_i\) --- попарно не пересекаются \( \iff \) ``\(A_i\) --- дизъюнктно''
\end{obozn}

\begin{definition}
    \(X\) --- множество, тогда \(2^X\) --- система всех подмножеств \(X\).

    \(\mathcal P \subset 2^X\) --- \textbf{полукольцо}, если:
    \begin{itemize}
        \item \(\text{\O}\subset \mathcal P\)
        \item \(\forall A,B \in \mathcal P \ \ A\cap B\in \mathcal{P}\)
        \item \(\forall A, A' \in \mathcal{P} \ \ \exists \text{ кон. и дизъюнктные } B_1\dots B_n\in \mathcal{P} : A\setminus A' = \bigsqcup\limits_i B_i\)
    \end{itemize}
\end{definition}

\begin{example}
    \begin{enumerate}
        \item \(2^X\) --- полукольцо
        \item \(X = \R^2, \mathcal{P} =\) ограниченные подмножества, в том числе \O
        \item \begin{definition}
                  Ячейка в \(\R^m\) \([a, b) = \{x\in\R^m : \forall i \ \ x_i\in[a_i, b_i)\} \)
              \end{definition}

              \(\mathcal{P}\) --- множество ячеек в \(\R^m\) --- полукольцо.

              \begin{proof}
                  \(\sphericalangle m = 2\)
                  \begin{enumerate}
                      \item Очевидно
                      \item \(A\cap B = [a, a') \cap [b, b') = \{(x_1, x_2) \in\R^2 : \forall i = 1, 2 \ \ \max(a_i, b_i) \leq x_i < \min(a_i', b_i')\} \)
                      \item \(A\setminus A' = \bigsqcup\limits_{i = 1}^8 B_i \)
                  \end{enumerate}
              \end{proof}

              \begin{figure}[h]
                  \centering
                  \includesvg[scale=1]{images/вырезаниеячейки.svg}
              \end{figure}
        \item \(A = \{1, 2, 3,4,5,6\} \)

              \[\forall i \ \ A_i : = A\]

              \[X = \bigotimes\limits_{i = 1}^{ +\infty} A_i = \{(a_1, a_2, a_3, \dots) , \forall i\ \ a_i\in A_i\} \]

              \[\sigma = \begin{pmatrix} i_1 & i_2 & \dots & i_k \\ \alpha_1 & \alpha_2 & \dots & \alpha_k \end{pmatrix}, k\in\N\cup \{0\} , \forall l \ \ \alpha_l\in A_{i_l} \]

              \[\mathcal{P} = \{X_\sigma\}_\sigma \]

              \[X_\sigma = \{a\in X : a_{i_1} = \alpha_1, \dots, a_{i_k} = \alpha_k\} \]

              \(\mathcal{P}\) --- полукольцо

              \begin{enumerate}
                  \item \(\text{\O} = X_\sigma, \sigma = \begin{pmatrix} 1 & 1 \\ 1 & 2 \end{pmatrix} \)
                  \item \(X_\sigma\cap X_{\sigma'} = X_{\sigma\cup \sigma'}\)
                  \item \(X_\sigma \setminus X_{\sigma'}\)

                        \(\sigma = \begin{pmatrix} 1 \\ 6 \end{pmatrix} , \sigma' = \begin{pmatrix} 2 \end{pmatrix} 1\)

                        \(X_\sigma \setminus X_{\sigma'}\) = на первой координате 6, на второй --- не 1 = \(X_{\sigma_2} \cap X_{\sigma_3} \cap \dots \cap X_{\sigma_6}, \sigma_k = \begin{pmatrix} 1 & 2 \\ 6 & k \end{pmatrix} \).
              \end{enumerate}
        \item Ячейки с рациональными координатами.
    \end{enumerate}
\end{example}

Свойства:
\begin{enumerate}
    \item Как показывают примеры:
          \begin{enumerate}
              \item \(A\in \mathcal{P} \not \Rightarrow A^c = X\setminus A\in \mathcal{P}\)
              \item \(A, B\in \mathcal{P}\), нельзя утверждать, что:
                    \begin{itemize}
                        \item \(A\cap B\not\in \mathcal{P}\)
                        \item \(A\setminus  B\not\in \mathcal{P}\)
                        \item \(A\triangle  B = (A\setminus B)\cup (B\setminus A)\not\in \mathcal{P}\)
                    \end{itemize}
          \end{enumerate}
    \item Модернизируем утверждение 3:

          \(A, A_1 \dots A_n \in \mathcal{P}\). Тогда \(A\setminus (A_1\cup A_2 \cup \dots \cup A_n)\) --- представимо в виде дизъюнктного объединения элементом \(\mathcal{P}\)

          \begin{proof}
              Докажем по индукции по \(n\).

              База ( \(n = 1\)) --- аксиома 3.

              Переход:

              \begin{align*}
                  A\setminus (A_1\cup \dots \cup A_n) & = (A\setminus (A_1\cup \dots \cup A_{n - 1}))\setminus A_n \\
                                                      & = \left( \bigsqcup_{i = 1}^k B_i \right)\setminus A_n      \\
                                                      & = \bigsqcup_{i = 1}^k (B_i \setminus A_n)                  \\
                                                      & = \bigsqcup_{i = 1}^k \bigsqcup_{j = 1}^{l_i} D_{ij}       \\
              \end{align*}
          \end{proof}
\end{enumerate}

\begin{definition}
    \(\vartheta \subset 2^X\) --- алгебра подмножеств в \(X\):
    \begin{enumerate}
        \item \(\forall A,B\in \vartheta \ \ A\setminus B\in\vartheta\)
        \item \(X\in\vartheta\)
    \end{enumerate}
\end{definition}

Свойства:
\begin{enumerate}
    \item \(\text{\O} = X\setminus X\in \vartheta\)
    \item \(A\cap B = A\setminus (A\setminus B)\in \vartheta\)
    \item \(A^c = X\setminus A\vartheta\)
    \item \(A\cup B\in\vartheta\), потому что \((A\cup B)^c = A^c \cap B^c\)
    \item \(A_1\dots A_n \in \vartheta \Rightarrow \bigcup\limits_{i = 1}^n A_i, \bigcap\limits_{i = 1}^n A_i\in \vartheta\) --- по индукции
    \item Всякая алгебра есть полукольцо

          Обратное неверно, см. пример 2.
\end{enumerate}

\begin{example}
    \begin{enumerate}
        \item \(2^X\)
        \item \(X = \R^2, \vartheta\) --- ограниченные подмножества или их дополнения.
    \end{enumerate}
\end{example}

\end{document}