\documentclass[12pt, a4paper]{article}

%<*preamble>
% Math symbols
\usepackage{amsmath, amsthm, amsfonts, amssymb}
\usepackage{accents}
\usepackage{esvect}
\usepackage{mathrsfs}
\usepackage{mathtools}
\mathtoolsset{showonlyrefs}
\usepackage{cmll}
\usepackage{stmaryrd}
\usepackage{physics}
\usepackage[normalem]{ulem}
\usepackage{ebproof}
\usepackage{extarrows}

% Page layout
\usepackage{geometry, a4wide, parskip, fancyhdr}

% Font, encoding, russian support
\usepackage[russian]{babel}
\usepackage[sb]{libertine}
\usepackage{xltxtra}

% Listings
\usepackage{listings}
\lstset{basicstyle=\ttfamily,breaklines=true}
\setmonofont{Inconsolata}

% Miscellaneous
\usepackage{array}
\usepackage{calc}
\usepackage{caption}
\usepackage{subcaption}
\captionsetup{justification=centering,margin=2cm}
\usepackage{catchfilebetweentags}
\usepackage{enumitem}
\usepackage{etoolbox}
\usepackage{float}
\usepackage{lastpage}
\usepackage{minted}
\usepackage{svg}
\usepackage{wrapfig}
\usepackage{xcolor}
\usepackage[makeroom]{cancel}

\newcolumntype{L}{>{$}l<{$}}
    \newcolumntype{C}{>{$}c<{$}}
\newcolumntype{R}{>{$}r<{$}}

% Footnotes
\usepackage[hang]{footmisc}
\setlength{\footnotemargin}{2mm}
\makeatletter
\def\blfootnote{\gdef\@thefnmark{}\@footnotetext}
\makeatother

% References
\usepackage{hyperref}
\hypersetup{
    colorlinks,
    linkcolor={blue!80!black},
    citecolor={blue!80!black},
    urlcolor={blue!80!black},
}

% tikz
\usepackage{tikz}
\usepackage{tikz-cd}
\usetikzlibrary{arrows.meta}
\usetikzlibrary{decorations.pathmorphing}
\usetikzlibrary{calc}
\usetikzlibrary{patterns}
\usepackage{pgfplots}
\pgfplotsset{width=10cm,compat=1.9}
\newcommand\irregularcircle[2]{% radius, irregularity
    \pgfextra {\pgfmathsetmacro\len{(#1)+rand*(#2)}}
    +(0:\len pt)
    \foreach \a in {10,20,...,350}{
            \pgfextra {\pgfmathsetmacro\len{(#1)+rand*(#2)}}
            -- +(\a:\len pt)
        } -- cycle
}

\providetoggle{useproofs}
\settoggle{useproofs}{false}

\pagestyle{fancy}
\lfoot{M3137y2019}
\cfoot{}
\rhead{стр. \thepage\ из \pageref*{LastPage}}

\newcommand{\R}{\mathbb{R}}
\newcommand{\Q}{\mathbb{Q}}
\newcommand{\Z}{\mathbb{Z}}
\newcommand{\B}{\mathbb{B}}
\newcommand{\N}{\mathbb{N}}
\renewcommand{\Re}{\mathfrak{R}}
\renewcommand{\Im}{\mathfrak{I}}

\newcommand{\const}{\text{const}}
\newcommand{\cond}{\text{cond}}

\newcommand{\teormin}{\textcolor{red}{!}\ }

\DeclareMathOperator*{\xor}{\oplus}
\DeclareMathOperator*{\equ}{\sim}
\DeclareMathOperator{\sign}{\text{sign}}
\DeclareMathOperator{\Sym}{\text{Sym}}
\DeclareMathOperator{\Asym}{\text{Asym}}

\DeclarePairedDelimiter{\ceil}{\lceil}{\rceil}

% godel
\newbox\gnBoxA
\newdimen\gnCornerHgt
\setbox\gnBoxA=\hbox{$\ulcorner$}
\global\gnCornerHgt=\ht\gnBoxA
\newdimen\gnArgHgt
\def\godel #1{%
    \setbox\gnBoxA=\hbox{$#1$}%
    \gnArgHgt=\ht\gnBoxA%
    \ifnum     \gnArgHgt<\gnCornerHgt \gnArgHgt=0pt%
    \else \advance \gnArgHgt by -\gnCornerHgt%
    \fi \raise\gnArgHgt\hbox{$\ulcorner$} \box\gnBoxA %
    \raise\gnArgHgt\hbox{$\urcorner$}}

% \theoremstyle{plain}

\theoremstyle{definition}
\newtheorem{theorem}{Теорема}
\newtheorem*{definition}{Определение}
\newtheorem{axiom}{Аксиома}
\newtheorem*{axiom*}{Аксиома}
\newtheorem{lemma}{Лемма}

\theoremstyle{remark}
\newtheorem*{remark}{Примечание}
\newtheorem*{exercise}{Упражнение}
\newtheorem{corollary}{Следствие}[theorem]
\newtheorem*{statement}{Утверждение}
\newtheorem*{corollary*}{Следствие}
\newtheorem*{example}{Пример}
\newtheorem{observation}{Наблюдение}
\newtheorem*{prop}{Свойства}
\newtheorem*{obozn}{Обозначение}

% subtheorem
\makeatletter
\newenvironment{subtheorem}[1]{%
    \def\subtheoremcounter{#1}%
    \refstepcounter{#1}%
    \protected@edef\theparentnumber{\csname the#1\endcsname}%
    \setcounter{parentnumber}{\value{#1}}%
    \setcounter{#1}{0}%
    \expandafter\def\csname the#1\endcsname{\theparentnumber.\Alph{#1}}%
    \ignorespaces
}{%
    \setcounter{\subtheoremcounter}{\value{parentnumber}}%
    \ignorespacesafterend
}
\makeatother
\newcounter{parentnumber}

\newtheorem{manualtheoreminner}{Теорема}
\newenvironment{manualtheorem}[1]{%
    \renewcommand\themanualtheoreminner{#1}%
    \manualtheoreminner
}{\endmanualtheoreminner}

\newcommand{\dbltilde}[1]{\accentset{\approx}{#1}}
\newcommand{\intt}{\int\!}

% magical thing that fixes paragraphs
\makeatletter
\patchcmd{\CatchFBT@Fin@l}{\endlinechar\m@ne}{}
{}{\typeout{Unsuccessful patch!}}
\makeatother

\newcommand{\get}[2]{
    \ExecuteMetaData[#1]{#2}
}

\newcommand{\getproof}[2]{
    \iftoggle{useproofs}{\ExecuteMetaData[#1]{#2proof}}{}
}

\newcommand{\getwithproof}[2]{
    \get{#1}{#2}
    \getproof{#1}{#2}
}

\newcommand{\import}[3]{
    \subsection{#1}
    \getwithproof{#2}{#3}
}

\newcommand{\given}[1]{
    Дано выше. (\ref{#1}, стр. \pageref{#1})
}

\renewcommand{\ker}{\text{Ker }}
\newcommand{\im}{\text{Im }}
\renewcommand{\grad}{\text{grad}}
\newcommand{\rg}{\text{rg}}
\newcommand{\defeq}{\stackrel{\text{def}}{=}}
\newcommand{\defeqfor}[1]{\stackrel{\text{def } #1}{=}}
\newcommand{\itemfix}{\leavevmode\makeatletter\makeatother}
\newcommand{\?}{\textcolor{red}{???}}
\renewcommand{\emptyset}{\varnothing}
\newcommand{\longarrow}[1]{\xRightarrow[#1]{\qquad}}
\DeclareMathOperator*{\esup}{\text{ess sup}}
\newcommand\smallO{
    \mathchoice
    {{\scriptstyle\mathcal{O}}}% \displaystyle
    {{\scriptstyle\mathcal{O}}}% \textstyle
    {{\scriptscriptstyle\mathcal{O}}}% \scriptstyle
    {\scalebox{.6}{$\scriptscriptstyle\mathcal{O}$}}%\scriptscriptstyle
}
\renewcommand{\div}{\text{div}\ }
\newcommand{\rot}{\text{rot}\ }
\newcommand{\cov}{\text{cov}}

\makeatletter
\newcommand{\oplabel}[1]{\refstepcounter{equation}(\theequation\ltx@label{#1})}
\makeatother

\newcommand{\symref}[2]{\stackrel{\oplabel{#1}}{#2}}
\newcommand{\symrefeq}[1]{\symref{#1}{=}}

% xrightrightarrows
\makeatletter
\newcommand*{\relrelbarsep}{.386ex}
\newcommand*{\relrelbar}{%
    \mathrel{%
        \mathpalette\@relrelbar\relrelbarsep
    }%
}
\newcommand*{\@relrelbar}[2]{%
    \raise#2\hbox to 0pt{$\m@th#1\relbar$\hss}%
    \lower#2\hbox{$\m@th#1\relbar$}%
}
\providecommand*{\rightrightarrowsfill@}{%
    \arrowfill@\relrelbar\relrelbar\rightrightarrows
}
\providecommand*{\leftleftarrowsfill@}{%
    \arrowfill@\leftleftarrows\relrelbar\relrelbar
}
\providecommand*{\xrightrightarrows}[2][]{%
    \ext@arrow 0359\rightrightarrowsfill@{#1}{#2}%
}
\providecommand*{\xleftleftarrows}[2][]{%
    \ext@arrow 3095\leftleftarrowsfill@{#1}{#2}%
}

\allowdisplaybreaks

\newcommand{\unfinished}{\textcolor{red}{Не дописано}}

% Reproducible pdf builds 
\special{pdf:trailerid [
<00112233445566778899aabbccddeeff>
<00112233445566778899aabbccddeeff>
]}
%</preamble>


\lhead{Математический анализ}
\cfoot{}
\rfoot{30.11.2020}

\begin{document}

\begin{theorem}
    \[\forall z, w\in \C \ \ \exp(z + w) = \exp z \cdot \exp w\]
\end{theorem}
\begin{proof}
    \[\exp z \cdot \exp w = \sum_{n = 0}^{ +\infty} \frac{z^n}{n!} \cdot \sum_{n = 0}^{ +\infty} \frac{w^n}{n!} = \sum C_n\]
    \begin{align*}
        C_n & = \frac{1}{n!}\sum_{k = 0}^n \frac{z^k}{k!} \frac{w^{n - k}}{(n - k)!} n! \\
            & = \frac{1}{n!} \sum_{k = 0}^n z^k w^{n - k} C_n^k                         \\
            & = \frac{(z + w)^n}{n!}
    \end{align*}
    \[\sum_{n = 0}^{ +\infty} \frac{z^n}{n!} \cdot \sum_{n = 0}^{ +\infty} \frac{w^n}{n!} = \sum \frac{(z + w)^n}{n!} = \exp(z + w)\]
\end{proof}

\begin{corollary}
    \(\forall z\in \C\ \ \exp z \neq 0\)
\end{corollary}

\(\overline{\exp(z)} = \exp(\overline z)\), потому что коэффициенты вещественные и
\[\overline{\sum_{n = 0}^N \frac{z^n}{n!}} = \sum_{n = 0}^N \frac{(\overline z)^n}{n!}\]

\begin{remark}[о тригонометрических функциях]
    Пусть \(\exp(ix) = \text{Cos}(x) + i\text{Sin}(x), x\in\R\)

    Тогда \(\exp( - ix) = \text{Cos}(x) - i\text{Sin}(x)\)

    \[\text{Cos}(x) = \frac{\exp(ix) + \exp( - ix)}{2} \quad \text{Sin}(x) = \frac{\exp(ix) - \exp(-ix)}{2i}\]

    Следовательно:

    \[\text{Cos}(x) = 1 - \frac{x^2}{2!} + \frac{x^4}{4!} + \dots \quad \text{Sin}(x) = x - \frac{x^3}{3!} + \dots \]

    Пусть \(T(x) = \exp(ix)\). Тогда \(T(x + y) = T(x) T(y)\).

    \[\text{Cos}(x + y) + i\text{Sin}(x + y) = (\text{Cos}(x) + i\text{Sin}(x))(\text{Cos}(y) + i\text{Sin}(y))\]
    \[\text{Cos}(x + y) = \text{Cos} x \text{Cos} y - \text{Sin} x \text{Sin} y\]
    \[\text{Sin}(x + y) = \text{Cos} x \text{Sin} y + \text{Sin} x \text{Cos} y\]
    \[|T(x)|^2 = T(x) \overline{T(x)} = \exp(ix) \exp( - ix) = \exp(0) = 1\]
    То есть \((\text{Cos} x, \text{Sin} x)\) --- точка на единичной окружности.
    \(T' = iT\), то есть \(x \mapsto T(x)\) --- движение по единичной окружности с единичным вектором скорости, перпендикулярным радиус-вектору.

    Мы неформально показали, что \(\text{Cos} = \cos, \text{Sin} = \sin\). Для более строго доказательства см. Рудин ``Основы математического анализа''.
\end{remark}

\section*{Ряды Тейлора}

В этом параграфе все вещественно.

\begin{definition}
    %<*разлагаетсявстепеннойряд>
    \(f\) --- \textbf{разлагается в степенной ряд} в окрестности \(x_0\), если:
    \begin{equation}
        \exists \varepsilon > 0\ \ \exists C_n \text{ --- вещ. посл.}\ \ \forall x\in(x_0 - \varepsilon, x_0 + \varepsilon)\ \ f(x) = \sum_{n=0}^{+\infty} C_n(x - x_0)^n \label{степеннойряд}
    \end{equation}
    %</разлагаетсявстепеннойряд>
\end{definition}
\begin{remark}
    Тогда \(f\in C^{\infty}(x_0 - \varepsilon, x_0 + \varepsilon)\)
\end{remark}

%<*оединственности>
\begin{theorem}[о единственности]
    \(f\) разлагается в степенной ряд в окрестности \(x_0\). Тогда разложение единственно.
\end{theorem}
%</оединственности>
%<*оединственностиproof>
\begin{proof}
    Выполняется \eqref{степеннойряд}.

    База:
    \[C_0 = f(x_0) \quad f'(x) = \sum_{n = 1}^{ +\infty} n C_n(x - x_0)^{n - 1} \]

    Переход:

    \[f^{(k)} = \sum_{n = k}^{ +\infty} n(n - 1)\dots (n - k + 1)C_n (x - x_0)^{n - k} \Rightarrow C_k = \frac{f^{(k)}(x_0)}{k!}\]
\end{proof}
%</оединственностиproof>
\begin{definition}
    Ряд Тейлора функции \(f\) в точке \(x_0\) --- формальный ряд \(\sum \frac{f^{(n)}(x_0)}{n!} (x - x_0)^n\)
\end{definition}

\begin{remark}\itemfix
    \begin{enumerate}
        \item Ряд Тейлора может оказаться сходящимся только при \(x = x_0\)
        \item Ряд Тейлора может сходиться \textbf{не туда}.
    \end{enumerate}
\end{remark}

\begin{example}[2]
    \[f(x) = \begin{cases}
            e^{-1/x^2} & , x\neq 0 \\
            0          & , x = 0   \\
        \end{cases}, f\in C^{\infty}(\R)\]

    При \(x = 0\ \ \forall n \ \ f^{(n)}(0) = 0\) --- мы это доказывали в прошлом семестре.

    Ряд Тейлора в \(x_0 = 0\) тождественно равен нулю. Очевидно в других точках ряд не сходится к \(f\).
\end{example}

\begin{example}[1, Кошмарный сон КПК]
    \[f(t) = \int_0^{ +\infty} \frac{e^{ - x}}{1 + t^2x} dx, t\in\R\]
    \begin{align*}
        f(t) & = \int_0^{ +\infty} \sum_{n = 0}^{ +\infty} e^{ - x}( - 1)^n t^{2n} x^n dx  \\
             & = \sum_{n = 0}^{ +\infty} \int_0^{ +\infty} ( - 1)^n t^{2n} x^n e^{ - x} dx \\
             & = \sum_{n = 0}^{ +\infty} ( - 1)^n t^{2n} \int_0^{ +\infty} x^n e^{ - x} dx \\
             & = \sum_{n = 0}^{ +\infty} ( - 1)^n t^{2n} \Gamma(n + 1)                     \\
             & = \sum_{n = 0}^{ +\infty} ( - 1)^n t^{2n} n!                                \\
    \end{align*}

    Этот ряд расходится при \(t\neq 0\), поэтому все равенства неверные.

    \(f(t)\in C^{\infty}(\R)\) по обобщенному правилу Лейбница, \(s = \frac{t^2}{t^2x + 1}, f(t) = \dots \int^{t^2} \frac{1}{s} e^{ - 1 / s} ds\)

    \[\frac{1}{1 + t^2x} = 1 - t^2x + \dots + ( - 1)^n (t^2x)^n + \frac{( - 1)^{n + 1} (t^2x)^{n + 1}}{1 + t^2x}\]
    \begin{align*}
        f(t)
         & = \int_0^{ +\infty} \sum_{k = 0}^n e^{ - x}( - 1)^k t^{2k} x^k dx + \int_0^{ +\infty} ( - 1)^{n + 1} \frac{(t^2x)^{n + 1} e^{ - x}}{1 + t^2 x} dx                                                                                \\
         & = \sum_{k = 0}^n \int_0^{ +\infty} e^{ - x}( - 1)^k t^{2k} x^k dx + ( - 1)^{n + 1} t^{2n + 2} \underbrace{\int_0^{ +\infty} \frac{x^{n + 1} e^{ - x}}{1 + t^2 x} dx}_{\text{огр.}, \le \int_0^{+\infty} e^{-x} x^{n+1} = (n+1)!} \\
         & = \sum_{k = 0}^n ( - 1)^k k! t^{2n} + O(t^{2n + 2})                                                                                                                                                                              \\
    \end{align*}

    Это формула Тейлора для \(f\) в точке \(t_0 = 0.1\), т.е. \(f^{(2n)}(0) = ( - 1)^n n!2n!\)
\end{example}

\begin{definition}
    %<*сигмаалгебра>
    \(\sigma\)-алгебра \(\mathfrak A \subset 2^X\):
    \begin{enumerate}
        \item \(\mathfrak A\) --- алгебра
        \item \(A_1, A_2, \dots \in \mathfrak A \Rightarrow \bigcup\limits_{i = 1}^{\infty} A_i \in \mathfrak A\)
    \end{enumerate}
    %</сигмаалгебра>
\end{definition}

\begin{remark}
    \(A_1, A_2, \dots \in \mathfrak A\). Тогда \(\bigcap\limits_{i = 1}^{ +\infty} A_i\in \mathfrak A\)
\end{remark}
\begin{proof}
    \[X\setminus \bigcap\limits_{i = 1}^{ +\infty} A_i = \bigcup\limits_{i = 1}^{ +\infty} X\setminus A_i \in \mathfrak A\]
\end{proof}

\begin{remark}
    \(E\in \mathfrak A\), \(\mathfrak A\) --- \(\sigma\)-алгебра

    Тогда \(\mathfrak A_E : = \{A\in \mathfrak A : A\subset E\} \) --- \(\sigma\)-алгебра подмножеств \(E\)
\end{remark}

\begin{example}\itemfix
    \begin{enumerate}
        \item \(2^X\)
        \item \(X\) --- бесконечное множество. \(\mathfrak A\) --- не более чем счётные множества и их дополнения
        \item \(X =\R^2, \mathfrak A\) --- ограниченные множества и их дополнения --- не \(\sigma\)-алгебра
    \end{enumerate}
\end{example}

\begin{exercise}
    \begin{enumerate}
        \item \(A_1, A_2, \dots \subset X\)

              \(B_1 = A_1, B_2 = A_2\setminus A_1 \dots B_k = A_k \setminus \bigcup\limits_{i = 1}^{k - 1}A_i\), тогда \(B_k\) --- дизъюнктны. \(\bigsqcup B_k \stackrel{?}{ = } \bigcup A_i\)

        \item \(\mathcal{P}\) --- полукольцо. \(\mathfrak A_0 : =\) конечные объединения множеств в \(\mathcal{P}\) и их дополнений. Доказать: \(\mathfrak A_0\) --- алгебра.
        \item \(\mathcal{P}\) --- полукольцо. \(\mathfrak A \supset \mathcal{P}\) --- алгебра. Доказать: \(\mathfrak A\supset \mathfrak A_0\)
    \end{enumerate}
\end{exercise}

\begin{definition}
    %<*аддитивнаяфункция>
    \(\mu : \underbrace{\mathcal{P}}_{\text{полукольцо}} \to \overline \R\) --- \textbf{аддитивная функция множества}, если:
    \begin{enumerate}
        \item \(\mu\) не должна принимать значения \(\pm \infty\) одновременно
        \item \(\mu(\text{\O}) = 0\)
        \item \(\forall A_1 \dots A_n \in \mathcal{P}\), дизъюнктны. Если \(A = \bigsqcup A_i \in \mathcal{P}\), то \(\mu A = \sum\limits_{i = 1}^n \mu A_i\)
    \end{enumerate}
    %</аддитивнаяфункция>
\end{definition}

\begin{definition}
    %<*объем>
    \(\mu : \mathcal{P} \to \overline \R\) --- \textbf{объем}, если \(\mu \ge 0\) и \(\mu\) --- аддитивная.
    %</объем>
\end{definition}

\begin{remark}
    \begin{enumerate}
        \item Если \(X\in \mathcal{P}, \mu X < +\infty\), то говорят, что \(\mu\) --- конечный объем
        \item \(\mu\) --- задано на \(\mathfrak{A}\) : 3 \(\leftrightarrow\) 3':
        \item [3'] \(\forall A, B\in \mathfrak{A}, A\cap B =\text{\O} \ \ \mu(A\cup B) = \mu A + \mu B\)
    \end{enumerate}
\end{remark}

\begin{example}\itemfix
    \begin{enumerate}
        \item \(\mathcal{P}^1\) --- ячейки в \(\R^1\). \(\mu[a, b) = b - a, b\ge a\)

              \[\sphericalangle x_0 = a < x_1 < x_2 < \dots < x_n = b\]
              \[[a, b) = \bigcup_{k = 1}^n [x_{k - 1}, x_k)\]
              \[\mu[a, b) = b - a = x_n - x_0 = \sum_k (x_k - x_{k - 1}) = \sum_k \mu[x_{k - 1}, x_k)\]

        \item \(g : \R \to \R\) --- возрастает, непрерывно

              \(\forall [\alpha, \beta) \ \ \vartheta : \mathcal{P}^1 \to \R, \vartheta [\alpha, \beta) = g(\beta) - g(\alpha)\) --- тоже объем.

        \item %<*классическийобъем>
              Классический объем в \(\R^m\) \(\mu : \mathcal{P}^m \to \R\)

              \[\mu [a, b) = \prod\limits_{i = 1}^m (b_i - a_i)\]

              Этот объем не конечный.
              %</классическийобъем>
        \item Наглый пример. \(\sphericalangle \R^2\)

              \(\mathfrak{A}\) --- алгебра ограниченных множеств и их дополнений.

              \[\mu A = \begin{cases}
                      0 & , A \text{ --- огр.}                  \\
                      1 & , A \text{ --- имеет огр. дополнение}
                  \end{cases}\]

              Наглость заключается в том, что \(\mu\) принимает значение 1, аддитивно, но не принимает значение 2. Это происходит, потому что нет двух дизъюнктных множеств, которые имеют ограниченное дополнение.

              Этот объем конечный.
    \end{enumerate}
\end{example}

\begin{definition}
    Свойство \(A\subset B \Rightarrow \mu A \le \mu B\) называется \textbf{монотонностью объема}.
\end{definition}

\begin{theorem}
    %<*свойстваобъема>
    \(\mu : \mathcal{P} \to \overline \R\) --- объем. Тогда \(\mu\) имеет свойства:
    \begin{enumerate}
        \item Усиленная монотонность
              \[\forall A, \underbrace{A_1, A_2, \dots A_n}_{\text{дизъюнктны}} \in \mathcal{P} \ \ \bigsqcup_{i = 1}^n A_i \subset A \ \ \sum_{i = 1}^n \mu A_i \le \mu A\]
        \item Конечная полуаддитивность
              \[\forall A, A_1, A_2, \dots A_n \in \mathcal{P} \ \ A\subset \bigcup_{i = 1}^n A_i \ \ \mu A \le \sum_{i = 1}^n \mu A_i\]
        \item \(\forall A, B \in \mathcal{P}\) пусть ещё известно \(A\setminus B \in \mathcal{P}, \mu(B)\) --- конечно. Тогда \(\mu(A\setminus B) \ge \mu A - \mu B\)
    \end{enumerate}
    %</свойстваобъема>
\end{theorem}

%<*свойстваобъемаremark>
\begin{remark}
    \begin{enumerate}
        \item В пунтах 1 и 2 не предполагается, что \(\bigcup A_i \in \mathcal{P}\)
        \item В пункте 3 если \(\mathcal{P}\) --- алгебра, условие \(A\setminus B\in \mathcal{P}\) можно убрать.
    \end{enumerate}
\end{remark}
%</свойстваобъемаremark>

%<*свойстваобъемаproof>
\begin{proof}\itemfix
    \begin{enumerate}
        \item Усиление аксиомы 3 из определения полукольца:
              \[A\setminus \left( \bigcup_{i = 1}^n A_i \right) = \bigsqcup_{l = 1}^S B_l\]
              Это было доказано ранее.
              Теорема \? \(A = \left( \bigsqcup A_i \right) \cup \left( \bigsqcup B_l \right)\) --- дизъюнктное объединение конечного числа множеств.
              \[\mu A = \sum \mu A_i + \sum \mu B_l \ge \sum \mu A_i\]

              \get{13.tex}{второесвойствообъемаproof}
        \item [3.] \begin{enumerate}
                  \item \(B \subset A \quad A = B \cup (A\setminus B) \quad \mu A = \mu B + \mu(A\setminus B)\)
                  \item \(B \not\subset A \quad A\setminus B = A\setminus (A\cap B) \quad \mu(A\setminus B) = \mu A - \mu(A\cap B) \ge \mu A - \mu B\)
              \end{enumerate}
    \end{enumerate}
\end{proof}
%</свойстваобъемаproof>

\end{document}