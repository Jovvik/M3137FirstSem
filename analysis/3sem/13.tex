\documentclass[12pt, a4paper]{article}

%<*preamble>
% Math symbols
\usepackage{amsmath, amsthm, amsfonts, amssymb}
\usepackage{accents}
\usepackage{esvect}
\usepackage{mathrsfs}
\usepackage{mathtools}
\mathtoolsset{showonlyrefs}
\usepackage{cmll}
\usepackage{stmaryrd}
\usepackage{physics}
\usepackage[normalem]{ulem}
\usepackage{ebproof}
\usepackage{extarrows}

% Page layout
\usepackage{geometry, a4wide, parskip, fancyhdr}

% Font, encoding, russian support
\usepackage[russian]{babel}
\usepackage[sb]{libertine}
\usepackage{xltxtra}

% Listings
\usepackage{listings}
\lstset{basicstyle=\ttfamily,breaklines=true}
\setmonofont{Inconsolata}

% Miscellaneous
\usepackage{array}
\usepackage{calc}
\usepackage{caption}
\usepackage{subcaption}
\captionsetup{justification=centering,margin=2cm}
\usepackage{catchfilebetweentags}
\usepackage{enumitem}
\usepackage{etoolbox}
\usepackage{float}
\usepackage{lastpage}
\usepackage{minted}
\usepackage{svg}
\usepackage{wrapfig}
\usepackage{xcolor}
\usepackage[makeroom]{cancel}

\newcolumntype{L}{>{$}l<{$}}
    \newcolumntype{C}{>{$}c<{$}}
\newcolumntype{R}{>{$}r<{$}}

% Footnotes
\usepackage[hang]{footmisc}
\setlength{\footnotemargin}{2mm}
\makeatletter
\def\blfootnote{\gdef\@thefnmark{}\@footnotetext}
\makeatother

% References
\usepackage{hyperref}
\hypersetup{
    colorlinks,
    linkcolor={blue!80!black},
    citecolor={blue!80!black},
    urlcolor={blue!80!black},
}

% tikz
\usepackage{tikz}
\usepackage{tikz-cd}
\usetikzlibrary{arrows.meta}
\usetikzlibrary{decorations.pathmorphing}
\usetikzlibrary{calc}
\usetikzlibrary{patterns}
\usepackage{pgfplots}
\pgfplotsset{width=10cm,compat=1.9}
\newcommand\irregularcircle[2]{% radius, irregularity
    \pgfextra {\pgfmathsetmacro\len{(#1)+rand*(#2)}}
    +(0:\len pt)
    \foreach \a in {10,20,...,350}{
            \pgfextra {\pgfmathsetmacro\len{(#1)+rand*(#2)}}
            -- +(\a:\len pt)
        } -- cycle
}

\providetoggle{useproofs}
\settoggle{useproofs}{false}

\pagestyle{fancy}
\lfoot{M3137y2019}
\cfoot{}
\rhead{стр. \thepage\ из \pageref*{LastPage}}

\newcommand{\R}{\mathbb{R}}
\newcommand{\Q}{\mathbb{Q}}
\newcommand{\Z}{\mathbb{Z}}
\newcommand{\B}{\mathbb{B}}
\newcommand{\N}{\mathbb{N}}
\renewcommand{\Re}{\mathfrak{R}}
\renewcommand{\Im}{\mathfrak{I}}

\newcommand{\const}{\text{const}}
\newcommand{\cond}{\text{cond}}

\newcommand{\teormin}{\textcolor{red}{!}\ }

\DeclareMathOperator*{\xor}{\oplus}
\DeclareMathOperator*{\equ}{\sim}
\DeclareMathOperator{\sign}{\text{sign}}
\DeclareMathOperator{\Sym}{\text{Sym}}
\DeclareMathOperator{\Asym}{\text{Asym}}

\DeclarePairedDelimiter{\ceil}{\lceil}{\rceil}

% godel
\newbox\gnBoxA
\newdimen\gnCornerHgt
\setbox\gnBoxA=\hbox{$\ulcorner$}
\global\gnCornerHgt=\ht\gnBoxA
\newdimen\gnArgHgt
\def\godel #1{%
    \setbox\gnBoxA=\hbox{$#1$}%
    \gnArgHgt=\ht\gnBoxA%
    \ifnum     \gnArgHgt<\gnCornerHgt \gnArgHgt=0pt%
    \else \advance \gnArgHgt by -\gnCornerHgt%
    \fi \raise\gnArgHgt\hbox{$\ulcorner$} \box\gnBoxA %
    \raise\gnArgHgt\hbox{$\urcorner$}}

% \theoremstyle{plain}

\theoremstyle{definition}
\newtheorem{theorem}{Теорема}
\newtheorem*{definition}{Определение}
\newtheorem{axiom}{Аксиома}
\newtheorem*{axiom*}{Аксиома}
\newtheorem{lemma}{Лемма}

\theoremstyle{remark}
\newtheorem*{remark}{Примечание}
\newtheorem*{exercise}{Упражнение}
\newtheorem{corollary}{Следствие}[theorem]
\newtheorem*{statement}{Утверждение}
\newtheorem*{corollary*}{Следствие}
\newtheorem*{example}{Пример}
\newtheorem{observation}{Наблюдение}
\newtheorem*{prop}{Свойства}
\newtheorem*{obozn}{Обозначение}

% subtheorem
\makeatletter
\newenvironment{subtheorem}[1]{%
    \def\subtheoremcounter{#1}%
    \refstepcounter{#1}%
    \protected@edef\theparentnumber{\csname the#1\endcsname}%
    \setcounter{parentnumber}{\value{#1}}%
    \setcounter{#1}{0}%
    \expandafter\def\csname the#1\endcsname{\theparentnumber.\Alph{#1}}%
    \ignorespaces
}{%
    \setcounter{\subtheoremcounter}{\value{parentnumber}}%
    \ignorespacesafterend
}
\makeatother
\newcounter{parentnumber}

\newtheorem{manualtheoreminner}{Теорема}
\newenvironment{manualtheorem}[1]{%
    \renewcommand\themanualtheoreminner{#1}%
    \manualtheoreminner
}{\endmanualtheoreminner}

\newcommand{\dbltilde}[1]{\accentset{\approx}{#1}}
\newcommand{\intt}{\int\!}

% magical thing that fixes paragraphs
\makeatletter
\patchcmd{\CatchFBT@Fin@l}{\endlinechar\m@ne}{}
{}{\typeout{Unsuccessful patch!}}
\makeatother

\newcommand{\get}[2]{
    \ExecuteMetaData[#1]{#2}
}

\newcommand{\getproof}[2]{
    \iftoggle{useproofs}{\ExecuteMetaData[#1]{#2proof}}{}
}

\newcommand{\getwithproof}[2]{
    \get{#1}{#2}
    \getproof{#1}{#2}
}

\newcommand{\import}[3]{
    \subsection{#1}
    \getwithproof{#2}{#3}
}

\newcommand{\given}[1]{
    Дано выше. (\ref{#1}, стр. \pageref{#1})
}

\renewcommand{\ker}{\text{Ker }}
\newcommand{\im}{\text{Im }}
\renewcommand{\grad}{\text{grad}}
\newcommand{\rg}{\text{rg}}
\newcommand{\defeq}{\stackrel{\text{def}}{=}}
\newcommand{\defeqfor}[1]{\stackrel{\text{def } #1}{=}}
\newcommand{\itemfix}{\leavevmode\makeatletter\makeatother}
\newcommand{\?}{\textcolor{red}{???}}
\renewcommand{\emptyset}{\varnothing}
\newcommand{\longarrow}[1]{\xRightarrow[#1]{\qquad}}
\DeclareMathOperator*{\esup}{\text{ess sup}}
\newcommand\smallO{
    \mathchoice
    {{\scriptstyle\mathcal{O}}}% \displaystyle
    {{\scriptstyle\mathcal{O}}}% \textstyle
    {{\scriptscriptstyle\mathcal{O}}}% \scriptstyle
    {\scalebox{.6}{$\scriptscriptstyle\mathcal{O}$}}%\scriptscriptstyle
}
\renewcommand{\div}{\text{div}\ }
\newcommand{\rot}{\text{rot}\ }
\newcommand{\cov}{\text{cov}}

\makeatletter
\newcommand{\oplabel}[1]{\refstepcounter{equation}(\theequation\ltx@label{#1})}
\makeatother

\newcommand{\symref}[2]{\stackrel{\oplabel{#1}}{#2}}
\newcommand{\symrefeq}[1]{\symref{#1}{=}}

% xrightrightarrows
\makeatletter
\newcommand*{\relrelbarsep}{.386ex}
\newcommand*{\relrelbar}{%
    \mathrel{%
        \mathpalette\@relrelbar\relrelbarsep
    }%
}
\newcommand*{\@relrelbar}[2]{%
    \raise#2\hbox to 0pt{$\m@th#1\relbar$\hss}%
    \lower#2\hbox{$\m@th#1\relbar$}%
}
\providecommand*{\rightrightarrowsfill@}{%
    \arrowfill@\relrelbar\relrelbar\rightrightarrows
}
\providecommand*{\leftleftarrowsfill@}{%
    \arrowfill@\leftleftarrows\relrelbar\relrelbar
}
\providecommand*{\xrightrightarrows}[2][]{%
    \ext@arrow 0359\rightrightarrowsfill@{#1}{#2}%
}
\providecommand*{\xleftleftarrows}[2][]{%
    \ext@arrow 3095\leftleftarrowsfill@{#1}{#2}%
}

\allowdisplaybreaks

\newcommand{\unfinished}{\textcolor{red}{Не дописано}}

% Reproducible pdf builds 
\special{pdf:trailerid [
<00112233445566778899aabbccddeeff>
<00112233445566778899aabbccddeeff>
]}
%</preamble>


\lhead{Математический анализ}
\cfoot{}
\rfoot{7.12.2020}

\begin{document}

\section*{Ряды Тейлора}

\begin{example}
    \begin{align*}
        e^x             & = \sum_{n = 0}^{ +\infty},\ x\in\R                                             \\
        \sin x          & = \sum_{n = 1}^{ +\infty} ( - 1)^{n - 1} \frac{x^{2n - 1}}{(2n - 1)!},\ x\in\R \\
        \cos x          & = \sum ( - 1)^n \frac{x^{2n}}{(2n)!},\ x\in\R                                  \\
        \frac{1}{1 + x} & = \sum_{n = 0}^{ +\infty} ( -1)^nx^n,\ x\in( - 1, 1)                           \\
        \ln(1 + x)      & = \sum_{n = 0}^{ +\infty} ( - 1)^{n - 1}\frac{x^n}{n},\ x\in( - 1, 1)          \\
    \end{align*}
\end{example}

\begin{theorem}
    %<*разложениебинома>
    \(\forall \sigma\in\R \ \ \forall x\in( - 1, 1)\)
    \[(1 + x)^{\sigma} = 1 + \sigma x + \frac{\sigma(\sigma - 1)}{2} x^2 + \dots + \binom{\sigma}{n} x^n + \dots\]
    %</разложениебинома>
\end{theorem}
%<*разложениебиномаproof>
\begin{proof}
    При \(|x|< 1\) ряд сходится по признаку Даламбера:

    \[\left|\frac{a_{n+1}}{a_n}\right| = \left|\frac{\frac{\sigma!}{(n + 1)!(n + 1 - \sigma)!}x^{n + 1}}{\frac{\sigma!}{n!(n - \sigma)!}x^n} \right| = \left|\frac{(\sigma - n)x}{n + 1} \right| \xrightarrow{n\to +\infty}|x|< 1\]

    Обозначим сумму ряда через \(S(x)\).

    Наблюдение: \(S'(x)(1 + x) = \sigma S(x)\)
    \[S'(x) = \dots + \frac{\sigma(\sigma - 1)\dots (\sigma - n)}{n!} x^n + \dots \]
    \[S(x) = \dots + \frac{\sigma(\sigma - 1)\dots (\sigma - n + 1)}{n!} x^n + \dots \]
    \begin{align*}
        (1 + x)S' & = \dots + \left( \frac{\sigma(\sigma - 1)\dots (\sigma - n)}{n!} + \frac{\sigma(\sigma - 1)\dots (\sigma - n + 1)}{n!}n \right)x^n + \dots \\
                  & = \dots + \frac{\sigma(\sigma - 1)\dots (\sigma - n + 1)}{n!}\sigma x^n + \dots
    \end{align*}

    \[f(x) = \frac{S(x)}{(1 + x)^\sigma} \quad f'(x) = \frac{S'(1 + x)^\sigma - \sigma(1 + x)^{\sigma - 1}S}{(1 + x)^{2\sigma}} = 0\]
    \(\Rightarrow f = \const, f(0) = 1\Rightarrow f\equiv 1 \Rightarrow S(x) = (1 + x)^\sigma\)
\end{proof}
%</разложениебиномаproof>

\begin{corollary}
    \[\arcsin x = \sum {}^{\textcolor{red}{**}} \frac{(2n - 1)!!}{(2n)!!} \frac{x^{2n + 1}}{2n + 1},\ x\in( - 1, 1)\]
    \[(\arcsin x)' = \frac{1}{\sqrt{1 - x^2}} = \sum_{n = 0}^{ +\infty} \binom{\sigma}{n} ( - x^2)^n\Big|_{\sigma =- \frac{1}{2}} = \sum {}^{\textcolor{red}{*}} \frac{(2n - 1)!!}{(2n)!!}x^{2n}\]
    При \(n = 0\) \textcolor{red}{*} это \(1\), и тогда \textcolor{red}{**}: \(\arcsin x = x + \dots \)
\end{corollary}

\begin{corollary}
    \[\sum_{n = m}^{ +\infty} n(n - 1)\dots (n - m + 1)t^{n - m} = \frac{m!}{(1 - t)^{m + 1}},\ |t|< 1\]
\end{corollary}
\begin{proof}
    \[\sum_{n = 0}^{ +\infty} t^n = \frac{1}{1 - t}\]
    Дифференцируем \(m\) раз, получим искомое. Слагаемые с \(n < m\) пропадут, т.к. они \( = 0\)
\end{proof}

\begin{theorem}
    %<*оразложимости>
    \(f\in C^{\infty} (x_0 - h, x_0 + h)\)

    Тогда \(f\) --- раскладывается в ряд Тейлора в окрестности \(x_0\) \(\iff\)
    \[\exists \delta, C, A > 0 \ \ \forall n \ \ \forall x : |x - x_0| < \delta \ \ |f^{(n)}(x)| < C A^n n!\]
    %</оразложимости>
\end{theorem}

\begin{remark}
    В ``Кошмарном сне'' \textit{(см. лекцию 12)} \(f^{(n)} \approx n! 2n! \Rightarrow f\) не раскладывается.
\end{remark}

%<*оразложимостиproof>
\begin{proof}\itemfix
    \begin{enumerate}
        \item [ \( \Leftarrow \) ] формула Тейлора в \(x_0\) : \(f(x) = \sum\limits_{k = 0}^{n - 1} \frac{f^{(k)}(x_0)}{k!}(x - x_0)^k + \frac{f^{(n)}(c)}{n!} (x - x_0)^n\)

              Если
              \[\left|\frac{f^{(n)}(c)}{n!} (x - x_0)^n\right| \xrightarrow{n\to +\infty} 0\]
              , то \(f\) раскладывается в ряд Тейлора. Из условия мы знаем:
              \[\left|\frac{f^{(n)}(c)}{n!} (x - x_0)^n\right| \leq C|A(x - x_0)|^n\]
              Тогда при \(C|A(x - x_0)|^n \to 0\) \(f\) раскладывается в ряд Тейлора.
              \[C|A(x - x_0)|^n \to 0 \Leftrightarrow |x - x_0|<\min(\delta, \frac{1}{A})\]
              Таким образом, \(f\) раскладывается в ряд Тейлора в области \((x_0 - \min(\delta, \frac{1}{A}), x_0 + \min(\delta, \frac{1}{A}))\)
        \item [ \( \Rightarrow \) ] \(f(x) = \sum \frac{f^{(n)}(x_0)}{n!}(x - x_0)^n\)

              Возьмём \(x_1\neq x_0\), для которого разложение верно.

              \begin{enumerate}
                  \item при \(x = x_1\), ряд сходится \(\Rightarrow\) слагаемые \(\to 0\) \(\Rightarrow\) слагаемые ограничены:

                        \[\left|\frac{f^{(n)}(x_0)}{n!}(x_1 - x_0)^n \right| \leq C_1 \Leftrightarrow |f^{(n)}(x_0)| \leq C_1 n! B^n\]
                        , где \(B = \frac{1}{|x_1 - x_0|}\)

                        Таким образом, мы оценили производную в \(x_0\), но нужно уметь оценивать и производную в окрестности \(x_0\).
                  \item \begin{align*}
                            f^{(m)}(x) & = \sum \frac{f^{(n)}(x_0)}{n!}n(n - 1)\dots (n - m + 1)(x - x_0)^{n - m}  \\
                                       & = \sum_{n = m}^{ +\infty} \frac{f^{(n)}(x_0)}{(n - m)!} (x - x_0)^{n - m}
                        \end{align*}

                        Пусть \(|x - x_0|< \frac{1}{2B}\)

                        \begin{align}
                            |f^{(m)}(x)| & \leq \sum \left|\frac{f^{(n)}(x_0)}{(n - m)!} |x - x_0|^{n - m}\right|                     \\
                                         & \leq \sum \frac{C_1 n! B^n}{(n - m)!} |x - x_0|^{n - m}                                    \\
                                         & = C_1 B^m \sum \frac{n!}{(n - m)!} (\underbrace{B|x - x_0|}_{ < \frac{1}{2}})^{n - m}      \\
                                         & = \frac{C_1B^m m!}{(\underbrace{1 - B|x - x_0|}_{ > \frac{1}{2}})^{m + 1}}  \label{по сл2} \\
                                         & < C_1 2^{m + 1} B^m m!                                                                     \\
                                         & = \underbrace{(2C_1)}_{C}(\underbrace{2B}_{A})^m m!
                        \end{align}

                        \eqref{по сл2}: по следствию 2.

                        Эта оценка выполняется при \(|x - x_0|< \delta = \frac{1}{2B}\)
              \end{enumerate}
    \end{enumerate}
\end{proof}
%</оразложимостиproof>

\section*{Теория меры}

Продолжим доказательство с прошлой лекции.

\begin{proof}\itemfix
    \begin{enumerate}
        %<*второесвойствообъемаproof>
        \item [2.]
              \[B_k : = A\cap A_k \in \mathcal{P}\ \ A = \bigcup_{\text{кон.}} B_k\]

              Сделаем это множество дизъюнктным.

              \[C_1 : = B_1, \dots , C_k : = B_k \setminus \left(\bigcup_{i = 1}^{k - 1} B_i\right)\ \ A = \bigsqcup_{\text{кон.}} C_k\]

              Но эти \(C_k\) вообще говоря \(\not\in \mathcal{P}\)

              \[C_k = B_k \setminus \left(\bigcup_{i = 1}^{k - 1} B_i\right) = \bigsqcup_j D_{k_j}\in \mathcal{P}\]

              Тогда \(A = \bigsqcup\limits_{k, j} D_{k_j}\ \ \mu A = \sum \mu D_{k_j}\)

              При этом \(\forall k \ \ \sum\limits_j \mu D_{k_j} = \mu C_k \stackrel{\text{монот.} \mu}{\le} \mu A_k\)

              Итого \(\mu A = \sum\limits_k \sum\limits_j \mu D_{k_j} = \sum \mu C_k \le \sum \mu A_k\)
              %</второесвойствообъемаproof>
    \end{enumerate}
\end{proof}

\begin{exercise}\itemfix
    \begin{enumerate}
        \item \(X = \{1,2,3\}, \mathcal{P} = 2^X\). Задайте объем \(\mu\) на \(\mathcal{P}\): \(\mu \{1\} = 10, \mu \{1,2,3\} = 2021\)
        \item \(\mu\) --- объем на алгебре \(\mathfrak{A}\), \(\mu X < +\infty \ \ \forall X\).

              Доказать: \(\forall A, B, C \in \mathfrak{A} : \mu(A\cup B\cup C) = \mu A + \mu B + \mu C - \mu(A\cap B) - \mu(B\cap C) - \mu(A\cap C) + \mu(A\cap B\cap C)\)
    \end{enumerate}
\end{exercise}

\begin{definition}
    %<*мера>
    \(\mu : \underbrace{\mathcal{P}}_{\text{полукольцо}} \to \overline \R\) --- \textbf{мера}, если \(\mu\) --- объем и \(\mu\) счётно-аддитивна:
    \[A, A_1, A_2, \dots \in \mathcal{P} : A = \bigsqcup_{i = 1}^{ +\infty} A_i \quad \mu A = \sum_i \mu A_i\]
    %</мера>
\end{definition}

\begin{remark}
    \((a_\omega)_{\omega\in\Omega}\) --- счётное семейство чисел \textit{( \(\Omega\) --- счётно)}, \(\forall \omega \ \ a_\omega \geq 0\)

    Тогда определена \(\sum\limits_{\omega\in\Omega} a_\omega = \sup \sum\limits_{\text{кон.}} a_\omega\)

    Значит, можно счётную аддитивность понимать обобщенно:
    \[A = \bigsqcup_{\text{кон.}} A_\omega \Rightarrow \mu A = \sum \mu A_\omega\]
\end{remark}

\begin{remark}
    Счётная аддитивность не следует из конечной аддитивности.
    \begin{example}[не меры]
        \(X =\R^2, \mathcal{P} =\) ограниченные множества и их дополнения.

        \[\mu A = \begin{cases}
                0 & , A \text{ --- огр.}                  \\
                1 & , A \text{ --- имеет огр. дополнение}
            \end{cases}\]
        Пусть \(\R^2 = \bigcup\limits_{\text{счётное}} \text{клеток} = \bigsqcup\limits_{\text{счётное}} \text{ячеек} = \bigsqcup A_i\)

        \(\mu(\R^2) = 1, \sum \mu A_i = 0 \Rightarrow \mu\) --- не счётно аддитивная и не мера.
    \end{example}
\end{remark}

\begin{example}[меры]
    %<*дискретнаямера>
    \(X\) --- (бесконечное) множество.

    \(a_1, a_2, a_3 \dots \) --- набор попарно различных точек.

    \(h_1, h_2, h_3 \dots \) --- положительные числа.

    Для \(A\subset X\) \(\mu A : = \sum\limits_{k : a_k\in A} h_k\).

    Физический смысл \(\mu\): каждой точке \(a_i\) сопоставляется ``масса'' \(h_i\). Мера множества точек есть сумма ``масс'' точек.

    Счётная аддитивность \(\mu \Leftrightarrow\) теореме о группировке слагаемых (в ряду можно ставить скобки).

    Эта мера называется дискретной.
    %</дискретнаямера>
\end{example}

\begin{theorem}
    %<*обэквивалентностисчетнойаддитивностиисчетнойполуаддитивности>
    \(\mu : \underbrace{\mathcal{P}}_{\text{полукольцо}} \to \overline \R\) --- объем.

    Тогда эквивалентно:
    \begin{enumerate}
        \item \(\mu\) --- мера, т.е. \(\mu\) --- счётно-аддитивна.
        \item \(\mu\) --- счётно-полуаддитивна:
              \[A, A_1, A_2, \dots \in \mathcal{P} \ \ A\subset \bigcup A_i \Rightarrow \mu A\le \sum \mu A_i\]
    \end{enumerate}
    %</обэквивалентностисчетнойаддитивностиисчетнойполуаддитивности>
\end{theorem}
%<*обэквивалентностисчетнойаддитивностиисчетнойполуаддитивностиproof>
\begin{proof}\itemfix
    \begin{enumerate}
        \item [1 \(\Rightarrow\) 2] как в предыдущей теореме, пункт 2, но вместо конечного объединения по \(k\) используется счётное.
        \item [2 \(\Rightarrow\) 1] \(A = \bigsqcup A_i \xRightarrow{?} \mu A = \sum \mu A_i\)

              \[\forall N \ \ A\supset \bigsqcup_{i = 1}^N A_i \ \ \mu A \ge \sum_{i = 1}^N \mu A_i\]

              \(A\subset \bigcup A_i\) (на самом деле \(A =\bigsqcup A_i\)) \(\Rightarrow \mu A \le \sum \mu A_i\)

              \(\Rightarrow \mu A = \sum \mu A_i\)
    \end{enumerate}
\end{proof}
%</обэквивалентностисчетнойаддитивностиисчетнойполуаддитивностиproof>

\begin{corollary}
    \(A\in \mathcal{P}, A_n\in \mathcal{P} : A \subset \bigcup A_n, \mu A_n = 0, \mu\) --- мера. Тогда \(\mu A = 0\)

    Это очевидно, т.к. \(\mu A \le \sum \mu A_i = 0\)
\end{corollary}

\begin{theorem}\itemfix
    %<*онепрерывностиснизу>
    \begin{itemize}
        \item \(\mathfrak{A}\) --- алгебра
        \item \(\mu : \mathfrak{A} \to \overline \R\) --- объем.
    \end{itemize}
    Тогда эквивалентно:
    \begin{enumerate}
        \item \(\mu\) --- мера
        \item \(\mu\) --- непрерывна снизу:
              \[A, A_1, A_2\dots \in \mathfrak{A} \ \ A_1 \subset A_2 \subset \dots , A = \bigcup_{i = 1}^{ +\infty} A_i \Rightarrow \mu A = \lim_{i\to +\infty} \mu A_i\]
    \end{enumerate}
    %</онепрерывностиснизу>
\end{theorem}

\begin{theorem}\itemfix
    %<*онепрерывностисверху>
    \begin{itemize}
        \item \(\mathfrak{A}\) --- алгебра
        \item \(\mu : \mathfrak{A} \to \R\) --- объем.
        \item \(\mu\) --- конечный объем.
    \end{itemize}

    Тогда эквивалентно:
    \begin{enumerate}
        \item \(\mu\) --- мера, т.е. \(\mu\) счётно-аддитивная.
        \item \(\mu\) --- непрерывна сверху:
              \[A, A_1, A_2\dots \in \mathfrak{A} \ \ A_1 \supset A_2 \supset \dots , A = \bigcap_{i = 1}^{ +\infty} A_i \Rightarrow \mu A = \lim_{i\to +\infty} \mu A_i\]
    \end{enumerate}
    %</онепрерывностисверху>
\end{theorem}

%<*онепрерывностисверхуproof>
\begin{proof}\itemfix
    \begin{enumerate}
        \item [1 \(\Rightarrow\) 2] \(B_k = A_k \setminus A_{k + 1}, A_1 = \bigsqcup B_k \cup A\)

              \(\mu A_1 = \sum \mu B_k + \mu A\)

              \[A_n = \bigsqcup\limits_{k \ge n} B_k \cup A \quad \mu A_n = \sum_{k\ge n} \mu B_k + \mu A \xrightarrow{n\to +\infty} \mu A\]
        \item [2 \(\Rightarrow\) 1] Дана непрерывность сверху. Воспользуемся ей для случая \(A = \text{\O}\)

              Проверим, что \(C = \bigsqcup C_i \xRightarrow{?} \mu C = \sum \mu C_i\).

              Пусть \(A_k : = \bigsqcup_{i = k + 1}^{ +\infty} C_i\). Тогда \(A_k \in \mathfrak{A}\), т.к. \(A_k = C \setminus \bigsqcup\limits_{i = 1}^k C_i\) --- конечное объединение.

              \[A_1\supset A_2\supset \dots \quad \bigcap A_k = \text{\O} \Rightarrow \mu A_k \xrightarrow{k\to+\infty} \mu \text{\O} = 0\]

              \[C = \bigsqcup_{i = 1}^k C_i \sqcup A_k \ \ \mu C = \sum_{i = 1}^k \mu C_i + \mu A_k \xrightarrow{k\to +\infty} \sum \mu C_i\]
    \end{enumerate}
\end{proof}
%</онепрерывностисверхуproof>

\subsection*{Теорема о продолжении меры}

\begin{definition}
    %<*сигмаконечнаямера>
    \(\mu : \mathcal{P} \to \overline \R\) --- мера, \(\mathcal{P} \subset 2^X\)

    \(\mu\) --- \textbf{ \(\sigma\)-конечна}, если \(\exists A_1, A_2 \dots \in \mathcal{P} : X = \bigcup A_i, \mu A_i < +\infty\)
    %</сигмаконечнаямера>
\end{definition}

\begin{example}
    \(X =\R^m, \mathcal{P} = \mathcal{P}^m\) --- полукольцо ячеек, \(\mu\) --- классический объем, \(\mu\) --- \(\sigma\)-конечный объем.
    \begin{align*}
        \R^m & = \bigcup \text{Куб}(0, 2R)          \\
             & = \bigcup \text{целочисл. ед. ячеек}
    \end{align*}
\end{example}

\begin{definition}
    %<*полнаямера>
    \(\mu: \mathcal{P} \to \overline \R\) --- мера.

    \(\mu\) --- \textbf{полная} в \(\mathcal{P}\), если \(\forall A\in \mathcal{P} \ \ \mu A = 0 \ \ \forall B\subset A\) выполняется: \(B\in \mathcal{P}\) и \textit{(тогда автоматически)} \(\mu B = 0\) \textit{(по монотонности)}

    Это совместное свойство \(\mu\) и \(\mathcal{P}\).
    %</полнаямера>
\end{definition}

Должок.
%<*пространствосмерой>
\textbf{Пространство с мерой} --- тройка \((\underbrace{X}_{\text{множество}}, \underbrace{\mathfrak{A}}_{\sigma\text{-алгебра}}, \underbrace{\mu}_{\text{мера на \(\mathfrak A\)} })\)
%</пространствосмерой>

\end{document}