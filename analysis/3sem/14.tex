\documentclass[12pt, a4paper]{article}

%<*preamble>
% Math symbols
\usepackage{amsmath, amsthm, amsfonts, amssymb}
\usepackage{accents}
\usepackage{esvect}
\usepackage{mathrsfs}
\usepackage{mathtools}
\mathtoolsset{showonlyrefs}
\usepackage{cmll}
\usepackage{stmaryrd}
\usepackage{physics}
\usepackage[normalem]{ulem}
\usepackage{ebproof}
\usepackage{extarrows}

% Page layout
\usepackage{geometry, a4wide, parskip, fancyhdr}

% Font, encoding, russian support
\usepackage[russian]{babel}
\usepackage[sb]{libertine}
\usepackage{xltxtra}

% Listings
\usepackage{listings}
\lstset{basicstyle=\ttfamily,breaklines=true}
\setmonofont{Inconsolata}

% Miscellaneous
\usepackage{array}
\usepackage{calc}
\usepackage{caption}
\usepackage{subcaption}
\captionsetup{justification=centering,margin=2cm}
\usepackage{catchfilebetweentags}
\usepackage{enumitem}
\usepackage{etoolbox}
\usepackage{float}
\usepackage{lastpage}
\usepackage{minted}
\usepackage{svg}
\usepackage{wrapfig}
\usepackage{xcolor}
\usepackage[makeroom]{cancel}

\newcolumntype{L}{>{$}l<{$}}
    \newcolumntype{C}{>{$}c<{$}}
\newcolumntype{R}{>{$}r<{$}}

% Footnotes
\usepackage[hang]{footmisc}
\setlength{\footnotemargin}{2mm}
\makeatletter
\def\blfootnote{\gdef\@thefnmark{}\@footnotetext}
\makeatother

% References
\usepackage{hyperref}
\hypersetup{
    colorlinks,
    linkcolor={blue!80!black},
    citecolor={blue!80!black},
    urlcolor={blue!80!black},
}

% tikz
\usepackage{tikz}
\usepackage{tikz-cd}
\usetikzlibrary{arrows.meta}
\usetikzlibrary{decorations.pathmorphing}
\usetikzlibrary{calc}
\usetikzlibrary{patterns}
\usepackage{pgfplots}
\pgfplotsset{width=10cm,compat=1.9}
\newcommand\irregularcircle[2]{% radius, irregularity
    \pgfextra {\pgfmathsetmacro\len{(#1)+rand*(#2)}}
    +(0:\len pt)
    \foreach \a in {10,20,...,350}{
            \pgfextra {\pgfmathsetmacro\len{(#1)+rand*(#2)}}
            -- +(\a:\len pt)
        } -- cycle
}

\providetoggle{useproofs}
\settoggle{useproofs}{false}

\pagestyle{fancy}
\lfoot{M3137y2019}
\cfoot{}
\rhead{стр. \thepage\ из \pageref*{LastPage}}

\newcommand{\R}{\mathbb{R}}
\newcommand{\Q}{\mathbb{Q}}
\newcommand{\Z}{\mathbb{Z}}
\newcommand{\B}{\mathbb{B}}
\newcommand{\N}{\mathbb{N}}
\renewcommand{\Re}{\mathfrak{R}}
\renewcommand{\Im}{\mathfrak{I}}

\newcommand{\const}{\text{const}}
\newcommand{\cond}{\text{cond}}

\newcommand{\teormin}{\textcolor{red}{!}\ }

\DeclareMathOperator*{\xor}{\oplus}
\DeclareMathOperator*{\equ}{\sim}
\DeclareMathOperator{\sign}{\text{sign}}
\DeclareMathOperator{\Sym}{\text{Sym}}
\DeclareMathOperator{\Asym}{\text{Asym}}

\DeclarePairedDelimiter{\ceil}{\lceil}{\rceil}

% godel
\newbox\gnBoxA
\newdimen\gnCornerHgt
\setbox\gnBoxA=\hbox{$\ulcorner$}
\global\gnCornerHgt=\ht\gnBoxA
\newdimen\gnArgHgt
\def\godel #1{%
    \setbox\gnBoxA=\hbox{$#1$}%
    \gnArgHgt=\ht\gnBoxA%
    \ifnum     \gnArgHgt<\gnCornerHgt \gnArgHgt=0pt%
    \else \advance \gnArgHgt by -\gnCornerHgt%
    \fi \raise\gnArgHgt\hbox{$\ulcorner$} \box\gnBoxA %
    \raise\gnArgHgt\hbox{$\urcorner$}}

% \theoremstyle{plain}

\theoremstyle{definition}
\newtheorem{theorem}{Теорема}
\newtheorem*{definition}{Определение}
\newtheorem{axiom}{Аксиома}
\newtheorem*{axiom*}{Аксиома}
\newtheorem{lemma}{Лемма}

\theoremstyle{remark}
\newtheorem*{remark}{Примечание}
\newtheorem*{exercise}{Упражнение}
\newtheorem{corollary}{Следствие}[theorem]
\newtheorem*{statement}{Утверждение}
\newtheorem*{corollary*}{Следствие}
\newtheorem*{example}{Пример}
\newtheorem{observation}{Наблюдение}
\newtheorem*{prop}{Свойства}
\newtheorem*{obozn}{Обозначение}

% subtheorem
\makeatletter
\newenvironment{subtheorem}[1]{%
    \def\subtheoremcounter{#1}%
    \refstepcounter{#1}%
    \protected@edef\theparentnumber{\csname the#1\endcsname}%
    \setcounter{parentnumber}{\value{#1}}%
    \setcounter{#1}{0}%
    \expandafter\def\csname the#1\endcsname{\theparentnumber.\Alph{#1}}%
    \ignorespaces
}{%
    \setcounter{\subtheoremcounter}{\value{parentnumber}}%
    \ignorespacesafterend
}
\makeatother
\newcounter{parentnumber}

\newtheorem{manualtheoreminner}{Теорема}
\newenvironment{manualtheorem}[1]{%
    \renewcommand\themanualtheoreminner{#1}%
    \manualtheoreminner
}{\endmanualtheoreminner}

\newcommand{\dbltilde}[1]{\accentset{\approx}{#1}}
\newcommand{\intt}{\int\!}

% magical thing that fixes paragraphs
\makeatletter
\patchcmd{\CatchFBT@Fin@l}{\endlinechar\m@ne}{}
{}{\typeout{Unsuccessful patch!}}
\makeatother

\newcommand{\get}[2]{
    \ExecuteMetaData[#1]{#2}
}

\newcommand{\getproof}[2]{
    \iftoggle{useproofs}{\ExecuteMetaData[#1]{#2proof}}{}
}

\newcommand{\getwithproof}[2]{
    \get{#1}{#2}
    \getproof{#1}{#2}
}

\newcommand{\import}[3]{
    \subsection{#1}
    \getwithproof{#2}{#3}
}

\newcommand{\given}[1]{
    Дано выше. (\ref{#1}, стр. \pageref{#1})
}

\renewcommand{\ker}{\text{Ker }}
\newcommand{\im}{\text{Im }}
\renewcommand{\grad}{\text{grad}}
\newcommand{\rg}{\text{rg}}
\newcommand{\defeq}{\stackrel{\text{def}}{=}}
\newcommand{\defeqfor}[1]{\stackrel{\text{def } #1}{=}}
\newcommand{\itemfix}{\leavevmode\makeatletter\makeatother}
\newcommand{\?}{\textcolor{red}{???}}
\renewcommand{\emptyset}{\varnothing}
\newcommand{\longarrow}[1]{\xRightarrow[#1]{\qquad}}
\DeclareMathOperator*{\esup}{\text{ess sup}}
\newcommand\smallO{
    \mathchoice
    {{\scriptstyle\mathcal{O}}}% \displaystyle
    {{\scriptstyle\mathcal{O}}}% \textstyle
    {{\scriptscriptstyle\mathcal{O}}}% \scriptstyle
    {\scalebox{.6}{$\scriptscriptstyle\mathcal{O}$}}%\scriptscriptstyle
}
\renewcommand{\div}{\text{div}\ }
\newcommand{\rot}{\text{rot}\ }
\newcommand{\cov}{\text{cov}}

\makeatletter
\newcommand{\oplabel}[1]{\refstepcounter{equation}(\theequation\ltx@label{#1})}
\makeatother

\newcommand{\symref}[2]{\stackrel{\oplabel{#1}}{#2}}
\newcommand{\symrefeq}[1]{\symref{#1}{=}}

% xrightrightarrows
\makeatletter
\newcommand*{\relrelbarsep}{.386ex}
\newcommand*{\relrelbar}{%
    \mathrel{%
        \mathpalette\@relrelbar\relrelbarsep
    }%
}
\newcommand*{\@relrelbar}[2]{%
    \raise#2\hbox to 0pt{$\m@th#1\relbar$\hss}%
    \lower#2\hbox{$\m@th#1\relbar$}%
}
\providecommand*{\rightrightarrowsfill@}{%
    \arrowfill@\relrelbar\relrelbar\rightrightarrows
}
\providecommand*{\leftleftarrowsfill@}{%
    \arrowfill@\leftleftarrows\relrelbar\relrelbar
}
\providecommand*{\xrightrightarrows}[2][]{%
    \ext@arrow 0359\rightrightarrowsfill@{#1}{#2}%
}
\providecommand*{\xleftleftarrows}[2][]{%
    \ext@arrow 3095\leftleftarrowsfill@{#1}{#2}%
}

\allowdisplaybreaks

\newcommand{\unfinished}{\textcolor{red}{Не дописано}}

% Reproducible pdf builds 
\special{pdf:trailerid [
<00112233445566778899aabbccddeeff>
<00112233445566778899aabbccddeeff>
]}
%</preamble>


\lhead{Математический анализ}
\cfoot{}
\rfoot{14.12.2020}

\begin{document}

\section*{Теория меры}

\begin{definition}
    \(\mu : \mathcal{P}\to\R\) --- \textbf{продолжает} \(\mu_0 : \mathcal{P}_0 \to \overline \R, \mathcal{P}_0\subset \mathcal{P}\), если \(\mu\big|_{\mathcal{P}_0} = \mu_0\)
\end{definition}

\begin{theorem}[о Лебеговском продолжении меры]\itemfix
    \begin{itemize}
        \item \(\mathcal{P}_0 \subset X\) --- полукольцо
        \item \(\mu_0 : \mathcal{P}_0\to\R\) --- \(\sigma\)-конечная мера
    \end{itemize}
    Тогда \(\exists \sigma\)-алгебра \(\mathfrak{A}\supset \mathcal{P}_0\), \(\exists \mu\) --- мера на \(\mathfrak{A}\):
    \begin{enumerate}
        \item \(\mu\) --- продолжение \(\mu_0\) на \(\mathfrak{A}\)
        \item \(\mu\) --- полная мера
        \item Если \(\tilde \mu\) --- полная мера на \(\sigma\)-алгебре \(\tilde{\mathfrak{A}}\) и \(\tilde{\mu}\) --- продолжение \(\mu_0\), то \(\tilde{\mathfrak{A}}\supset \mathfrak{A}\) и при этом \(\tilde \mu\) продолжает \(\mu : \tilde \mu\big|_{\mathfrak{A}} = \mu\)
        \item Если \(\mathcal{P}\) --- полукольцо, такое что \(\mathcal{P}_0 \subset \mathcal{P} \subset \mathfrak{A}\) и мера \(\upsilon\) --- продолжение \(\mu_0\) на \(\mathcal{P}\), то \(\forall A\in \mathcal{P}\ \ \upsilon(A) = \mu(A)\)
        \item \(\forall A\in \mathfrak{A}\ \ \mu A = \inf \{\sum \mu P_k : P_k \in \mathcal{P} : A \subset \bigcup\limits_{k = 1}^{ +\infty} P_k\} \)
    \end{enumerate}
\end{theorem}
\begin{proof}
    Не будет, это слишком сложно.

    Общая идея следующая: \(\forall A\subset X\) положим \(\mu^*(A) = \inf \{\dots \} \) --- не аддитивна. \(A\subset \bigcup A_k \ \ \mu^*A = \sum \mu^* A_k\)
    %TODO
\end{proof}

\begin{corollary}\itemfix
    \begin{itemize}
        \item \(A\in \mathfrak{A}\)
        \item \(\mu A < +\infty\)
        \item \(\varepsilon > 0\)
    \end{itemize}
    %TODO
    Тогда \(\exists P_k \in \mathcal{P}_0 : A\subset \bigcup P_k \ \ \mu A < \sum \mu P_k < \mu A + \varepsilon\)
\end{corollary}

\subsection*{Мера Лебега}

\begin{theorem}\itemfix
    \begin{itemize}
        \item \(\mu : \mathcal{P}^m \to \R\) --- классический объем в \(\R^m\)
    \end{itemize}
    Тогда \(\sigma\) --- конечная мера
\end{theorem}
\begin{proof}
    \(\sigma\)-конечность очевидна.

    Докажем счётную аддитивность \(\mu\).

    Для этого достаточно проверить счётную полуаддитивность:
    \[P = [a, b), P_n = [a_n, b_n) \ \ P\subset \bigcup P_n \ \ \mu P \stackrel{?}{\le} \sum \mu P_n\]

    Если \(P = \text{\O}\), то утверждение тривиально. Пусть \(P\) непустое.

    Фиксируем \(\varepsilon > 0\). Чуть уменьшим координаты вектора \(b\), так что \([a, b'] \subset [a, b)\) и \(\mu(P\setminus [a, b')) < \varepsilon\). Последняя формула некорректна, т.к. \(P\setminus [a, b')\) не обязательно ячейка. Но оно представимо в виде \(\bigsqcup D_j\), поэтому под \(\mu(P\setminus [a, b'))\) подразумевается \(\sum \mu D_j\). Также можно было записать \(\mu P - \mu[a, b') < \varepsilon\) вместо этих трюков.

    Уменьшим слегка координаты векторов \(a_n\), так что \((a'_n, b_n) \supset [a_n, b_n)\), \(\mu([a'_n, b_n) \setminus [a_n, b_n)) < \frac{\varepsilon}{2^n}\). Эта запись также некорректна, поэтому напишем \(\mu[a_n', b_n) - \mu[a_n, b_n) < \frac{\varepsilon}{2^n}\)

    \(\underbrace{[a, b']}_{\text{комп.}} \subset \bigcup (a_n', b_n) \Rightarrow \exists \) конечное подпокрытие: \([a, b'] \subset \bigcup\limits_{n = 1}^N (a_n', b_n) \Rightarrow [a, b') \subset \bigcup\limits_{n = 1}^{N} [a_n', b_n)\)

    Тогда \(\mu[a, b') \le \sum\limits_{n = 1}^{N} \mu[a_n', b_n)\)

    \begin{align}
        \mu P - \varepsilon & \leq \sum_{n = 1}^N \left( \mu P_n + \frac{\varepsilon}{2^n} \right) \\
        \mu P - \varepsilon & \leq \sum_{n = 1}^{ +\infty} \mu P_n + \varepsilon
    \end{align}
\end{proof}

\begin{definition}
    \textbf{Мера Лебега} в \(\R^m\) --- лебеговское продолжение классического объема.

    \(\mathfrak{M}^m\) --- \(\sigma\)-алгебра, на которой задана мера Лебега. Тогда множество называется \textbf{измеримым по Лебегу}

    %TODO \lambda
\end{definition}

Свойства меры Лебега:
\begin{enumerate}
    \item \begin{enumerate}
              \item \(A_1, A_2 \dots \) --- измеримы \( \Rightarrow A_1\cap A_2, A_1\cup A_2, A_1\cap A_2\cap A_3 \dots , A_1\cup A_2\cup A_3 \dots\) --- измеримы.
              \item \(\forall n \ \ \lambda A_n = 0 \Rightarrow \lambda(\bigcup A_n) = 0\)
              \item \(\lambda A_n = 0, B\subset A \Rightarrow B\) --- измеримо, \(\lambda B = 0\)
          \end{enumerate}

          \begin{example}
              \(\Q \subset \R\) --- измеримо, \(\lambda_1 \Q = 0\)
          \end{example}
          \begin{proof}
              \(\forall x\in \R \ \ \{x\} = \bigcap_n[x, x + \frac{1}{n})\)
              \[0 \leq \lambda \{x\} \leq \lambda[x, x + \frac{1}{n}) = \frac{1}{n} \Rightarrow \lambda \{x\} = 0\]
              \(\Q\) --- объединение одноточечных множеств.
          \end{proof}
    \item \(\mathfrak{M}^m\) содержит все открытые и замкнутые множества.
          \begin{lemma}\itemfix
              \begin{enumerate}
                  \item \(O\subset \R^m\) --- открытое. Тогда \(O = \bigsqcup Q_i\), где \(Q_i\) --- ячейки с рациональными координатами. Можно считать, что ячейки кубические.
                  \item Можно считать, что \(\overline Q_i \subset O\)
                  \item \(E\) --- измеримо, \(\lambda E = 0\). Тогда \(\forall \varepsilon > 0 \ \ E\subset \bigcup Q_i : Q_i\) --- кубические ячейки и \(\sum \lambda Q_i < \varepsilon\)
              \end{enumerate}
          \end{lemma}
          \begin{remark}
              \(\forall \varepsilon > 0\ \ \exists (B_i)\) --- шары: \(E\subset \bigcup B_i, \sum \lambda B_i < \varepsilon\)

              \[Q\left(x, \frac{R}{\sqrt{m}}\right) \subset B(x, R) \subset Q(x, R)\]
              \[\left( \frac{2R}{\sqrt{m}} \right)^m \leq \lambda B \leq \lambda Q(x, R) = (2R)^m\]
              %TODO: pic
          \end{remark}
          \begin{proof}\itemfix
              \begin{enumerate}
                  \item [(a, b)] \(\forall x\in O\) пусть \(Q(x)\) --- какая угодно ячейка с рациональными координатами, \(Q(x)\subset O\) \textit{(можно потребовать \(\overline{Q(x)}\subset O, Q\) --- куб, координаты двоично-рациональны для второго пункта)}.

                        \(O = \bigcup\limits_{x\in O} Q(x)\) --- здесь не более чем счётное множество различных ячеек.

                        \(\Rightarrow O = \bigcap_{i = 1}^{\infty} Q(x_i)\). Сделаем ячейки дизъюнктными: \(Q_1 : = Q(x_1), Q(x_2)\setminus Q(x_1) = \bigcup D_j\). Переобозначим \(D_j\) как \(Q_2, Q_3 \dots Q_k\). Аналогично для всех \(Q(x_i)\).

                        Можно считать, что координаты всех ячеек двоично рациональны.

                        Ячейки можно подразбить, чтобы они стали кубическими: пусть \(2^l\) --- самый крупный знаменатель. Тогда \([a_i, b_i]\) --- конечное объединение кубических ячеек со стороной \(\frac{1}{2^l}\)
                  \item [(c)] Следует из пункта 5 теоремы о продолжении Лебега:

                        \[\forall \varepsilon > 0 \ \ \exists \text{ ячейки } P_k \ \ E\subset \bigcup P_k \ \ 0 = \lambda E \le \sum \lambda P_k \leq \varepsilon\]
                        \(\exists \tilde P_k\) --- двоично-рациональные ячейки:
                        \[P_k \subset \tilde P_k \ \ 0 = \lambda E \le \sum \lambda_k \tilde P_k \le 2 \varepsilon\]
                        Можно разбить \(\tilde P_k\) на конечное число кубов.
              \end{enumerate}
          \end{proof}

          \begin{definition}
              \(\mathfrak{B}\) --- \textbf{борелевская \(\sigma\)-алгебра} (в \(\R^m\) или в метрическом пространстве) --- минимальная \(\sigma\)-алгебра, которая содержит все открытые множества.

              \(\mathfrak{M}^m\supset \mathfrak{B}\)
          \end{definition}

          \begin{example}
              Канторово множество в \(\R\):
              \begin{align}
                  K_0 & = [0, 1]                                                                                                                                             \\
                  K_1 & = \left[0, \frac{1}{3}\right] \cup \left[\frac{2}{3}, 1\right]                                                                                       \\
                  K_2 & = \left[0, \frac{1}{9}\right] \cup \left[\frac{2}{9}, \frac{1}{3}\right] \cup \left[\frac{2}{3}, \frac{7}{9}\right] \cup \left[\frac{8}{9}, 1\right] \\
                  \vdots
              \end{align}

              \begin{figure}[h]
                  \centering
                  \includesvg{images/множество_кантора.svg}
                  \caption{Множество Кантора}
              \end{figure}

              \(\mathcal{K} = \bigcup K_i\) --- измеримо, \(\lambda \mathcal{K} = 0, \lambda(K_i) = \left( \frac{2}{3} \right)^i\)

              Также можно задать множество Кантора следующим образом:
              \[\mathcal{K} = \{x\in[0, 1] : \text{ троичная запись \(x\) содержит только цифр \(0\) и \(2\)}\} \]
          \end{example}

    \item \(\exists \) неизмеримые по Лебегу множества, т.е. не принадлежащие \(\mathfrak{M}\)

          Зададим отношение \(\sim\) на \(\R\): \(x\sim y \Leftrightarrow x - y\in \Q\)

          \(\R / \Q = A\) --- из каждого класса эквивалентности взяли по одной точке. Можно считать, что \(A\subset [0, 1]\)

          Очевидно, что \(\bigsqcup\limits_{q\in\Q} (A + q) = \R\)

          \[[0, 1] \stackrel{(1)}{\subset} \bigsqcup_{q\in \Q\cap [ - 1, 1]}(A + q) \stackrel{(2)}{\subset} [ - 1, 2]\]

          Измеримо ли \(A\)? Предпололжим, что да.

          Очевидно \(\forall q \ \ \lambda A = \lambda(A + q)\) по пункту 5 теоремы о продолжении меры.

          Из (1):
          \[\lambda[0, 1] = 1 \le \sum_q \lambda(A + q) = \sum_q \lambda(A) \Rightarrow \lambda A > 0\]
          Из (2):
          \[\lambda\left( \bigcup(A + q) \right) = \sum_q \lambda A \leq \lambda[ - 1, 2] = 3 \Rightarrow \lambda A = 0\]

          Противоречие \( \Rightarrow \) \(A\) неизмеримо.
    \item \(A \in \mathfrak{M}\) % TODO: ^m?

          \begin{itemize}
              \item \(A\) --- ограничено \( \Rightarrow \lambda A < +\infty\)
              \item \(A\) --- открыто \( \Rightarrow \lambda A > 0\) --- из леммы. % TODO: какой
              \item \(\lambda A = 0 \Rightarrow A\) не имеет внутренних точек.
          \end{itemize}

    \item \(A\in \mathfrak{M}^m\). Тогда \(\forall \varepsilon > 0\)

          \begin{itemize}
              \item \(\exists \) открытое \(G_\varepsilon\supset A : \lambda(G_\varepsilon \setminus A) < \varepsilon\)
              \item \(\exists \) замкнутое \(F_\varepsilon\subset A : \lambda(A \setminus F_\varepsilon) < \varepsilon\)
          \end{itemize}

          \begin{proof}\itemfix
              \begin{enumerate}
                  \item \(\lambda A\) --- кон.
                        \[\lambda A = \inf \left\{\sum_i \lambda P_i : A\subset \bigcup P_i, P_i \in \mathcal{P}\right\} \]
                        Чуть ``раздуем'' эти \(P_i = [a_i, b_i) \rightsquigarrow (a'_i, b_i) \subset [a'_i, b_i)\)
                        \[\lambda[a'_i, b_i) \leq \lambda P_i + \frac{\varepsilon}{2^i}\]
                        \[A\subset \underbrace{\bigcup(a'_i, b_i)}_{G_{2\varepsilon}} \subset [a'_i, b_i]\]
                        \[\lambda A \leq \lambda G_{2\varepsilon} \leq \sum \lambda [a'_i, b_i) \leq \sum \left( \lambda P_i + \frac{\varepsilon}{2^i} \right) \leq \lambda A + 2 \varepsilon\]
                  \item \(\lambda A = +\infty\). Используем \(\sigma\)-конечность: \(\R^m =\bigsqcup\limits_{j = 1}^{ +\infty} Q_j\)
                        \[\exists G_{\varepsilon, j} \text{ --- откр. } \ \ (A\cap Q)j\subset G_{\varepsilon, j} \ \ \lambda(G_{\varepsilon, j} \setminus (A\cap Q_j)) < \frac{\varepsilon}{2^j} \]
                        \[A = \bigsqcup (A\cap Q_j) \subset \bigcup G_{\varepsilon, j} = : G_\varepsilon\]
                        \[\lambda(G_\varepsilon \setminus A) \leq \sum \lambda(G_{\varepsilon, j} \setminus (A\cap Q_j)) \leq \varepsilon\]
                        Очевидно: \(G_\varepsilon \setminus A \subset \bigcup\limits_j (G_{\varepsilon, j} \setminus (A\cap Q_j))\)
                \item Для \(F_\varepsilon\) --- переходим к дополнению:
                
                    Для \(A^c\) подбираем \(G_\varepsilon\), \(A^c\subset G_\varepsilon\)

                    \[A\supset (G_\varepsilon)^c = : F_\varepsilon\]

                    \[G_\varepsilon \setminus  A^C = A\setminus (G_\varepsilon)^c\]
                    
                    \[\lambda(G_\varepsilon\setminus A^c) < \varepsilon \Rightarrow \lambda(A\setminus F) < \varepsilon\]
              \end{enumerate}
          \end{proof}
\end{enumerate}

\end{document}