\documentclass[12pt, a4paper]{article}

%<*preamble>
% Math symbols
\usepackage{amsmath, amsthm, amsfonts, amssymb}
\usepackage{accents}
\usepackage{esvect}
\usepackage{mathrsfs}
\usepackage{mathtools}
\mathtoolsset{showonlyrefs}
\usepackage{cmll}
\usepackage{stmaryrd}
\usepackage{physics}
\usepackage[normalem]{ulem}
\usepackage{ebproof}
\usepackage{extarrows}

% Page layout
\usepackage{geometry, a4wide, parskip, fancyhdr}

% Font, encoding, russian support
\usepackage[russian]{babel}
\usepackage[sb]{libertine}
\usepackage{xltxtra}

% Listings
\usepackage{listings}
\lstset{basicstyle=\ttfamily,breaklines=true}
\setmonofont{Inconsolata}

% Miscellaneous
\usepackage{array}
\usepackage{calc}
\usepackage{caption}
\usepackage{subcaption}
\captionsetup{justification=centering,margin=2cm}
\usepackage{catchfilebetweentags}
\usepackage{enumitem}
\usepackage{etoolbox}
\usepackage{float}
\usepackage{lastpage}
\usepackage{minted}
\usepackage{svg}
\usepackage{wrapfig}
\usepackage{xcolor}
\usepackage[makeroom]{cancel}

\newcolumntype{L}{>{$}l<{$}}
    \newcolumntype{C}{>{$}c<{$}}
\newcolumntype{R}{>{$}r<{$}}

% Footnotes
\usepackage[hang]{footmisc}
\setlength{\footnotemargin}{2mm}
\makeatletter
\def\blfootnote{\gdef\@thefnmark{}\@footnotetext}
\makeatother

% References
\usepackage{hyperref}
\hypersetup{
    colorlinks,
    linkcolor={blue!80!black},
    citecolor={blue!80!black},
    urlcolor={blue!80!black},
}

% tikz
\usepackage{tikz}
\usepackage{tikz-cd}
\usetikzlibrary{arrows.meta}
\usetikzlibrary{decorations.pathmorphing}
\usetikzlibrary{calc}
\usetikzlibrary{patterns}
\usepackage{pgfplots}
\pgfplotsset{width=10cm,compat=1.9}
\newcommand\irregularcircle[2]{% radius, irregularity
    \pgfextra {\pgfmathsetmacro\len{(#1)+rand*(#2)}}
    +(0:\len pt)
    \foreach \a in {10,20,...,350}{
            \pgfextra {\pgfmathsetmacro\len{(#1)+rand*(#2)}}
            -- +(\a:\len pt)
        } -- cycle
}

\providetoggle{useproofs}
\settoggle{useproofs}{false}

\pagestyle{fancy}
\lfoot{M3137y2019}
\cfoot{}
\rhead{стр. \thepage\ из \pageref*{LastPage}}

\newcommand{\R}{\mathbb{R}}
\newcommand{\Q}{\mathbb{Q}}
\newcommand{\Z}{\mathbb{Z}}
\newcommand{\B}{\mathbb{B}}
\newcommand{\N}{\mathbb{N}}
\renewcommand{\Re}{\mathfrak{R}}
\renewcommand{\Im}{\mathfrak{I}}

\newcommand{\const}{\text{const}}
\newcommand{\cond}{\text{cond}}

\newcommand{\teormin}{\textcolor{red}{!}\ }

\DeclareMathOperator*{\xor}{\oplus}
\DeclareMathOperator*{\equ}{\sim}
\DeclareMathOperator{\sign}{\text{sign}}
\DeclareMathOperator{\Sym}{\text{Sym}}
\DeclareMathOperator{\Asym}{\text{Asym}}

\DeclarePairedDelimiter{\ceil}{\lceil}{\rceil}

% godel
\newbox\gnBoxA
\newdimen\gnCornerHgt
\setbox\gnBoxA=\hbox{$\ulcorner$}
\global\gnCornerHgt=\ht\gnBoxA
\newdimen\gnArgHgt
\def\godel #1{%
    \setbox\gnBoxA=\hbox{$#1$}%
    \gnArgHgt=\ht\gnBoxA%
    \ifnum     \gnArgHgt<\gnCornerHgt \gnArgHgt=0pt%
    \else \advance \gnArgHgt by -\gnCornerHgt%
    \fi \raise\gnArgHgt\hbox{$\ulcorner$} \box\gnBoxA %
    \raise\gnArgHgt\hbox{$\urcorner$}}

% \theoremstyle{plain}

\theoremstyle{definition}
\newtheorem{theorem}{Теорема}
\newtheorem*{definition}{Определение}
\newtheorem{axiom}{Аксиома}
\newtheorem*{axiom*}{Аксиома}
\newtheorem{lemma}{Лемма}

\theoremstyle{remark}
\newtheorem*{remark}{Примечание}
\newtheorem*{exercise}{Упражнение}
\newtheorem{corollary}{Следствие}[theorem]
\newtheorem*{statement}{Утверждение}
\newtheorem*{corollary*}{Следствие}
\newtheorem*{example}{Пример}
\newtheorem{observation}{Наблюдение}
\newtheorem*{prop}{Свойства}
\newtheorem*{obozn}{Обозначение}

% subtheorem
\makeatletter
\newenvironment{subtheorem}[1]{%
    \def\subtheoremcounter{#1}%
    \refstepcounter{#1}%
    \protected@edef\theparentnumber{\csname the#1\endcsname}%
    \setcounter{parentnumber}{\value{#1}}%
    \setcounter{#1}{0}%
    \expandafter\def\csname the#1\endcsname{\theparentnumber.\Alph{#1}}%
    \ignorespaces
}{%
    \setcounter{\subtheoremcounter}{\value{parentnumber}}%
    \ignorespacesafterend
}
\makeatother
\newcounter{parentnumber}

\newtheorem{manualtheoreminner}{Теорема}
\newenvironment{manualtheorem}[1]{%
    \renewcommand\themanualtheoreminner{#1}%
    \manualtheoreminner
}{\endmanualtheoreminner}

\newcommand{\dbltilde}[1]{\accentset{\approx}{#1}}
\newcommand{\intt}{\int\!}

% magical thing that fixes paragraphs
\makeatletter
\patchcmd{\CatchFBT@Fin@l}{\endlinechar\m@ne}{}
{}{\typeout{Unsuccessful patch!}}
\makeatother

\newcommand{\get}[2]{
    \ExecuteMetaData[#1]{#2}
}

\newcommand{\getproof}[2]{
    \iftoggle{useproofs}{\ExecuteMetaData[#1]{#2proof}}{}
}

\newcommand{\getwithproof}[2]{
    \get{#1}{#2}
    \getproof{#1}{#2}
}

\newcommand{\import}[3]{
    \subsection{#1}
    \getwithproof{#2}{#3}
}

\newcommand{\given}[1]{
    Дано выше. (\ref{#1}, стр. \pageref{#1})
}

\renewcommand{\ker}{\text{Ker }}
\newcommand{\im}{\text{Im }}
\renewcommand{\grad}{\text{grad}}
\newcommand{\rg}{\text{rg}}
\newcommand{\defeq}{\stackrel{\text{def}}{=}}
\newcommand{\defeqfor}[1]{\stackrel{\text{def } #1}{=}}
\newcommand{\itemfix}{\leavevmode\makeatletter\makeatother}
\newcommand{\?}{\textcolor{red}{???}}
\renewcommand{\emptyset}{\varnothing}
\newcommand{\longarrow}[1]{\xRightarrow[#1]{\qquad}}
\DeclareMathOperator*{\esup}{\text{ess sup}}
\newcommand\smallO{
    \mathchoice
    {{\scriptstyle\mathcal{O}}}% \displaystyle
    {{\scriptstyle\mathcal{O}}}% \textstyle
    {{\scriptscriptstyle\mathcal{O}}}% \scriptstyle
    {\scalebox{.6}{$\scriptscriptstyle\mathcal{O}$}}%\scriptscriptstyle
}
\renewcommand{\div}{\text{div}\ }
\newcommand{\rot}{\text{rot}\ }
\newcommand{\cov}{\text{cov}}

\makeatletter
\newcommand{\oplabel}[1]{\refstepcounter{equation}(\theequation\ltx@label{#1})}
\makeatother

\newcommand{\symref}[2]{\stackrel{\oplabel{#1}}{#2}}
\newcommand{\symrefeq}[1]{\symref{#1}{=}}

% xrightrightarrows
\makeatletter
\newcommand*{\relrelbarsep}{.386ex}
\newcommand*{\relrelbar}{%
    \mathrel{%
        \mathpalette\@relrelbar\relrelbarsep
    }%
}
\newcommand*{\@relrelbar}[2]{%
    \raise#2\hbox to 0pt{$\m@th#1\relbar$\hss}%
    \lower#2\hbox{$\m@th#1\relbar$}%
}
\providecommand*{\rightrightarrowsfill@}{%
    \arrowfill@\relrelbar\relrelbar\rightrightarrows
}
\providecommand*{\leftleftarrowsfill@}{%
    \arrowfill@\leftleftarrows\relrelbar\relrelbar
}
\providecommand*{\xrightrightarrows}[2][]{%
    \ext@arrow 0359\rightrightarrowsfill@{#1}{#2}%
}
\providecommand*{\xleftleftarrows}[2][]{%
    \ext@arrow 3095\leftleftarrowsfill@{#1}{#2}%
}

\allowdisplaybreaks

\newcommand{\unfinished}{\textcolor{red}{Не дописано}}

% Reproducible pdf builds 
\special{pdf:trailerid [
<00112233445566778899aabbccddeeff>
<00112233445566778899aabbccddeeff>
]}
%</preamble>


\lhead{Математический анализ}
\cfoot{}
\rfoot{14.9.2020}

\begin{document}

\begin{theorem}[Лагранжа для отображений]\itemfix
    \begin{itemize}
        \item $E$ открыто
        \item $F : E\subset \R^m \to\R^l$
        \item $F$ дифф. на $E$
        \item $a, b\in E$
        \item $[a, b]\in E$
    \end{itemize}
    Тогда $\exists c\in[a, b]\ (c = a + \Theta(b-a)), \Theta\in[0, 1]:$
    $$|F(b)-F(a)|\le||F'(c)|||b-a|$$
\end{theorem}
\begin{proof}
    $f(t) := F(a + t(b-a)), t\in\R$

    $f'(t) = F'(a+t(b-a))(b-a)$

    Тогда по теореме Лагранжа для векторнозначных функций
    $$|f(1)-f(0)| \le |f'(c)||1-0|$$
    $$|F(b)-F(a)| \le |F'(a+c(b-a))(b-a)| \le ||F'(a+c(b-a))|||b-a|$$
\end{proof}

\begin{remark}
    Особенно удобная оценка происходит, когда $E$ выпукло, тогда
    $$|f(b)-f(a)| \le \sup_{x\in E} ||F'(x)|||b-a|$$
\end{remark}

$\Omega_m := \{L\in \mathcal L_{m, m} : \exists L^{-1}\}$

\begin{lemma}\itemfix
    \begin{itemize}
        \item $B\in \mathcal L_{m, m}$
        \item $\exists c > 0 \ \ \forall x\in\R^m \ \ |Bx|\ge c|x|$
    \end{itemize}
    Тогда $B\in\Omega_m$ и $||B^{-1}||\le\cfrac{1}{c}$
\end{lemma}
\begin{proof}
    $B$ --- биекция, т.к. его ядро $\{0_m\} \Rightarrow \exists B^{-1}$.

    $$|B^{-1} y| \le ||B^{-1}||\cdot |y|$$
    $$x := B^{-1}y$$
    $$|y| \ge c\cdot|B^{-1}y|$$
    $$|B^{-1}y| \le \frac{1}{c}|y| \Rightarrow ||B^{-1}||\le \frac{1}{c}$$
\end{proof}
\begin{remark}
    Есть геометрическое доказательство, отталкивающееся от того, что $B$ сжимает пространство на $c$.
\end{remark}
\begin{remark}
    $A\in\Omega_m$. Тогда $\exists c : |Ax| \ge c|x|$
\end{remark}
\begin{proof}
    $$|x| = |A^{-1}Ax| \le ||A^{-1}|| \cdot |Ax| \quad c := \frac{1}{||A^{-1}||}$$
\end{proof}

\begin{theorem}[об обратимости оператора, близкого к обратимому]\itemfix
    %<*обобратимости>
    \begin{itemize}
        \item $L\in\Omega_m$
        \item $M\in\mathcal L_{m, m}$
        \item $||L-M|| < \cfrac{1}{||L^{-1}||}$ --- $M$ ``близкий'' к $L$
    \end{itemize}
    Тогда:
    \begin{enumerate}
        \item $M\in\Omega_m$, т.е. $\Omega_m$ открыто в $\mathcal L_{m, m}$
        \item $||M^{-1}|| \le \cfrac{1}{||L^{-1}||^{-1} - ||L-M||}$
        \item $||L^{-1} - M^{-1}|| \le \cfrac{||L^{-1}||}{||L^{-1}||^{-1} - ||L-M||}||L-M||$
    \end{enumerate}
    %</обобратимости>
\end{theorem}
%<*обобратимостиproof>
\begin{proof}
    По неравенству треугольника $|a+b|\ge |a|-|b|$:
    \begin{align*}
        |Mx| & = |Lx + (M-L)x|                               \\
             & \ge |Lx| - |(M-L)x|                           \\
             & \ge \frac{1}{||L||^{-1}}|x| - ||M-L||\cdot|x| \\
             & \ge \left(||L^{-1}||^{-1} - ||M-L||\right)|x|
    \end{align*}
    Это доказало пункты 1 и 2, докажем 3:

    Аналогично равенству $\frac{1}{a} + \frac{1}{b} = \frac{a+b}{ab}$ в $\R$ выполняется следующее равенство в $\Omega_m$:
    $$M^{-1} - L^{-1} = M^{-1}(L-M)L^{-1}$$
    Это очевидно доказывается домножением на $M$ слева и на $L$ справа:
    \begin{proof}
        $$M^{-1} - L^{-1} \stackrel{?}{=} M^{-1}(L-M)L^{-1}$$
        $$E - L^{-1} \stackrel{?}{=} (L-M)L^{-1}$$
        $$L - M = L-M$$
    \end{proof}
    $$||M^{-1} - L^{-1}|| = ||M^{-1}(L-M)L^{-1}|| \le \frac{||L^{-1}||}{||L^{-1}||^{-1} - ||L-M||}||L-M||$$
\end{proof}
%</обобратимостиproof>

\begin{theorem}[о непрерывно дифференцируемом отображении]\itemfix
    \begin{itemize}
        \item $F:E\subset\R^m \to\R^l$
        \item $F$ дифф. на $E$
    \end{itemize}
    Тогда эквивалентны 1 и 2:
    \begin{enumerate}
        \item $F\in C^1(E)$, т.е. $\exists$ все $\frac{\partial F_i}{\partial x_j}$ и они непрерывны на $E$
        \item $F' : E\to\mathcal L_{m, l}$ --- непрерывно, т.е. $$\forall x\in E \ \ \forall \varepsilon > 0 \ \ \exists \delta = \delta(\varepsilon, x) \ \ \forall \overline x : |\overline x - x| < \delta \quad ||F'(x) - F'(\overline x)|| \le \varepsilon$$
    \end{enumerate}
\end{theorem}
\begin{proof}\itemfix
    \begin{itemize}
        \item 1 $\Rightarrow$ 2:

              Берем $x, \varepsilon$. $\exists \delta > 0 : \forall \overline x \ \ \left|\frac{\partial F_i}{\partial x_j}(x) - \frac{\partial F_i}{\partial x_j}(\overline x)\right| < \frac{\varepsilon}{\sqrt{ml}}$ для всех $i, j$.

              $$||F'(x)|-F'(\overline x)|| < \sqrt{\sum_{i, j} \frac{\varepsilon^2}{ml}} = \varepsilon$$

        \item 2 $\Rightarrow$ 1:

              $$\forall \varepsilon > 0 \ \ \exists \delta > 0 \ \ \forall \overline x : |x - \overline x| < \delta \quad ||F'(x)-F'(\overline x)|| < \varepsilon$$

              $\sphericalangle h = (0, 0, \ldots, 0, \underbrace{1}_{j}, 0, \ldots, 0)$
              $$|F'(x)h - F'(\overline x)h| \le ||F'(x)-F'(\overline x)|| \cdot |h| < \varepsilon$$
              $$|F'(x)h - F'(\overline x)h| = \sqrt{\sum_{i=1}^l \left( \frac{\partial F_i}{\partial x_j}(x) - \frac{\partial F_i}{\partial x_j}(\overline x) \right)^2}$$
              $$\sqrt{\sum_{i=1}^l \left( \frac{\partial F_i}{\partial x_j}(x) - \frac{\partial F_i}{\partial x_j}(\overline x) \right)^2} < \varepsilon \Rightarrow \forall i \ \ \left| \frac{\partial F_i}{\partial x_j}(x) - \frac{\partial F_i}{\partial x_j}(\overline x) \right| < \varepsilon$$
    \end{itemize}
\end{proof}

\section{Экстремумы}

\begin{definition}
    %<*экстремум>
    $f : E\subset\R^m \to \R, a\in E$ --- \textbf{локальный максимум}, если
    $$\exists U(a)\subset E \ \ \forall x\in U(a) \quad f(x)\le f(a)$$

    Аналогично определяется строгий локальный максимум, локальный минимум и строгий локальный минимум
    %</экстремум>
\end{definition}

\begin{theorem}[Ферма]\itemfix
    %<*ферма>
    \begin{itemize}
        \item $f : E\subset \R^m \to \R$
        \item $a\in Int E$
        \item $a$ --- точка локального экстремума
        \item $f$ --- дифф. в точке $a$
    \end{itemize}
    Тогда $\forall u\in\R^m : |u|=1 \quad \cfrac{\partial f}{\partial u} = 0$
    %</ферма>
\end{theorem}
%<*фермаproof>
\begin{proof}
    Для $f\Big|_{\text{прямая}(a, u)}$ $a$ остается локальным экстремумом, выполняется одномерная теорема Ферма.
\end{proof}
%</фермаproof>

\begin{corollary}[необходимое условие экстремума]
    $a$ --- локальный экстремум $f \Rightarrow \frac{\partial f}{\partial x_1}(a) \ldots \frac{\partial f}{\partial x_m}(a)$
\end{corollary}

\begin{corollary}[теорема Ролля]\itemfix
    %<*ролля>
    \begin{itemize}
        \item $f: E\subset \R^m \to \R$
        \item $K\subset E$ компакт
        \item $f$ дифф. в $Int K$
        \item $f$ непрерывно на $K$
        \item $f\Big|_{\text{граница} K} = \const$
    \end{itemize}

    Тогда $\exists a\in Int K : f'(a)=\vec 0$
    %</ролля>
\end{corollary}
%<*ролляproof>
\begin{proof}
    По теореме Вейерштрасса $f$ достигает минимального и максимального значения на компакте. Тогда либо $f$ на $K$ $\const$, либо $\exists a\in Int K$ --- точка экстремума. В первом случае $f'\equiv 0$, во втором по т. Ферма $f'(a)=0$
\end{proof}
%</ролляproof>

\begin{definition}
    \textbf{Квадратичная форма} $Q : \R^m \to\R$
    $$Q(h) = \sum_{1 \le i, j\le m} a_{ij}h_ih_j$$
\end{definition}

\begin{definition}
    \textbf{Положительно определенная кв. форма}: $\forall h\not=0 \ \ Q(h) > 0$
\end{definition}
\begin{definition}
    \textbf{Отрицательно определенная кв. форма}: $\forall h\not=0 \ \ Q(h) < 0$
\end{definition}
\begin{definition}
    \textbf{Неопределенная кв. форма}: $\exists \overline h : Q(h) < 0, \exists \tilde h : Q(h) > 0$
\end{definition}
\begin{definition}
    \textbf{Полуопределенная (\textit{положительно определенная вырожденная}) кв. форма}: $Q(h) \ge 0 \ \ \exists \overline h\not=0 : Q(\overline h) = 0$
\end{definition}

\begin{lemma}\itemfix
    \begin{itemize}
        \item $Q$ --- положительно определенная
    \end{itemize}
    Тогда $\exists \gamma_Q > 0 \ \ \forall h \quad Q(h) \ge \gamma_Q |h|^2$
\end{lemma}
\begin{proof}
    $S^{m-1} := \{x\in\R^m : |x| = 1\}$ --- компакт, поэтому $\min$ и $\max$ достигается по т. Вейерштрасса.
    $$\gamma_Q := \min\limits_{h\in S^{m-1}} Q(h) > 0$$
    \begin{align*}
        Q(h) & \ge \gamma_Q |h|^2                                              \\
        Q(h) & = Q\left(|h|\frac{h}{|h|}\right)=|h^2|Q\left(\frac{h}{h}\right)
    \end{align*}
    \textcolor{red}{Не дописано}
\end{proof}
\begin{lemma}\itemfix
    \begin{itemize}
        \item $p : \R^m \to \R$ --- норма
    \end{itemize}
    Тогда $\exists C_1,C_2 > 0 \ \ \forall x \ \ C_2|x| \le p(x) \le C_1|x|$
\end{lemma}
\begin{proof}
    $$C_1 := \min_{x\in S^{m-1}} p(x) \quad C_2 := \max_{x\in S^{m-1}} p(x)$$
    $$p(x) = p\left(|x|\frac{x}{|x|}\right) = |x|p\left(\frac{x}{|x|}\right)\begin{cases}
            \ge C_2|x| \\
            \le C_1|x|
        \end{cases}$$
    Существование $C_1$ и $C_2$ гарантируется теоремой Вейерштрасса, но она требует непрерывности $p(x)$.

    Докажем, что $p$ непрерывна.

    Введем базис $\{e_i\}_{i=1}^n$. Тогда
    \begin{align*}
        p(x-y) & = p\left(\sum (x_k - y_k)e_k \right) \\
               & \le \sum p((x_k - y_k)e_k)           \\
               & = \sum |x_k-y_k|p(e_k)               \\
               & \le |x-y| \sqrt{\sum p(e_k)^2}       \\
               & \le |x-y|M                           \\
    \end{align*}
\end{proof}

\subsection*{Напоминание}

$$d^2 f(a, h) = f''_{x_1x_1}(a) h_1^2 + \ldots + f''_{x_mx_m}(a) h_m^2 + 2 \sum_{1\le i < j \le m} f''_{x_ix_j}h_ih_j$$
Формула Тейлора с остатком в форме Пеано:
$$f(x) = f(a) + df(a, x-a) + \frac{1}{2!}d^2f(a, x-a) + o(|x-a|^2)$$
Формула Тейлора с остатком в форме Лагранжа:
$$f(a+h) = f(a) + df(a, h) + \frac{1}{2} d^2f(a + \Theta h, h) \quad 0\le\Theta\le 1$$

\begin{theorem}[достаточное условие экстремума]\itemfix
    \begin{itemize}
        \item $f : E\subset \R^m \to \R$
        \item $a\in Int E$
        \item $\frac{\partial f}{\partial x_1}(a) = 0, \ldots, \frac{\partial f}{\partial x_m}(a) = 0$
        \item $Q(h) := d^2 f(a, h)$
        \item $f\in C^2(E)$
    \end{itemize}
    Тогда:
    \begin{itemize}
        \item Если $Q(h)$ положительно определена, $a$ --- локальный минимум
        \item Если $Q(h)$ отрицательно определена, $a$ --- локальный максимум
        \item Если $Q(h)$ незнакоопределена, $a$ --- не экстремум
        \item Если $Q(h)$ положительно определена, но вырождена, недостаточно информации.
    \end{itemize}
\end{theorem}
\begin{proof}
    \begin{align*}
        f(a+h) & = f(a)                                                                                                                                                                                                                                                                                                    \\
               & = \frac{1}{2} d^2f(a+\Theta h, h)                                                                                                                                                                                                                                                                         \\
               & = \frac{1}{2} \left( Q(h) + \underbrace{\sum_{i=1}^n \underbrace{\left( f''_{x_ix_i} (a+\Theta h) - f''_{x_ix_i} (a)\right)}_{\to0} \underbrace{h_i^2}_{\le |h|^2} + 2 \sum_{i < j} \underbrace{\left(f_{x_ix_i}''(a+\Theta h) - f''_{x_ix_i} (a)\right)}_{\to0} \underbrace{h_ih_j}_{\substack{\le |h|^2 \\ \text{по модулю}}}}_{o(|h|^2) \Leftrightarrow < \frac{\gamma_Q}{2}|h|^2}  \right) \\
    \end{align*}
    $$f(a+h) - f(a) \ge \frac{1}{2}\left(\gamma_Q |h|^2 - \frac{\gamma_Q}{2}|h|^2\right) \ge \frac{1}{4}\gamma_Q|h|^2 > 0$$

    $\sphericalangle \overline h : Q(\overline h) > 0 \Rightarrow f(a+t\overline h) - f(a) = \frac{1}{2}d^2 f(a+\Theta t\overline h, \overline h)t^2$

    \textcolor{red}{Не дописано}
\end{proof}

\end{document}