\documentclass[12pt, a4paper]{article}

%<*preamble>
% Math symbols
\usepackage{amsmath, amsthm, amsfonts, amssymb}
\usepackage{accents}
\usepackage{esvect}
\usepackage{mathrsfs}
\usepackage{mathtools}
\mathtoolsset{showonlyrefs}
\usepackage{cmll}
\usepackage{stmaryrd}
\usepackage{physics}
\usepackage[normalem]{ulem}
\usepackage{ebproof}
\usepackage{extarrows}

% Page layout
\usepackage{geometry, a4wide, parskip, fancyhdr}

% Font, encoding, russian support
\usepackage[russian]{babel}
\usepackage[sb]{libertine}
\usepackage{xltxtra}

% Listings
\usepackage{listings}
\lstset{basicstyle=\ttfamily,breaklines=true}
\setmonofont{Inconsolata}

% Miscellaneous
\usepackage{array}
\usepackage{calc}
\usepackage{caption}
\usepackage{subcaption}
\captionsetup{justification=centering,margin=2cm}
\usepackage{catchfilebetweentags}
\usepackage{enumitem}
\usepackage{etoolbox}
\usepackage{float}
\usepackage{lastpage}
\usepackage{minted}
\usepackage{svg}
\usepackage{wrapfig}
\usepackage{xcolor}
\usepackage[makeroom]{cancel}

\newcolumntype{L}{>{$}l<{$}}
    \newcolumntype{C}{>{$}c<{$}}
\newcolumntype{R}{>{$}r<{$}}

% Footnotes
\usepackage[hang]{footmisc}
\setlength{\footnotemargin}{2mm}
\makeatletter
\def\blfootnote{\gdef\@thefnmark{}\@footnotetext}
\makeatother

% References
\usepackage{hyperref}
\hypersetup{
    colorlinks,
    linkcolor={blue!80!black},
    citecolor={blue!80!black},
    urlcolor={blue!80!black},
}

% tikz
\usepackage{tikz}
\usepackage{tikz-cd}
\usetikzlibrary{arrows.meta}
\usetikzlibrary{decorations.pathmorphing}
\usetikzlibrary{calc}
\usetikzlibrary{patterns}
\usepackage{pgfplots}
\pgfplotsset{width=10cm,compat=1.9}
\newcommand\irregularcircle[2]{% radius, irregularity
    \pgfextra {\pgfmathsetmacro\len{(#1)+rand*(#2)}}
    +(0:\len pt)
    \foreach \a in {10,20,...,350}{
            \pgfextra {\pgfmathsetmacro\len{(#1)+rand*(#2)}}
            -- +(\a:\len pt)
        } -- cycle
}

\providetoggle{useproofs}
\settoggle{useproofs}{false}

\pagestyle{fancy}
\lfoot{M3137y2019}
\cfoot{}
\rhead{стр. \thepage\ из \pageref*{LastPage}}

\newcommand{\R}{\mathbb{R}}
\newcommand{\Q}{\mathbb{Q}}
\newcommand{\Z}{\mathbb{Z}}
\newcommand{\B}{\mathbb{B}}
\newcommand{\N}{\mathbb{N}}
\renewcommand{\Re}{\mathfrak{R}}
\renewcommand{\Im}{\mathfrak{I}}

\newcommand{\const}{\text{const}}
\newcommand{\cond}{\text{cond}}

\newcommand{\teormin}{\textcolor{red}{!}\ }

\DeclareMathOperator*{\xor}{\oplus}
\DeclareMathOperator*{\equ}{\sim}
\DeclareMathOperator{\sign}{\text{sign}}
\DeclareMathOperator{\Sym}{\text{Sym}}
\DeclareMathOperator{\Asym}{\text{Asym}}

\DeclarePairedDelimiter{\ceil}{\lceil}{\rceil}

% godel
\newbox\gnBoxA
\newdimen\gnCornerHgt
\setbox\gnBoxA=\hbox{$\ulcorner$}
\global\gnCornerHgt=\ht\gnBoxA
\newdimen\gnArgHgt
\def\godel #1{%
    \setbox\gnBoxA=\hbox{$#1$}%
    \gnArgHgt=\ht\gnBoxA%
    \ifnum     \gnArgHgt<\gnCornerHgt \gnArgHgt=0pt%
    \else \advance \gnArgHgt by -\gnCornerHgt%
    \fi \raise\gnArgHgt\hbox{$\ulcorner$} \box\gnBoxA %
    \raise\gnArgHgt\hbox{$\urcorner$}}

% \theoremstyle{plain}

\theoremstyle{definition}
\newtheorem{theorem}{Теорема}
\newtheorem*{definition}{Определение}
\newtheorem{axiom}{Аксиома}
\newtheorem*{axiom*}{Аксиома}
\newtheorem{lemma}{Лемма}

\theoremstyle{remark}
\newtheorem*{remark}{Примечание}
\newtheorem*{exercise}{Упражнение}
\newtheorem{corollary}{Следствие}[theorem]
\newtheorem*{statement}{Утверждение}
\newtheorem*{corollary*}{Следствие}
\newtheorem*{example}{Пример}
\newtheorem{observation}{Наблюдение}
\newtheorem*{prop}{Свойства}
\newtheorem*{obozn}{Обозначение}

% subtheorem
\makeatletter
\newenvironment{subtheorem}[1]{%
    \def\subtheoremcounter{#1}%
    \refstepcounter{#1}%
    \protected@edef\theparentnumber{\csname the#1\endcsname}%
    \setcounter{parentnumber}{\value{#1}}%
    \setcounter{#1}{0}%
    \expandafter\def\csname the#1\endcsname{\theparentnumber.\Alph{#1}}%
    \ignorespaces
}{%
    \setcounter{\subtheoremcounter}{\value{parentnumber}}%
    \ignorespacesafterend
}
\makeatother
\newcounter{parentnumber}

\newtheorem{manualtheoreminner}{Теорема}
\newenvironment{manualtheorem}[1]{%
    \renewcommand\themanualtheoreminner{#1}%
    \manualtheoreminner
}{\endmanualtheoreminner}

\newcommand{\dbltilde}[1]{\accentset{\approx}{#1}}
\newcommand{\intt}{\int\!}

% magical thing that fixes paragraphs
\makeatletter
\patchcmd{\CatchFBT@Fin@l}{\endlinechar\m@ne}{}
{}{\typeout{Unsuccessful patch!}}
\makeatother

\newcommand{\get}[2]{
    \ExecuteMetaData[#1]{#2}
}

\newcommand{\getproof}[2]{
    \iftoggle{useproofs}{\ExecuteMetaData[#1]{#2proof}}{}
}

\newcommand{\getwithproof}[2]{
    \get{#1}{#2}
    \getproof{#1}{#2}
}

\newcommand{\import}[3]{
    \subsection{#1}
    \getwithproof{#2}{#3}
}

\newcommand{\given}[1]{
    Дано выше. (\ref{#1}, стр. \pageref{#1})
}

\renewcommand{\ker}{\text{Ker }}
\newcommand{\im}{\text{Im }}
\renewcommand{\grad}{\text{grad}}
\newcommand{\rg}{\text{rg}}
\newcommand{\defeq}{\stackrel{\text{def}}{=}}
\newcommand{\defeqfor}[1]{\stackrel{\text{def } #1}{=}}
\newcommand{\itemfix}{\leavevmode\makeatletter\makeatother}
\newcommand{\?}{\textcolor{red}{???}}
\renewcommand{\emptyset}{\varnothing}
\newcommand{\longarrow}[1]{\xRightarrow[#1]{\qquad}}
\DeclareMathOperator*{\esup}{\text{ess sup}}
\newcommand\smallO{
    \mathchoice
    {{\scriptstyle\mathcal{O}}}% \displaystyle
    {{\scriptstyle\mathcal{O}}}% \textstyle
    {{\scriptscriptstyle\mathcal{O}}}% \scriptstyle
    {\scalebox{.6}{$\scriptscriptstyle\mathcal{O}$}}%\scriptscriptstyle
}
\renewcommand{\div}{\text{div}\ }
\newcommand{\rot}{\text{rot}\ }
\newcommand{\cov}{\text{cov}}

\makeatletter
\newcommand{\oplabel}[1]{\refstepcounter{equation}(\theequation\ltx@label{#1})}
\makeatother

\newcommand{\symref}[2]{\stackrel{\oplabel{#1}}{#2}}
\newcommand{\symrefeq}[1]{\symref{#1}{=}}

% xrightrightarrows
\makeatletter
\newcommand*{\relrelbarsep}{.386ex}
\newcommand*{\relrelbar}{%
    \mathrel{%
        \mathpalette\@relrelbar\relrelbarsep
    }%
}
\newcommand*{\@relrelbar}[2]{%
    \raise#2\hbox to 0pt{$\m@th#1\relbar$\hss}%
    \lower#2\hbox{$\m@th#1\relbar$}%
}
\providecommand*{\rightrightarrowsfill@}{%
    \arrowfill@\relrelbar\relrelbar\rightrightarrows
}
\providecommand*{\leftleftarrowsfill@}{%
    \arrowfill@\leftleftarrows\relrelbar\relrelbar
}
\providecommand*{\xrightrightarrows}[2][]{%
    \ext@arrow 0359\rightrightarrowsfill@{#1}{#2}%
}
\providecommand*{\xleftleftarrows}[2][]{%
    \ext@arrow 3095\leftleftarrowsfill@{#1}{#2}%
}

\allowdisplaybreaks

\newcommand{\unfinished}{\textcolor{red}{Не дописано}}

% Reproducible pdf builds 
\special{pdf:trailerid [
<00112233445566778899aabbccddeeff>
<00112233445566778899aabbccddeeff>
]}
%</preamble>


\lhead{Математический анализ}
\cfoot{}
\rfoot{12.10.2020}

\begin{document}

\begin{lemma}\itemfix
    %<*касательноепространство>
    \begin{itemize}
        \item $\Phi : O\subset\R^k \to\R^m$
        \item $\Phi\in C^r$
        \item $\Phi$ --- параметризация многообразия $U(p)\cap M$, где $p\in M$, $M$ --- гладкое $k$-мерное многообразие $\Rightarrow U(p)\cap M$ --- простое многообразие
        \item $\Phi(t^0) = p$
    \end{itemize}
    Тогда образ $\Phi'(t^0) : \R^k \to \R^m$ есть $k$-мерное линейное подпространство в $\R^m$. Оно не зависит от $\Phi$.

    $\Phi'(t^0)$ --- \textbf{касательное пространство} к $M$ в точке $p$, обозначается $T_pM$.
    %</касательноепространство>
\end{lemma}
%<*касательноепространство-proof>
\begin{proof}
    $\rg \Phi'(t^0) = k$ по определению параметризации $\Rightarrow$ искомое очевидно. Если взять другую параметризацию $\Phi_1$, то по следствию о двух параметризациях $$\Phi = \Phi_1 \circ \Psi$$
    $$\Phi' = \Phi'_1 \Psi'$$
    $\Psi'(t^0)$ --- невырожденный оператор $\Rightarrow$ образ $\Phi'$ = образ $\Phi_1'$
\end{proof}
%</касательноепространство-proof>

\begin{example}
    $M$ --- окружность в $\R^2$, задается параметризацией $\Phi : t\mapsto \begin{pmatrix}
            \cos t \\
            \sin t
        \end{pmatrix}, t^0 := \frac{\pi}{4}$
    \begin{figure}[h]
        \begin{tikzpicture}
            \draw[-{>[length=2mm,width=3mm]}] (-3, 0) -- (3, 0) node[anchor=north west] {$x$};
            \draw[-{>[length=2mm,width=3mm]}] (0, -3) -- (0, 3) node[anchor=north west] {$y$};
            \draw (0, 0) circle (2);
            \filldraw[black] (1.41, 1.41) circle (1pt) node[anchor=south] {$a$};
            \draw[dashed] (0, 0) -- (1.41, 1.41);
            \draw (0.5, 0) arc (0:45:0.5);
            \node (b) at (0.7, 0.3) {$\frac{\pi}{4}$};
            \draw[blue, dashed, thick] (-3, 3) -- (3, -3);
            \draw[green, dashed, thick] (-1.59, 4.41) -- (4.41, -1.59);
            % \node (1) at (1, 0)
        \end{tikzpicture}
        \centering
        \caption{Синим --- касательное пространство к окружности в $a$.\\ Зеленым --- афинное (``сдвинутое'') линейное подпространство.}
        \centering
    \end{figure}
    $$\Phi'(t^0) = \begin{pmatrix}
            -\frac{\sqrt{2}}{2} \\
            \frac{\sqrt{2}}{2}
        \end{pmatrix}$$

    Тогда $h \mapsto \begin{pmatrix}
            -\frac{\sqrt{2}}{2} \\
            \frac{\sqrt{2}}{2}
        \end{pmatrix} h$ --- касательное подпространство.
\end{example}

\begin{definition}
    Афинное подпространство $\{p + v, v\in T_pM\}$ называется \textbf{афинным касательным подпространством}.
\end{definition}

\begin{remark}\itemfix
    \begin{enumerate}
        \item $v\in T_pM$. Тогда $\exists$ путь $\gamma_v : [-\varepsilon, \varepsilon] \to M$, такой что $\gamma(0)=p, \gamma'(0)=v$
              \begin{proof}
                  $u:=\left(\Phi'(t_0)\right)^{-1}(v)$

                  \begin{tikzpicture}
                      \draw[-{>[length=2mm,width=3mm]}] (0, 0) -- (5, 0) node[anchor=north west] {$x$};
                      \draw[-{>[length=2mm,width=3mm]}] (0, 0) -- (0, 5) node[anchor=north west] {$y$};
                      \draw[-{>[length=2mm,width=3mm]}] (0, 0) -- (-2, -2) node[anchor=north west] {$z$};
                      \draw (0.5, 3) .. controls (2, 2.5) .. (3, 1);
                      \draw (0.5, 3) .. controls (2.5, 3.5) .. (4, 3);
                      \draw (3, 1) -- (4, 3);
                      \node at (2, -2) {$\R^m$};

                      \draw (-5, 0) ellipse (2 and 1);
                      \node (a) at (-5, -1.5) {$\R^k$};
                      \filldraw[black] (-5, 0) circle (1pt) node[anchor=south] {$t_0$};
                      \draw[->] (-5, 0) -- (-4, 0.5) node[anchor=north] {$\vec u$};

                      \draw[->] (-4, 1) .. controls (-3, 2) and (-2, 2.5) .. (-0.5, 2.5) node[midway, above] {$\Phi$};

                      \draw[thick, rotate around={-30:(2, 3)}] (2, 2.5) rectangle (3.5, 3.5);
                      \filldraw[black] (2.5, 2.5) circle (1pt);
                      \draw[->] (2.5, 2.5) -- (3, 2.7) node[anchor=west] {$\vec v$};
                  \end{tikzpicture}

                  Определим путь в $\R^k$:
                  $$\tilde \gamma_v(s) := t_0 + su, s\in[-\varepsilon, \varepsilon]$$
                  Отобразим этот путь в $M$ и проверим, что такой путь подходит под условие:
                  $$\gamma_v(s) := \Phi(\tilde \gamma_v(s))$$
                  $$\gamma_v'(0) = \Phi'(\tilde \gamma_v(0))\cdot \tilde \gamma'_v(0) = \Phi'(\tilde \gamma_v(0))\cdot u = \Phi'(\tilde \gamma_v(0)) \left(\Phi'(t_0)\right)^{-1}(v) = v$$
              \end{proof}
        \item %<*касательноепространствовтерминахвекторовскоростигладкихпутей>
              Пусть $\gamma : [-\varepsilon, \varepsilon]\to M, \gamma(0) = p$ --- гладкий путь. Тогда $\gamma'(0) \in T_pM$
              %</касательноепространствовтерминахвекторовскоростигладкихпутей>

              %<*касательноепространствовтерминахвекторовскоростигладкихпутейproof>
              \begin{figure}[H]
                  \begin{tikzpicture}
                      \draw[-{>[length=2mm,width=3mm]}] (-1, 0) -- (5, 0) node[anchor=north west] {$x$};
                      \draw[-{>[length=2mm,width=3mm]}] (0, -1) -- (0, 5) node[anchor=south east] {$y$};

                      \node at (-0.5, -0.5) {$\R^2$};
                      \node (t) [draw, circle, minimum size = 20mm, dashed, label=above:$W(t^0)$] at (2.5, 2.5) {$t^0$};

                      \draw[-{>[length=2mm,width=3mm]}] (7, 0) -- (13, 0) node[anchor=north west] {$\R^k$};
                      \draw[-{>[length=2mm,width=3mm]}] (8, -1) -- (8, 5) node[anchor=south east] {$\R^{m-k}$};
                      \draw (10, 4) .. controls (12, 3.5) .. (13, 2);
                      \draw (9.5, 3) .. controls (11.5, 2.5) .. (12.5, 1);
                      \draw (10, 4) -- (9.5, 3);
                      \draw (13, 2) -- (12.5, 1);

                      \node (circ) at (10.5, 3.3) {};
                      \draw[] (circ) \irregularcircle{0.4cm}{0.5mm};
                      \draw[] ($ (circ) - (0, 5) $) \irregularcircle{0.4cm}{0.5mm};
                      \draw[->] (circ) -- ($ (circ) - (0, 5) $) node[midway, left] {$L$};

                      \draw[->] ($ (circ) - (0, 5) + (-0.3, 0.3) $) .. controls (7, 1) .. (t) node[midway, above] {$\Psi$};
                      \draw[->] (t) .. controls (6, 0) .. ($ (circ) - (0, 5) + (-0.5, 0) $) node[midway, below] {$L \circ \Phi$};

                      \draw[->] (4.5, 4) -- (6.5, 4) node[midway, above] {$\Phi$};
                  \end{tikzpicture}
              \end{figure}
              \begin{proof}
                  Из иллюстрации очевидно:
                  $$\gamma(s) = \Phi \circ \Psi \circ L \circ \gamma(s)$$
                  $$\gamma'(0) = \Phi'(\Psi(L(\gamma(0)))) \Psi'(L(\gamma(0))) L'(\gamma(0)) \gamma'(0)$$
                  Очевидно, что мы попадаем в образ \(\Phi'(\dots)\), поэтому \(\gamma'(0)\in T_pM\)
              \end{proof}
              %</касательноепространствовтерминахвекторовскоростигладкихпутейproof>
        \item %<*касательноепространствокграфику>
              $f : O\subset\R^m \to\R, f\in C(O), y=f(x)$ --- поверхность в $\R^{m+1}$, задается точками $(x, y)$.

              Тогда (аффинная) касательная плоскость в точке $(a, b)$ задается уравнением
              $$y-b = f'_{x_1}(a)(x_1-a_1) + f'_{x_2}(a)(x_2-a_2) + \ldots + f'_{x_m}(a)(x_m-a_m)$$
              %</касательноепространствокграфику>

              %<*касательноепространствокграфикуproof>
              \begin{proof}
                  $\Phi : O\subset\R^m \to\R^{m+1}$

                  $\Phi(x) = (x, f(x))$

                  $$\Phi' = \begin{bmatrix}
                          1        & 0        & \ldots & 0        \\
                          0        & 1        & \ldots & 0        \\
                          \vdots   & \vdots   & \ddots & \vdots   \\
                          0        & 0        & \ldots & 1        \\
                          f'_{x_1} & f'_{x_2} & \ldots & f'_{x_m}
                      \end{bmatrix}$$
                  Рассмотрим произвольный вектор $\begin{pmatrix}
                          \alpha_1 \\
                          \vdots   \\
                          \alpha_m \\
                          \beta
                      \end{pmatrix}$. В каких случаях он принадлежит образу $\Phi'$?

                  $$\Phi' \vec x = \begin{bmatrix}
                          1        & 0        & \ldots & 0        \\
                          0        & 1        & \ldots & 0        \\
                          \vdots   & \vdots   & \ddots & \vdots   \\
                          0        & 0        & \ldots & 1        \\
                          f'_{x_1} & f'_{x_2} & \ldots & f'_{x_m}
                      \end{bmatrix} \begin{pmatrix}
                          x_1    \\
                          \vdots \\
                          x_m
                      \end{pmatrix} = \begin{pmatrix}
                          x_1    \\
                          \vdots \\
                          x_m    \\
                          x_1 f'_{x_1} + \ldots + x_m f'_{x_m}
                      \end{pmatrix}$$
                  Таким образом, вектор принадлежит образу, если $\beta = \alpha_1 f'_{x_1} + \ldots + \alpha_m f'_{x_m}$

                  Сместив на \(a, b\), получаем искомое.
              \end{proof}
              %</касательноепространствокграфикуproof>
        \item %<*линииуровня>
              $\Phi(x_1\ldots x_m) = 0$

              $\Phi : O\subset\R^m \to\R$

              $\Phi(a) = 0$

              Тогда уравнение касательной плоскости $\Phi'_{x_1}(a)(x_1-a_1) + \ldots + \Phi'_{x_m}(a)(x_m-a_m) = 0$
              %</линииуровня>

              %<*линииуровняproof>
              \begin{proof}
                  $\gamma$ --- путь в $M$ : $\Phi(\gamma(s)) = 0, \Phi'(\gamma(s))\gamma'(s) = 0$. По предыдущим утверждениям такой путь есть, кроме того, любому вектору $x$ в касательном пространстве можно сопоставить $\gamma : \gamma'(s) = x$. Поэтому любой касательный вектор от точки $a$ должен быть подчинём искомому отношению.

                  Альтернативное доказательство:

                  По определению дифференцируемости $\Phi$ в точке $a$:
                  $$\Phi(x) = \Phi(a) + \Phi'_{x_1}(x_1-a_1) + \ldots + \Phi'_{x_m}(x_m-a_m) + \xcancel{o()}$$
                  Мы игнорируем $o$, потому что оно скомпенсируется тем, что мы берем не с поверхности $\Phi$, а с касательной плоскости. Это нестрогое утверждение.
              \end{proof}
              %</линииуровняproof>
    \end{enumerate}
\end{remark}

\begin{remark}
    $y(x) = f(a) + f'_{x_1}(a)(x_1-a_1)+\ldots + f'_{x_m}(a)(x_m-a_m)\Leftrightarrow$ дифференцирование, но без $o$.

    Таким образом, $f(x) - y(x) = o(x-a)$

    % TODO
    % \begin{tikzpicture}
    %     \draw[-{>[length=2mm,width=3mm]}] (0, 0) -- (5, 0);
    %     \draw[-{>[length=2mm,width=3mm]}] (0, 0) -- (0, 5) node[anchor=north west] {$y$};
    %     \draw[-{>[length=2mm,width=3mm]}] (0, 0) -- (-2, -2);
    % \end{tikzpicture}
\end{remark}

\subsection*{Относительный экстремум}

\begin{example}
    Найти наибольшее/наименьшее значение выражения $f(x, y) = x + y$ при условии $x^2+y^2=1$.

    Рассмотрим линии уровня, т.е. $f(x, y) = C$:

    \begin{figure}[h]
        \begin{tikzpicture}
            \draw[-{>[length=2mm,width=3mm]}] (-3, 0) -- (3, 0) node[anchor=north west] {$x$};
            \draw[-{>[length=2mm,width=3mm]}] (0, -3) -- (0, 3) node[anchor=north west] {$y$};
            \draw (0, 0) circle (30pt);
            \node at (1.2, 0.2) {$1$};
            \draw[gray, thin] (-3, 3) -- (3, -3);
            \draw[gray, thin] (-2.75, 3) -- (3, -2.75);
            \draw[gray, thin] (-2.50, 3) -- (3, -2.50);
            \draw[gray, thin] (-2.25, 3) -- (3, -2.25);
            \draw[gray, thin] (-2, 3) -- (3, -2);
            \draw[gray, thin] (-1.75, 3) -- (3, -1.75);
            \draw[gray, thin] (-1.50, 3) -- (3, -1.50);
            \draw[gray, thin] (-1.25, 3) -- (3, -1.25);
            \draw[gray, thin] (-1, 3) -- (3, -1);
            \draw[gray, thin] (-0.75, 3) -- (3, -0.75);
            \draw[gray, thin] (-0.50, 3) -- (3, -0.50);
            \draw[gray, thin] (-0.25, 3) -- (3, -0.25);
            \draw[gray, thin] (0, 3) -- (3, 0);
            \draw[gray, thin] (0.25, 3) -- (3, 0.25);
            \draw[gray, thin] (0.50, 3) -- (3, 0.50);
            \draw[gray, thin] (0.75, 3) -- (3, 0.75);
            \draw[gray, thin] (1, 3) -- (3, 1);
            \draw[gray, thin] (1.25, 3) -- (3, 1.25);
            \draw[gray, thin] (1.50, 3) -- (3, 1.50);
            \draw[gray, thin] (1.75, 3) -- (3, 1.75);
            \draw[gray, thin] (2, 3) -- (3, 2);
            \draw[gray, thin] (2.25, 3) -- (3, 2.25);
            \draw[gray, thin] (2.50, 3) -- (3, 2.50);
            \draw[gray, thin] (2.75, 3) -- (3, 2.75);

            \draw[gray, thin] (-3, 2.75) -- (2.75, -3);
            \draw[gray, thin] (-3, 2.50) -- (2.50, -3);
            \draw[gray, thin] (-3, 2.25) -- (2.25, -3);
            \draw[gray, thin] (-3, 2) -- (2, -3);
            \draw[gray, thin] (-3, 1.75) -- (1.75, -3);
            \draw[gray, thin] (-3, 1.50) -- (1.50, -3);
            \draw[gray, thin] (-3, 1.25) -- (1.25, -3);
            \draw[gray, thin] (-3, 1) -- (1, -3);
            \draw[gray, thin] (-3, 0.75) -- (0.75, -3);
            \draw[gray, thin] (-3, 0.50) -- (0.50, -3);
            \draw[gray, thin] (-3, 0.25) -- (0.25, -3);
            \draw[gray, thin] (-3, 0) -- (0, -3);
            \draw[gray, thin] (-3, -0.25) -- (-0.25, -3);
            \draw[gray, thin] (-3, -0.50) -- (-0.50, -3);
            \draw[gray, thin] (-3, -0.75) -- (-0.75, -3);
            \draw[gray, thin] (-3, -1) -- (-1, -3);
            \draw[gray, thin] (-3, -1.25) -- (-1.25, -3);
            \draw[gray, thin] (-3, -1.50) -- (-1.50, -3);
            \draw[gray, thin] (-3, -1.75) -- (-1.75, -3);
            \draw[gray, thin] (-3, -2) -- (-2, -3);
            \draw[gray, thin] (-3, -2.25) -- (-2.25, -3);
            \draw[gray, thin] (-3, -2.50) -- (-2.50, -3);
            \draw[gray, thin] (-3, -2.75) -- (-2.75, -3);

            \filldraw[black] (0.75, 0.75) circle (1pt) node[anchor=south] {$M$};
            \filldraw[black] (-0.75, -0.75) circle (1pt) node[anchor=north] {$m$};
        \end{tikzpicture}\centering
    \end{figure}

    $M$ и $m$ --- точки минимума и максимума, т.к. $M$ --- точка, в которой $f=\max$ и линия уровня касаются. Если они пересекаются, но не касаются, то есть точка больше. Аналогичное верно для минимума.

    В более формальных терминах: пусть условие $\Phi(x, y) = 0$.

    $\Phi'_x(x-a) + \Phi'_y(y-b) = 0$

    $(\Phi'_x, \Phi'_y)$ --- вектор нормали к касательной прямой. Тогда $(f'_x, f'_y)$ и $(\Phi'_x, \Phi'_y)$ --- параллельны.
\end{example}

\begin{definition}\itemfix
    %<*локальныйминимум>
    \begin{itemize}
        \item $f: O\subset\R^{m+n} \to\R$
        \item $M_\Phi \subset O := \{ x : \Phi(x) = 0 \}$
        \item $x_0\in M_\Phi$
    \end{itemize}
    $x_0$ --- точка \textbf{локального относительного} $\max, \min$, строгий $\max$, строгий $\min$, экстремума, если $\exists U(x_0)\subset\R^{m+n} : \forall x\in U(x_0)\cap M_\Phi \ \ f(x_0) \ge f(x)$, остальные --- аналогично.

    Уравнения $\Phi(x) = 0$ называются \textbf{уравнениями связи}.
    %</локальныйминимум>
\end{definition}

Как решать задачу нахождения локального относительного чего-то?

Если $\rg \Phi'(x_0) = n$, \textit{то есть $\rg$ максимален}, то выполнено условие теоремы о неявном отображении.

\begin{theorem}[необходимое условие относительного экстремума]\itemfix
    %<*необходимоеусловиеотносительногоэкстремума>
    \begin{itemize}
        \item $f: O\subset\R^{m+n} \to\R$ --- гладкое в $O$
        \item $M_\Phi \subset O := \{ x : \Phi(x) = 0 \}$ --- гладкое в $O$
        \item $a\in O$ --- точка относительного локального экстремума
        \item $\Phi(a) = 0$
        \item $\rg \Phi'(a) = n$
    \end{itemize}
    Тогда $\exists \lambda = (\lambda_1 \ldots \lambda_n)\in\R^n$ : $\begin{cases}
            f'(a) + \lambda\Phi'(a) = 0 \\
            \Phi(a) = 0
        \end{cases}$

    Второе условие дописано для удобства, оно не содержательно, т.к. оно уже дано.

    В координатах: $\begin{cases}
            f'_{x_1} + \lambda_1 (\Phi_1)_{x_1}' + \lambda_2 (\Phi_2)_{x_1}' + \ldots + \lambda_n (\Phi_n)_{x_1}' = 0                 \\
            \vdots                                                                                                                    \\
            f'_{x_{m+n}} + \lambda_1 (\Phi_1)_{x_{m+n}}' + \lambda_2 (\Phi_2)_{x_{m+n}}' + \ldots + \lambda_n (\Phi_n)_{x_{m+n}}' = 0 \\
            \Phi_1(a) = 0                                                                                                             \\
            \vdots                                                                                                                    \\
            \Phi_n(a) = 0                                                                                                             \\
        \end{cases}$

    Здесь неизвестны $a_1\ldots a_{m+n}, \lambda_1\ldots \lambda_n$, поэтому, если уравнения не вырождены, то решение есть.
    %</необходимоеусловиеотносительногоэкстремума>
\end{theorem}

%<*необходимоеусловиеотносительногоэкстремумаproof>
\begin{proof}
    $\rg \Phi'(a) = n$. Пусть ранг реализуется на столбцах $x_{m+1}\ldots x_{m+n}$.

    Обозначим $y_1 = x_{m+1} \ldots y_n = x_{m+n}$.

    $(x_1\ldots x_{m+n})\leftrightarrow (x, y), a \leftrightarrow (a_x, a_y)$.

    $\det \cfrac{\partial \Phi}{\partial y}(a) \not=0$. Тогда по теореме о неявном отображении $\exists U(a_x)\ \exists V(a_y)\ \exists \varphi : U(a_x) \to V(a_y) : \Phi(x, \varphi(x)) \equiv 0$ и отображение $x\mapsto (x, \varphi(x))$ есть параметризация простого гладкого многообразия $M_\varphi \cap (U(a_x) \times V(a_y))$.

    $a$ --- точка относительного локального экстремума $\Rightarrow a_x$ --- точка локального экстремума функции $g(x) = f(x, \varphi(x))$, потому что $(x, \varphi(x))\in U(a)$.

    Необходимое свойство экстремума для $a_x$:
    \begin{equation}
        (f'_x + f'_y \cdot \varphi')(a_x) = 0 \label{extr_prop}
    \end{equation}

    % TODO
    \begin{remark}
        Здесь и далее в этом доказательстве в функции и производные операторы подставляется $a$ и $a_x$, но не записывается ради краткости и запутанности.
    \end{remark}
    \begin{align*}
        \Phi(x, \varphi(x))              & \equiv 0 \\
        \Phi'_x + \Phi'_y \cdot \varphi' & = 0      \\
    \end{align*}
    Тогда $\forall \lambda \in\R^n$:
    \begin{equation}
        \lambda \Phi'_x + \lambda \Phi'_y \cdot \varphi' = 0 \label{bruh}
    \end{equation}

    \begin{equation}
        f'_x + \lambda \Phi'_x + (f'_y + \lambda \Phi'_y)\varphi' = 0 \label{total}
    \end{equation}

    \eqref{total} это \eqref{bruh} + \eqref{extr_prop}

    Пусть $\lambda = - f'_y \cdot (\Phi'_y(a))^{-1}$.

    Тогда $f'_y + \lambda \Phi'_y = f'_y - f'_y (\Phi'_y(a))^{-1}\Phi'_y(a) = 0$ и $f'_x + \lambda \Phi'_y = 0$ в силу \eqref{total}. Итого \eqref{total} выполнено, мы предъявили \(\lambda\), подходящее под искомое.
\end{proof}

%</необходимоеусловиеотносительногоэкстремумаproof>
\begin{definition}
    $G := f - \lambda_1 \Phi_1 - \lambda_2 \Phi_2 - \ldots - \lambda_n \Phi_n$ --- \textbf{функция Лагранжа}.
\end{definition}

$$f'(a) - \lambda\Phi'(a) = 0 \Leftrightarrow G'(a) = 0$$

\begin{remark}
    В определении выше можно писать ``$+$'' вместо ``$-$''.
\end{remark}

\begin{example}
    $A=(a_{ij})$ --- матрица $m \times n$, симметричная и вещественная.

    $f(x) := \langle Ax, x\rangle, x\in \R^m$ --- квадратичная форма.

    Найдём $\max f(x)$, когда $x\in S^{m-1}$ \textit{(единичной сфере в $\R^m$)}.

    Такой $\max \exists$ по теореме Вейерштрасса.

    $$G(x) = \sum_{i}^m \sum_{j}^n a_{ij} x_i x_j - \lambda \left(\sum_{i=1}^m x_i^2 - 1\right)$$
    $\Phi' = \begin{pmatrix}
            2x_1   \\
            2x_2   \\
            \vdots \\
            2x_m
        \end{pmatrix}$. На сфере $\rg \Phi'=1$.

    $$G'_{x_k} = 2\sum_{j=1}^m a_{kj} x_j - 2\lambda x_k \quad \forall k = 1\ldots m = 0$$

    То есть $Ax = \lambda x \Rightarrow \lambda$ --- собственное число $A, x$ --- собственный единичный вектор.

    $f(x) = \langle Ax, x\rangle = \langle \lambda x, x\rangle = \lambda x^2 = \lambda$
\end{example}

\begin{theorem}\itemfix
    %<*нормалоп>
    \begin{itemize}
        \item $A \in \text{Lin}(\R^m, \R^n)$
    \end{itemize}
    Тогда $||A|| = \max\{\sqrt{\lambda} : \lambda \text{ --- собственное число } A^TA\}$

    Такое число существует, т.к. $\langle Ax, y\rangle = \langle x, A^Ty\rangle \Rightarrow \langle A^T Ax, x\rangle = \langle Ax, Ax\rangle \ge 0 \Rightarrow \lambda \ge 0$.
    %</нормалоп>
\end{theorem}
%<*нормалопproof>
\begin{proof}
    $\sphericalangle x\in S^{m-1}$.

    $$|Ax|^2 = \langle Ax, Ax\rangle = \langle \underbrace{A^TA}_{\text{симм.}}x, x\rangle$$
    $$\max |Ax|^2 = \max \langle A^TAx, x\rangle = \lambda_{\max}$$
\end{proof}
%</нормалопproof>

\section*{Функциональные последовательности и ряды}

\subsection*{Равномерная сходимость последовательностей и функций}

\begin{definition}
    \textbf{Последовательность функций} : $\N \to \mathcal F$ --- пространство функций, $n\mapsto f_n$
\end{definition}
\begin{definition}
    $\mathcal F = \{f : \underbrace{X}_{\substack{\text{метрическое} \\ \text{пространство}}}\to\R\}$
\end{definition}

\begin{definition}
    %<*поточеченаясходимость>
    Пусть $E\subset X$. Последовательность $f_n$ \textbf{сходится поточечно} к $f$ на множестве $E$, если $\forall x\in E \quad f_n(x) \to f(x)$, т.е.:
    $$\forall x\in E \ \ \forall \varepsilon > 0 \ \ \exists N \ \ \forall n > N \quad |f_n(x) - f(x)| < \varepsilon$$
    %</поточеченаясходимость>
\end{definition}

\begin{example}
    $f_n : \R_+ \to \R, f_n(x) = \frac{x^n}{n}$

    Если $E = [0, 1] \Rightarrow f_n(x) \to 0$ --- тождественный ноль, не ноль.

    Если $E \cap(1, +\infty) \not=\text{\O}$, то нет поточечной сходимости ни к какой функции.
\end{example}

\begin{example}
    $f_n = \frac{n^\alpha x}{1 + n^2x^2}, x\in[0, 1], 0 < \alpha < 2$.

    Ясно, что $f_n(x)\to0 \ \ \forall \alpha$, поточечно сходится на $[0, 1]$.

    $$\max_{x\in[0, 1]}\frac{n^\alpha x}{1 + n^2x^2} = n^\alpha \max\frac{x}{1+n^2x^2} = n^\alpha \frac{1}{2n} = \frac{1}{2}n^{\alpha - 1}$$

    При $\alpha > 1 \ \ \frac{1}{2}n^{\alpha-1}\to+\infty$. Это странно.

    \begin{minipage}{\linewidth}
        \centering
        \begin{minipage}{0.3\linewidth}
            Теперь мы видим, что функции стремятся к тождественному нулю, хотя $\exists x : f(x)\to+\infty$. Придумаем определение, которое это запрещает.
        \end{minipage}
        \hspace{0.05\linewidth}
        \begin{minipage}{0.6\linewidth}
            \begin{tikzpicture}
                \begin{axis}[
                        axis lines = left,
                        xlabel = $x$,
                        ylabel = {$f(x)$}
                    ]
                    \addplot [
                        domain=0:10,
                        samples=100,
                        color=blue
                    ]
                    {1^1.5 * x / (1 + 1^2 * x^2)};
                    \addlegendentry{$n=1, \alpha=1.5$}
                    \addplot [
                        domain=0:10,
                        samples=100,
                        color=green
                    ]
                    {3^1.5 * x / (1 + 3^2 * x^2)};
                    \addlegendentry{$n=3, \alpha=1.5$}
                    \addplot [
                        domain=0:10,
                        samples=100,
                        color=red
                    ]
                    {10^1.5 * x / (1 + 10^2 * x^2)};
                    \addlegendentry{$n=10, \alpha=1.5$}
                \end{axis}
            \end{tikzpicture}
        \end{minipage}
    \end{minipage}
\end{example}

\begin{definition}
    %<*равномернаясходимость>
    $f_n$ \textbf{равномерно сходится} к $f$ на $E\subset X$, если $M_n := \sup\limits_{x\in E} |f_n(x)-f(x)|\xrightarrow{n\to+\infty}0$.
    $$\forall \varepsilon \ \ \exists N \ \ \forall n > N \ \ 0\le M_n < \varepsilon \text{ т.е. } \forall x\in E \ \ |f_n(x) - f(x)| < \varepsilon$$
    Обозначается $f_n\xrightrightarrows[E]{} f$
    %</равномернаясходимость>
\end{definition}

\begin{remark}\itemfix
    \begin{itemize}
        \item $x_0\in E$
        \item $f_n \xrightrightarrows[E]{} f$
    \end{itemize}
    Тогда $f_n(x_0)\to f(x_0)$. То есть равномерная сходимость $\Rightarrow$ поточечная сходимость и предел.
\end{remark}
\begin{remark}\itemfix
    \begin{itemize}
        \item $E_0\subset E$
        \item $f_n \xrightrightarrows[E]{} f$
    \end{itemize}
    Тогда $f_n \xrightrightarrows[E_0]{} f$
\end{remark}

\begin{remark}\itemfix
    \begin{itemize}
        \item $\mathcal F = \{f : X \to \R \text{ --- огр. функции}\}$
    \end{itemize}
    Тогда $\rho(f_1, f_2) := \sup\limits_{x\in X}|f_1(x) - f_2(x)|$ --- метрика в $\mathcal F$. Называется Чебышевское расстояние.
\end{remark}
\begin{proof}
    \begin{enumerate}
        \item $\rho(f_1, f_2) \ge 0$, $\rho(f_1, f_2) = 0 \Leftrightarrow f_1\not= f_2$ --- очевидно
        \item $\rho(f_1, f_2) = \rho(f_2, f_1)$ --- очевидно
        \item $\rho(f_1, f_2) \le \rho(f_1, f_3) + \rho(f_2, f_3)$

              $\sphericalangle \varepsilon > 0 \ \ \exists x$
              \begin{align*}
                  \rho(f_1, f_2) - \varepsilon & = \sup|f_1 - f_2| - \varepsilon           \\
                                               & < |f_1(x) - f_2(x)|                       \\
                                               & \le |f_1(x) - f_3(x)| + |f_2(x) - f_3(x)| \\
                                               & \le \rho(f_1, f_3) + \rho(f_2, f_3)
              \end{align*}
    \end{enumerate}
\end{proof}

$f_n \xrightrightarrows[E]{} f \Leftrightarrow f_n \to f$ по метрике $\rho_E$.

Можем заметить, что в $\mathcal F$ при различных метриках происходит различная сходимость или расходимость, в отличие от $\R^m$.

\begin{remark}\itemfix
    \begin{itemize}
        \item $E = E_1 \cup E_2$
    \end{itemize}
    $$\begin{rcases*}
            f_n \xrightrightarrows[E_1]{} f \\
            f_n \xrightrightarrows[E_2]{} f
        \end{rcases*} \Rightarrow f_n \xrightrightarrows[E]{} f$$
\end{remark}

\end{document}
