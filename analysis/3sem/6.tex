\documentclass[12pt, a4paper]{article}

%<*preamble>
% Math symbols
\usepackage{amsmath, amsthm, amsfonts, amssymb}
\usepackage{accents}
\usepackage{esvect}
\usepackage{mathrsfs}
\usepackage{mathtools}
\mathtoolsset{showonlyrefs}
\usepackage{cmll}
\usepackage{stmaryrd}
\usepackage{physics}
\usepackage[normalem]{ulem}
\usepackage{ebproof}
\usepackage{extarrows}

% Page layout
\usepackage{geometry, a4wide, parskip, fancyhdr}

% Font, encoding, russian support
\usepackage[russian]{babel}
\usepackage[sb]{libertine}
\usepackage{xltxtra}

% Listings
\usepackage{listings}
\lstset{basicstyle=\ttfamily,breaklines=true}
\setmonofont{Inconsolata}

% Miscellaneous
\usepackage{array}
\usepackage{calc}
\usepackage{caption}
\usepackage{subcaption}
\captionsetup{justification=centering,margin=2cm}
\usepackage{catchfilebetweentags}
\usepackage{enumitem}
\usepackage{etoolbox}
\usepackage{float}
\usepackage{lastpage}
\usepackage{minted}
\usepackage{svg}
\usepackage{wrapfig}
\usepackage{xcolor}
\usepackage[makeroom]{cancel}

\newcolumntype{L}{>{$}l<{$}}
    \newcolumntype{C}{>{$}c<{$}}
\newcolumntype{R}{>{$}r<{$}}

% Footnotes
\usepackage[hang]{footmisc}
\setlength{\footnotemargin}{2mm}
\makeatletter
\def\blfootnote{\gdef\@thefnmark{}\@footnotetext}
\makeatother

% References
\usepackage{hyperref}
\hypersetup{
    colorlinks,
    linkcolor={blue!80!black},
    citecolor={blue!80!black},
    urlcolor={blue!80!black},
}

% tikz
\usepackage{tikz}
\usepackage{tikz-cd}
\usetikzlibrary{arrows.meta}
\usetikzlibrary{decorations.pathmorphing}
\usetikzlibrary{calc}
\usetikzlibrary{patterns}
\usepackage{pgfplots}
\pgfplotsset{width=10cm,compat=1.9}
\newcommand\irregularcircle[2]{% radius, irregularity
    \pgfextra {\pgfmathsetmacro\len{(#1)+rand*(#2)}}
    +(0:\len pt)
    \foreach \a in {10,20,...,350}{
            \pgfextra {\pgfmathsetmacro\len{(#1)+rand*(#2)}}
            -- +(\a:\len pt)
        } -- cycle
}

\providetoggle{useproofs}
\settoggle{useproofs}{false}

\pagestyle{fancy}
\lfoot{M3137y2019}
\cfoot{}
\rhead{стр. \thepage\ из \pageref*{LastPage}}

\newcommand{\R}{\mathbb{R}}
\newcommand{\Q}{\mathbb{Q}}
\newcommand{\Z}{\mathbb{Z}}
\newcommand{\B}{\mathbb{B}}
\newcommand{\N}{\mathbb{N}}
\renewcommand{\Re}{\mathfrak{R}}
\renewcommand{\Im}{\mathfrak{I}}

\newcommand{\const}{\text{const}}
\newcommand{\cond}{\text{cond}}

\newcommand{\teormin}{\textcolor{red}{!}\ }

\DeclareMathOperator*{\xor}{\oplus}
\DeclareMathOperator*{\equ}{\sim}
\DeclareMathOperator{\sign}{\text{sign}}
\DeclareMathOperator{\Sym}{\text{Sym}}
\DeclareMathOperator{\Asym}{\text{Asym}}

\DeclarePairedDelimiter{\ceil}{\lceil}{\rceil}

% godel
\newbox\gnBoxA
\newdimen\gnCornerHgt
\setbox\gnBoxA=\hbox{$\ulcorner$}
\global\gnCornerHgt=\ht\gnBoxA
\newdimen\gnArgHgt
\def\godel #1{%
    \setbox\gnBoxA=\hbox{$#1$}%
    \gnArgHgt=\ht\gnBoxA%
    \ifnum     \gnArgHgt<\gnCornerHgt \gnArgHgt=0pt%
    \else \advance \gnArgHgt by -\gnCornerHgt%
    \fi \raise\gnArgHgt\hbox{$\ulcorner$} \box\gnBoxA %
    \raise\gnArgHgt\hbox{$\urcorner$}}

% \theoremstyle{plain}

\theoremstyle{definition}
\newtheorem{theorem}{Теорема}
\newtheorem*{definition}{Определение}
\newtheorem{axiom}{Аксиома}
\newtheorem*{axiom*}{Аксиома}
\newtheorem{lemma}{Лемма}

\theoremstyle{remark}
\newtheorem*{remark}{Примечание}
\newtheorem*{exercise}{Упражнение}
\newtheorem{corollary}{Следствие}[theorem]
\newtheorem*{statement}{Утверждение}
\newtheorem*{corollary*}{Следствие}
\newtheorem*{example}{Пример}
\newtheorem{observation}{Наблюдение}
\newtheorem*{prop}{Свойства}
\newtheorem*{obozn}{Обозначение}

% subtheorem
\makeatletter
\newenvironment{subtheorem}[1]{%
    \def\subtheoremcounter{#1}%
    \refstepcounter{#1}%
    \protected@edef\theparentnumber{\csname the#1\endcsname}%
    \setcounter{parentnumber}{\value{#1}}%
    \setcounter{#1}{0}%
    \expandafter\def\csname the#1\endcsname{\theparentnumber.\Alph{#1}}%
    \ignorespaces
}{%
    \setcounter{\subtheoremcounter}{\value{parentnumber}}%
    \ignorespacesafterend
}
\makeatother
\newcounter{parentnumber}

\newtheorem{manualtheoreminner}{Теорема}
\newenvironment{manualtheorem}[1]{%
    \renewcommand\themanualtheoreminner{#1}%
    \manualtheoreminner
}{\endmanualtheoreminner}

\newcommand{\dbltilde}[1]{\accentset{\approx}{#1}}
\newcommand{\intt}{\int\!}

% magical thing that fixes paragraphs
\makeatletter
\patchcmd{\CatchFBT@Fin@l}{\endlinechar\m@ne}{}
{}{\typeout{Unsuccessful patch!}}
\makeatother

\newcommand{\get}[2]{
    \ExecuteMetaData[#1]{#2}
}

\newcommand{\getproof}[2]{
    \iftoggle{useproofs}{\ExecuteMetaData[#1]{#2proof}}{}
}

\newcommand{\getwithproof}[2]{
    \get{#1}{#2}
    \getproof{#1}{#2}
}

\newcommand{\import}[3]{
    \subsection{#1}
    \getwithproof{#2}{#3}
}

\newcommand{\given}[1]{
    Дано выше. (\ref{#1}, стр. \pageref{#1})
}

\renewcommand{\ker}{\text{Ker }}
\newcommand{\im}{\text{Im }}
\renewcommand{\grad}{\text{grad}}
\newcommand{\rg}{\text{rg}}
\newcommand{\defeq}{\stackrel{\text{def}}{=}}
\newcommand{\defeqfor}[1]{\stackrel{\text{def } #1}{=}}
\newcommand{\itemfix}{\leavevmode\makeatletter\makeatother}
\newcommand{\?}{\textcolor{red}{???}}
\renewcommand{\emptyset}{\varnothing}
\newcommand{\longarrow}[1]{\xRightarrow[#1]{\qquad}}
\DeclareMathOperator*{\esup}{\text{ess sup}}
\newcommand\smallO{
    \mathchoice
    {{\scriptstyle\mathcal{O}}}% \displaystyle
    {{\scriptstyle\mathcal{O}}}% \textstyle
    {{\scriptscriptstyle\mathcal{O}}}% \scriptstyle
    {\scalebox{.6}{$\scriptscriptstyle\mathcal{O}$}}%\scriptscriptstyle
}
\renewcommand{\div}{\text{div}\ }
\newcommand{\rot}{\text{rot}\ }
\newcommand{\cov}{\text{cov}}

\makeatletter
\newcommand{\oplabel}[1]{\refstepcounter{equation}(\theequation\ltx@label{#1})}
\makeatother

\newcommand{\symref}[2]{\stackrel{\oplabel{#1}}{#2}}
\newcommand{\symrefeq}[1]{\symref{#1}{=}}

% xrightrightarrows
\makeatletter
\newcommand*{\relrelbarsep}{.386ex}
\newcommand*{\relrelbar}{%
    \mathrel{%
        \mathpalette\@relrelbar\relrelbarsep
    }%
}
\newcommand*{\@relrelbar}[2]{%
    \raise#2\hbox to 0pt{$\m@th#1\relbar$\hss}%
    \lower#2\hbox{$\m@th#1\relbar$}%
}
\providecommand*{\rightrightarrowsfill@}{%
    \arrowfill@\relrelbar\relrelbar\rightrightarrows
}
\providecommand*{\leftleftarrowsfill@}{%
    \arrowfill@\leftleftarrows\relrelbar\relrelbar
}
\providecommand*{\xrightrightarrows}[2][]{%
    \ext@arrow 0359\rightrightarrowsfill@{#1}{#2}%
}
\providecommand*{\xleftleftarrows}[2][]{%
    \ext@arrow 3095\leftleftarrowsfill@{#1}{#2}%
}

\allowdisplaybreaks

\newcommand{\unfinished}{\textcolor{red}{Не дописано}}

% Reproducible pdf builds 
\special{pdf:trailerid [
<00112233445566778899aabbccddeeff>
<00112233445566778899aabbccddeeff>
]}
%</preamble>


\lhead{Математический анализ}
\cfoot{}
\rfoot{19.10.2020}

\begin{document}

\begin{theorem}[достаточное условие экстремума]\itemfix
    %<*достаточноеусловиеэкстремума>
    Выполняется условие теоремы о необходимом условии экстремума, то есть:
    \begin{itemize}
        \item $f: O\subset\R^{m+n} \to\R$ --- гладкое в $O$
        \item $M_\Phi \subset O := \{ x : \Phi(x) = 0 \}$ --- гладкое в $O$
        \item $a\in O$ --- точка относительного локального экстремума
        \item $\Phi(a) = 0$
        \item $\rg \Phi'(a) = n$
    \end{itemize}
    $\forall h=(h_x, h_y)\in\R^{m+n}$: если $\Phi'(a)h=0$, то можно выразить $h_y = \Psi(h_x)$.

    Рассмотрим квадратную форму $Q(h_x) = d^2 G(a, (h_x, \Psi(h_x)))$.

    Тогда:
    \begin{enumerate}
        \item Если $Q(h)$ положительно определена, $a$ --- точка минимума
        \item Если $Q(h)$ отрицательно определена, $a$ --- точка максимума
        \item Если $Q(h)$ незнакоопределена, $a$ --- не экстремум
        \item Если $Q(h)$ положительно определена, но вырождена, недостаточно информации
    \end{enumerate}
    %</достаточноеусловиеэкстремума>
\end{theorem}
\begin{proof}
    \begin{align}
        f(a+h) - f(a) & = G(a+h) - G(a)                                \\
                      & = dG(a, h) + \frac{1}{2}d^2 G(a, h) + o(|h|^2) \\
                      & = \frac{1}{2}d^2G(a, \tilde h) + o(|h|^2) > 0
    \end{align}
    Объяснение переходов:
    \begin{enumerate}
        \item $a+h\in M_\Phi$
        \item Формула Тейлора
        \item $a + \tilde h$ лежит на касательной поверхности, $dG(a, h) = 0$, $h\simeq \tilde h$
    \end{enumerate}

    Это нестрогое доказательство, но этого нам достаточно.
\end{proof}

\begin{example}\itemfix
    \begin{itemize}
        \item $f = x^2z^2 + y^3$
        \item $\Phi(x, y, z) = xyz - 6$
        \item $a = (1, 2, 3)$
        \item $\lambda = 1$
    \end{itemize}
    Найдем экстремум.

    \begin{enumerate}
        \item $a$ --- подозрительная точка?

              $$G = x^2z^2 + y^3 - 12x - 9y - 4z - xyz + 6$$
              $$G'_x = 0 \quad 2xz^2 - 12 - yz = 0 \text{ --- подходит}$$
              В $G'_y = 0, G'_z = 0$ --- подходит

        \item \begin{align*}
                  d^2 G & = 2z^2 dx^2 + 2x^2 dz^2 + 6ydy^2 + 2(4xz - y)dxdz + 2(-x)dydz - 2zdxdy          \\
                        & \stackrel{\text{подст. } a}{=} 18dx^2 + 2dz^2 + 12dy^2 + 20dxdz - 2dydz - 6dxdy
              \end{align*}

              Найдём знак этого выражения, если $(dx, dy, dz)$ удовлетворяет $d\Phi=0$

              $$yz dx + xz dy + xy dz = 0 \xrightarrow{\text{в точке } a} 6dx + 3dy + 2dz = 0 \Rightarrow dz = -3dx - \frac{3}{2}dy$$
              \begin{align*}
                  d^2G\Big|_{d\Phi=0} & = 18dx^2 + 2\left(3dx + \frac{3}{2}\right)^2 + 12dy^2 - 10dx(6dx + 3dy) + dy(6dx + 3dy) - 6dxdy \\
                                      & = -24 dx^2 + 19.5dy^2 + \ldots dxdy
              \end{align*}
              Экстремума в $a$ нет, т.к. форма неопределена, т.к. $\begin{cases}
                      dx = 1, dy = 0 \Rightarrow d^2G<0 \\
                      dx = 0, dy = 1 \Rightarrow d^2G>0
                  \end{cases}$
    \end{enumerate}
\end{example}

Такие задачи \textit{(где параметр --- функция)} называются \textbf{вариационное исчисление}.

\section*{Функциональные последовательности и ряды}

%<*стоксазайдля>
\begin{manualtheorem}{1}[Стокса-Зайдля]\itemfix
    \begin{itemize}
        \item $f_n, f : X\to\R$
        \item $X$ --- метрическое пространство
        \item $x_0\in X$
        \item $f_n$ непрерывна в $x_0$
        \item $f_n\xrightrightarrows[X]{} f$
    \end{itemize}
    Тогда $f$ непрерывна в $x_0$.
\end{manualtheorem}
%</стоксазайдля>
%<*стоксазайдляproof>
\begin{proof}
    $|f(x) - f(x_0)| \le \underline{|f(x) - f_n(x)|} + |f_n(x) - f_n(x_0)| + \underline{|f_n(x_0) - f(x_0)|}$ --- верно $\forall x, \forall n$

    $f_n\xrightrightarrows[X]{} f \xLeftrightarrow{\text{def}} \forall \varepsilon > 0 \ \ \exists N \ \ \forall n > N \quad \sup\limits_X |f_n(x) - f(x)| < \frac{\varepsilon}{3} \quad \text{(1)}$

    Берем $\forall \varepsilon > 0$ возьмём любой $n$, для которого выполняется (1). Тогда подчеркнутые слагаемые $\le \frac{\varepsilon}{3}$. Теперь для этого $n$ подбираем $U(x_0) : \forall x \in U(x_0) \ \ |f_n(x) - f_n(x_0)| < \frac{\varepsilon}{3}$

    $$|f(x) - f(x_0)| \le \frac{\varepsilon}{3} + \frac{\varepsilon}{3} + \frac{\varepsilon}{3} = \varepsilon$$
\end{proof}
%</стоксазайдляproof>

\begin{remark}
    То же верно, если $f_n, f : X \to Y$, где $Y$ --- метрическое пространство.
\end{remark}

\begin{remark}
    То же верно, если $X$ --- топологическое пространство, т.е. в нём определены открытые множества.
\end{remark}

\begin{corollary}
    Если $f_n\in C(X), f_n\xrightrightarrows[X]{} f$, тогда $f\in C(X)$
\end{corollary}

\begin{remark}
    В теореме достаточно требовать $f_n \xrightrightarrows[W(x_0)]{} f$

    В следствии достаточно требовать локальную равномерную сходимость, т.е. $$\forall x\in X \ \ \exists W(x) \ \ f_n\xrightrightarrows[W(x)]{} f$$
\end{remark}

\begin{example}
    $f_n(x) = x^n, x = (0, 1)$.

    $f_n(x)\to 0$ точечно на $X$

    $f_n\not\rightrightarrows 0$ на $X$

    Но есть локальная равномерная сходимость:
    $$\sup_{x\in(\alpha, \beta)} |f_n(x) - f(x)| = \sup_{x\in(\alpha, \beta)} x^n = \beta^n \xrightarrow{n\to+\infty} 0 \Rightarrow f_n\xrightrightarrows[(\alpha, \beta)]{} f$$
\end{example}

\begin{theorem}\itemfix
    %<*метрикавcx>
    \begin{itemize}
        \item $X$ компакт
        \item $\rho(f_1, f_2)=\sup\limits_{x\in X}|f_1(x) - f_2(x)|$, где $f_1, f_2 \in C(X)$
    \end{itemize}
    Тогда пространство $C(X)$ --- полное метрическое пространство с метрикой $\rho$.
    %</метрикавcx>
\end{theorem}
%<*метрикавcxproof>
\begin{proof}
    $f_n$ --- фундаментальная в $C(X) \xLeftrightarrow{\text{def}}$:
    \begin{equation}
        \forall \varepsilon > 0 \ \ \exists N \ \ \forall n, m > N \ \ \forall x\in X \ \ |f_n(x) - f_m(x)| < \varepsilon \label{фундпоследовательность}
    \end{equation}
    $\Rightarrow \forall x_0\in X$ вещественная последовательность $(f_n(x_0))$ фундаментальная $\Rightarrow \exists\lim\limits_{n\to+\infty} f_n(x_0) =: f(x_0)$, тогда $f$ --- поточечный предел $f_n$. Проверим это.

    В \eqref{фундпоследовательность} перейдем к пределу при $m\to+\infty$:
    $$\forall \varepsilon > 0 \ \ \exists N \ \ \forall n \ \ \forall x\in X \ \ |f_n(x) - f(x)| \le \varepsilon \Rightarrow f_n\xrightrightarrows[X]{}f \xRightarrow{\text{сл. из Стокс}} f\in C(X)$$
\end{proof}
%</метрикавcxproof>

$(x_n)$ --- последовательность в полном метрическом пространстве $X$, $x_n$ сходится $\Leftrightarrow x_n$ фундаментальная.

$f : \underbrace{X}_{\text{м.п.}}\to \underbrace{Y}_{\substack{\text{м.п.} \\ \text{полное}}}, f(x) \xrightarrow{x\to a} L \xLeftrightarrow[\text{Больцано-Коши}]{\text{критерий}} \forall \varepsilon > 0 \ \ \exists U(a) \ \ \forall x_1, x_2 \in \dot U(a) \quad \rho(f(x_1), f(x_2)) < \varepsilon$

В $C(X)$ $f_n\xrightrightarrows[X]{}f \Leftrightarrow$ фундаментальность:
\begin{equation}
    \forall \varepsilon > 0 \ \ \exists N \ \ \forall n, m > N \ \ \forall x \ \ |f_n(x) - f_m(x)| < \varepsilon \label{fund}
\end{equation}
\begin{equation}
    \sup_{x\in X} |f_n - f| < \varepsilon \label{sup_fund}
\end{equation}

\begin{itemize}
    \item \eqref{sup_fund} $\Rightarrow$ \eqref{fund}
    \item \eqref{fund} $\Rightarrow \sup\limits_{x\in X}|f_n - f| \le \varepsilon$, но по двойной бухгалтерии это $\Leftrightarrow$ \eqref{sup_fund}
\end{itemize}

\subsection*{Предельный переход под знаком интеграла}

``\textbf{Теорема}'' $f_n \to f \Rightarrow \int_a^b f_n \to \int_a^b f$

Эта теорема неверная.

\begin{example}
    $[a, b] = [0, 1]$

    $$f_n(x) = n x^{n-1} (1-x^n) \xrightarrow{n\to+\infty} f(x) \equiv 0$$
    $$\int_a^b f_n = \int_0^1 nx^{n-1}(1-x^n) dx \stackrel{y:=x^n}{=} \int_0^1 (1-y)dy = \frac{1}{2} \not= \int_0^1 f(x) = 0$$
\end{example}

\begin{manualtheorem}{2}\itemfix
    %<*предельныйпереходподинтегралом>
    \begin{itemize}
        \item $f, f_n \in C[a, b]$
        \item $f_n\rightrightarrows f$ на $[a, b]$
    \end{itemize}
    Тогда $\int_a^b f_n \to \int_a^b f$
    %</предельныйпереходподинтегралом>
\end{manualtheorem}
%<*предельныйпереходподинтеграломproof>
\begin{proof}
    $$\left|\int_a^b f_n - \int_a^b f\right| \le \int_a^b |f_n - f| \le \sup_{[a, b]} |f_n - f|(b-a) = \rho(f_n, f)(b-a) \to0$$
\end{proof}
%</предельныйпереходподинтеграломproof>
\begin{corollary}[Правило Лейбница]\itemfix
    %<*правилолейбница>
    \begin{itemize}
        \item $f : [a,b]\times[c, d] \to\R$
        \item $f, f'_y$ --- непр. на $[a, b]\times [c, d]$
        \item $\Phi(y) = \int_a^b f(x, y) dx$
    \end{itemize}
    Тогда $\Phi$ дифференцируема на $[c, d]$ и $\Phi'(y) = \int_a^b f'_y(x, y) dx$
    %</правилолейбница>
\end{corollary}
%<*правилолейбницаproof>
\begin{proof}
    \begin{align}
        \frac{\Phi\left(y + \frac{1}{n}\right) - \Phi(y)}{\frac{1}{n}} & = \int_a^b \frac{f\left(x, y + \frac{1}{n}\right) - f(x, y)}{\frac{1}{n}}dx \\
                                                                       & = \int_a^b f'_y\left(x, y + \frac{\Theta}{n}\right) dx \label{лагранж}      \\
                                                                       & = \int_a^b g_n(x, y) dx
    \end{align}

    \eqref{лагранж}: по т. Лагранжа.

    $g_n(x, y) \xrightrightarrows[]{n\to+\infty} f'_y(x, y)$ на $x\in[a, b]$ по теореме Кантора о равномерной непрерывности, и мы считаем $y$ фиксированным.

    Таким образом, $\Phi'(y)\leftarrow\cfrac{\Phi\left(y + \frac{1}{n}\right) - \Phi(y)}{\frac{1}{n}}\rightarrow \int_a^b f'_y(x, y)dx$
\end{proof}
%</правилолейбницаproof>

\begin{manualtheorem}{3}[о предельном переходе под знаком производной]\itemfix
    %<*предельныйпереходподпроизводной>
    \begin{itemize}
        \item $f_n\in C^1\langle a, b\rangle$
        \item $f_n\to f$ поточечно на $\langle a, b\rangle$
        \item $f'_n \xrightrightarrows[\langle a, b\rangle]{}\varphi$
    \end{itemize}
    Тогда $f\in C^1\langle a, b\rangle$

    То есть пунктирное преобразование верно:
    $$\begin{tikzcd}[ampersand replacement=\&]
            f_n \arrow{r}{n\to+\infty} \arrow[swap]{d}{D} \& f \arrow[dashed]{d}{} \\
            f_n' \arrow[r, yshift=0.7ex]\arrow[r, yshift=-0.7ex] \& \varphi
        \end{tikzcd}$$
    %</предельныйпереходподпроизводной>
\end{manualtheorem}

%<*предельныйпереходподпроизводнойproof>
\begin{proof}
    $\forall x_0, x_1\in \langle a,b\rangle$:

    $$f'_n\xrightrightarrows{[x_0, x_1]} \varphi \xRightarrow{\text{теорема } 2} \int_{x_0}^{x_1} f'_n \xrightarrow{n\to+\infty} \int_{x_0}^{x_1} \varphi$$
    \begin{align*}
        \int_{x_0}^{x_1} f'_n                                        & \xrightarrow{n\to+\infty} \int_{x_0}^{x_1} \varphi \\
        f(x_1) - f(x_0) \xleftarrow{n\to+\infty} f_n(x_1) - f_n(x_0) & \xrightarrow{n\to+\infty} \int_{x_0}^{x_1} \varphi \\
        f(x_1) - f(x_0)                                              & \to \int_{x_0}^{x_1} \varphi                       \\
    \end{align*}

    Тогда $\begin{cases}
            f \text{ --- первообразная } \varphi \\
            \varphi \text{ --- непр.}
        \end{cases} \Rightarrow f' = \varphi$
\end{proof}
%</предельныйпереходподпроизводнойproof>

\subsection*{Равномерная сходимость функциональных рядов}

\begin{definition}\itemfix
    \begin{itemize}
        \item $X$ --- произвольное множество
        \item $u_n : X\to\R (\R^n)$
    \end{itemize}
    $\sum u_n(x)$ \textbf{сходится поточечно} (к сумме $S(x)$) на $X$, если $S_N(x) := \sum_{n=0}^N u_n(x), S_N(x) \to S(x)$ поточечно на $X$.
\end{definition}
\begin{definition}\itemfix
    %<*равномернаясходимостьряда>
    \begin{itemize}
        \item $X$ --- произвольное множество
        \item $u_n : X\to Y$ --- нормированное пространство
    \end{itemize}
    $\sum_{n=0}^{+\infty} u_n(x)$ сходится к $S(x)$ \textbf{равномерно} на $E\subset X : S_N\xrightrightarrows[E]{N\to+\infty} S$
    %</равномернаясходимостьряда>
\end{definition}

\begin{remark}
    $\sum u_n(x)$ равномерно сходится $\Rightarrow \sum u_n(x)$ поточечно сходится к той же сумме.
\end{remark}
\begin{proof}
    $$\sup_{x\in E} |S_N - S|\xrightarrow{N\to+\infty} 0 \Rightarrow \forall x_0 \in E : |S_N(x_0) - S(x_0)| \le \sup_{x\in E}|S_N - S| \to 0$$
\end{proof}
\begin{remark}
    %<*критерийбольцанокоши>
    Остаток ряда: $R_N(x) = \sum_{n=N+1}^{+\infty} u_n(x), S(x) = S_N(x) + R_N(x)$

    Ряд сходится на $E \Leftrightarrow R_N \xrightrightarrows[E]{} \mathbf 0$ --- тождественный ноль.
    %</критерийбольцанокоши>
\end{remark}


\begin{remark}
    Необходимое условие равномерной сходимости: $\sum u_n(x)$ --- сходится на $E \Rightarrow u_n(x) \xrightrightarrows[]{n\to+\infty} 0$
\end{remark}
\begin{proof}
    $u_n = R_{n-1} - R_n$
\end{proof}

\end{document}