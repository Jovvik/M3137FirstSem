\documentclass[12pt, a4paper]{article}

%<*preamble>
% Math symbols
\usepackage{amsmath, amsthm, amsfonts, amssymb}
\usepackage{accents}
\usepackage{esvect}
\usepackage{mathrsfs}
\usepackage{mathtools}
\mathtoolsset{showonlyrefs}
\usepackage{cmll}
\usepackage{stmaryrd}
\usepackage{physics}
\usepackage[normalem]{ulem}
\usepackage{ebproof}
\usepackage{extarrows}

% Page layout
\usepackage{geometry, a4wide, parskip, fancyhdr}

% Font, encoding, russian support
\usepackage[russian]{babel}
\usepackage[sb]{libertine}
\usepackage{xltxtra}

% Listings
\usepackage{listings}
\lstset{basicstyle=\ttfamily,breaklines=true}
\setmonofont{Inconsolata}

% Miscellaneous
\usepackage{array}
\usepackage{calc}
\usepackage{caption}
\usepackage{subcaption}
\captionsetup{justification=centering,margin=2cm}
\usepackage{catchfilebetweentags}
\usepackage{enumitem}
\usepackage{etoolbox}
\usepackage{float}
\usepackage{lastpage}
\usepackage{minted}
\usepackage{svg}
\usepackage{wrapfig}
\usepackage{xcolor}
\usepackage[makeroom]{cancel}

\newcolumntype{L}{>{$}l<{$}}
    \newcolumntype{C}{>{$}c<{$}}
\newcolumntype{R}{>{$}r<{$}}

% Footnotes
\usepackage[hang]{footmisc}
\setlength{\footnotemargin}{2mm}
\makeatletter
\def\blfootnote{\gdef\@thefnmark{}\@footnotetext}
\makeatother

% References
\usepackage{hyperref}
\hypersetup{
    colorlinks,
    linkcolor={blue!80!black},
    citecolor={blue!80!black},
    urlcolor={blue!80!black},
}

% tikz
\usepackage{tikz}
\usepackage{tikz-cd}
\usetikzlibrary{arrows.meta}
\usetikzlibrary{decorations.pathmorphing}
\usetikzlibrary{calc}
\usetikzlibrary{patterns}
\usepackage{pgfplots}
\pgfplotsset{width=10cm,compat=1.9}
\newcommand\irregularcircle[2]{% radius, irregularity
    \pgfextra {\pgfmathsetmacro\len{(#1)+rand*(#2)}}
    +(0:\len pt)
    \foreach \a in {10,20,...,350}{
            \pgfextra {\pgfmathsetmacro\len{(#1)+rand*(#2)}}
            -- +(\a:\len pt)
        } -- cycle
}

\providetoggle{useproofs}
\settoggle{useproofs}{false}

\pagestyle{fancy}
\lfoot{M3137y2019}
\cfoot{}
\rhead{стр. \thepage\ из \pageref*{LastPage}}

\newcommand{\R}{\mathbb{R}}
\newcommand{\Q}{\mathbb{Q}}
\newcommand{\Z}{\mathbb{Z}}
\newcommand{\B}{\mathbb{B}}
\newcommand{\N}{\mathbb{N}}
\renewcommand{\Re}{\mathfrak{R}}
\renewcommand{\Im}{\mathfrak{I}}

\newcommand{\const}{\text{const}}
\newcommand{\cond}{\text{cond}}

\newcommand{\teormin}{\textcolor{red}{!}\ }

\DeclareMathOperator*{\xor}{\oplus}
\DeclareMathOperator*{\equ}{\sim}
\DeclareMathOperator{\sign}{\text{sign}}
\DeclareMathOperator{\Sym}{\text{Sym}}
\DeclareMathOperator{\Asym}{\text{Asym}}

\DeclarePairedDelimiter{\ceil}{\lceil}{\rceil}

% godel
\newbox\gnBoxA
\newdimen\gnCornerHgt
\setbox\gnBoxA=\hbox{$\ulcorner$}
\global\gnCornerHgt=\ht\gnBoxA
\newdimen\gnArgHgt
\def\godel #1{%
    \setbox\gnBoxA=\hbox{$#1$}%
    \gnArgHgt=\ht\gnBoxA%
    \ifnum     \gnArgHgt<\gnCornerHgt \gnArgHgt=0pt%
    \else \advance \gnArgHgt by -\gnCornerHgt%
    \fi \raise\gnArgHgt\hbox{$\ulcorner$} \box\gnBoxA %
    \raise\gnArgHgt\hbox{$\urcorner$}}

% \theoremstyle{plain}

\theoremstyle{definition}
\newtheorem{theorem}{Теорема}
\newtheorem*{definition}{Определение}
\newtheorem{axiom}{Аксиома}
\newtheorem*{axiom*}{Аксиома}
\newtheorem{lemma}{Лемма}

\theoremstyle{remark}
\newtheorem*{remark}{Примечание}
\newtheorem*{exercise}{Упражнение}
\newtheorem{corollary}{Следствие}[theorem]
\newtheorem*{statement}{Утверждение}
\newtheorem*{corollary*}{Следствие}
\newtheorem*{example}{Пример}
\newtheorem{observation}{Наблюдение}
\newtheorem*{prop}{Свойства}
\newtheorem*{obozn}{Обозначение}

% subtheorem
\makeatletter
\newenvironment{subtheorem}[1]{%
    \def\subtheoremcounter{#1}%
    \refstepcounter{#1}%
    \protected@edef\theparentnumber{\csname the#1\endcsname}%
    \setcounter{parentnumber}{\value{#1}}%
    \setcounter{#1}{0}%
    \expandafter\def\csname the#1\endcsname{\theparentnumber.\Alph{#1}}%
    \ignorespaces
}{%
    \setcounter{\subtheoremcounter}{\value{parentnumber}}%
    \ignorespacesafterend
}
\makeatother
\newcounter{parentnumber}

\newtheorem{manualtheoreminner}{Теорема}
\newenvironment{manualtheorem}[1]{%
    \renewcommand\themanualtheoreminner{#1}%
    \manualtheoreminner
}{\endmanualtheoreminner}

\newcommand{\dbltilde}[1]{\accentset{\approx}{#1}}
\newcommand{\intt}{\int\!}

% magical thing that fixes paragraphs
\makeatletter
\patchcmd{\CatchFBT@Fin@l}{\endlinechar\m@ne}{}
{}{\typeout{Unsuccessful patch!}}
\makeatother

\newcommand{\get}[2]{
    \ExecuteMetaData[#1]{#2}
}

\newcommand{\getproof}[2]{
    \iftoggle{useproofs}{\ExecuteMetaData[#1]{#2proof}}{}
}

\newcommand{\getwithproof}[2]{
    \get{#1}{#2}
    \getproof{#1}{#2}
}

\newcommand{\import}[3]{
    \subsection{#1}
    \getwithproof{#2}{#3}
}

\newcommand{\given}[1]{
    Дано выше. (\ref{#1}, стр. \pageref{#1})
}

\renewcommand{\ker}{\text{Ker }}
\newcommand{\im}{\text{Im }}
\renewcommand{\grad}{\text{grad}}
\newcommand{\rg}{\text{rg}}
\newcommand{\defeq}{\stackrel{\text{def}}{=}}
\newcommand{\defeqfor}[1]{\stackrel{\text{def } #1}{=}}
\newcommand{\itemfix}{\leavevmode\makeatletter\makeatother}
\newcommand{\?}{\textcolor{red}{???}}
\renewcommand{\emptyset}{\varnothing}
\newcommand{\longarrow}[1]{\xRightarrow[#1]{\qquad}}
\DeclareMathOperator*{\esup}{\text{ess sup}}
\newcommand\smallO{
    \mathchoice
    {{\scriptstyle\mathcal{O}}}% \displaystyle
    {{\scriptstyle\mathcal{O}}}% \textstyle
    {{\scriptscriptstyle\mathcal{O}}}% \scriptstyle
    {\scalebox{.6}{$\scriptscriptstyle\mathcal{O}$}}%\scriptscriptstyle
}
\renewcommand{\div}{\text{div}\ }
\newcommand{\rot}{\text{rot}\ }
\newcommand{\cov}{\text{cov}}

\makeatletter
\newcommand{\oplabel}[1]{\refstepcounter{equation}(\theequation\ltx@label{#1})}
\makeatother

\newcommand{\symref}[2]{\stackrel{\oplabel{#1}}{#2}}
\newcommand{\symrefeq}[1]{\symref{#1}{=}}

% xrightrightarrows
\makeatletter
\newcommand*{\relrelbarsep}{.386ex}
\newcommand*{\relrelbar}{%
    \mathrel{%
        \mathpalette\@relrelbar\relrelbarsep
    }%
}
\newcommand*{\@relrelbar}[2]{%
    \raise#2\hbox to 0pt{$\m@th#1\relbar$\hss}%
    \lower#2\hbox{$\m@th#1\relbar$}%
}
\providecommand*{\rightrightarrowsfill@}{%
    \arrowfill@\relrelbar\relrelbar\rightrightarrows
}
\providecommand*{\leftleftarrowsfill@}{%
    \arrowfill@\leftleftarrows\relrelbar\relrelbar
}
\providecommand*{\xrightrightarrows}[2][]{%
    \ext@arrow 0359\rightrightarrowsfill@{#1}{#2}%
}
\providecommand*{\xleftleftarrows}[2][]{%
    \ext@arrow 3095\leftleftarrowsfill@{#1}{#2}%
}

\allowdisplaybreaks

\newcommand{\unfinished}{\textcolor{red}{Не дописано}}

% Reproducible pdf builds 
\special{pdf:trailerid [
<00112233445566778899aabbccddeeff>
<00112233445566778899aabbccddeeff>
]}
%</preamble>


\lhead{Математический анализ}
\cfoot{}
\rfoot{26.10.2020}

\begin{document}

\subsection*{Функциональные последовательности и ряды}

\begin{example}
    $\sum x^n, x\in (0, 1)$ --- нет равномерной сходимости

    $\exists \varepsilon = 0.1 \ \ \forall N \ \ \exists n>N$ --- подходит любое $>100$ $\exists p = 1 \ \ \exists x = 1 - \frac{1}{n+1} : |u_{n+1}(x)| \ge \varepsilon$, т.е. $\left(1 - \frac{1}{n+1}\right)^{n+1} \approx \frac{1}{e} > \frac{1}{10}$
\end{example}

\begin{theorem}[признак Вейерштрасса]\itemfix
    \begin{itemize}
        \item $\sum u_n(x)$
        \item $x\in X$
    \end{itemize}
    Пусть $\exists c_n$ --- вещественная:
    \begin{itemize}
        \item $|u_n(x)| \le c_n$ при $x\in E$
        \item $\sum c_n$ --- сходится
    \end{itemize}
    Тогда $\sum u_n(x)$ равномерно сходится на $E$
\end{theorem}
\begin{proof}
    $|u_{n+1}(x) + \ldots + u_{n+p}(x)| \le c_{n+1} + \ldots + c_{n+p}$ --- тривиально

    $\sum c_n$ --- сх. $\Rightarrow$ $c_n$ удовлетворяет критерию Больцано-Коши : $$\forall \varepsilon > 0 \ \ \exists N \ \ \forall n > N \ \ \forall p\in \N \ \ \forall x\in E \ \ c_{n+1} + \ldots c_{n+p} < \varepsilon$$

    Тогда $\sum u_n(x)$ удовлетворяет критерию Больцано-Коши равномерной сходимости.
\end{proof}

\begin{example}
    $\sum\limits_{n=1}^{+\infty} \frac{x}{1+n^2x^2}, x\in\R$. Попытаемся применить признак.

    $c_n := \sup\limits_{x\in \R} \left|\frac{x}{1+n^2x^2}\right|$ --- это минимальное возможное $c_n$, если для него не сработает признак, до ни для какого $c_n$ не сработает.

    $\sup$ достигается в точке $x_0 = \frac{1}{n}$, $\sup = \frac{1}{2n}$. $\sum \frac{1}{2n}$ расходится $\Rightarrow$ признак не сработал.

    Построим отрицание критерия Больцано-Коши:
    $$\exists \varepsilon = \frac{1}{6} \ \ \forall N \ \ \exists n>N \ \ p=n\in\N \ \ \exists x=\frac{1}{n} \ \ |u_{n+1}(x) + u_{2n}(x)| = \frac{\frac{1}{n}}{1+(n+1)^2\frac{1}{n^2}} + \ldots \frac{\frac{1}{n}}{1+(2n)^2\frac{1}{n^2}} \ge$$
    $$\ge n \frac{\frac{1}{n}}{1+(2n)^2\frac{1}{n^2}} = \frac{1}{5} > \frac{1}{6} = \varepsilon$$
\end{example}

\begin{example}
    $\sum \frac{x}{1+x^2n^2}, x\in\left(\frac{1}{2020}, 2020\right)$

    $$c_n := \sup \frac{x}{1+x^2n^2} \le \frac{2020}{1+\frac{1}{2020^2}n^2} \equ_{n\to+\infty} \frac{\?}{n^2}$$

    $\sum c_n$ сходится $\Rightarrow$ есть равномерная сходимость.
\end{example}

\subsection*{Приложения равномерной сходимости для рядов}

\begin{manualtheorem}{1'}[Стокса-Зайдля для рядов]\itemfix
    \begin{itemize}
        \item $u_n : X\to Y$
        \item $X$ --- метрическое пространство
        \item $Y$ --- нормированное пространство
        \item $x_0\in X$
        \item $u_n$ --- непрерывно в $x_0$
        \item $\sum u_n(x)$ \textbf{равномерно} сходится на $X$
        \item $S(x) := \sum u_n(x)$
    \end{itemize}
    Тогда $S(x)$ --- непрерывно в $x_0$.
\end{manualtheorem}
\begin{proof}
    По теореме 1 $S_n(x) \rightrightarrows S(x), S_n(x)$ --- непр. в $x_0 \xRightarrow{\text{т. 1}} S(x)$ непр. в $x_0$
\end{proof}

\begin{remark}
    Достаточно равномерной сходимости $u_n(x)$ на некоторой окрестности $x_0$
\end{remark}
\begin{remark}
    $u_n \in C(x), \sum u_n$ --- равномерно сходится на $X \Rightarrow S(x) \in C(x)$
\end{remark}

\begin{manualtheorem}{2'}{О почленном интегрировании ряда}\itemfix
    \begin{itemize}
        \item $u_n : [a, b] \to\R$
        \item $u_n$ --- непр. на $[a, b]$
        \item $\sum\limits_{n=0}^{+\infty} u_n(x)$ \textbf{равномерно} сходится на $[a, b]$
        \item $S(x) = \sum u_n(x)$
    \end{itemize}
    Тогда $\int_a^b S(x)dx = \sum_{n=0}^{+\infty} \int_a^b u_n(x)dx$

    Можно интегрировать, т.к. $S(x)$ --- непр. на $[a, b]$ по теореме 1'
\end{manualtheorem}
\begin{proof}
    По теореме 2

    $S_n\xrightrightarrows{[a, b]} S$

    По теореме 2 $\int_a^b S_n(x) dx \to \int_a^b S(x)dx$

    $\int_a^b \sum\limits_{k=0}^{n} u_k(x) dx = \sum\limits_{k=0}^{n} \int_a^b u_k(x) dx \to \sum\limits_{k=0}^{n} \?$
\end{proof}

\begin{example}
    $\sum\limits_{n=0}^{+\infty} (-1)^n x^n$ --- равномерно сходится при $|x| \le q < 1$ по Вейерштрассу: $|(-1)^n x^n|\le q^n, \sum q^n$ сходится.

    Проинтегрируем от $0$ до $t$ ($|t|\le q$)

    $$\sum_{n=0}^{+\infty} (-1)^n x^n = \frac{1}{1+x}$$
    $$\ln(1+t) = \sum_{n=0}^{+\infty} (-1)^n \frac{t^{n+1}}{n+1} = \sum_{k=1}^{+\infty} (-1)^{k+1} \frac{t^k}{k}$$
    Это верно при $t\in[-q, q]\ \ \forall q : 0<q<1$, т.е. верно при $t\in(-1, 1)$

    При $t=-1$ $\sum -\frac{1}{k}$ расходится

    При $t\to 1$ ряд $\sum (-1)^{k+1} \frac{t^k}{k}$ равномерно сходится на $[0, 1]$, слагаемые непрерывны в $t_0=1 \xRightarrow{\text{т. 1}}$ сумма ряда непрерывна в точке $t_0=1 \Rightarrow \ln 2 = \sum\limits_{k=1}^{+\infty} \frac{(-1)^{k+1}}{k}$
\end{example}

\end{document}