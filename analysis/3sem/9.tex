\documentclass[12pt, a4paper]{article}

%<*preamble>
% Math symbols
\usepackage{amsmath, amsthm, amsfonts, amssymb}
\usepackage{accents}
\usepackage{esvect}
\usepackage{mathrsfs}
\usepackage{mathtools}
\mathtoolsset{showonlyrefs}
\usepackage{cmll}
\usepackage{stmaryrd}
\usepackage{physics}
\usepackage[normalem]{ulem}
\usepackage{ebproof}
\usepackage{extarrows}

% Page layout
\usepackage{geometry, a4wide, parskip, fancyhdr}

% Font, encoding, russian support
\usepackage[russian]{babel}
\usepackage[sb]{libertine}
\usepackage{xltxtra}

% Listings
\usepackage{listings}
\lstset{basicstyle=\ttfamily,breaklines=true}
\setmonofont{Inconsolata}

% Miscellaneous
\usepackage{array}
\usepackage{calc}
\usepackage{caption}
\usepackage{subcaption}
\captionsetup{justification=centering,margin=2cm}
\usepackage{catchfilebetweentags}
\usepackage{enumitem}
\usepackage{etoolbox}
\usepackage{float}
\usepackage{lastpage}
\usepackage{minted}
\usepackage{svg}
\usepackage{wrapfig}
\usepackage{xcolor}
\usepackage[makeroom]{cancel}

\newcolumntype{L}{>{$}l<{$}}
    \newcolumntype{C}{>{$}c<{$}}
\newcolumntype{R}{>{$}r<{$}}

% Footnotes
\usepackage[hang]{footmisc}
\setlength{\footnotemargin}{2mm}
\makeatletter
\def\blfootnote{\gdef\@thefnmark{}\@footnotetext}
\makeatother

% References
\usepackage{hyperref}
\hypersetup{
    colorlinks,
    linkcolor={blue!80!black},
    citecolor={blue!80!black},
    urlcolor={blue!80!black},
}

% tikz
\usepackage{tikz}
\usepackage{tikz-cd}
\usetikzlibrary{arrows.meta}
\usetikzlibrary{decorations.pathmorphing}
\usetikzlibrary{calc}
\usetikzlibrary{patterns}
\usepackage{pgfplots}
\pgfplotsset{width=10cm,compat=1.9}
\newcommand\irregularcircle[2]{% radius, irregularity
    \pgfextra {\pgfmathsetmacro\len{(#1)+rand*(#2)}}
    +(0:\len pt)
    \foreach \a in {10,20,...,350}{
            \pgfextra {\pgfmathsetmacro\len{(#1)+rand*(#2)}}
            -- +(\a:\len pt)
        } -- cycle
}

\providetoggle{useproofs}
\settoggle{useproofs}{false}

\pagestyle{fancy}
\lfoot{M3137y2019}
\cfoot{}
\rhead{стр. \thepage\ из \pageref*{LastPage}}

\newcommand{\R}{\mathbb{R}}
\newcommand{\Q}{\mathbb{Q}}
\newcommand{\Z}{\mathbb{Z}}
\newcommand{\B}{\mathbb{B}}
\newcommand{\N}{\mathbb{N}}
\renewcommand{\Re}{\mathfrak{R}}
\renewcommand{\Im}{\mathfrak{I}}

\newcommand{\const}{\text{const}}
\newcommand{\cond}{\text{cond}}

\newcommand{\teormin}{\textcolor{red}{!}\ }

\DeclareMathOperator*{\xor}{\oplus}
\DeclareMathOperator*{\equ}{\sim}
\DeclareMathOperator{\sign}{\text{sign}}
\DeclareMathOperator{\Sym}{\text{Sym}}
\DeclareMathOperator{\Asym}{\text{Asym}}

\DeclarePairedDelimiter{\ceil}{\lceil}{\rceil}

% godel
\newbox\gnBoxA
\newdimen\gnCornerHgt
\setbox\gnBoxA=\hbox{$\ulcorner$}
\global\gnCornerHgt=\ht\gnBoxA
\newdimen\gnArgHgt
\def\godel #1{%
    \setbox\gnBoxA=\hbox{$#1$}%
    \gnArgHgt=\ht\gnBoxA%
    \ifnum     \gnArgHgt<\gnCornerHgt \gnArgHgt=0pt%
    \else \advance \gnArgHgt by -\gnCornerHgt%
    \fi \raise\gnArgHgt\hbox{$\ulcorner$} \box\gnBoxA %
    \raise\gnArgHgt\hbox{$\urcorner$}}

% \theoremstyle{plain}

\theoremstyle{definition}
\newtheorem{theorem}{Теорема}
\newtheorem*{definition}{Определение}
\newtheorem{axiom}{Аксиома}
\newtheorem*{axiom*}{Аксиома}
\newtheorem{lemma}{Лемма}

\theoremstyle{remark}
\newtheorem*{remark}{Примечание}
\newtheorem*{exercise}{Упражнение}
\newtheorem{corollary}{Следствие}[theorem]
\newtheorem*{statement}{Утверждение}
\newtheorem*{corollary*}{Следствие}
\newtheorem*{example}{Пример}
\newtheorem{observation}{Наблюдение}
\newtheorem*{prop}{Свойства}
\newtheorem*{obozn}{Обозначение}

% subtheorem
\makeatletter
\newenvironment{subtheorem}[1]{%
    \def\subtheoremcounter{#1}%
    \refstepcounter{#1}%
    \protected@edef\theparentnumber{\csname the#1\endcsname}%
    \setcounter{parentnumber}{\value{#1}}%
    \setcounter{#1}{0}%
    \expandafter\def\csname the#1\endcsname{\theparentnumber.\Alph{#1}}%
    \ignorespaces
}{%
    \setcounter{\subtheoremcounter}{\value{parentnumber}}%
    \ignorespacesafterend
}
\makeatother
\newcounter{parentnumber}

\newtheorem{manualtheoreminner}{Теорема}
\newenvironment{manualtheorem}[1]{%
    \renewcommand\themanualtheoreminner{#1}%
    \manualtheoreminner
}{\endmanualtheoreminner}

\newcommand{\dbltilde}[1]{\accentset{\approx}{#1}}
\newcommand{\intt}{\int\!}

% magical thing that fixes paragraphs
\makeatletter
\patchcmd{\CatchFBT@Fin@l}{\endlinechar\m@ne}{}
{}{\typeout{Unsuccessful patch!}}
\makeatother

\newcommand{\get}[2]{
    \ExecuteMetaData[#1]{#2}
}

\newcommand{\getproof}[2]{
    \iftoggle{useproofs}{\ExecuteMetaData[#1]{#2proof}}{}
}

\newcommand{\getwithproof}[2]{
    \get{#1}{#2}
    \getproof{#1}{#2}
}

\newcommand{\import}[3]{
    \subsection{#1}
    \getwithproof{#2}{#3}
}

\newcommand{\given}[1]{
    Дано выше. (\ref{#1}, стр. \pageref{#1})
}

\renewcommand{\ker}{\text{Ker }}
\newcommand{\im}{\text{Im }}
\renewcommand{\grad}{\text{grad}}
\newcommand{\rg}{\text{rg}}
\newcommand{\defeq}{\stackrel{\text{def}}{=}}
\newcommand{\defeqfor}[1]{\stackrel{\text{def } #1}{=}}
\newcommand{\itemfix}{\leavevmode\makeatletter\makeatother}
\newcommand{\?}{\textcolor{red}{???}}
\renewcommand{\emptyset}{\varnothing}
\newcommand{\longarrow}[1]{\xRightarrow[#1]{\qquad}}
\DeclareMathOperator*{\esup}{\text{ess sup}}
\newcommand\smallO{
    \mathchoice
    {{\scriptstyle\mathcal{O}}}% \displaystyle
    {{\scriptstyle\mathcal{O}}}% \textstyle
    {{\scriptscriptstyle\mathcal{O}}}% \scriptstyle
    {\scalebox{.6}{$\scriptscriptstyle\mathcal{O}$}}%\scriptscriptstyle
}
\renewcommand{\div}{\text{div}\ }
\newcommand{\rot}{\text{rot}\ }
\newcommand{\cov}{\text{cov}}

\makeatletter
\newcommand{\oplabel}[1]{\refstepcounter{equation}(\theequation\ltx@label{#1})}
\makeatother

\newcommand{\symref}[2]{\stackrel{\oplabel{#1}}{#2}}
\newcommand{\symrefeq}[1]{\symref{#1}{=}}

% xrightrightarrows
\makeatletter
\newcommand*{\relrelbarsep}{.386ex}
\newcommand*{\relrelbar}{%
    \mathrel{%
        \mathpalette\@relrelbar\relrelbarsep
    }%
}
\newcommand*{\@relrelbar}[2]{%
    \raise#2\hbox to 0pt{$\m@th#1\relbar$\hss}%
    \lower#2\hbox{$\m@th#1\relbar$}%
}
\providecommand*{\rightrightarrowsfill@}{%
    \arrowfill@\relrelbar\relrelbar\rightrightarrows
}
\providecommand*{\leftleftarrowsfill@}{%
    \arrowfill@\leftleftarrows\relrelbar\relrelbar
}
\providecommand*{\xrightrightarrows}[2][]{%
    \ext@arrow 0359\rightrightarrowsfill@{#1}{#2}%
}
\providecommand*{\xleftleftarrows}[2][]{%
    \ext@arrow 3095\leftleftarrowsfill@{#1}{#2}%
}

\allowdisplaybreaks

\newcommand{\unfinished}{\textcolor{red}{Не дописано}}

% Reproducible pdf builds 
\special{pdf:trailerid [
<00112233445566778899aabbccddeeff>
<00112233445566778899aabbccddeeff>
]}
%</preamble>


\lhead{Математический анализ}
\cfoot{}
\rfoot{9.11.2020}

\begin{document}

\subsection*{Интеграл локального потенциального векторного поля по непрерывному пути}

\begin{lemma}[о гусенице]\itemfix
    %<*огусенице>
    \begin{itemize}
        \item \(\gamma : [a, b] \to O \subset \R^m\) --- непр.
    \end{itemize}
    Тогда \(\exists \) дробление \(a = t_0 < t_1 < \dots < t_n = b\) и \(\exists \) шары \(B_1 \dots B_n \subset O : \gamma[t_{k - 1}, t_k] \subset B_k\).

    \begin{figure}[h]
        \centering
        \includesvg[scale=0.7]{images/огусенице.svg}
        \caption{``Гусеница'' --- покрытие пути шарами}
    \end{figure}
    %</огусенице>
\end{lemma}
%<*огусеницеproof>
\begin{proof}
    \(\forall c\in[a, b]\) возьмём \(B_c := B(\gamma(c), \underbrace{r_c}_{\text{произвольн.}}) \subset O\).

    \(\overline{\alpha_c} : = \inf\{ \alpha \in [a, b] : \gamma[\alpha, c] \subset B_c\}\)

    \(\overline{\beta_c} : = \inf\{ \alpha \in [a, b] : \gamma[c, \beta] \subset B_c\}\) --- момент первого выхода после посещения точки $\gamma(c)$

    Возьмём \((\alpha_c, \beta_c) : \overline \alpha_c < \alpha_c < c < \beta_c < \overline \beta_c\)

    Таким образом \(c \mapsto (\alpha_c, \beta_c)\) --- открытое покрытие $[a, b]$, если для \(c = a\) или \(c = b\) вместо \(\alpha_c, \beta_c\) брать \([a, \beta_a), (\alpha_b, b]\)

    \([a, b]\) --- компактно \( \implies [a, b] \subset \bigcup_{\text{кон.}} (\alpha_c, \beta_c)\)

    \? ни один интервал не накрывается целиком остальными \( \Leftrightarrow \forall (\alpha_c, \beta_c)\: \exists d_c\), принадлежащая ``только этому'' интервалу.

    \begin{figure}[h]
        \centering
        \includesvg[scale=0.7]{images/t_choice.svg}
        \caption{Выбор точек $t_k$}
    \end{figure}

    Точка \(t_k\) выбирается на \(d_k, d_{k + 1}\) и \(t_k \in (\alpha_k, \beta_k) \cap (\alpha_{k + 1}, \alpha_{k + 1})\).

    \(\gamma([t_{k - 1}, t_k]) = \gamma(\alpha_k, \beta_k) \subset B_k\)
\end{proof}
%</огусеницеproof>

\begin{remark}
    \(\forall \delta > 0 \) мы можем требовать, чтобы все \(r_k < \delta\)
\end{remark}
\begin{remark}
    В силу произвольности \(r_c\) можно требовать, чтобы шары \(B_c\) удовлетворяют некоторому локальному условию.

    Например пусть \(V\) --- локально потенциальное поле в \(O\). Мы можем требовать, чтобы во всех шарах существовал потенциал \(V\). Тогда будем называть \(\{B_k\}\) \(V\)-гусеницей.
\end{remark}

\begin{definition}\itemfix
    \begin{itemize}
        \item \(V\) --- локально потенциальное поле в \(O\subset\R^m\)
    \end{itemize}
    \(\gamma, \tilde \gamma : [a, b] \to O\) называются похожими \textit{(\(V\)-похожими)}, если у них есть общая \(V\)-гусеница:

    \(\exists t_0 = a < t_1 < t_2 < \dots < t_n = b\ \ \exists \) шары \( B_k \subset O :\)
    \[
        \gamma[t_{k - 1}, t_k] \subset B_k, \tilde \gamma [t_{k - 1}, t_k] \subset B_k
    \]
\end{definition}
\begin{corollary}\itemfix
    \begin{itemize}
        \item \(V\) --- локально потенциальное поле в \(O\subset\R^m\)
    \end{itemize}

    Тогда любой путь \(V\)-похож на ломаную:

    \begin{figure}[h]
        \centering
        \includesvg[scale=0.7]{images/ломаная.svg}
        \caption{Построение ломаной \textit{(розовая)} по пути \textit{(чёрный)} с помощью \(V\)-гусеницы \textit{(круги)}}.
    \end{figure}
\end{corollary}

\begin{lemma}[о равенстве интегралов локально-потенциальных векторных путей по похожим путям]\itemfix
    \label{о равенстве интегралов}
    \begin{itemize}
        \item \(V\) --- локально-потенциальное векторное поле в \(O\subset \R^m\)
        \item \(\gamma, \tilde \gamma : [a, b] \to O\) --- \(V\)-похожие, кусочно гладкие
        \item \(\gamma(a) = \tilde \gamma(a), \gamma(b) = \tilde \gamma(b)\)
    \end{itemize}
    Тогда \(\int_\gamma \sum V_i dx_i = \int_{\tilde \gamma} \sum V_i dx_i\)
\end{lemma}
\begin{proof}
    Рассмотрим общую \(V\)-гусеницу. Пусть \(f_k\) --- потенциал \(V\) в шаре \(B_k\), \(a = t_0 < t_1 < \dots < t_n = b\)

    Сдвинем потенциалы прибавлением константы, так что \(f_k(\gamma(t_k)) = f_{k + 1}(\gamma(t_k))\) при \(k = 1 \dots n\)

    Тогда
    \begin{align}
        \int_\gamma \sum_i V_i dx_i & = \sum \int_{[t_{k - 1}, t_k]} \dots   \nonumber                   \\
                                    & = \sum f_k(\gamma(t_k)) - f_k(\gamma(t_{k - 1})) \label{обобщ. НЛ} \\
                                    & = f_n(\gamma(b)) - f_1(\gamma(a))\nonumber
    \end{align}

    \ref{обобщ. НЛ}: По обобщенной формуле Ньютона-Лейбница.

    Для \(\tilde \gamma\) воспользуемся свойством: \(f_k\Big|_{B_k \cap B_{k + 1}} = f_{k + 1}\Big|_{B_k \cap B_{k + 1}}\) и тогда аналогично
    \[\int_{\tilde \gamma} \sum v_i dx_i = f_n(\tilde \gamma(b)) - f_1(\tilde \gamma(a))\]
\end{proof}

\begin{remark}
    Вместо условия ``\(\gamma(a) = \tilde \gamma(a), \gamma(b) = \tilde \gamma(b)\)'' можно взять условие: \(\gamma, \tilde \gamma\) --- петли. Тогда утверждение леммы тоже верно.
\end{remark}

\begin{lemma}\itemfix
    \label{лемма 3, лекция 9}
    \begin{itemize}
        \item \(\gamma : [a, b] \to O\) --- непр.
        \item \(V\) --- локально-потенциальное векторное поле в \(O\subset \R^m\)
    \end{itemize}
    Тогда \(\exists \delta > 0 :\) если \(\tilde \gamma, \dbltilde \gamma : [a, b] \to O\) таковы, что:
    \[
        \forall t \in [a, b] \ \ |\gamma(t) - \tilde\gamma(t)| < \delta, |\gamma(t) - \dbltilde\gamma(t)| < \delta
    \]
    Тогда \(\gamma, \tilde\gamma, \dbltilde\gamma\) \(V\)-похожи.
\end{lemma}
\begin{proof}
    Берём \(V\)-гусеницу для \(\gamma\).

    \(\delta_k\)-окрестность множества \(A\) \( : = \{x : \exists a\in A \ \ \rho(a, x) < \delta\} = \bigcap\limits_{a\in A} B(a, \delta)\)

    \begin{figure}[h]
        \centering
        \includesvg[scale=0.7]{images/deltakокрестность.svg}
        \caption{\(\delta_k\)-окрестность множества \(\gamma[t_{k - 1}, t_k]\)}
    \end{figure}

    \[
        \forall k \ \ \exists \delta_k > 0 : \left( \delta_k\text{-окрестность } \gamma[t_{k_1}, t_k] \right) \subset B_k
    \]

    Это следует из компактности:

    Пусть \(B_k = B(w, r)\), функция \(t \in [\gamma_{k - 1}m\, \gamma_k] \mapsto \rho(\gamma(t), w)\) непрерывна \( \Rightarrow \) достигается \(\max \), \(\rho(\gamma(t), w) \le r_0 < r\)

    \(\delta_k : = \frac{r - r_0}{2}, \delta : = \min(\delta_1 \dots \delta_k) \)
\end{proof}

\begin{definition}[Интеграл локального потенциального векторного поля \(V\) по непрерывному пути \(\gamma\)]
    Возьмём \(\delta > 0\) из леммы \ref{лемма 3, лекция 9}.

    Пусть \(\tilde \gamma\) --- \(\delta\)-близкий кусочно-гладкий путь, т.е. \(\forall t \ \ | \gamma(t) - \tilde \gamma(t) | < \delta \).

    Полагаем \(I(V, \gamma) : = I(V, \tilde \gamma)\).

    Корректность \textit{(нет произвольности)} следует из лемм \ref{лемма 3, лекция 9} и \ref{о равенстве интегралов}
\end{definition}

\end{document}