\documentclass[12pt, a4paper]{article}

%<*preamble>
% Math symbols
\usepackage{amsmath, amsthm, amsfonts, amssymb}
\usepackage{accents}
\usepackage{esvect}
\usepackage{mathrsfs}
\usepackage{mathtools}
\mathtoolsset{showonlyrefs}
\usepackage{cmll}
\usepackage{stmaryrd}
\usepackage{physics}
\usepackage[normalem]{ulem}
\usepackage{ebproof}
\usepackage{extarrows}

% Page layout
\usepackage{geometry, a4wide, parskip, fancyhdr}

% Font, encoding, russian support
\usepackage[russian]{babel}
\usepackage[sb]{libertine}
\usepackage{xltxtra}

% Listings
\usepackage{listings}
\lstset{basicstyle=\ttfamily,breaklines=true}
\setmonofont{Inconsolata}

% Miscellaneous
\usepackage{array}
\usepackage{calc}
\usepackage{caption}
\usepackage{subcaption}
\captionsetup{justification=centering,margin=2cm}
\usepackage{catchfilebetweentags}
\usepackage{enumitem}
\usepackage{etoolbox}
\usepackage{float}
\usepackage{lastpage}
\usepackage{minted}
\usepackage{svg}
\usepackage{wrapfig}
\usepackage{xcolor}
\usepackage[makeroom]{cancel}

\newcolumntype{L}{>{$}l<{$}}
    \newcolumntype{C}{>{$}c<{$}}
\newcolumntype{R}{>{$}r<{$}}

% Footnotes
\usepackage[hang]{footmisc}
\setlength{\footnotemargin}{2mm}
\makeatletter
\def\blfootnote{\gdef\@thefnmark{}\@footnotetext}
\makeatother

% References
\usepackage{hyperref}
\hypersetup{
    colorlinks,
    linkcolor={blue!80!black},
    citecolor={blue!80!black},
    urlcolor={blue!80!black},
}

% tikz
\usepackage{tikz}
\usepackage{tikz-cd}
\usetikzlibrary{arrows.meta}
\usetikzlibrary{decorations.pathmorphing}
\usetikzlibrary{calc}
\usetikzlibrary{patterns}
\usepackage{pgfplots}
\pgfplotsset{width=10cm,compat=1.9}
\newcommand\irregularcircle[2]{% radius, irregularity
    \pgfextra {\pgfmathsetmacro\len{(#1)+rand*(#2)}}
    +(0:\len pt)
    \foreach \a in {10,20,...,350}{
            \pgfextra {\pgfmathsetmacro\len{(#1)+rand*(#2)}}
            -- +(\a:\len pt)
        } -- cycle
}

\providetoggle{useproofs}
\settoggle{useproofs}{false}

\pagestyle{fancy}
\lfoot{M3137y2019}
\cfoot{}
\rhead{стр. \thepage\ из \pageref*{LastPage}}

\newcommand{\R}{\mathbb{R}}
\newcommand{\Q}{\mathbb{Q}}
\newcommand{\Z}{\mathbb{Z}}
\newcommand{\B}{\mathbb{B}}
\newcommand{\N}{\mathbb{N}}
\renewcommand{\Re}{\mathfrak{R}}
\renewcommand{\Im}{\mathfrak{I}}

\newcommand{\const}{\text{const}}
\newcommand{\cond}{\text{cond}}

\newcommand{\teormin}{\textcolor{red}{!}\ }

\DeclareMathOperator*{\xor}{\oplus}
\DeclareMathOperator*{\equ}{\sim}
\DeclareMathOperator{\sign}{\text{sign}}
\DeclareMathOperator{\Sym}{\text{Sym}}
\DeclareMathOperator{\Asym}{\text{Asym}}

\DeclarePairedDelimiter{\ceil}{\lceil}{\rceil}

% godel
\newbox\gnBoxA
\newdimen\gnCornerHgt
\setbox\gnBoxA=\hbox{$\ulcorner$}
\global\gnCornerHgt=\ht\gnBoxA
\newdimen\gnArgHgt
\def\godel #1{%
    \setbox\gnBoxA=\hbox{$#1$}%
    \gnArgHgt=\ht\gnBoxA%
    \ifnum     \gnArgHgt<\gnCornerHgt \gnArgHgt=0pt%
    \else \advance \gnArgHgt by -\gnCornerHgt%
    \fi \raise\gnArgHgt\hbox{$\ulcorner$} \box\gnBoxA %
    \raise\gnArgHgt\hbox{$\urcorner$}}

% \theoremstyle{plain}

\theoremstyle{definition}
\newtheorem{theorem}{Теорема}
\newtheorem*{definition}{Определение}
\newtheorem{axiom}{Аксиома}
\newtheorem*{axiom*}{Аксиома}
\newtheorem{lemma}{Лемма}

\theoremstyle{remark}
\newtheorem*{remark}{Примечание}
\newtheorem*{exercise}{Упражнение}
\newtheorem{corollary}{Следствие}[theorem]
\newtheorem*{statement}{Утверждение}
\newtheorem*{corollary*}{Следствие}
\newtheorem*{example}{Пример}
\newtheorem{observation}{Наблюдение}
\newtheorem*{prop}{Свойства}
\newtheorem*{obozn}{Обозначение}

% subtheorem
\makeatletter
\newenvironment{subtheorem}[1]{%
    \def\subtheoremcounter{#1}%
    \refstepcounter{#1}%
    \protected@edef\theparentnumber{\csname the#1\endcsname}%
    \setcounter{parentnumber}{\value{#1}}%
    \setcounter{#1}{0}%
    \expandafter\def\csname the#1\endcsname{\theparentnumber.\Alph{#1}}%
    \ignorespaces
}{%
    \setcounter{\subtheoremcounter}{\value{parentnumber}}%
    \ignorespacesafterend
}
\makeatother
\newcounter{parentnumber}

\newtheorem{manualtheoreminner}{Теорема}
\newenvironment{manualtheorem}[1]{%
    \renewcommand\themanualtheoreminner{#1}%
    \manualtheoreminner
}{\endmanualtheoreminner}

\newcommand{\dbltilde}[1]{\accentset{\approx}{#1}}
\newcommand{\intt}{\int\!}

% magical thing that fixes paragraphs
\makeatletter
\patchcmd{\CatchFBT@Fin@l}{\endlinechar\m@ne}{}
{}{\typeout{Unsuccessful patch!}}
\makeatother

\newcommand{\get}[2]{
    \ExecuteMetaData[#1]{#2}
}

\newcommand{\getproof}[2]{
    \iftoggle{useproofs}{\ExecuteMetaData[#1]{#2proof}}{}
}

\newcommand{\getwithproof}[2]{
    \get{#1}{#2}
    \getproof{#1}{#2}
}

\newcommand{\import}[3]{
    \subsection{#1}
    \getwithproof{#2}{#3}
}

\newcommand{\given}[1]{
    Дано выше. (\ref{#1}, стр. \pageref{#1})
}

\renewcommand{\ker}{\text{Ker }}
\newcommand{\im}{\text{Im }}
\renewcommand{\grad}{\text{grad}}
\newcommand{\rg}{\text{rg}}
\newcommand{\defeq}{\stackrel{\text{def}}{=}}
\newcommand{\defeqfor}[1]{\stackrel{\text{def } #1}{=}}
\newcommand{\itemfix}{\leavevmode\makeatletter\makeatother}
\newcommand{\?}{\textcolor{red}{???}}
\renewcommand{\emptyset}{\varnothing}
\newcommand{\longarrow}[1]{\xRightarrow[#1]{\qquad}}
\DeclareMathOperator*{\esup}{\text{ess sup}}
\newcommand\smallO{
    \mathchoice
    {{\scriptstyle\mathcal{O}}}% \displaystyle
    {{\scriptstyle\mathcal{O}}}% \textstyle
    {{\scriptscriptstyle\mathcal{O}}}% \scriptstyle
    {\scalebox{.6}{$\scriptscriptstyle\mathcal{O}$}}%\scriptscriptstyle
}
\renewcommand{\div}{\text{div}\ }
\newcommand{\rot}{\text{rot}\ }
\newcommand{\cov}{\text{cov}}

\makeatletter
\newcommand{\oplabel}[1]{\refstepcounter{equation}(\theequation\ltx@label{#1})}
\makeatother

\newcommand{\symref}[2]{\stackrel{\oplabel{#1}}{#2}}
\newcommand{\symrefeq}[1]{\symref{#1}{=}}

% xrightrightarrows
\makeatletter
\newcommand*{\relrelbarsep}{.386ex}
\newcommand*{\relrelbar}{%
    \mathrel{%
        \mathpalette\@relrelbar\relrelbarsep
    }%
}
\newcommand*{\@relrelbar}[2]{%
    \raise#2\hbox to 0pt{$\m@th#1\relbar$\hss}%
    \lower#2\hbox{$\m@th#1\relbar$}%
}
\providecommand*{\rightrightarrowsfill@}{%
    \arrowfill@\relrelbar\relrelbar\rightrightarrows
}
\providecommand*{\leftleftarrowsfill@}{%
    \arrowfill@\leftleftarrows\relrelbar\relrelbar
}
\providecommand*{\xrightrightarrows}[2][]{%
    \ext@arrow 0359\rightrightarrowsfill@{#1}{#2}%
}
\providecommand*{\xleftleftarrows}[2][]{%
    \ext@arrow 3095\leftleftarrowsfill@{#1}{#2}%
}

\allowdisplaybreaks

\newcommand{\unfinished}{\textcolor{red}{Не дописано}}

% Reproducible pdf builds 
\special{pdf:trailerid [
<00112233445566778899aabbccddeeff>
<00112233445566778899aabbccddeeff>
]}
%</preamble>


\usepackage{sectsty}

\allsectionsfont{\raggedright}
\subsectionfont{\fontsize{14}{15}\selectfont}

\lhead{Итоговый конспект}
\cfoot{}
\rfoot{}

\settoggle{useproofs}{true}

\begin{document}

\section{Определения}

\import{Мультииндекс и обозначения с ним}{1.tex}{мультииндекс}

\import{\teormin Формула Тейлора (различные виды записи)}{1.tex}{формулатейлорадифференциал}
\get{1.tex}{формулатейлорамультииндекс}

\import{$n$-й дифференциал}{1.tex}{дифференциал}

\import{\teormin Норма линейного оператора}{1.tex}{нормалоп}

\import{Положительно-, отрицательно-, незнако- определенная квадратичная форма}{2.tex}{формы}

\import{Локальный максимум, минимум, экстремум}{2.tex}{экстремум}

\import{Диффеоморфизм}{3.tex}{диффеоморфизм}

\subsection{Формулировка теоремы о локальной обратимости}
\get{3.tex}{олокальнойобратимости}

\subsection{Формулировка теоремы о локальной обратимости в терминах систем уравнений}
\get{3.tex}{олокальнойобратимостисистема}

\import{Формулировка теоремы о неявном отображении в терминах систем уравнений}{4.tex}{онеявномотображениивтерминахсистемы}

\import{\teormin Простое $k$-мерное гладкое многообразие в $\R^m$}{4.tex}{простоеkмерноегладкоемногообразие}

\import{Касательное пространство к $k$-мерному многообразию в $\R^m$}{5.tex}{касательноепространство}

\import{Относительный локальный максимум, минимум, экстремум}{5.tex}{локальныйминимум}

\import{\teormin Формулировка достаточного условия относительного экстремума}{6.tex}{достаточноеусловиеэкстремума}

\import{Поточечная сходимость последовательности функций на множестве }{5.tex}{поточеченаясходимость}

\import{\teormin Равномерная сходимость последовательности функций на множестве }{5.tex}{равномернаясходимость}

\import{Равномерная сходимость функционального ряда}{6.tex}{равномернаясходимостьряда}

\import{Формулировка критерия Больцано-Коши для равномерной сходимости}{6.tex}{критерийбольцанокоши}

\import{\teormin Степенной ряд, радиус сходимости степенного ряда, формула Адамара}{9.tex}{}
\get{9.tex}{степеннойряд}
\get{9.tex}{формулаадамара}

\import{Признак Абеля равномерной сходимости функционального ряда}{9.tex}{}
Ряд \(\sum a_n(x) b_n(x)\) равномерно сходится, если:
\begin{enumerate}
    \item \(a_n\) равномерно ограничена и монотонна по \(n\) для любого \(x\)
    \item \(\sum b_n\) равномерно сходится
\end{enumerate}

\import{Кусочно-гладкий путь}{7.tex}{путь}
\get{7.tex}{кусочногладкоеотображение}

\import{Векторное поле}{7.tex}{векторноеполе}

\import{Интеграл векторного поля по кусочно-гладкому пути}{7.tex}{интегралвекторногополяпокусочногладкомупути}

\import{\teormin Потенциал, потенциальное векторное поле}{7.tex}{потенциальноевекторноеполе}

\import{Локально потенциальное векторное поле}{8.tex}{локальнопотенциальноеполе}

\import{Интеграл локально-потенциального векторного поля по произвольному пути}{9.tex}{интеграллокальнопотенциальноговекторногопляпонепрерывномупути}

\import{Гомотопия путей связанная и петельная}{10.tex}{гомотопия}

\import{Односвязная область}{10.tex}{односвязнаяобласть}

\import{\teormin Полукольцо, алгебра, сигма-алгебра}{11.tex}{полукольцо}
\get{11.tex}{алгебраподмножеств}
\get{12.tex}{сигмаалгебра}

\import{\teormin Объем}{12.tex}{объем}

\import{\teormin Ячейка}{11.tex}{ячейка}

\import{Классический объем в $\mathbb R^m$}{12.tex}{классическийобъем}

\import{Формулировка теорема о непрерывности снизу}{13.tex}{онепрерывностиснизу}

\import{\teormin Мера, пространство с мерой}{13.tex}{мера}
\get{13.tex}{пространствосмерой}

\import{Полная мера}{13.tex}{полнаямера}

\import{\teormin Сигма-конечная мера}{13.tex}{сигмаконечнаямера}

\import{Дискретная мера}{13.tex}{дискретнаямера}

\import{Формулировка теоремы о лебеговском продолжении меры}{14.tex}{олебеговскомпродолжениимеры}

\import{\teormin Мера Лебега, измеримое по Лебегу множество}{14.tex}{мералебега}

\import{Борелевская сигма-алгебра}{14.tex}{борелевскаясигмаалгебра}

\import{Формулировка теоремы о мерах, инвариантных относительно сдвигов}{15.tex}{инвариантныемеры}

\section{Теоремы}

\import{Лемма о дифференцировании ``сдвига''}{1.tex}{леммаодиффсдвига}

\import{\teormin Многомерная формула Тейлора (с остатком в форме Лагранжа и Пеано)}{1.tex}{}
\subsubsection{В форме Лагранжа}
\getwithproof{1.tex}{тейлорлагранж}
\subsubsection{В форме Пеано}
\getwithproof{1.tex}{тейлорпеано}

\import{Теорема о пространстве линейных отображений}{1.tex}{опространствелинейныхотображений}

\import{Лемма об условиях, эквивалентных непрерывности линейного оператора}{1.tex}{эквивалентностьнепрерывности}

\import{Теорема Лагранжа для отображений}{2.tex}{лагранжадляотображений}

\import{Теорема об обратимости линейного отображения, близкого к обратимому}{2.tex}{обобратимости}

\import{Теорема о непрерывно дифференцируемых отображениях}{2.tex}{онепрдифф}

\import{Теорема Ферма. Необходимое условие экстремума. Теорема Ролля}{2.tex}{ферма}
\get{2.tex}{необходимоеусловиеэкстремума}
\getwithproof{2.tex}{ролля}

\import{Лемма об оценке квадратичной формы и об эквивалентных нормах}{2.tex}{обоценкеформы}
\getwithproof{2.tex}{обэквивалентныхнормах}

\import{\teormin Достаточное условие экстремума}{2.tex}{достаточноеусловиеэкстремума}

\import{Лемма о ``почти локальной инъективности''}{3.tex}{опочтилокальнойиньективности}

\import{Теорема о сохранении области}{3.tex}{осохраненииобласти}

\import{Следствие о сохранении области для отображений в пространство меньшей размерности}{3.tex}{следствиеосохраненииобласти}

\import{Теорема о гладкости обратного отображения}{3.tex}{огладкостиобратногоотображения}

\import{\teormin Теорема о неявном отображении}{4.tex}{онеявномотображении}

\import{Теорема о задании гладкого многообразия системой уравнений}{4.tex}{озаданиигладкогомногообразиясистемойуравнений}

\import{Следствие о двух параметризациях}{4.tex}{одвухпараметризациях}

\import{Лемма о корректности определения касательного пространства}{5.tex}{касательноепространство}
\get{5.tex}{касательноепространство-proof}

\import{Касательное пространство в терминах векторов скорости гладких путей}{5.tex}{касательноепространствовтерминахвекторовскоростигладкихпутей}

\import{Касательное пространство к графику функции и к поверхности уровня}{5.tex}{касательноепространствокграфику}
\getwithproof{5.tex}{линииуровня}

\import{\teormin Необходимое условие относительного локального экстремума}{5.tex}{необходимоеусловиеотносительногоэкстремума}

\import{Вычисление нормы линейного оператора с помощью собственных чисел}{5.tex}{нормалоп}

\import{\teormin Теорема Стокса-Зайдля о непрерывности предельной функции. Следствие для рядов}{6.tex}{стоксазайдля}
\getwithproof{7.tex}{стоксазайдлярядов}

\import{Метрика в пространстве непрерывных функций на компакте, его полнота}{6.tex}{метрикавcx}

\import{Теорема о предельном переходе под знаком интеграла. Следствие для рядов}{6.tex}{предельныйпереходподинтегралом}
\getwithproof{7.tex}{предельныйпереходподинтегралом2}

\import{Правило Лейбница дифференцирования интеграла по параметру}{6.tex}{правилолейбница}

\import{Теорема о предельном переходе под знаком производной. Дифференцирование функционального ряда}{6.tex}{предельныйпереходподпроизводной}
\getwithproof{8.tex}{дифференцированиеряда}

\import{\teormin Признак Вейерштрасса равномерной сходимости функционального ряда}{7.tex}{признаквейерштрасса}

\import{Дифференцируемость $\Gamma$ функции}{8.tex}{дифференцируемостьгаммафункции}
\getwithproof{8.tex}{дифференцируемостьгаммафункции2}

\import{Теорема о предельном переходе в суммах}{8.tex}{определьномпереходевсуммах}

\import{Теорема о перестановке двух предельных переходов}{8.tex}{оперестановкедвухпредельныхпереходов}

\import{\teormin Признак Дирихле равномерной сходимости функционального ряда}{9.tex}{признакдирихле}

\import{Теорема о круге сходимости степенного ряда}{9.tex}{кругстепенногоряда}

\import{Теорема о непрерывности степенного ряда}{10.tex}{оравномернойсходимостиинепрерывностистепенногоряда}

\import{Теорема о дифференцировании степенного ряда. Следствие об интегрировании. Пример.}{10.tex}{одифференцируемостистепенногоряда}
\get{11.tex}{следствиеобинтегрировании}
\get{11.tex}{следствиеобинтегрированииexample}

\import{Свойства экспоненты}{11.tex}{свойстваэкспоненты}

\import{Метод Абеля суммирования рядов. Следствие}{11.tex}{методабеля}
\getwithproof{11.tex}{теоремаабеля}

\import{Единственность разложения функции в ряд}{12.tex}{разлагаетсявстепеннойряд}
\getwithproof{12.tex}{оединственности}

\import{Разложение бинома в ряд Тейлора}{13.tex}{разложениебинома}

\import{Теорема о разложимости функции в ряд Тейлора}{13.tex}{оразложимости}

\import{Простейшие свойства интеграла векторного поля по кусочно-гладкому пути}{7.tex}{свойства}

\import{\teormin Обобщенная формула Ньютона--Лейбница}{7.tex}{обобщеннаяформуланьютоналейбница}

\import{Характеризация потенциальных векторных полей в терминах интегралов}{8.tex}{характеризацияпотенциальныхвекторныхполейвтерминахинтегралов}

\import{\teormin Необходимое условие потенциальности гладкого поля. Лемма Пуанкаре}{8.tex}{необходимоеусловиепотенциальности}
Лемма Пуанкаре:
\getwithproof{8.tex}{леммапуанкаре}
Лемма Пуанкаре о локальной потенциальности:
\get{8.tex}{леммапуанкаре2}

\import{Лемма о гусенице}{9.tex}{огусенице}

\import{Лемма о равенстве интегралов по похожим путям}{9.tex}{оравенствеинтегралов}

\import{Лемма о похожести путей, близких к данному}{9.tex}{опохожестиблизкихпутей}

\import{Равенство интегралов по гомотопным путям}{10.tex}{интегралпосвязанногомотопнымпутям}

\import{\teormin Теорема Пуанкаре для односвязной области}{10.tex}{леммапуанкареводносвязнойобласти}

\import{Теорема о веревочке}{10.tex}{оверевочке}

\import{Свойства объема: усиленная монотонность, конечная полуаддитивность}{12.tex}{свойстваобъема}
\get{12.tex}{свойстваобъемаremark}

\import{Теорема об эквивалентности счетной аддитивности и счетной полуаддитивности}{13.tex}{обэквивалентностисчетнойаддитивностиисчетнойполуаддитивности}

\import{Теорема о непрерывности сверху}{13.tex}{онепрерывностисверху}

\import{Счетная аддитивность классического объема}{14.tex}{классическийобъемаддитивность}

\import{Лемма о структуре открытых множеств и множеств меры 0}{14.tex}{леммаоструктуреоткрытыхимеры0}

\import{Пример неизмеримого по Лебегу множества}{14.tex}{неизмеримоемножество}

\import{\teormin Регулярность меры Лебега}{15.tex}{регулярностьмерылебега}

\import{Лемма о сохранении измеримости при непрерывном отображении}{15.tex}{непрерывноеотображениеизмеримость}

\import{Лемма о сохранении измеримости при гладком отображении. Инвариантность меры Лебега относительно сдвигов}{15.tex}{гладкоеотображениеизмеримость}
\getwithproof{15.tex}{инвариантностьмерылебега}

\import{Инвариантность меры Лебега при ортогональном преобразовании}{15.tex}{инвариантностьмерылебегаотносительнолинейногоортогональногопреобразования}


\end{document}
