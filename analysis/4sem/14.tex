\chapter{17 мая}

\begin{obozn}
    \([ - \pi, \pi] \setminus [ - \delta, \delta] = E_\delta\)
\end{obozn}

\begin{definition}[аппроксимативная единица]\itemfix
    %<*аппроксимативнаяединица>
    \begin{itemize}
        \item \(D \subset \R\)
        \item \(h_0\) --- предельная точка \(D\) в \(\overline\R\)
    \end{itemize}

    Семейство функций \(\{K_h\}_{h \in D}\), удовлетворяющее нижеуказанным аксиомам, называется \textbf{аппроксимативной единицей}.

    \begin{axiom}
        \(\forall h \in D \ \ K_h \in L^1[ - \pi, \pi], \int_{ - \pi}^\pi K_h = 1\)
    \end{axiom}

    \begin{axiom}
        \(L_1\) нормы функций \(K_h\) ограничены в совокупности:
        \[ \exists M \ \ \forall h \ \ \int_{[-\pi, \pi]} |K_h| \le M \]
    \end{axiom}

    \begin{axiom}
        \(\forall \delta \in (0, \pi)\)
        \[ \int_{E_\delta} |K_h|\,dx \xrightarrow[h \to h_0]{} 0 \]
    \end{axiom}
    %</аппроксимативнаяединица>
\end{definition}

\begin{remark}\itemfix
    \begin{enumerate}
        \item Если \(K_h \geq 0 \ \ \forall h\), то аксиома 1 \( \Rightarrow \) аксиома 2.
        \item %<*усиленнаяаппроксимативнаяединица>
              Рассмотрим аксиому 3': \(K_h \in L^{+\infty}[ - \pi, \pi]\) и \(\forall \delta \in (0, \pi) \ \ \esup_{t \in E_\delta} |K_h(t)| \xrightarrow{h \to h_0} 0\)

              \begin{statement}
                  Аксиома 3' \( \Rightarrow \) аксиома 3.
              \end{statement}

              \begin{definition}
                  Семейство функций, удовлетворяющее аксиомам 1, 2 и 3' называется \textbf{усиленной аппроксимативной единицей}.
              \end{definition}
              %</усиленнаяаппроксимативнаяединица>
        \item Если \(K_h\) --- \textit{(возможно усиленная)} аппроксимативная единица, то \(\frac{|K_h|}{\norm{K_h}_1}\) --- тоже \textit{(возможно усиленная)} аппроксимативная единица.
              \begin{proof}
                  Первая аксиома очевидна.

                  Вторая аксиома следует из того, что \(\norm{K_h}_1 \geq 1\), т.к. \(\int K_n = 1\). Аксиомы 3 и 3' тоже следуют из этого соображения.
              \end{proof}
    \end{enumerate}
\end{remark}

\begin{theorem}\itemfix
    %<*свойстваединицы>
    \begin{itemize}
        \item \(K_h\) --- аппроксимативная единица
    \end{itemize}

    Тогда:
    \begin{enumerate}
        \item \(f \in \widetilde{C}[ - \pi, \pi] \Rightarrow f * K_h \xrightrightarrows[h \to h_0]{[-\pi, \pi]} f\)
        \item \(f \in L^1[ - \pi, \pi] \Rightarrow \norm{f * K_h - f}_1 \xrightarrow{h \to +\infty} 0\)
        \item \(K\) --- усиленная аппроксимативная единица, \(f \in L^1[ - \pi, \pi], f\) непрерывно в \(x\).

              Тогда \(f * K_h\) непрерывно в \(x\) и \(f * K_h(x) \xrightarrow{h \to h_0} f(x)\)
    \end{enumerate}
    %</свойстваединицы>
\end{theorem}
%<*свойстваединицыproof>
\begin{proof}
    \[ f*K_h(x) - f(x) = \int_{-\pi}^\pi (f(x - t) - f(x))K_h(t)\,dt \]
    \begin{enumerate}
        \item \(\sphericalangle \varepsilon > 0, f\) --- равномерно непрерывна, т.к. \([ - \pi, \pi]\) --- компакт.
              \[ \exists \delta > 0 \ \ \forall t : |t| < \delta \ \ \forall x \ \ |f(x - t) - f(x)| < \frac{\varepsilon}{2M} \]
              \(M\) взято из аксиомы 2.
              \[ |f*K_h(x) - f(x)| \le \int_{-\pi}^\pi |f(x - t) - f(x)| |K_h(t)| \,dt = \int_{-\delta}^\delta + \int_{E_\delta}  = I_1 + I_2 \color{gray} < \varepsilon ? \]
              \[I_1 \leq \frac{\varepsilon}{2M} \cdot \int_{ - \delta}^\delta |K_h| \leq \frac{\varepsilon}{2M}\int_{ - \pi}^\pi |K_h| \leq \frac{\varepsilon}{2}\]
              \[I_2 \leq 2 \cdot \norm{f}_{\infty} \cdot \int_{E_\delta} |K_h| \xrightarrow[\text{акс. 3}]{h \to h_0} 0\]
              Тогда \(\exists V(h_0) \ \ \forall h \in V(h_0) \ \ I_2 \leq \frac{\varepsilon}{2}\).

        \item [3.] \(f \in L^1\), \(K_h \in L^\infty \Rightarrow f*K_h\) --- непрерывна  \textit{(в том числе и в \(x\))}.

              Для данного \(x\) проверим утверждение \(\varepsilon > 0;\ I_1 + I_2 < \varepsilon;\ \exists V(h_0)\ \forall h\in V(h_0)\)

              \(f\) непрерывна в \(x\):
              \[ \forall \varepsilon > 0 \ \ \exists \delta > 0 \ \ \forall t : |t| < \delta \quad |f(x - t) - f(x)| < \frac{\varepsilon}{2M} \]
              Как в пункте 1:
              \[ I_1 \le \frac{\varepsilon}{2} \]
              \[ I_2 \le \int_{E_\delta} |f(x - t)|\cdot |K_h(t)|\,dt + |f(x)| \int_{E_\delta} |K_h(t)|\,dt \]
              \[ \le \esup_{E_\delta}|K_h|\cdot \left(\norm{f}_1 + 2\pi |f(x)|\right) \xrightarrow[h \to h_0]{\text{акс. 3'}} 0 \]
              Тогда \(\exists V(h_0) \ \ \forall h \in V(h_0) \ \ I_2 \leq \frac{\varepsilon}{2}\).
        \item [2.] \[ \norm{f * K_h - f}_1 = \int_{-\pi}^\pi \left| \int_{-\pi}^\pi(f(x - t) - f(x)) K_h \,dt\right|\,dx \le \]
              \[ \le \int_{-\pi}^\pi \int_{-\pi}^\pi |f(x - t) - f(x)| \cdot |K_h(t)|\,dx\,dt = \]
              \[ = \norm{K_h}_1 \cdot \int_{-\pi}^\pi g(-t) \frac{|K_h(t)|}{\norm{K_h}_1}\,dt \]
              , где \(g(t) = \int_{-\pi}^\pi |f(x + t) - f(x)|\) --- непрерывна (по теореме о непрерывности сдвига)
              \[ \defeq \norm{K_h}_1 \underbrace{\left(g * \frac{|K_h|}{\norm{K_h}}\right)(0)}_{ \to g(0) = 0 \text{ по п.1}} \]
    \end{enumerate}
\end{proof}
%</свойстваединицыproof>

\begin{remark}[модификация пункта 2]
    \(f \in L^p[ - \pi, \pi] \Rightarrow ||f * K_h - f||_p \xrightarrow{h \to h_0} 0\)
\end{remark}
\begin{proof}
    Аналогично пункту 2, но хуже.
\end{proof}

\begin{remark}[модификация пункта 3]
    \(f \in L^1, \exists f(x - 0), f(x + 0)\). \(K_h\) --- усиленная аппроксимативная единица, \(\forall h \ \ K_h\) чётная. Тогда \((f * K_h)(x) \to \frac{1}{2} (f(x - 0) + f(x + 0))\)
\end{remark}

\section{Суммирование рядов Фурье}

\subsection{Метод средних арифметических \textit{(Чезаро)}}

\begin{definition}
    %<*среднихарифметических>
    \[ \sum a_n \quad S_n \coloneqq \sum_{k = 0}^n a_k \]
    \[ \sigma_n \coloneqq \frac{1}{n + 1} (S_0 + S_1 + \dots + S_n) \]
    \[ \sum a_n \xlongequal{\text{сред. арифм.}} S \]
    , если \(\sigma_n \to S\)
    %</среднихарифметических>
\end{definition}

\begin{theorem}[о перманентности метода средних арифметических]
    %<*оперманентности>
    \[\sum a_n = S \Rightarrow \sum a_n \xlongequal{\text{сред. арифм.}} S\]
    %</оперманентности>
\end{theorem}

\begin{definition}[суммы Фейера]
    %<*суммафейера>
    \(f \in L^1 [ - \pi, \pi], S_n(f)\) --- част. сумма ряда Фурье.
    \[\sigma_n(f) = \frac{1}{n + 1} \sum_{k = 0}^n S_k(f)\]
    %</суммафейера>
\end{definition}
\begin{remark}
    \[S_n(f) = \int_{ - \pi}^\pi f(x + t) D_n(t) dt\]
    \[\sigma_n(f, x) = \int_{ - \pi}^\pi f(x + t) \Phi_n(t) dt \xlongequal{\Phi_n \text{ чётно}} \int_{ - \pi}^\pi f(x - t) \Phi_n(t) dt\]
\end{remark}

\begin{example}
    Для расходящегося факта \(1 - 1 + 1 - 1\dots \) \(\sigma_n \to \frac{1}{2}\)
\end{example}

%<*оперманентностиproof>
\begin{proof}[Доказательство теоремы]
    \[\forall \varepsilon > 0 \ \ \exists N_1 \ \ \forall n > N_1 \ \ |S_n - S| < \frac{\varepsilon}{2}\]
    \[|\sigma_n - S| = \left|\frac{1}{n + 1} \sum_{k = 0}^n (S_k - S)\right| \leq \frac{1}{n + 1} \sum |S_k - S| = \underbrace{\frac{\sum_{k = 0}^{N_1} |S_k - S|}{n + 1}}_{\xrightarrow{n \to +\infty} 0} + \underbrace{\frac{\sum_{k = N_1 + 1}^{n} |S_k - S|}{n + 1}}_{ < \frac{\varepsilon}{2}} \]
\end{proof}
%</оперманентностиproof>

\begin{theorem}[Фейера]\itemfix
    %<*теоремафейера>
    \begin{enumerate}
        \item \(f \in \widetilde{C}[ - \pi, \pi]\). Тогда \(\sigma_n(f) \xrightrightarrows{[ - \pi, \pi]} f\)
        \item \(f \in L^p[ - \pi, \pi], 1 \leq p \leq +\infty\). Тогда \(\norm{\sigma_n(f) - f}_{\infty} \to 0\)
        \item \(f \in L^1, f\) непрерывно в \(x\). Тогда \(\sigma_n(f, x) \xrightarrow{n \to +\infty} f(x)\)
    \end{enumerate}
    %</теоремафейера>
\end{theorem}
