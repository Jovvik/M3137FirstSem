\documentclass[12pt, a4paper]{article}

%<*preamble>
% Math symbols
\usepackage{amsmath, amsthm, amsfonts, amssymb}
\usepackage{accents}
\usepackage{esvect}
\usepackage{mathrsfs}
\usepackage{mathtools}
\mathtoolsset{showonlyrefs}
\usepackage{cmll}
\usepackage{stmaryrd}
\usepackage{physics}
\usepackage[normalem]{ulem}
\usepackage{ebproof}
\usepackage{extarrows}

% Page layout
\usepackage{geometry, a4wide, parskip, fancyhdr}

% Font, encoding, russian support
\usepackage[russian]{babel}
\usepackage[sb]{libertine}
\usepackage{xltxtra}

% Listings
\usepackage{listings}
\lstset{basicstyle=\ttfamily,breaklines=true}
\setmonofont{Inconsolata}

% Miscellaneous
\usepackage{array}
\usepackage{calc}
\usepackage{caption}
\usepackage{subcaption}
\captionsetup{justification=centering,margin=2cm}
\usepackage{catchfilebetweentags}
\usepackage{enumitem}
\usepackage{etoolbox}
\usepackage{float}
\usepackage{lastpage}
\usepackage{minted}
\usepackage{svg}
\usepackage{wrapfig}
\usepackage{xcolor}
\usepackage[makeroom]{cancel}

\newcolumntype{L}{>{$}l<{$}}
    \newcolumntype{C}{>{$}c<{$}}
\newcolumntype{R}{>{$}r<{$}}

% Footnotes
\usepackage[hang]{footmisc}
\setlength{\footnotemargin}{2mm}
\makeatletter
\def\blfootnote{\gdef\@thefnmark{}\@footnotetext}
\makeatother

% References
\usepackage{hyperref}
\hypersetup{
    colorlinks,
    linkcolor={blue!80!black},
    citecolor={blue!80!black},
    urlcolor={blue!80!black},
}

% tikz
\usepackage{tikz}
\usepackage{tikz-cd}
\usetikzlibrary{arrows.meta}
\usetikzlibrary{decorations.pathmorphing}
\usetikzlibrary{calc}
\usetikzlibrary{patterns}
\usepackage{pgfplots}
\pgfplotsset{width=10cm,compat=1.9}
\newcommand\irregularcircle[2]{% radius, irregularity
    \pgfextra {\pgfmathsetmacro\len{(#1)+rand*(#2)}}
    +(0:\len pt)
    \foreach \a in {10,20,...,350}{
            \pgfextra {\pgfmathsetmacro\len{(#1)+rand*(#2)}}
            -- +(\a:\len pt)
        } -- cycle
}

\providetoggle{useproofs}
\settoggle{useproofs}{false}

\pagestyle{fancy}
\lfoot{M3137y2019}
\cfoot{}
\rhead{стр. \thepage\ из \pageref*{LastPage}}

\newcommand{\R}{\mathbb{R}}
\newcommand{\Q}{\mathbb{Q}}
\newcommand{\Z}{\mathbb{Z}}
\newcommand{\B}{\mathbb{B}}
\newcommand{\N}{\mathbb{N}}
\renewcommand{\Re}{\mathfrak{R}}
\renewcommand{\Im}{\mathfrak{I}}

\newcommand{\const}{\text{const}}
\newcommand{\cond}{\text{cond}}

\newcommand{\teormin}{\textcolor{red}{!}\ }

\DeclareMathOperator*{\xor}{\oplus}
\DeclareMathOperator*{\equ}{\sim}
\DeclareMathOperator{\sign}{\text{sign}}
\DeclareMathOperator{\Sym}{\text{Sym}}
\DeclareMathOperator{\Asym}{\text{Asym}}

\DeclarePairedDelimiter{\ceil}{\lceil}{\rceil}

% godel
\newbox\gnBoxA
\newdimen\gnCornerHgt
\setbox\gnBoxA=\hbox{$\ulcorner$}
\global\gnCornerHgt=\ht\gnBoxA
\newdimen\gnArgHgt
\def\godel #1{%
    \setbox\gnBoxA=\hbox{$#1$}%
    \gnArgHgt=\ht\gnBoxA%
    \ifnum     \gnArgHgt<\gnCornerHgt \gnArgHgt=0pt%
    \else \advance \gnArgHgt by -\gnCornerHgt%
    \fi \raise\gnArgHgt\hbox{$\ulcorner$} \box\gnBoxA %
    \raise\gnArgHgt\hbox{$\urcorner$}}

% \theoremstyle{plain}

\theoremstyle{definition}
\newtheorem{theorem}{Теорема}
\newtheorem*{definition}{Определение}
\newtheorem{axiom}{Аксиома}
\newtheorem*{axiom*}{Аксиома}
\newtheorem{lemma}{Лемма}

\theoremstyle{remark}
\newtheorem*{remark}{Примечание}
\newtheorem*{exercise}{Упражнение}
\newtheorem{corollary}{Следствие}[theorem]
\newtheorem*{statement}{Утверждение}
\newtheorem*{corollary*}{Следствие}
\newtheorem*{example}{Пример}
\newtheorem{observation}{Наблюдение}
\newtheorem*{prop}{Свойства}
\newtheorem*{obozn}{Обозначение}

% subtheorem
\makeatletter
\newenvironment{subtheorem}[1]{%
    \def\subtheoremcounter{#1}%
    \refstepcounter{#1}%
    \protected@edef\theparentnumber{\csname the#1\endcsname}%
    \setcounter{parentnumber}{\value{#1}}%
    \setcounter{#1}{0}%
    \expandafter\def\csname the#1\endcsname{\theparentnumber.\Alph{#1}}%
    \ignorespaces
}{%
    \setcounter{\subtheoremcounter}{\value{parentnumber}}%
    \ignorespacesafterend
}
\makeatother
\newcounter{parentnumber}

\newtheorem{manualtheoreminner}{Теорема}
\newenvironment{manualtheorem}[1]{%
    \renewcommand\themanualtheoreminner{#1}%
    \manualtheoreminner
}{\endmanualtheoreminner}

\newcommand{\dbltilde}[1]{\accentset{\approx}{#1}}
\newcommand{\intt}{\int\!}

% magical thing that fixes paragraphs
\makeatletter
\patchcmd{\CatchFBT@Fin@l}{\endlinechar\m@ne}{}
{}{\typeout{Unsuccessful patch!}}
\makeatother

\newcommand{\get}[2]{
    \ExecuteMetaData[#1]{#2}
}

\newcommand{\getproof}[2]{
    \iftoggle{useproofs}{\ExecuteMetaData[#1]{#2proof}}{}
}

\newcommand{\getwithproof}[2]{
    \get{#1}{#2}
    \getproof{#1}{#2}
}

\newcommand{\import}[3]{
    \subsection{#1}
    \getwithproof{#2}{#3}
}

\newcommand{\given}[1]{
    Дано выше. (\ref{#1}, стр. \pageref{#1})
}

\renewcommand{\ker}{\text{Ker }}
\newcommand{\im}{\text{Im }}
\renewcommand{\grad}{\text{grad}}
\newcommand{\rg}{\text{rg}}
\newcommand{\defeq}{\stackrel{\text{def}}{=}}
\newcommand{\defeqfor}[1]{\stackrel{\text{def } #1}{=}}
\newcommand{\itemfix}{\leavevmode\makeatletter\makeatother}
\newcommand{\?}{\textcolor{red}{???}}
\renewcommand{\emptyset}{\varnothing}
\newcommand{\longarrow}[1]{\xRightarrow[#1]{\qquad}}
\DeclareMathOperator*{\esup}{\text{ess sup}}
\newcommand\smallO{
    \mathchoice
    {{\scriptstyle\mathcal{O}}}% \displaystyle
    {{\scriptstyle\mathcal{O}}}% \textstyle
    {{\scriptscriptstyle\mathcal{O}}}% \scriptstyle
    {\scalebox{.6}{$\scriptscriptstyle\mathcal{O}$}}%\scriptscriptstyle
}
\renewcommand{\div}{\text{div}\ }
\newcommand{\rot}{\text{rot}\ }
\newcommand{\cov}{\text{cov}}

\makeatletter
\newcommand{\oplabel}[1]{\refstepcounter{equation}(\theequation\ltx@label{#1})}
\makeatother

\newcommand{\symref}[2]{\stackrel{\oplabel{#1}}{#2}}
\newcommand{\symrefeq}[1]{\symref{#1}{=}}

% xrightrightarrows
\makeatletter
\newcommand*{\relrelbarsep}{.386ex}
\newcommand*{\relrelbar}{%
    \mathrel{%
        \mathpalette\@relrelbar\relrelbarsep
    }%
}
\newcommand*{\@relrelbar}[2]{%
    \raise#2\hbox to 0pt{$\m@th#1\relbar$\hss}%
    \lower#2\hbox{$\m@th#1\relbar$}%
}
\providecommand*{\rightrightarrowsfill@}{%
    \arrowfill@\relrelbar\relrelbar\rightrightarrows
}
\providecommand*{\leftleftarrowsfill@}{%
    \arrowfill@\leftleftarrows\relrelbar\relrelbar
}
\providecommand*{\xrightrightarrows}[2][]{%
    \ext@arrow 0359\rightrightarrowsfill@{#1}{#2}%
}
\providecommand*{\xleftleftarrows}[2][]{%
    \ext@arrow 3095\leftleftarrowsfill@{#1}{#2}%
}

\allowdisplaybreaks

\newcommand{\unfinished}{\textcolor{red}{Не дописано}}

% Reproducible pdf builds 
\special{pdf:trailerid [
<00112233445566778899aabbccddeeff>
<00112233445566778899aabbccddeeff>
]}
%</preamble>


\lhead{Математический анализ}
\cfoot{}
\rfoot{15.2.2021}

\begin{document}

\begin{theorem}[характеризация измеримых функций с помощью ступенчатых]\itemfix
    %<*характеризацияизмеримыхфункцийспомощьюступенчатых>
    \begin{itemize}
        \item \(f: X \to \R\)
        \item \(f \geq 0\)
        \item \(f\) измеримо
    \end{itemize}
    Тогда \(\exists f_n\) --- ступенчатые:
    \begin{enumerate}
        \item \(0 \leq f_1 \leq f_2 \leq f_3 \leq \dots \)
        \item \(\forall x \ \ f(x) = \lim\limits_{n \to +\infty} f_n(x)\)
    \end{enumerate}
    %</характеризацияизмеримыхфункцийспомощьюступенчатых>
\end{theorem}

\begin{proof}
    %<*характеризацияизмеримыхфункцийспомощьюступенчатыхproof>
    \[e^{(n)}_k = X\left( \frac{k - 1}{n} \leq f < \frac{k}{n} \right) \quad k = 1 \dots n^2\]
    \[e^{(n)}_{n^2 + 1} : = X(n \leq f)\]
    \[g_n: = \sum_{k = 1}^{n^2 + 1} \frac{k - 1}{n} \chi_{e_k^{(n)}}\]

    \begin{figure}[h]
        \centering
        \includesvg[scale=0.7]{images/приближение_ступенчатой.svg}
    \end{figure}

    \[g_n \geq 0\]
    \[\lim\limits_{n \to +\infty} g_n(x) = f(x) : g_n(x) \leq f(x)\]

    \textcolor{red}{Не дописано.}

    % TODO: #66 дописать
    %</характеризацияизмеримыхфункцийспомощьюступенчатыхproof>
\end{proof}

\begin{corollary}\itemfix
    \begin{itemize}
        \item \(f\) --- измеримо
    \end{itemize}
    Тогда \(\exists f_n\) --- измеримые : \(f_n \xrightarrow[n \to +\infty]{} f\) всюду и \(|f_n| \leq |f|\)
\end{corollary}
\begin{proof}
    Рассмотрим срезки \(f^{+}, f^{-}\), дальше очевидно.
\end{proof}

\begin{corollary}\itemfix
    \begin{itemize}
        \item \(f, g\) --- измеримо
    \end{itemize}
    Тогда \(fg\) --- измеримо, если \(0\cdot \infty = 0\).
\end{corollary}
\begin{proof}
    \[\underbrace{f_n}_{\text{ступ.}} \to f, \underbrace{g_n}_{\text{ступ.}} \to g\]
    \[f_n g_n \text{ --- ступ.} \quad f_n g_n \to fg\]
    Измеримость выполняется в силу измеримости предела.
\end{proof}

\begin{corollary}\itemfix
    \begin{itemize}
        \item \(f, g\) --- измеримо
    \end{itemize}

    Тогда \(f + g\) измеримо.

    \begin{remark}
        Считаем, что \(\forall x\) не может быть одновременно \(f(x) = \pm \infty, g(x) = \pm \infty\).
    \end{remark}
\end{corollary}

\begin{proof}
    \[f_n + g_n \to f + g\]
\end{proof}

\begin{theorem}[об измеримости функций, непрерывных на множестве полной меры]
    %<*обизмеримостифункцийнепрерывныхнамножествеполноймеры>
    \begin{remark}
        \(A\subset X\) --- \textbf{полной меры}, если \(\mu(X\setminus A) = 0\).
    \end{remark}

    \begin{itemize}
        \item \(f: E \to \R, E\subset \R^m\)
        \item \(e\subset E\)
        \item \(\lambda_m e = 0\)
        \item \(f\) --- непрерывно на \(E' = E \setminus e\)
    \end{itemize}

    Тогда \(f\) --- измеримо.
    %</обизмеримостифункцийнепрерывныхнамножествеполноймеры>
\end{theorem}
%<*обизмеримостифункцийнепрерывныхнамножествеполноймерыproof>
\begin{proof}
    \(f\) --- измеримо на \(E'\), т.к. \(E'(f < a)\) открыто в \(E'\) по топологическому определению непрерывности.

    \(e(f < a) \subset e\), \(\lambda_m\) --- полная \( \Rightarrow e(f < a)\) --- измеримо в \(E\).

    \(E(f < a) = E'(f < a)\cup e(f < a)\), объединение измеримых множеств измеримо.
\end{proof}
%</обизмеримостифункцийнепрерывныхнамножествеполноймерыproof>

\begin{example}
    \(E = \R, f = \chi_{\text{Irr}}\), где \(\text{Irr}\) --- множество иррациональных чисел. \(f\) непр. на \(\text{Irr}\) и разрывно на \(\R\).
\end{example}

\begin{corollary}\itemfix
    \begin{itemize}
        \item \(f : E \to \R\)
        \item \(e \subset E \subset X\)
        \item \(\mu e = 0\)
        \item \(E' = E\setminus e\)
        \item \(f\) измеримо на \(E'\)
    \end{itemize}

    Тогда можно так переопределить \(f\) на \(e\), что полученная функция \(\tilde{f}\) будет измерима.
\end{corollary}
\begin{proof}
    Пусть \(\tilde{f}(x) = \begin{cases}
        f(x), x\in E' \\
        \const, x\in e
    \end{cases}\)

    \[E(\tilde{f} < a) = \underbrace{E'(\tilde{f} < a)}_{E'(f < a)} \subset \underbrace{e(\tilde{f} < a)}_{\emptyset \text{ или } e}\]
\end{proof}

\begin{corollary}
    \(f : \ev{a, b} \to \R\) --- монотонна.

    Тогда \(f\) измерима.
\end{corollary}
\begin{proof}
    \(f\) --- непрерывно на \(\ev{a, b}\) за исключением, возможно, счётного множества точек.
\end{proof}

% TODO: #67 решить задачи

\begin{exercise}
    \(f, g : \R \to \R\) --- измеримо.

    \(\varphi : \R^2 \to \R\) --- непрерывна.

    Доказать: \(x \mapsto \varphi(f(x), g(x))\) --- измеримо.
\end{exercise}

\begin{exercise}
    \(f : \R \to \R\) --- измеримо.

    Доказать: \(\R^2 \to \R : (x, y) \mapsto f(x, y)\) --- измеримо.
\end{exercise}

\begin{exercise}
    Доказать, что \(\exists \) измеримая функция \(f:\R \to \R\)

    \(\forall e\subset \R : \lambda e = 0\), если \(f\) непрерывно на \(e\), то полученная \(\tilde{f}\) разрывна всюду.
\end{exercise}

\subsection*{Сходимость почти везде и по мере}

\begin{definition}\itemfix
    %<*свойствопочтивезде>
    \begin{itemize}
        \item \((X, \mathfrak{A}, \mu)\)
        \item \(E\in \mathfrak{A}\)
        \item \(W(x)\) --- высказывание \((x \in X)\)
    \end{itemize}

    \(W(x)\) --- верно при почти всех из \(E\) = почти всюду на \(E\) = почти везде на \(E\) = п.в. \(E\), если:

    \(\exists e\in E : \mu e = 0 \ \ W(x)\) --- истинно при \(x\in E\setminus e\)
    %</свойствопочтивезде>
\end{definition}

\begin{example}
    \(X =\R\), \(W\) = иррационально.
\end{example}

\begin{example}
    \(f_n(x) \xrightarrow[x \to +\infty]{} f(x)\) при п.в. \(x\in E\)
\end{example}

\begin{prop}\itemfix
    \begin{enumerate}
        \item \begin{itemize}
                  \item \(\mu\) --- полная
                  \item \(f_n, f : X \to \overline R\) п.в. \(X\)
                  \item \(f_n\) измеримо
              \end{itemize}

              Тогда \(f\) измеримо.

              \begin{proof}
                  \(f_n \to f\) на \(X'\), где \(e = X\setminus X', \mu e = 0\)

                  \(f\) --- измеримо на \(X\)

                  \(\mu\) --- полная \( \Rightarrow \) \(f\) измеримо на \(X\), т.к. \(X(f < a) = \underbrace{X'(f < a)}_{\text{изм.}} \cup \underbrace{e(f < a)}_{\subset e}\)
              \end{proof}

        \item \? % TODO #68
        \item Пусть \(\forall n \ \ W_n(x)\) истинно при почти всех \(x\).

              Тогда утверждение `` \(\forall n \ \ W_n\) истинно'' --- верно при почти всех \(X\)

              \begin{proof}
                  \(\sphericalangle e_n : \mu(e_n) = 0\). Искомое высказывание верно при \(x\in X \setminus \left( \bigcup\limits_{i = 1}^{+\infty} e_i \right), \mu(\bigcup e_i) = 0\)
              \end{proof}
    \end{enumerate}
\end{prop}

\begin{definition}
    %<*сходимостьпомере>
    \(f_n, f : X \to \overline \R\) --- почти везде конечны.

    \(f_n\) \textbf{сходится к} \(f\) \textbf{по мере} \(\mu\), обозначается \(f_n \longarrow{\mu} f : \forall \varepsilon > 0 \ \ \mu X(|f_n - f| \geq \varepsilon)\xrightarrow[n\to +\infty]{} 0\)
    %</сходимостьпомере>
\end{definition}

\begin{remark}
    \(f_n\) и \(f\) можно изменить на множестве меры \(0\), т.е. предел не задан однозначно.
\end{remark}

\begin{exercise} % TODO: #69 доказать
    \(f_n \longarrow{\mu} f; f_n \longarrow{\mu} g\). Тогда \(f\) и \(g\) эквивалентны.
\end{exercise}

\begin{example}\itemfix
    \begin{enumerate}
        \item \(f_n(x) = \frac{1}{nx}, x > 0, X = \R_{ +}, f \equiv 0\)

              \(f_n \to f\) всюду на \((0, +\infty)\)

              \(f_n \longarrow{\mu} f\)

              \[X(|f_n - f| \geq \varepsilon) = X\left(\frac{1}{nx} \geq \varepsilon\right) = X(x \leq \frac{1}{\varepsilon n})\]
              \[\lambda(\dots ) = \frac{1}{\varepsilon n} \to 0\]

        \item \(f_n(x) : = e^{ -(n - x)^2}, x\in\R\)

              \(f_n(x) \to 0\) при всех \(x\)

              \(f_n(x) \longarrow{} 0\)

              \[\mu(\R(e^{ -(n - x)^2} \geq \varepsilon)) = \const \not\to 0\]

        \item \(n = 2^k + l, 0 \leq l < 2^k, X = [0, 1], \lambda\)

              \[f_n(x) : = \chi_{[\frac{l}{2^k}, \frac{l + 1}{2^k} ]}\]

              \(\lim f_n(x)\) не существует ни при каком \(x\)!

              \[X(f_n \geq \varepsilon) = \frac{1}{2^k} \to 0 \Rightarrow f_n \longarrow{\lambda} 0\]
    \end{enumerate}
\end{example}

\begin{theorem}[Лебега]\itemfix
    %<*лебега>
    \begin{itemize}
        \item \((X, \mathfrak{A}, \mu)\)
        \item \(\mu X\) конечно
        \item \(f_n, f\) --- измеримо, п.в. конечно
        \item \(f_n \to f\) п.в.
    \end{itemize}

    Тогда \(f_n \longarrow{\mu} f\)
    %</лебега>
\end{theorem}

%<*лебегаproof>
\begin{proof}
    Переопределим \(f_n, f\) на множестве меры \(0\), чтобы сходимость была всюду.

    Рассмотрим частный случай: \(\forall x\) последовательность \(f_n(x)\) монотонно убывает к \(0\), то есть \(f\equiv 0\)
    \[X(|f_n| \geq \varepsilon) = X(f_n \geq \varepsilon) \supset X(f_{n+1} \geq \varepsilon)\]
    \[\bigcap X(f_n \geq \varepsilon)\]

    Таким образом, по теореме о непрерывности меры сверху, \(\mu X(f_n \geq \varepsilon) \to 0\)

    Рассмотрим общий случай: \(f_n \to f\), \(\varphi(x) : = \sup\limits_{k \geq n} |f_k(x) - f(x)|\)

    Тогда \(\varphi_n \to 0, \varphi_n \geq 0\) и монотонно, таким образом мы попали в частный случай.

    \[X(|f_n - f| \geq \varepsilon) \subset X(\varphi_n \geq \varepsilon)\]
    \[\mu X(|f_n - f| \geq \varepsilon) \leq \mu X(\varphi_n \geq \varepsilon) \to 0\]
\end{proof}
%</лебегаproof>

\begin{theorem}[Рисс]\itemfix
    %<*рисса>
    \begin{itemize}
        \item \((X, \mathfrak{A}, \mu)\)
        \item \(f_n \longarrow{\mu} f\).
    \end{itemize}
    Тогда \(\exists n_k : f_{n_k} \to f\) почти везде.
    %</рисса>
\end{theorem}

%<*риссаproof>
\begin{proof}
    \[\forall k \ \ \mu X\left(|f_n - f| \geq \frac{1}{k}\right) \to 0\]
    \[\exists n_k : \text{при } n \geq n_k \ \ \mu X\left( |f_n - f| \geq \frac{1}{k} \right) < \frac{1}{2^k}\]
    Можно считать, что \(n_1 < n_2 < n_3\)

    Проверим, что \(f_{n_k} \to f\) почти везде.

    \[E_k : = \bigcup_{j = k}^{+\infty} X\left( |f_{n_j} - f| \geq \frac{1}{j} \right) \quad E = \bigcap E_k\]
    \[E_k \supset E_{k+1} \quad \mu E_k \symref{счётная полуаддитивность}{\leq} \sum_{j = k}^{+\infty} \mu X\left( |f_{n_j} - f| \geq \frac{1}{j} \right) < \sum_{j = k}^{+\infty} \frac{1}{2^j} \leq \frac{2}{2^k} \to 0\]
    \[\mu E_k \to \mu E \Rightarrow \mu E = 0\]

    \ref{счётная полуаддитивность}: по счётной полуаддитивности меры.


    Покажем, что при \(x\not\in E \ \ f_{n_k} \to f\).

    \[x\not\in E \ \ \exists N \ \ x\not\in E_k \text{ при } k > N \ \ |f_{n_k}(x) - f(x)|< \frac{1}{k}\]

    То есть \(f_{n_k}(x) \to f(a)\).

    Т.к. \(\mu E = 0\), искомое выполнено.
\end{proof}
%</риссаproof>

\begin{corollary}
    \(f_n \longarrow{\mu} f \ \ |f_n| \leq g\) почти всюду. Тогда \(|f| \leq g\) почти всюду.
\end{corollary}
\begin{proof}
    \(\exists n_k \ \ f_{n_k} \to f\) почти всюду.

    % TODO: отпарсить
\end{proof}

\[f_n \rightrightarrows f  \Rightarrow f_n(x) \to f(x) \ \  \forall x \Rightarrow f_n \longarrow{} f\]

\begin{theorem}[Егорова]\itemfix
    %<*теоремаегорова>
    \begin{itemize}
        \item \(X, \mathfrak{A}, \mu\)
        \item \(\mu X < +\infty\)
        \item \(f_n, f\) --- почти везде конечно, измеримо
    \end{itemize}

    Тогда
    \[\forall \varepsilon > 0 \ \ \exists e\subset X : \mu e < \varepsilon \quad f_n \xrightrightarrows[X\setminus e]{} f\]
    %</теоремаегорова>
\end{theorem}
\begin{proof}
    Упражнение. % TODO: #70 доказать
\end{proof}

\section*{Интеграл}

\(\sphericalangle (X, \mathfrak{A}, \mu)\) --- зафиксировали.

\begin{definition}[1]\itemfix
    %<*интеграл1>
    \begin{itemize}
        \item \(f = \sum \alpha_k \chi_{E_k}\)
        \item \(E_k\) --- допустимое разбиение
        \item \(\alpha_k \geq 0\)
    \end{itemize}

    \[\int_X f d_{\mu(x)} : = \sum \alpha_k \mu E_k\]

    И пусть \(0\cdot \infty = 0\)
    %</интеграл1>
\end{definition}

\begin{prop}\itemfix
    \begin{enumerate}
        \item Не зависит от представления \(f\) в виде суммы, т.е.:

              \[f = \sum \alpha_k \chi_{E_k} = \sum \alpha'_k \chi_{E'_k} = \sum_{k, j} \alpha_k \chi_{E_k\cap E'_j}\]

              \begin{remark}
                  При \(E_k \cap E'_j \neq \emptyset\) \(\alpha_k = \alpha_j \Rightarrow \) можно писать любое из них.
              \end{remark}

              \[\int f = \sum \alpha_k \mu E_k = \sum_{k, j} \alpha_k \mu(E_k\cap E'_j) = \sum \alpha'_k \mu E'_k\] % TODO: пояснить

        \item \(\underbrace{f}_{\text{ст.}} \leq \underbrace{g}_{\text{ст.}} \Rightarrow \int_X f \leq \int_X g\)
    \end{enumerate}
\end{prop}

\begin{definition}[2]\itemfix
    %<*интеграл2>
    \begin{itemize}
        \item \(f \geq 0\)
        \item \(f\) измеримо
    \end{itemize}
    \[\int_X f d\mu : = \sup_{\substack{g \text{ --- ступ.} \\ 0 \leq g \leq f}} \int g d\mu\]
    %</интеграл2>
\end{definition}

\begin{prop}\itemfix
    \begin{itemize}
        \item Если \(f\) ступенчатая, то определение 2 = определение 1.
        \item \(0 \leq \int_X f \leq +\infty\)
        \item \(g \leq f, f\) --- измеримая, \(g\) --- измеримая \( \Rightarrow \int_X g \leq \int_X f\)
    \end{itemize}
\end{prop}

\begin{definition}[3]\itemfix
    \begin{itemize}
        \item \(f\) измеримо
        \item \(\int f^{+}\) или \(\int f^{-}\) конечен
    \end{itemize}

    \[\int_X f d\mu = \int_X f^{+} d\mu - \int_X f^{-} d\mu\]

    Требование о конечности необходимо для избегания неопределенностей.
\end{definition}

\begin{theorem}[Тонелли]\itemfix
    \begin{itemize}
        \item \(f : \R^{m + n} \to \overline \R\)
        \item \(f \geq 0\)
        \item \(f\) измерима
        \item Записывается как \(f(x, y)\), где \(x\in\R^m, y\in\R^n\)
        \item \(E\subset \R^{m + n}\)
    \end{itemize}

    \begin{obozn}
        \[\forall x\in\R^{m + n} \ \ E_x : = \{y\in\R^n : (x, y)\in E\} \]
        % TODO: #71 иллюстрация
    \end{obozn}

    Тогда:
    \begin{enumerate}
        \item При почти всех \(x\in \R^m\) функция \(y \mapsto f(x,y)\) измерима на \(\R^n\)
        \item Функция \(x \mapsto \int_{E_x} f(x, y) d\lambda_n(y) \geq 0\), измерима и корректно задана.
        \item \[\int_E f(x, y) d\mu = \int_{\R^m} \left( \int_{E_x} f(x, y) d\lambda_n(y) \right) d\lambda_m (x)\]
    \end{enumerate}
\end{theorem}

\begin{remark}
    Неформально говоря, можно разбить \(\R^{m + n}\) на \(\R^{m}\) и \(\R^n\) и интегрировать сначала по одной переменной, потом по другой.
\end{remark}

\end{document}