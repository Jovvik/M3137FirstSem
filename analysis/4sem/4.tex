\documentclass[12pt, a4paper]{article}

%<*preamble>
% Math symbols
\usepackage{amsmath, amsthm, amsfonts, amssymb}
\usepackage{accents}
\usepackage{esvect}
\usepackage{mathrsfs}
\usepackage{mathtools}
\mathtoolsset{showonlyrefs}
\usepackage{cmll}
\usepackage{stmaryrd}
\usepackage{physics}
\usepackage[normalem]{ulem}
\usepackage{ebproof}
\usepackage{extarrows}

% Page layout
\usepackage{geometry, a4wide, parskip, fancyhdr}

% Font, encoding, russian support
\usepackage[russian]{babel}
\usepackage[sb]{libertine}
\usepackage{xltxtra}

% Listings
\usepackage{listings}
\lstset{basicstyle=\ttfamily,breaklines=true}
\setmonofont{Inconsolata}

% Miscellaneous
\usepackage{array}
\usepackage{calc}
\usepackage{caption}
\usepackage{subcaption}
\captionsetup{justification=centering,margin=2cm}
\usepackage{catchfilebetweentags}
\usepackage{enumitem}
\usepackage{etoolbox}
\usepackage{float}
\usepackage{lastpage}
\usepackage{minted}
\usepackage{svg}
\usepackage{wrapfig}
\usepackage{xcolor}
\usepackage[makeroom]{cancel}

\newcolumntype{L}{>{$}l<{$}}
    \newcolumntype{C}{>{$}c<{$}}
\newcolumntype{R}{>{$}r<{$}}

% Footnotes
\usepackage[hang]{footmisc}
\setlength{\footnotemargin}{2mm}
\makeatletter
\def\blfootnote{\gdef\@thefnmark{}\@footnotetext}
\makeatother

% References
\usepackage{hyperref}
\hypersetup{
    colorlinks,
    linkcolor={blue!80!black},
    citecolor={blue!80!black},
    urlcolor={blue!80!black},
}

% tikz
\usepackage{tikz}
\usepackage{tikz-cd}
\usetikzlibrary{arrows.meta}
\usetikzlibrary{decorations.pathmorphing}
\usetikzlibrary{calc}
\usetikzlibrary{patterns}
\usepackage{pgfplots}
\pgfplotsset{width=10cm,compat=1.9}
\newcommand\irregularcircle[2]{% radius, irregularity
    \pgfextra {\pgfmathsetmacro\len{(#1)+rand*(#2)}}
    +(0:\len pt)
    \foreach \a in {10,20,...,350}{
            \pgfextra {\pgfmathsetmacro\len{(#1)+rand*(#2)}}
            -- +(\a:\len pt)
        } -- cycle
}

\providetoggle{useproofs}
\settoggle{useproofs}{false}

\pagestyle{fancy}
\lfoot{M3137y2019}
\cfoot{}
\rhead{стр. \thepage\ из \pageref*{LastPage}}

\newcommand{\R}{\mathbb{R}}
\newcommand{\Q}{\mathbb{Q}}
\newcommand{\Z}{\mathbb{Z}}
\newcommand{\B}{\mathbb{B}}
\newcommand{\N}{\mathbb{N}}
\renewcommand{\Re}{\mathfrak{R}}
\renewcommand{\Im}{\mathfrak{I}}

\newcommand{\const}{\text{const}}
\newcommand{\cond}{\text{cond}}

\newcommand{\teormin}{\textcolor{red}{!}\ }

\DeclareMathOperator*{\xor}{\oplus}
\DeclareMathOperator*{\equ}{\sim}
\DeclareMathOperator{\sign}{\text{sign}}
\DeclareMathOperator{\Sym}{\text{Sym}}
\DeclareMathOperator{\Asym}{\text{Asym}}

\DeclarePairedDelimiter{\ceil}{\lceil}{\rceil}

% godel
\newbox\gnBoxA
\newdimen\gnCornerHgt
\setbox\gnBoxA=\hbox{$\ulcorner$}
\global\gnCornerHgt=\ht\gnBoxA
\newdimen\gnArgHgt
\def\godel #1{%
    \setbox\gnBoxA=\hbox{$#1$}%
    \gnArgHgt=\ht\gnBoxA%
    \ifnum     \gnArgHgt<\gnCornerHgt \gnArgHgt=0pt%
    \else \advance \gnArgHgt by -\gnCornerHgt%
    \fi \raise\gnArgHgt\hbox{$\ulcorner$} \box\gnBoxA %
    \raise\gnArgHgt\hbox{$\urcorner$}}

% \theoremstyle{plain}

\theoremstyle{definition}
\newtheorem{theorem}{Теорема}
\newtheorem*{definition}{Определение}
\newtheorem{axiom}{Аксиома}
\newtheorem*{axiom*}{Аксиома}
\newtheorem{lemma}{Лемма}

\theoremstyle{remark}
\newtheorem*{remark}{Примечание}
\newtheorem*{exercise}{Упражнение}
\newtheorem{corollary}{Следствие}[theorem]
\newtheorem*{statement}{Утверждение}
\newtheorem*{corollary*}{Следствие}
\newtheorem*{example}{Пример}
\newtheorem{observation}{Наблюдение}
\newtheorem*{prop}{Свойства}
\newtheorem*{obozn}{Обозначение}

% subtheorem
\makeatletter
\newenvironment{subtheorem}[1]{%
    \def\subtheoremcounter{#1}%
    \refstepcounter{#1}%
    \protected@edef\theparentnumber{\csname the#1\endcsname}%
    \setcounter{parentnumber}{\value{#1}}%
    \setcounter{#1}{0}%
    \expandafter\def\csname the#1\endcsname{\theparentnumber.\Alph{#1}}%
    \ignorespaces
}{%
    \setcounter{\subtheoremcounter}{\value{parentnumber}}%
    \ignorespacesafterend
}
\makeatother
\newcounter{parentnumber}

\newtheorem{manualtheoreminner}{Теорема}
\newenvironment{manualtheorem}[1]{%
    \renewcommand\themanualtheoreminner{#1}%
    \manualtheoreminner
}{\endmanualtheoreminner}

\newcommand{\dbltilde}[1]{\accentset{\approx}{#1}}
\newcommand{\intt}{\int\!}

% magical thing that fixes paragraphs
\makeatletter
\patchcmd{\CatchFBT@Fin@l}{\endlinechar\m@ne}{}
{}{\typeout{Unsuccessful patch!}}
\makeatother

\newcommand{\get}[2]{
    \ExecuteMetaData[#1]{#2}
}

\newcommand{\getproof}[2]{
    \iftoggle{useproofs}{\ExecuteMetaData[#1]{#2proof}}{}
}

\newcommand{\getwithproof}[2]{
    \get{#1}{#2}
    \getproof{#1}{#2}
}

\newcommand{\import}[3]{
    \subsection{#1}
    \getwithproof{#2}{#3}
}

\newcommand{\given}[1]{
    Дано выше. (\ref{#1}, стр. \pageref{#1})
}

\renewcommand{\ker}{\text{Ker }}
\newcommand{\im}{\text{Im }}
\renewcommand{\grad}{\text{grad}}
\newcommand{\rg}{\text{rg}}
\newcommand{\defeq}{\stackrel{\text{def}}{=}}
\newcommand{\defeqfor}[1]{\stackrel{\text{def } #1}{=}}
\newcommand{\itemfix}{\leavevmode\makeatletter\makeatother}
\newcommand{\?}{\textcolor{red}{???}}
\renewcommand{\emptyset}{\varnothing}
\newcommand{\longarrow}[1]{\xRightarrow[#1]{\qquad}}
\DeclareMathOperator*{\esup}{\text{ess sup}}
\newcommand\smallO{
    \mathchoice
    {{\scriptstyle\mathcal{O}}}% \displaystyle
    {{\scriptstyle\mathcal{O}}}% \textstyle
    {{\scriptscriptstyle\mathcal{O}}}% \scriptstyle
    {\scalebox{.6}{$\scriptscriptstyle\mathcal{O}$}}%\scriptscriptstyle
}
\renewcommand{\div}{\text{div}\ }
\newcommand{\rot}{\text{rot}\ }
\newcommand{\cov}{\text{cov}}

\makeatletter
\newcommand{\oplabel}[1]{\refstepcounter{equation}(\theequation\ltx@label{#1})}
\makeatother

\newcommand{\symref}[2]{\stackrel{\oplabel{#1}}{#2}}
\newcommand{\symrefeq}[1]{\symref{#1}{=}}

% xrightrightarrows
\makeatletter
\newcommand*{\relrelbarsep}{.386ex}
\newcommand*{\relrelbar}{%
    \mathrel{%
        \mathpalette\@relrelbar\relrelbarsep
    }%
}
\newcommand*{\@relrelbar}[2]{%
    \raise#2\hbox to 0pt{$\m@th#1\relbar$\hss}%
    \lower#2\hbox{$\m@th#1\relbar$}%
}
\providecommand*{\rightrightarrowsfill@}{%
    \arrowfill@\relrelbar\relrelbar\rightrightarrows
}
\providecommand*{\leftleftarrowsfill@}{%
    \arrowfill@\leftleftarrows\relrelbar\relrelbar
}
\providecommand*{\xrightrightarrows}[2][]{%
    \ext@arrow 0359\rightrightarrowsfill@{#1}{#2}%
}
\providecommand*{\xleftleftarrows}[2][]{%
    \ext@arrow 3095\leftleftarrowsfill@{#1}{#2}%
}

\allowdisplaybreaks

\newcommand{\unfinished}{\textcolor{red}{Не дописано}}

% Reproducible pdf builds 
\special{pdf:trailerid [
<00112233445566778899aabbccddeeff>
<00112233445566778899aabbccddeeff>
]}
%</preamble>


\lhead{Математический анализ}
\cfoot{}
\rfoot{1.3.2021}

\begin{document}

\begin{theorem}[об абсолютной непрерывности интеграла]\itemfix
    \begin{itemize}
        \item \((X, \mathfrak{A}, \mu)\) --- пространство с мерой
        \item \(f : X \to \overline \R\)
        \item \(f\) суммируемо
    \end{itemize}

    Тогда \(\forall \varepsilon > 0 \ \ \exists \delta > 0 \ \ \forall E \text{ --- изм. } \mu E < \delta : \left|\int_E f\right|< \varepsilon\)
\end{theorem}
\begin{corollary}
    \?
\end{corollary}
\begin{proof}
    \[X_n : = X(|f| \geq n)\]
    \[X_n \subset X_{n+1} \subset \dots \quad \mu\left( \bigcap X_n \right) \symrefeq{почти везде конечна} 0\]
    \begin{equation}
        \forall \varepsilon > 0 \ \ \exists n_\varepsilon \ \ \int_{X_{n_\varepsilon}} |f| < \frac{\varepsilon}{2} \label{непрерывность сверху меры}
    \end{equation}

    Пусть \(\delta : = \frac{\varepsilon}{2n_\varepsilon} \). Тогда при \(\mu E < \delta\):
    \[|\int_E f| \leq \int_E |f| = \int_{E\cap X_{n_\varepsilon}} |f| + \int_{E\cap X_{n_\varepsilon}^c} \leq \int_{X_{n_\varepsilon}} |f| + \int_{E\cap X_{n_\varepsilon}^c} n_\varepsilon < \frac{\varepsilon}{2} + \underbrace{\mu E}_{\delta} \cdot n_\varepsilon \leq \varepsilon\]

    \begin{itemize}
        \item \eqref{непрерывность сверху меры}: По непрерывности сверху меры \(A \mapsto \int_A |f| d\mu\)
        \item \eqref{почти везде конечна}: Т.к. \(f\) почти везде конечна.
    \end{itemize}
\end{proof}

\begin{remark}
    Следующие два свойства не эквивалентны:
    \begin{enumerate}
        \item \(f_n \xRightarrow{\mu} f \ \ \forall \varepsilon > 0 \ \ \mu X(|f_n - f| > \varepsilon) \to 0\)
        \item \(\int_X |f_n - f| d\mu \to 0\)
    \end{enumerate}

    Из 1 не следует 2: пусть \((X, \mathfrak{A}, \mu) = (\R, \mathfrak{M}, \lambda), f_n = \frac{1}{nx}\). Тогда \(f_n \xRightarrow{\lambda} 0\), но \(\int |f_n - f|\) \?
\end{remark}

\begin{theorem}[Лебега]\itemfix
    \begin{itemize}
        \item \((X, \mathfrak{A}, \mu)\) --- пространство с мерой
        \item \(f_n, f\) --- измеримо и почти везде конечно
        \item \(f_n \xRightarrow{\mu} f\)
        \item \(\exists g\) : \begin{enumerate}
                  \item \(\forall n \ \ |f_n| \symref{``1''}{\leq} g\) почти везде
                  \item  \(g\) --- суммируемо на \(X\)
              \end{enumerate}
    \end{itemize}

    Тогда: \(f_n, f\) --- суммируемы и \(\int_X |f_n - f| d\mu \xrightarrow{n \to +\infty} 0\), и тем более \(\int_X f_n d\mu \to \int_X f d\mu\)
\end{theorem}
\begin{proof}
    \(f_n\) --- суммируемы в силу неравенства \eqref{``1''}, \(f\) суммируемо в силу следствия теоремы Рисса, тем более \(|\int_X f_n - \int_X f| \leq  \int_X |f_n - f| \to 0\)

    \begin{enumerate}
        \item \(\mu X < +\infty\)

              Зафиксируем \(\varepsilon\). \(X_n : = X(|f_n - f| > \varepsilon)\)

              \(f_n \Rightarrow f\), т.е. \(\mu X_n \to 0\)

              \[\int_X |f_n - f| = \int_{X_n} + \int_{X_n^c} = \underbrace{\int_{X_n} 2g}_{\xrightarrow[\text{сл. т. об абс. непр.}]{n \to +\infty} 0} + \int_{X_n^c} \varepsilon d\mu < \varepsilon + \varepsilon \mu X\]

        \item \(\mu X = +\infty\)
    \end{enumerate}
\end{proof}

\textcolor{red}{Не дописано}

\end{document}