\chapter{22 марта}

\subsection{Сферические координаты в \(\R^m\)}

Координаты задаются \(r, \varphi_1, \varphi_2 \dots \varphi_{m - 1}\). Зададим их по индукции:

\begin{itemize}
    \item \(\varphi_1\) --- угол между \(\overline e_1\) и \(\overline{OX}\in[0, \pi]\)
    \item \(\varphi_2\) --- угол между \(\overline e_2\) и \(P_{2_{(e_2 \dots e_n)}}(x)\in[0, \pi]\)
    \item \(\vdots\)
    \item \(\varphi_{m - 1}\) --- полярный угол в \(\R^2\)
\end{itemize}

\begin{align*}
    x_1       & = r \cos \varphi_1                                                 \\
    x_2       & = r \sin \varphi_1 \cos \varphi_2                                  \\
    x_3       & = r \sin \varphi_1 \sin \varphi_2 \cos \varphi_3                   \\
    \vdots                                                                         \\
    x_{n - 1} & = r \sin \varphi_1 \dots \sin \varphi_{m - 2} \cos \varphi_{m - 1} \\
    x_n       & = r \sin \varphi_1 \dots \sin \varphi_{m - 2} \sin \varphi_{m - 1}
\end{align*}

\[J = r^{m - 1} \sin^{m - 2}\varphi_1 \sin^{m - 3}\varphi_2 \dots \sin \varphi_{m - 2}\]

\begin{remark}
    В \(\R^3\) ``географические'' координаты имеют якобиан \(J = r^2 \cos \psi\)
\end{remark}

Поймём, почему якобиан именно такой. Можно его посчитать руками, но это трудно.

\begin{itemize}
    \item [1 шаг] \begin{align*}
              x_m       & = \rho_{m - 1} \sin \varphi_{m - 1} \\
              x_{m - 1} & = \rho_{m - 1} \cos \varphi_{m - 1}
          \end{align*}
          \[(x_1 \dots x_m) \rightsquigarrow (x_1 \dots x_{m - 2}, \rho_{m - 1}, \varphi_{m - 1})\]
          \[J = \begin{vmatrix} E & 0 \\ 0 & J_2 \end{vmatrix} = \rho_{m - 1}\]
    \item [2 шаг] \begin{align*}
              \rho_{m - 1} & = \rho_{m - 2} \sin \varphi_{m - 2} \\
              x_{m - 2}    & = \rho_{m - 2} \cos \varphi_{m - 2}
          \end{align*}
          \[(x_1 \dots x_{m - 2}, \rho_{m - 1}, \varphi_{m - 1}) \rightsquigarrow (x_1 \dots x_{m - 3}, \rho_{m - 2}, \varphi_{m - 2}, \varphi_{m - 1})\]

    \item [последний шаг]
          \[(x_1 \rho_2, \varphi_2 \dots \varphi_{m - 1}) \rightsquigarrow (r, \varphi_1 \dots \varphi_{m - 1})\]
          \begin{align*}
              \rho_2 & = r \sin \varphi_1 \\
              x_1    & = r \cos \varphi_1
          \end{align*}
\end{itemize}

\begin{align*}
    \lambda_m(\Omega) & = \int_\Omega 1 d\lambda_m                                                                                                                                                                                                     \\
                      & \stackrel{\text{1 шаг}}{=} \int_{\Omega_1} \rho_{m - 1}                                                                                                                                                                        \\
                      & \stackrel{\text{2 шаг}}{=} \int_{\Omega_2} \underbrace{\rho_{m - 2} \sin \varphi_{m - 2}}_{\text{замена } \rho_{m - 1}} \cdot \underbrace{\rho_{m_2}}_J                                                                        \\
                      & \stackrel{\text{3 шаг}}{=} \int_{\Omega_3} \underbrace{\rho^2_{m - 3} \sin^2 \varphi_{m - 3}}_{\text{замена } \rho_{m - 2}^2} \cdot \underbrace{\sin \varphi_{m - 2}}_{\text{с прошлого шага}} \cdot \underbrace{\rho_{m_3}}_J \\
                      & = \dots                                                                                                                                                                                                                        \\
                      & = \int_{\Omega_{m - 1}} r^{m - 1} \sin^{m - 2}\varphi_1 \sin^{m - 3}\varphi_2 \dots \sin \varphi_{m - 2} d\lambda
\end{align*}

Тогда по теореме о единственности плотности искомое верно.

\section{Произведение мер}

\(\sphericalangle (X, \mathfrak{A}, \mu)\), \((Y, \mathfrak{B}, \nu)\) --- пространства с мерой

\begin{lemma}
    \(\mathfrak{A}, \mathfrak{B}\) --- полукольца \( \Rightarrow \mathfrak{A} \times \mathfrak{B} = \{A \times B \subset X \times Y : A\in \mathfrak{A}, B \in \mathfrak{B}\}\) --- полукольцо.
\end{lemma}
\begin{proof}
    Тривиально. % TODO: доказать
\end{proof}

\begin{notation}
    \(\mathcal{P} = \mathfrak{A} \times \mathfrak{B}\) --- называем \textbf{измеримыми прямоугольниками}.

    \(m_0(A \times B) = \mu(A) \cdot \nu(B)\), при этом \(0 \cdot \infty\) принимаем за \(0\).
\end{notation}
\begin{theorem}\itemfix
    %<*опроизведениимер>
    \begin{enumerate}
        \item \(m_0\) --- мера на \(\mathcal{P}\)
        \item \(\mu, \nu\) --- \(\sigma\)-конечны \( \Rightarrow m_0\) тоже \(\sigma\)-конечно\footnote{Т.е. пространство можно представить в виде счётного объединения множеств конечной меры.}.
    \end{enumerate}
    %</опроизведениимер>
\end{theorem}
%<*опроизведениимерproof>
\begin{proof}\itemfix
    \begin{enumerate}
        \item Проверим счётную аддитивность \(m_0\), т.е. \(m_0 P = \sum_{k = 1}^{+\infty} m_0 P_k\)\footnote{Прочие суммы/объединения также счётны в рамках данного доказательства.}, если \(A \times B = P = \bigsqcup P_k\), где \(P_k = A_k \times B_k\)

              Заметим, что \(\chi_{A \times B}(x, y) = \chi_A(x) \cdot \chi_B(y)\).

              Тогда \(\chi_P = \sum \chi_{P_k}\), где \(\forall x\in X, y\in Y \ \ \chi_A(x) \chi_B(y) = \sum \chi_{A_k}(x) \chi_{B_k}(y)\)

              Слева измеримая функция, справа --- неотрицательный ряд \(\Rightarrow\) можем интегрировать.

              Проинтегрируем по \(y\) по мере \(\nu\) по пространству \(Y\):
              \[\chi_A(x) \nu B = \sum \chi_{A_k}(x) \cdot \nu B_k\]

              Проинтегрируем по \(x\) по мере \(\mu\) по пространству \(X\):
              \[\mu A \nu B = \sum \mu A_k \nu B_k\]
              Это и есть искомое.

        \item Очевидно, т.к.:
              \begin{itemize}
                  \item \(\mu\) \(\sigma\)-конечно \( \Rightarrow X = \bigcup X_k, \mu X_k\) --- конечно \(\forall k\)
                  \item \(\nu\) \(\sigma\)-конечно \( \Rightarrow Y = \bigcup Y_n, \nu Y_n\) --- конечно \(\forall n\)
              \end{itemize}

              Тогда \(X \times Y = \bigcup X_k \times Y_n, m_0(X_k \times Y_n) = \mu X_k \nu Y_n\). Конечное произведение конечных конечно, поэтому \(m_0\) \(\sigma\)-конечно.
    \end{enumerate}
\end{proof}
%</опроизведениимерproof>

\begin{definition}\itemfix
    %<*произведениемер>
    \begin{itemize}
        \item \(\sphericalangle (X, \mathfrak{A}, \mu)\), \((Y, \mathfrak{B}, \nu)\) --- пространства с мерой
        \item \(\mu, \nu\) \(\sigma\)-конечны
    \end{itemize}

    Пусть \(m\) --- лебеговское продолжение меры \(m_0\) на \(\sigma\)-алгебру, которую будем обозначать \(\mathfrak{A} \otimes \mathfrak{B}\)\footnote{\(\otimes\) --- не тензорное произведение}

    \begin{notation}
        \(m = \mu \times \nu\)
    \end{notation}

    \((X \times Y, \mathfrak{A} \otimes \mathfrak{B}, \mu \times \nu)\) --- \textbf{произведение пространств с мерой} \((X, \mathfrak{A}, \mu)\) и \((Y, \mathfrak{B}, \nu)\)
    %</произведениемер>
\end{definition}

\begin{remark}\itemfix
    \begin{itemize}
        \item Это произведение ассоциативно.
        \item \(\sigma\)-конечность нужна для единственности произведения, которая выполняется по теореме о продолжении меры.
    \end{itemize}
\end{remark}

\begin{theorem}
    \(\lambda_m \times \lambda_n = \lambda_{m + n}\)
\end{theorem}
\begin{proof}
    Не будет. % и не надо пытаться доказать, это страшно.
\end{proof}

\begin{definition}
    \(X, Y\) --- множества, \(C \subset X \times Y\)

    \[\forall x\in X \ \ C_x : = \{y\in Y : (x, y)\in C\}\]
    \[\forall y\in Y \ \ C^y : = \{x\in X : (x, y)\in C\}\]

    \(C_x, C^y\) называется \textbf{сечением}.
\end{definition}

\begin{remark}
    \[\left( \bigcup_\alpha C_\alpha \right)_x = \bigcup (C_\alpha)_x \quad \left( \bigcap C_\alpha \right)_x = \bigcap (C_\alpha)_x \quad (C \setminus C')_x = C_x \setminus C'_x \]
\end{remark}

\begin{theorem}[принцип Кавальери]\itemfix
    %<*кавальери>
    \footnote{Кавальери имеет к этой теореме косвенное отношение, т.к. он жил за пару веков до появления теории меры.}
    \label{принцип Кавальери}
    \begin{itemize}
        \item \((X, \mathfrak{A}, \mu)\)
        \item \((Y, \mathfrak{B}, \nu)\)
        \item \(\mu, \nu\) --- \(\sigma\)-конечны.
        \item \(\mu, \nu\) --- полные.
        \item \(m = \mu \times \nu\)
        \item \(C \in \mathfrak{A} \otimes \mathfrak{B}\)
    \end{itemize}

    Тогда:
    \begin{enumerate}
        \item \(C_x \in \mathfrak{B}\) при почти всех \(x\)
        \item \(x \mapsto \nu(C_x)\) --- измеримая\footnote{Функция задана при почти всех \(X\); она равна п.в. некоторой измеримой функции, заданной всюду.} функция на \(X\)
        \item \(mC = \int_X \nu(C_x) d \mu(x)\)
    \end{enumerate}

    Аналогичное верно для \(C^y\).
    %</кавальери>
\end{theorem}

\begin{example}
    \? % TODO #93
\end{example}

%<*кавальериproof>
\begin{proof}
    Пусть \(\mathfrak{D}\) --- система множеств, для которых выполнено 1.-3.

    \begin{enumerate}
        \item \(C = A \times B\), где \(A\) и \(B\) измеримы в соответствующих пространствах \(\Rightarrow C \in \mathfrak{D}\), так как:
              \begin{enumerate}
                  \item \(C_x = \begin{cases}
                            \emptyset, x\notin A \\
                            B, x\in A
                        \end{cases}\) и оба случая очевидно \(\in \mathfrak{B}\)
                  \item \(x \mapsto \nu(C_x)\) --- функция \(\nu B \cdot \chi_A\)
                  \item \(\int \nu(C_x) d\mu = \int_X \nu B \cdot \chi_A d \mu = \nu B \cdot \mu A = mC\)
              \end{enumerate}

        \item \(E_i \in \mathfrak{D}\), дизъюнктны \( \xRightarrow{?} \bigsqcup E_i\in \mathfrak{D}\). Обозначим \(E = \bigsqcup E_i\)

              \(E_i \in \mathfrak{D} \Rightarrow (E_i)_x\) измеримы почти везде \( \Rightarrow \) при почти всех \(x\) все \((E_i)_x\) измеримы.

              Тогда при этих \(x\) \(E_x = \bigsqcup (E_i)_x \in \mathfrak{B}\) по определению \(\sigma\)-алгебры --- это 1.

              \(\nu E_x = \sum \underbrace{\nu(E_i)_x}_{\substack{\text{измеримая} \\ \text{функция}}} \Rightarrow\) Функция \(x \mapsto \nu E_x\) измерима --- это 2.

              \(\int_X \nu E_x d\mu = \sum_i \int_X \nu(E_i) x = \sum_i m E_i = m E\) --- это 3.

        \item \(E_i \in \mathfrak{D}, E_1 \supset E_2 \supset \dots , E = \bigcap_i E_i, \mu E_i < +\infty\). Тогда \(E\in \mathfrak{D}\).

              \(\int_X \nu(E_i)_x d\mu = m E_i < +\infty \Rightarrow \nu(E_i)_x\) --- конечно при почти всех \(x\).

              \(\forall x\) верно \((E_1)_x \supset (E_2)_x \supset \dots , E_x = \bigcap (E_i)_x\)

              Тогда \(E_x\) измеримо п.в. \textit{(это 1.)} и \(\lim_{i \to +\infty} \nu(E_i)_x = \nu E_x\) при п.в. \(x\) --- непрерывность сверху \(\nu\).

              Таким образом, \(x \mapsto \nu E_x\) измерима --- это 2.

              \(\int_x \nu E_x d\mu = \lim \int_X \nu(E_i)_x d\mu = \lim m E_i = m E\) --- это 3.

              По теореме Лебега о предельном переходе под знаком интеграла
              \(|\nu(E_i) x| \leq \nu(E_i)x\) суммируемо.

              Итого: Если \(A_{ij} \in \mathcal{P} = \mathfrak{A} \times \mathfrak{B}\), то \(\bigcap \bigcup A_{ij} \in \mathfrak{D}\). Строго говоря, мы это не доказали, т.к. ещё нужно упомянуть процесс дизъюнктнизации в полукольце и то, что пересечение множеств лежит в полукольце, следовательно любое пересечение можно свести к тому, которое мы рассматривали.

        \item \(E \subset X \times Y, m E = 0 \Rightarrow E \in \mathfrak{D}\)

              \(m E = \inf \left\{\sum m_0 P_k : E \subset \bigcup P_k, P_k \in \mathcal{P}\right\} \) --- из пункта 5 теоремы о лебеговском продолжении.

              \(\exists\) множество \(H\) вида \(\bigcap_l \bigcup_k P_{kl}\), т.е. пересечение аппроксимаций. По пункту 3 \(H \in \mathfrak{D}\). При этом \(E \subset H, m H = m E = 0\).

              \(0 = mH = \int_X \underbrace{\nu H_x}_{ \geq 0} d\mu \Rightarrow \nu H_x = 0\) про почти всех \(x\).

              \(E_x \subset H_x, \nu\) --- полная \( \Rightarrow E_x\) --- измеримо при почти всех \(x\) --- это 1 и \(\nu E_x = 0\) почти везде, это 2.

              \(\int \nu E_x d\mu = 0 = m E\) --- это 3.

        \item \(C\) --- \(m\)-измеримо, \(mC < +\infty\). Тогда \(C\in \mathfrak{D}\).

              \(C = H \setminus e\), где \(H\) имеет вид \(\bigcap \bigcup P_{kl}, me = 0\). Почему? Из предыдущих соображений \(C \subset H\), а нулевая мера \(H \setminus C\) следует из того, что мера \(C\) конечна. \textcolor{red}{Как оно следует?}

              \(mC = mH - 0 = mH\)

              \begin{enumerate}
                  \item \(C_x = H_x \setminus e_x\) --- оба ``слагаемых'' измеримы при почти всех \(x\), т.к. \(H_x\) по третьему пункту \(\in \mathfrak{B}\), а \(e_x\) измеримы по полноте \(\nu\). В силу замкнутости по вычитанию \(C_x \in \mathfrak{B}\) п.в.
                  \item \(\nu e_x = 0\) при почти всех \(x\) \( \Rightarrow \nu C_x = \nu H_x - \nu E_x = \nu H_x\) п.в. \(\Rightarrow \) измеримо.
                  \item \(\int_X \nu C_x d\mu = \int_X \nu H_x d\mu = mH = mC\)
              \end{enumerate}

        \item \(C\) --- произвольное измеримое множество в \(X \times Y \Rightarrow C \in \mathfrak{D}\)

              \(X = \bigsqcup X_k, \mu X_k < +\infty, Y = \bigsqcup Y_j, \nu Y_j < +\infty\) по полноте обеих мер.

              \(C = \bigsqcup (\underbrace{C \cap (X_k \times Y_j)}_{m(\dots) < +\infty})\), тогда по пункту 5 все элементы объединения \(\in \mathfrak{D}\) и по пункту 2 объединение лежит в \(\mathfrak{D}\).
    \end{enumerate}
\end{proof}
%</кавальериproof>

%<*кавальериcorollary>
\begin{corollary}
    \(C\) измеримо в \(X \times Y\). Пусть \(P_1(C) = \{x\in X, C_x \neq \emptyset\}\) --- проекция~\(C\) на~\(X\).

    Если \(P_1(C)\) измеримо, то:
    \[mC = \int_{P_1(C)} \nu(C_x) d\mu\]

    Аналогично для проекции на \(y\).
\end{corollary}
%</кавальериcorollary>
%<*кавальериcorollaryproof>
\begin{proof}
    При \(x\notin P_1(C) \ \ \nu(C_x) = 0\)
\end{proof}

%</кавальериcorollaryproof>
\begin{remark}\itemfix
    \begin{enumerate}
        \item \(C\) измеримо \(\nRightarrow P_1(C)\) измеримо.

              \begin{example}
                  Пусть \(C = (\) неизмеримое множество в \(X) \times \{y\}\), где \(y\) --- фиксированный элемент в \(Y\). Тогда \(C\) измеримо в \(X \times Y\) и имеет меру \(0\).
              \end{example}
        \item \(C\) измеримо \(\nRightarrow \forall x \ \ C_x\) измеримо.

              \begin{example}
                  Пусть \(C = \{\tilde{x}\} \times (\) неизмеримое множество в \(y)\), где \(\tilde{x}\) --- фиксированный элемент в \(X\). Тогда \(C\) измеримо в \(X \times Y\) и имеет меру \(0\), но при этом \(C_{\tilde{x}}\) неизмеримо.
              \end{example}
        \item \(\forall x, \forall y \ \ C_x, C^y\) измеримо \(\nRightarrow C\) измеримо.

              Пример Серпинского.
    \end{enumerate}
\end{remark}
