\chapter{5 апреля}

\begin{definition}
    \(M \subset \R^3\) --- \textbf{кусочно-гладкое двумерное многообразие}, если \(M\) --- конечное объединение:
    \begin{itemize}
        \item Простых гладких многообразий \(M_i\)
        \item Гладких кривых
        \item Точек
    \end{itemize}
\end{definition}

\begin{definition}
    \(E \subset M\) измеримо, если \(E \cap M_i\) измеримо.

    \[S(E) : = \sum_i S(E \cap M_i)\]
    \[\int_E fds : = \sum_i \int_{E\cap M_i} f ds\]
\end{definition}

\subsection{Поверхностный интеграл II рода}

\begin{obozn}
    Будем называть простое двумерное гладкое многообразие в \(\R^3\) поверхностью.
\end{obozn}

\begin{definition}
    \textbf{Сторона поверхности} есть непрерывное семейство единичных нормалей к этой поверхности.
\end{definition}

\begin{definition}\itemfix
    \begin{itemize}
        \item \(M\) --- поверхность в \(\R^3\)
        \item \(n_0\) --- сторона % чья?
        \item \(\gamma\) --- контур \textit{(петля)} в \(M\), ориентированная
    \end{itemize}

    Говорят, что сторона поверхности \(n_0\) согласована с ориентацией \(\gamma\), если:
    \[(\gamma' \times N_{\text{внутр.}}) \parallel n_0\]
    Т.е. если ориентация \(\gamma\) задаёт сторону \(n_0\).
\end{definition}

\begin{definition}[интеграл II рода]\itemfix
    \begin{itemize}
        \item \(M\) --- простое двумерное гладкое многообразие
        \item \(n_0\) --- сторона \(M\)
        \item \(F : M \to \R^3\) --- непрерывное векторное поле
    \end{itemize}
    Тогда \(\int_M \ev{F, n_0} dS\) --- интеграл II рода векторного поля \(F\) по поверхности \(M\).
\end{definition}

\begin{remark}\itemfix
    \begin{itemize}
        \item Смена стороны = смена знака
        \item Не зависит от параметризации
        \item \(F = (P, Q, R)\), тогда интеграл обозначается \(\iint P dy dz + Q dz dx + R dx dy\)
        \item \(\Phi\) - параметризация, \(n = \Phi'_u \times \Phi'_v \rightsquigarrow n_0\) - нормирование

              Пусть \(\Phi(u, v) = (x(u, v), y(u, v), z(u, v))\)

              \begin{align*}
                  \int_M \ev{F, n_0} ds & = \int_O \ev{F, \frac{\Phi'_u \times \Phi'_v}{|\Phi'_u \times \Phi'_v|}} |\Phi'_u \times \Phi'_v| du dv                  \\
                                        & = \int_O \underbrace{\ev{F, \Phi'_u \times \Phi'_v}}_{\substack{\text{смешанное}                                         \\ \text{произведение}:~\eqref{смешанное произведение}}} du dv \\
                                        & = \int_O P \cdot \begin{vmatrix} y'_u & y'_v \\ z'_u & z'_v \end{vmatrix} + Q \cdot \begin{vmatrix} z'_u & z'_v \\ x'_u & x'_v \end{vmatrix} + R \cdot \begin{vmatrix} x'_u & x'_v \\ y'_u & y'_v \end{vmatrix} du dv
              \end{align*}

              \begin{equation}
                  \ev{F, \Phi'_u \times \Phi'_v} = \begin{vmatrix} P & x'_u & x'_v \\ Q & y'_u & y'_v \\ R & z'_u & z'_v \end{vmatrix} \label{смешанное произведение}
              \end{equation}

              Сторона поверхности учитывается в порядке переменных \(u, v\).
    \end{itemize}
\end{remark}

\begin{example}
    Рассмотрим график функции \(z(x, y)\) над областью \(G\) по верхней стороне.
    \[n_0 = \begin{pmatrix} - \frac{z'_x}{\sqrt{1 + z_x^{\prime 2} + z_y^{\prime 2}}} & - \frac{z'_y}{\sqrt{1 + z_x^{\prime 2} + z_y^{\prime 2}}} & \frac{1}{\sqrt{1 + z_x^{\prime 2} + z_y^{\prime 2}}} \end{pmatrix} \]
    \begin{align*}
        \int_{\Gamma_z} R dx dy & = \int_{\Gamma_z} 0 dy dz + 0 dz dx + R(x, y, z) dx dy                                           \\
                                & \defeq \iint_{\Gamma_z} R(x, y, z) \cdot \frac{1}{\sqrt{1 + z_x^{\prime 2} + z_y^{\prime 2}}} dS \\
                                & = \iint_G R(x, y, z(x, y)) dx dy                                                                 \\
                                & = \iint_G R dx dy
    \end{align*}

    Т.е. этот интеграл II рода равен интегралу по проекции.
\end{example}

\begin{corollary*}
    \(V \subset \R^3, M = \partial V\) --- гладкая двумерная поверхность, \(n_0\) --- внешняя нормаль.

    \[\lambda_3 V = \iint_{\partial V} z dx dy = \frac{1}{3} \iint_{\partial V} x dy dz + y dz dx + z dx dy\]
\end{corollary*}

\begin{corollary*}
    \(\Omega\) --- гладкая кривая в \(\R^2\), \(M\) --- цилиндр над \(\Omega\), т.е. \(M = \Omega \times [z_0, z_1]\)

    Тогда \(\int_M R dx dy = 0\) по любой стороне.
\end{corollary*}
\begin{proof}
    \(n_0 \perp (0, 0, R)\)
\end{proof}

\section{Ряды Фурье}

\subsection{Пространства \(L^p\)}

\begin{enumerate}
    \item \begin{itemize}
              \item \((X, \mathfrak{A}, \mu)\) --- пространство с мерой
              \item \(f : X \to \mathbb C\), т.е. \(x = f(x) = u(x) + iv(x), u = \Re f, v = \Im f\)
          \end{itemize}

          \(f\) \textbf{измеримо}, если \(u\) и \(v\) измеримы\footnote{Или измеримы почти везде.}.

          \(f\) \textbf{суммируемо}, если \(u\) и \(v\) суммируемы.

          Если \(f\) суммируемо, то \(\int_E f = \int_E u + i \int_E v\)

    \item Неравенство Гёльдера.

          \begin{itemize}
              \item \(p, q, > 1, \frac{1}{p} + \frac{1}{q} = 1\)
              \item \((X, \mathfrak{A}, \mu)\)
              \item \(E\) --- измеримо
              \item \(f, g : E \to \mathbb C\)
              \item \(f, g\) --- измеримы
          \end{itemize}

          Тогда \(\int_E |fg| d\mu \leq \left( \int_E |f|^p \right)^{\frac{1}{p}} \left( \int_E |g|^q \right)^{\frac{1}{q}}\)
          \begin{proof}
              Не будет, но общая идея следующая:
              \begin{enumerate}
                  \item Для ступенчатых функций --- из неравенства Гёльдера для сумм\footnote{Мы его рассматривали во втором семестре.}
                  \item Для суммируемых функций --- по теореме \nameref{леви}.
              \end{enumerate}
          \end{proof}

    \item Неравенство Минковского.

          В тех же условиях \(\left( \int_E |f + g|^p \right)^{\frac{1}{p}} \leq \left( \int_E |f|^p \right)^{\frac{1}{p}} + \left( \int_E |g|^p \right)^{\frac{1}{p}}\)

          \begin{proof}
              Не будет, можно вывести аналогично выводу во втором семестре.
          \end{proof}

          \begin{remark}
          	Для \(p = 1\) тоже верно
          \end{remark}
    \item Определение пространства \(L^p, 1 \leq p < +\infty\)

          \begin{itemize}
              \item \((X, \mathfrak{A}, \mu)\) --- пространство с мерой.
              \item \(E \subset X\) --- измеримо.
          \end{itemize}

          \(\mathcal{L}^p(E, \mu) : = \{f : \text{почти везде } E \to \overline{\R}(\overline{\mathbb C}\footnote{ \(\overline{\mathbb C} = \mathbb C \cup \{\infty\} \)}), f \text{ --- изм.}\footnote{Или измерима почти везде.}, \int_E |f|^p d\mu < +\infty\} \) --- это линейное пространство по неравенству Минковского.

          Зададим отношение эквивалентности \(\sim\) на \(\mathcal{L}^p(E, \mu)\): \(f \sim g \Leftrightarrow f = g\) почти везде.

          \(\mathcal{L}^p /_\sim = L^p(E, \mu)\) --- линейное пространство.

          Задаём норму на \(L^p\): \(||f||_{L^p(E, \mu)} = \left( \int_E |f|^p \right)^{\frac{1}{p}}\), обозначается \(||f||_p\)

    \item \(L^{\infty}(E, \mu)\)

          \begin{itemize}
              \item \((X, \mathfrak{A}, \mu)\) --- пространство с мерой.
              \item \(E \subset X\) --- измеримо.
              \item \(f : \) почти везде на \(E \to \overline{\R}\) --- измеримо
          \end{itemize}

          \begin{definition}[существенный супремум]
              \[\esup_{x\in E} f = \inf \{A \in \overline\R, f \leq A \text{ почти везде}\} \]
              При этом \(A\) называется существенной вещественной границей.
          \end{definition}

          \begin{prop}\itemfix
              \begin{itemize}
                  \item \(\esup f \leq \sup f\) --- очевидно.
                  \item \(f \leq \esup f\) почти везде --- пусть \(B = \esup f\), тогда \(\forall n \ \ f \leq B + \frac{1}{n}\) почти везде.
                  \item \(f\) --- суммируемо, \(\esup_E |g| < +\infty\). Тогда \(|\int_E fg| \leq \esup |g| \cdot \int_E |f|\)
                        \begin{proof}
                            \begin{equation}
                                \left|\int_E fg\right| \leq \int_E |fg| \leq \int_E \esup |g| \cdot |f| \label{брух 2}
                            \end{equation}
                        \end{proof}
              \end{itemize}
          \end{prop}

          \(L^{\infty}(E, \mu) = \{f : \text{почти везде } E \to \overline\R(\overline{\mathbb C}), \text{ изм.}, \esup |f| < +\infty\}/_\sim \) --- линейное пространство.

          \(||f||_{L^{\infty}(E, \mu)}: = \esup_E |f| = ||f||_{\infty}\)

          \begin{remark}\itemfix
              \begin{enumerate}
                  \item В новых обозначениях неравенство Гёльдера: \(||fg||_1 \leq ||f||_p \cdot ||g||_q\) --- здесь можно брать \(p = 1, q = +\infty\) --- это~\eqref{брух 2}.
                  \item \(f\in L^p \Rightarrow f\) --- почти везде конечно, если \(1 \leq p \leq +\infty \Rightarrow \) можно считать, что \(f\) задана всюду на \(E\) и всюду конечна. % почему, как интерполировать?
              \end{enumerate}
          \end{remark}
\end{enumerate}

