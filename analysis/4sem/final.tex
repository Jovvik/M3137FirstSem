\documentclass[12pt, a4paper]{article}

%<*preamble>
% Math symbols
\usepackage{amsmath, amsthm, amsfonts, amssymb}
\usepackage{accents}
\usepackage{esvect}
\usepackage{mathrsfs}
\usepackage{mathtools}
\mathtoolsset{showonlyrefs}
\usepackage{cmll}
\usepackage{stmaryrd}
\usepackage{physics}
\usepackage[normalem]{ulem}
\usepackage{ebproof}
\usepackage{extarrows}

% Page layout
\usepackage{geometry, a4wide, parskip, fancyhdr}

% Font, encoding, russian support
\usepackage[russian]{babel}
\usepackage[sb]{libertine}
\usepackage{xltxtra}

% Listings
\usepackage{listings}
\lstset{basicstyle=\ttfamily,breaklines=true}
\setmonofont{Inconsolata}

% Miscellaneous
\usepackage{array}
\usepackage{calc}
\usepackage{caption}
\usepackage{subcaption}
\captionsetup{justification=centering,margin=2cm}
\usepackage{catchfilebetweentags}
\usepackage{enumitem}
\usepackage{etoolbox}
\usepackage{float}
\usepackage{lastpage}
\usepackage{minted}
\usepackage{svg}
\usepackage{wrapfig}
\usepackage{xcolor}
\usepackage[makeroom]{cancel}

\newcolumntype{L}{>{$}l<{$}}
    \newcolumntype{C}{>{$}c<{$}}
\newcolumntype{R}{>{$}r<{$}}

% Footnotes
\usepackage[hang]{footmisc}
\setlength{\footnotemargin}{2mm}
\makeatletter
\def\blfootnote{\gdef\@thefnmark{}\@footnotetext}
\makeatother

% References
\usepackage{hyperref}
\hypersetup{
    colorlinks,
    linkcolor={blue!80!black},
    citecolor={blue!80!black},
    urlcolor={blue!80!black},
}

% tikz
\usepackage{tikz}
\usepackage{tikz-cd}
\usetikzlibrary{arrows.meta}
\usetikzlibrary{decorations.pathmorphing}
\usetikzlibrary{calc}
\usetikzlibrary{patterns}
\usepackage{pgfplots}
\pgfplotsset{width=10cm,compat=1.9}
\newcommand\irregularcircle[2]{% radius, irregularity
    \pgfextra {\pgfmathsetmacro\len{(#1)+rand*(#2)}}
    +(0:\len pt)
    \foreach \a in {10,20,...,350}{
            \pgfextra {\pgfmathsetmacro\len{(#1)+rand*(#2)}}
            -- +(\a:\len pt)
        } -- cycle
}

\providetoggle{useproofs}
\settoggle{useproofs}{false}

\pagestyle{fancy}
\lfoot{M3137y2019}
\cfoot{}
\rhead{стр. \thepage\ из \pageref*{LastPage}}

\newcommand{\R}{\mathbb{R}}
\newcommand{\Q}{\mathbb{Q}}
\newcommand{\Z}{\mathbb{Z}}
\newcommand{\B}{\mathbb{B}}
\newcommand{\N}{\mathbb{N}}
\renewcommand{\Re}{\mathfrak{R}}
\renewcommand{\Im}{\mathfrak{I}}

\newcommand{\const}{\text{const}}
\newcommand{\cond}{\text{cond}}

\newcommand{\teormin}{\textcolor{red}{!}\ }

\DeclareMathOperator*{\xor}{\oplus}
\DeclareMathOperator*{\equ}{\sim}
\DeclareMathOperator{\sign}{\text{sign}}
\DeclareMathOperator{\Sym}{\text{Sym}}
\DeclareMathOperator{\Asym}{\text{Asym}}

\DeclarePairedDelimiter{\ceil}{\lceil}{\rceil}

% godel
\newbox\gnBoxA
\newdimen\gnCornerHgt
\setbox\gnBoxA=\hbox{$\ulcorner$}
\global\gnCornerHgt=\ht\gnBoxA
\newdimen\gnArgHgt
\def\godel #1{%
    \setbox\gnBoxA=\hbox{$#1$}%
    \gnArgHgt=\ht\gnBoxA%
    \ifnum     \gnArgHgt<\gnCornerHgt \gnArgHgt=0pt%
    \else \advance \gnArgHgt by -\gnCornerHgt%
    \fi \raise\gnArgHgt\hbox{$\ulcorner$} \box\gnBoxA %
    \raise\gnArgHgt\hbox{$\urcorner$}}

% \theoremstyle{plain}

\theoremstyle{definition}
\newtheorem{theorem}{Теорема}
\newtheorem*{definition}{Определение}
\newtheorem{axiom}{Аксиома}
\newtheorem*{axiom*}{Аксиома}
\newtheorem{lemma}{Лемма}

\theoremstyle{remark}
\newtheorem*{remark}{Примечание}
\newtheorem*{exercise}{Упражнение}
\newtheorem{corollary}{Следствие}[theorem]
\newtheorem*{statement}{Утверждение}
\newtheorem*{corollary*}{Следствие}
\newtheorem*{example}{Пример}
\newtheorem{observation}{Наблюдение}
\newtheorem*{prop}{Свойства}
\newtheorem*{obozn}{Обозначение}

% subtheorem
\makeatletter
\newenvironment{subtheorem}[1]{%
    \def\subtheoremcounter{#1}%
    \refstepcounter{#1}%
    \protected@edef\theparentnumber{\csname the#1\endcsname}%
    \setcounter{parentnumber}{\value{#1}}%
    \setcounter{#1}{0}%
    \expandafter\def\csname the#1\endcsname{\theparentnumber.\Alph{#1}}%
    \ignorespaces
}{%
    \setcounter{\subtheoremcounter}{\value{parentnumber}}%
    \ignorespacesafterend
}
\makeatother
\newcounter{parentnumber}

\newtheorem{manualtheoreminner}{Теорема}
\newenvironment{manualtheorem}[1]{%
    \renewcommand\themanualtheoreminner{#1}%
    \manualtheoreminner
}{\endmanualtheoreminner}

\newcommand{\dbltilde}[1]{\accentset{\approx}{#1}}
\newcommand{\intt}{\int\!}

% magical thing that fixes paragraphs
\makeatletter
\patchcmd{\CatchFBT@Fin@l}{\endlinechar\m@ne}{}
{}{\typeout{Unsuccessful patch!}}
\makeatother

\newcommand{\get}[2]{
    \ExecuteMetaData[#1]{#2}
}

\newcommand{\getproof}[2]{
    \iftoggle{useproofs}{\ExecuteMetaData[#1]{#2proof}}{}
}

\newcommand{\getwithproof}[2]{
    \get{#1}{#2}
    \getproof{#1}{#2}
}

\newcommand{\import}[3]{
    \subsection{#1}
    \getwithproof{#2}{#3}
}

\newcommand{\given}[1]{
    Дано выше. (\ref{#1}, стр. \pageref{#1})
}

\renewcommand{\ker}{\text{Ker }}
\newcommand{\im}{\text{Im }}
\renewcommand{\grad}{\text{grad}}
\newcommand{\rg}{\text{rg}}
\newcommand{\defeq}{\stackrel{\text{def}}{=}}
\newcommand{\defeqfor}[1]{\stackrel{\text{def } #1}{=}}
\newcommand{\itemfix}{\leavevmode\makeatletter\makeatother}
\newcommand{\?}{\textcolor{red}{???}}
\renewcommand{\emptyset}{\varnothing}
\newcommand{\longarrow}[1]{\xRightarrow[#1]{\qquad}}
\DeclareMathOperator*{\esup}{\text{ess sup}}
\newcommand\smallO{
    \mathchoice
    {{\scriptstyle\mathcal{O}}}% \displaystyle
    {{\scriptstyle\mathcal{O}}}% \textstyle
    {{\scriptscriptstyle\mathcal{O}}}% \scriptstyle
    {\scalebox{.6}{$\scriptscriptstyle\mathcal{O}$}}%\scriptscriptstyle
}
\renewcommand{\div}{\text{div}\ }
\newcommand{\rot}{\text{rot}\ }
\newcommand{\cov}{\text{cov}}

\makeatletter
\newcommand{\oplabel}[1]{\refstepcounter{equation}(\theequation\ltx@label{#1})}
\makeatother

\newcommand{\symref}[2]{\stackrel{\oplabel{#1}}{#2}}
\newcommand{\symrefeq}[1]{\symref{#1}{=}}

% xrightrightarrows
\makeatletter
\newcommand*{\relrelbarsep}{.386ex}
\newcommand*{\relrelbar}{%
    \mathrel{%
        \mathpalette\@relrelbar\relrelbarsep
    }%
}
\newcommand*{\@relrelbar}[2]{%
    \raise#2\hbox to 0pt{$\m@th#1\relbar$\hss}%
    \lower#2\hbox{$\m@th#1\relbar$}%
}
\providecommand*{\rightrightarrowsfill@}{%
    \arrowfill@\relrelbar\relrelbar\rightrightarrows
}
\providecommand*{\leftleftarrowsfill@}{%
    \arrowfill@\leftleftarrows\relrelbar\relrelbar
}
\providecommand*{\xrightrightarrows}[2][]{%
    \ext@arrow 0359\rightrightarrowsfill@{#1}{#2}%
}
\providecommand*{\xleftleftarrows}[2][]{%
    \ext@arrow 3095\leftleftarrowsfill@{#1}{#2}%
}

\allowdisplaybreaks

\newcommand{\unfinished}{\textcolor{red}{Не дописано}}

% Reproducible pdf builds 
\special{pdf:trailerid [
<00112233445566778899aabbccddeeff>
<00112233445566778899aabbccddeeff>
]}
%</preamble>


\usepackage{sectsty}

\allsectionsfont{\raggedright}
\subsectionfont{\fontsize{14}{15}\selectfont}

\lhead{Итоговый конспект}
\rfoot{}

\settoggle{useproofs}{true}

\renewcommand{\import}[3]{
    \subsection{#1}
    \getwithproof{#2}{#3}
}

\begin{document}

\section{Определения}

\import{Ступенчатая функция}{1}{ступенчатаяфункция}
\label{ступенчатая функция}

\import{Разбиение, допустимое для ступенчатой функции}{}{}
\given{ступенчатая функция}

\import{\teormin Измеримая функция}{1}{измеримаяфункция}

\import{Свойство, выполняющееся почти везде}{2}{свойствопочтивезде}

\import{\teormin Сходимость почти везде}{2}{сходимостьпочтивезде}

\import{Сходимость по мере}{2}{сходимостьпомере}

\import{Теорема Егорова о сходиомсти почти везде и почти равномерной сходиомсти}{2}{теоремаегорова}

\import{Интеграл ступенчатой функции}{2}{интеграл1}

\import{\teormin Интеграл неотрицательной измеримой функции}{2}{интеграл2}

\import{\teormin Суммируемая функция}{3}{суммируемаяфункция}

\import{Интеграл суммируемой функции}{2}{интеграл3}

\import{Образ меры при отображении}{4}{образприотображении}

\import{Взвешенный образ меры}{4}{взвешенныйобразмеры}

\import{Плотность одной меры по отношению к другой}{4}{плотностьмеры}

\import{Измеримое множество на простой двумерной поверхности в \(\R^3\)}{7}{измеримоемножествонадвумернойповерхности}
\get{7}{алгебраповерхностей}

\import{Мера Лебега на простой двумерной поверхности в \(\R^3\)}{7}{мераповерхности}

\import{\teormin Поверхностный интеграл первого рода}{7}{поверхностныйинтегралпервогорода}

\import{Произведение мер}{6}{произведениемер}

\import{\teormin Теорема Фубини}{7}{фубини}

\import{Сторона поверхности}{8}{сторонаповерхости}

\import{Задание стороны поверхности с помощью касательных реперов}{8}{реперы}

\import{\teormin Интеграл II рода}{8}{интегралвторогорода}

\import{Ориентация контура, согласованная со стороной поверхности}{8}{согласованнаяориентация}

\import{Интегральные неравенства Гельдера и Минковского}{8}{неравенствогёльдера}
\get{8}{неравенствоминеовского}

\import{Интеграл комплекснозначной функции}{8}{интегралкомплекснозначнойфункции}

\import{\teormin Пространство $L^p(E,\mu)$}{8}{пространствоlp}

\import{\teormin Пространство $L^\infty(E,\mu)$}{8}{пространствоlinfty}

\import{\teormin Существенный супремум}{8}{существенныйсупремум}

% \import{\teormin Ротор, дивергенция векторного поля}{2}{}

% \import{Соленоидальное векторное поле}{2}{}

% \import{Бескоординатное определение ротора и дивергенции}{2}{}

% \import{\teormin Гильбертово пространство}{2}{}

% \import{Ортогональный ряд}{2}{}

% \import{Сходящийся ряд в гильбертовом пространстве}{2}{}

% \import{Ортогональная система (семейство) векторов}{2}{}

% \import{\teormin Ортонормированная система }{2}{}

% \import{Коэффициенты Фурье}{2}{}

% \import{Ряд Фурье в Гильбертовом пространстве}{2}{}

% \import{Базис, полная, замкнутая ОС}{2}{}

% \import{Тригонометрический ряд}{2}{}

% \import{Коэффициенты Фурье функции}{2}{}

% \import{Класс Липшица с константой M и показателем альфа}{2}{}

% \import{Ядро Дирихле, ядро Фейера}{2}{}

% \import{\teormin Свертка}{2}{}

% \import{\teormin Аппроксимативная единица}{2}{}

% \import{Усиленная аппроксимативная единица}{2}{}

% \import{Метод суммирования средними арифметическими}{2}{}

% \import{Суммы Фейера}{2}{}


\section{Теоремы}

\import{Лемма ``о структуре компактного оператора''}{1}{оструктурекомпактногооператора}

\import{\teormin Теорема о преобразовании меры Лебега при линейном отображении}{1}{опреобразованиимерылебегаподдействиемлинейногоотображения}

\import{Теорема об измеримости пределов и супремумов}{1}{обизмеримостипределовисупрмемумов}

\import{\teormin Характеризация измеримых функций с помощью ступенчатых. Следствия}{2}{характеризацияизмеримыхфункцийспомощьюступенчатых}
\get{2}{характеризацияизмеримыхфункцийспомощьюступенчатыхcorollary}

\import{Измеримость функции, непрерывной на множестве полной меры}{2}{обизмеримостифункцийнепрерывныхнамножествеполноймеры}

\import{Теорема Лебега о сходимости почти везде и сходимости по мере}{2}{лебега}

\import{Теорема Рисса о сходимости по мере и сходимости почти везде}{2}{рисса}

\import{Простейшие свойства интеграла Лебега}{3}{свойстваинтегралалебега}

\import{Счетная аддитивность интеграла (по множеству)}{3}{счётнаяадиитивностьинтегралалемма}
\getwithproof{3}{счётнаяадиитивностьинтеграла}
\get{3}{счётнаяадиитивностьинтегралаcorollary}

\import{\teormin Теорема Леви}{3}{леви}

\import{Линейность интеграла Лебега}{3}{линейностьинтеграла}
\getwithproof{3}{линейностьинтегралаcorollary}

\import{Теорема об интегрировании положительных рядов. Следствие о рядах, сходящихся почти везде}{3}{обинтегрированииположительныхрядов}
\getwithproof{3}{обинтегрированииположительныхрядовcorollary}

\import{Абсолютная непрерывность интеграла}{4}{обабсолютнойнепрерывностиинтеграла}
\get{4}{обабсолютнойнепрерывностиинтегралаcorollary}

\import{\teormin Теорема Лебега о мажорированной сходимости для случая сходимости по мере}{4}{лебега1}

\import{\teormin Теорема Лебега о мажорированной сходимости для случая сходимости почти везде}{4}{лебега2}

\import{Теорема Фату. Следствия}{4}{фату}
\get{4}{фатуcorollary}

\import{Теорема о вычислении интеграла по взвешенному образу меры}{4}{овычисленииинтегралаповзвешенномуобразумерынаблюдение}
\getwithproof{4}{овычисленииинтегралаповзвешенномуобразумеры}
\getwithproof{4}{овычисленииинтегралаповзвешенномуобразумерыcorollary}

\import{Критерий плотности}{5}{критерийплотности}

\import{Лемма о единственности плотности}{5}{единственностьплотности}

\import{Лемма об оценке мер образов малых кубов}{5}{обоценкемеробразовмалыхкубов}

\import{Теорема о преобразовании меры при диффеоморфизме}{5}{леммабезимени}
\getwithproof{5}{мералебегапридиффеоморфизме}

\import{Теорема о гладкой замене переменной в интеграле Лебега}{5}{огладкойзамене}

\import{Теорема о произведении мер}{6}{опроизведениимер}

\import{Принцип Кавальери}{6}{кавальери}
\getwithproof{6}{кавальериcorollary}

\import{Теорема Тонелли}{7}{тонелли}

\import{Формула для Бета-функции}{7}{формулабетафункции}

\import{Объем шара в $\R^m$}{7}{объемшара}

% \import{Формула Грина}{2}{}

% \import{\teormin Формула Стокса}{2}{}

% \import{\teormin Формула Гаусса--Остроградского}{2}{}

% \import{Соленоидальность бездивергентного векторного поля}{2}{}

% \import{Теорема о вложении пространств $L^p$}{2}{}

% \import{Теорема о сходимости в $L_p$ и по мере}{2}{}

% \import{Полнота $L^p$}{2}{}

% \import{Плотность в $L^p$ множества ступенчатых функций}{2}{}

% \import{Лемма Урысона}{2}{}

% \import{Плотность в $L^p$ непрерывных финитных функций}{2}{}

% \import{\teormin Теорема о непрерывности сдвига}{2}{}

% \import{Теорема о свойствах сходимости в гильбертовом пространстве}{2}{}

% \import{Теорема о коэффициентах разложения по ортогональной системе}{2}{}

% \import{Теорема о свойствах частичных сумм ряда Фурье. Неравенство Бесселя}{2}{}

% \import{Теорема Рисса -- Фишера о сумме ряда Фурье. Равенство Парсеваля}{2}{}

% \import{Теорема о характеристике базиса}{2}{}

% \import{Лемма о вычислении коэффициентов тригонометрического ряда}{2}{}

% \import{Теорема Римана--Лебега}{2}{}

% \import{Три следствия об оценке коэффициентов Фурье}{2}{}

% \import{Принцип локализации Римана}{2}{}

% \import{\teormin Признак Дини. Следствия}{2}{}

% \import{Корректность определения свертки}{2}{}

% \import{Свойства свертки }{2}{}

% \import{Теорема о свойствах аппроксимативной единицы}{2}{}

% \import{Теорема о перманентности метода средних арифметических}{2}{}

% \import{Теорема Фейера }{2}{}

% \import{Следствия из теоремы Фейера}{2}{}

% \import{Теорема об интегрировании ряда Фурье}{2}{}

% \import{Лемма о слабой сходимости сумм Фурье}{2}{}

\end{document}
