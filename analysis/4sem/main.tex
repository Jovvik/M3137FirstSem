\documentclass[12pt, a4paper, oneside]{book}

\usepackage{catchfilebetweentags}
\ExecuteMetaData[../../preamble.sty]{preamble}

\addto\captionsrussian{\renewcommand{\chaptername}{Лекция}}

\let\MakeUppercase\relax

\lfoot{}
\cfoot{}
\rfoot{}
\lhead{\leftmark}

\usepackage{tocloft}
\advance\cftsecnumwidth -1em\relax
\advance\cftsubsecindent -1em\relax
\advance\cftsubsecnumwidth -1em\relax
\advance\cftsubsubsecindent -3em\relax
\advance\cftsubsubsecnumwidth -1em\relax

\setcounter{tocdepth}{3}
\setcounter{secnumdepth}{3}

\usepackage{titletoc}
\titlecontents*{chapter}% <section-type>
[0pt]% <left>
{}% <above-code>
{\bfseries\chaptername\ \thecontentslabel\quad}% <numbered-entry-format>
{}% <numberless-entry-format>
{\bfseries\hfill\contentspage}% <filler-page-format>

\let\endtitlepage\relax

\patchcmd{\chapter}{\thispagestyle{plain}}{\thispagestyle{fancy}}{}{}

\usepackage{remreset}
\makeatletter
\@removefromreset{section}{chapter}
\makeatother

\usepackage{chngcntr}
\counterwithout{equation}{chapter}

\renewcommand{\thesection}{\arabic{section}}
\author{Михайлов Максим}


\begin{document}

% а где
\title{Математический анализ}
\titlepage

% TODO: названия частей
\tableofcontents

\chapter{8 февраля}

\begin{lemma}[о структуре компактного оператора]\itemfix
    %<*оструктурекомпактногооператора>
    \begin{itemize}
        \item \(V : \R^m \to \R^m\) --- линейный оператор
        \item \(\det V \neq 0\)
    \end{itemize}
    Тогда \(\exists \) ортонормированные базисы \(g_1 \dots g_m\) и \(h_1 \dots h_m\), а также \(\exists s_1 \dots s_m > 0\), такие что:
    \[\forall x\in\R^m \quad V(x) = \sum_{i = 1}^m s_i\langle x, g_i\rangle h_i\]

    И \(|\det V| = s_1s_2 \dots s_m\).
    %</оструктурекомпактногооператора>
\end{lemma}

\begin{remark}
    Эта лемма из функционального анализа, что такое компактный оператор --- мы не знаем.
\end{remark}

%<*оструктурекомпактногооператораproof>
\begin{proof}
    \(W : = V^* V\) --- самосопряженный оператор \textit{(матрица симметрична относительно диагонали)}.

    Из линейной алгебры мы знаем, что такой оператор имеет:
    \begin{itemize}
        \item Собственные числа: \(c_1 \dots c_m\) --- вещественные \textit{(возможно с повторениями)}
        \item Собственные векторы: \(g_1 \dots g_m\) --- ортонормированные
    \end{itemize}

    \begin{remark}
        Пока мы в \(\R^m\) \textit{(а не в \(\mathbb{C}^m\))}, \({}^*\) есть транспонирование. В комплексном случае ещё берется сопряжение.
    \end{remark}

    \[c_i \ev{g_i, g_i} \symrefeq{собственный вектор} \ev{W g_i, g_i} \symrefeq{из линала} \ev{V g_i, V g_i} > 0\]

    \begin{itemize}
        \item \eqref{собственный вектор}: т.к. \(g_i\) --- собственный вектор для \(W\) с собственным значением \(c_i\).
        \item \eqref{из линала}: из линейной алгебры:
              \[W_{kl} = \sum_{i = 1}^m V_{ik}V_{il}\]
              \[\ev{W g_i, g_i} = \sum_{k, l, j} V_{jk} V_{jl} g_k^{(i)} g_l^{(i)} = \ev{V g_i, V g_i}\]
    \end{itemize}

    Таким образом, \(c_i > 0\).

    \[s_i : = \sqrt{c_i}\]
    \[h_i : = \frac{1}{s_i} V g_i\]

    \[\ev{h_i, h_j} \defeqfor{h_i} \frac{1}{s_i s_j}\ev{V g_i, V g_j} \symrefeq{из линала 2} \frac{1}{s_i s_j}\ev{W g_i, g_j} \symrefeq{собственный вектор 2} \frac{c_i}{s_i s_j}\ev{g_i, g_j} \symrefeq{кронекер} \delta_{ij}\]

    \begin{itemize}
        \item \eqref{из линала 2}: из линейной алгебры, аналогично предыдущему.
        \item \eqref{собственный вектор 2}: т.к. \(g_i\) --- собственный вектор для \(W\) с собственным значением \(c_i\).
        \item \eqref{кронекер}: при \(i \neq j\) \(\ev{g_i, g_j} = 0\) в силу ортогональности, а при \(i = j\) \(\ev{g_i, g_j} = 1\) в силу ортонормированности и \(\frac{c_i}{s_i s_j} = \frac{c_i}{\sqrt{c_i}\sqrt{c_i}} = 1\)
    \end{itemize}

    \begin{remark}
        \(\delta_{ij} = \begin{cases}
            1, & i = j    \\
            0, & i \neq j
        \end{cases}\) --- символ Кронекера.
    \end{remark}

    Таким образом, \(\{h_i\}\) ортонормирован.

    \[V(x) \defeqfor{x} V\left( \sum_{i = 1}^m \ev{x, g_i} g_i \right) \symrefeq{линейность V} \sum_{i = 1}^m \ev{x, g_i} V(g_i) \defeqfor{h_i} \sum s_i \ev{x, g_i} h_i\]

    \blfootnote{\eqref{линейность V}: в силу линейности \(V\)}

    \[(\det V)^2 \symrefeq{мультипликативность det} \det(V^* V) \defeqfor{W} \det W \symrefeq{определитель базис} c_1 \dots c_m\]

    \blfootnote{\eqref{мультипликативность det}: в силу мультипликативности \(\det\) и инвариантности относительно транспонирования.}
    \blfootnote{\eqref{определитель базис}: т.к. \(\det\) инвариантен по базису и в базисе собственных векторов \(\det W = c_1 \dots c_m\).}

    \[|\det V| = \sqrt{c_1} \dots \sqrt{c_m} = s_1 \dots s_m\]
\end{proof}
%</оструктурекомпактногооператораproof>

\begin{theorem}[о преобразовании меры Лебега под действием линейного отображения]\itemfix
    %<*опреобразованиимерылебегаподдействиемлинейногоотображения>
    \begin{itemize}
        \item \(V : \R^m \to \R^m\) --- линейное отображение
    \end{itemize}
    Тогда \(\forall E \in \mathfrak{M}^m \ \ V(E) \in \mathfrak{M}^m\) и \(\lambda(V(E)) = |\det V| \cdot \lambda E\)
    %</опреобразованиимерылебегаподдействиемлинейногоотображения>
\end{theorem}

%<*опреобразованиимерылебегаподдействиемлинейногоотображенияproof>
\begin{proof}\itemfix
    \begin{enumerate}
        \item Если \(\det V = 0 \quad \text{Im}(V)\) --- подпространство в \(\R^m\) \( \Rightarrow \lambda(\text{Im}(V)) = 0\) по следствию 6 лекции 15 третьего семестра. Тогда \(\forall E \ \ V(E)\subset \text{Im}(V) \Rightarrow \lambda(V(E)) = 0\)
        \item Если \(\det V \neq 0 \quad \mu E : = \lambda(V(E))\) --- мера, инвариантная относительно сдвигов. Это было доказано в конце прошлого семестра:
              \[\mu(E + a) = \lambda(V(E + a)) = \lambda(V(E) + V(a)) = \lambda(V(E)) = \mu E\]

              \( \Rightarrow \exists k : \mu = k \lambda\) по недоказанной теореме из прошлого семестра.

              Мы хотим найти \(k\), для этого нужно что-нибудь померять. Померяем что-то очень простое, например \(Q = \{\sum \alpha_i g_i\ |\ \alpha_i \in [0, 1]\} \) --- единичный куб на векторах \(g_i\).

              \(V(g_i) = s_i h_i\). Таким образом, \(V(Q) = \{\sum \alpha_i s_i h_i \ |\ \alpha_i \in [0, 1]\} \).

              \[\mu Q = \lambda(V(Q)) = s_1 \dots s_m = |\det V| = |\det V| \underbrace{\lambda Q}_{= 1}\]

              Таким образом, \(k = |\det V|\)
    \end{enumerate}
\end{proof}
%</опреобразованиимерылебегаподдействиемлинейногоотображенияproof>

\section{Интеграл}

\subsection{Измеримые функции}

\begin{definition}\itemfix
    \begin{enumerate}
        \item \(E\) --- множество, \(E = \bigsqcup\limits_{\text{кон.}} e_i\) --- разбиение множества.
        \item %<*ступенчатаяфункция>
              \(f : X \to \R\) --- \textbf{ступенчатая}, если:
              \[\exists\ \text{разбиение} \ X = \bigsqcup\limits_{\text{кон.}} e_i : \forall i \ \ f\Big|_{e_i} = \const_i = c_i\]

              При этом разбиение называется \textbf{допустимым} для этой функции.
              %</ступенчатаяфункция>
    \end{enumerate}
\end{definition}

\begin{example}\itemfix
    \begin{enumerate}
        \item Характеристическая функция множества \(E\subset X\) : \(\chi_E(x) = \begin{cases}
                  1, & x\in E            \\
                  0, & x\in X\setminus E
              \end{cases}\)
        \item \(f = \sum\limits_{\text{кон.}} c_i \chi_{e_i}\), где \(X = \bigsqcup e_i\)
    \end{enumerate}
\end{example}

\begin{figure}[h]
    \centering
    \includesvg{images/ступенчатая_функция.svg}
    \caption{Ступенчатая функция}
\end{figure}

\begin{prop}\itemfix
    \begin{enumerate}
        \item \(\forall f, g\) --- ступенчатые:

              \(\exists \) разбиение \(X\), допустимое и для \(f\), и для \(g\):

              \[f = \sum\limits_{\text{кон.}} c_i \chi_{e_i} \quad g = \sum\limits_{\text{кон.}} b_k \chi_{a_k}\]
              \[f = \sum\limits_{i, k} c_i \chi_{e_i \cap a_k} \quad g = \sum\limits_{i, k} b_k \chi_{e_i \cap a_k}\]
        \item \(f, g\) --- ступенчатые, \(\alpha\in\R\)

              Тогда \(f + g, \alpha f, fg, \max(f, g), \min(f, g), |f|\) --- ступенчатые.
    \end{enumerate}
\end{prop}

\begin{definition}
    \(f : E\subset X \to \overline \R, a\in\R\)

    \(E(f < a) = \{x\in E : f(x) < a\} \) --- лебегово множество функции \(f\)

    Аналогично можно использовать \(E(f \leq a), E(f > a), E(f \geq a)\)
\end{definition}

\begin{remark}
    \[E(f \geq a) = E(f < a)^c \quad E(f < a) = E(f \geq a)^c\]
    \[E(f \leq a) = \bigcap_{b > a} E(f < b) = \bigcap_{n\in\N} E\left(f < a + \frac{1}{n}\right)\]
\end{remark}

\begin{definition}\itemfix
    %<*измеримаяфункция>
    \begin{itemize}
        \item \((X, \mathfrak{A}, \mu)\) --- пространство с мерой
        \item \(f : X \to \overline \R\)
        \item \(E\in \mathfrak{A}\)
    \end{itemize}

    \(f\) \textbf{измерима} на множестве \(E\), если \(\forall a\in\R \ \ E(f < a)\) измеримо, т.е. \(\in \mathfrak{A}\)
    %</измеримаяфункция>
\end{definition}

Вместо ``\(f\) измерима на \(X\)'' говорят просто ``измерима''.

Если \(X = \R^m\), мера --- мера Лебега, тогда \(f\) --- измеримо по Лебегу.

\begin{remark}
    Эквивалентны:
    \begin{enumerate}
        \item \(\forall a \ \ E(f < a)\) --- измеримо
        \item \(\forall a \ \ E(f \leq a)\) --- измеримо
        \item \(\forall a \ \ E(f > a)\) --- измеримо
        \item \(\forall a \ \ E(f \geq a)\) --- измеримо
    \end{enumerate}
\end{remark}

\begin{proof}
    Тривиально по соображениям выше.
\end{proof}

\begin{example}\itemfix
    \begin{enumerate}
        \item \(E\subset X, E\) --- измеримо \( \Rightarrow \chi_E\) --- измеримо.
              \[E(\chi_E < a) = \begin{cases}
                      \emptyset,     & a < 0           \\
                      X \setminus E, & 0 \leq a \leq 1 \\
                      X ,            & a > 1
                  \end{cases}\]
        \item \(f:\R^m \to \R\) --- непрерывно. Тогда \(f\) --- измеримо по Лебегу.
              \begin{proof}
                  \(f^{ - 1}(( -\infty, a))\) открыто по топологическому определению открытости, а любое открытое множество измеримо по Лебегу.
              \end{proof}
    \end{enumerate}
\end{example}

\begin{prop}\itemfix
    \begin{enumerate}
        \item \(f\) измеримо на \(E\) \( \Rightarrow \forall a\in\R \ \ E(f = a)\) измеримо.

              В обратную сторону неверно, пример --- \(f(x) = x + \chi_\text{неизм.}\)

        \item \(f\) --- измеримо \( \Rightarrow \forall \alpha\in\R \ \ \alpha f\) --- измеримо.

              \begin{proof}
                  \(E(\alpha f < a) = \begin{cases}
                      E(f < \frac{a}{\alpha}), & \alpha > 0           \\
                      E(f > \frac{a}{\alpha}), & \alpha < 0           \\
                      E,                       & \alpha = 0, a \geq 0 \\
                      \emptyset,               & \alpha = 0, a < 0    \\
                  \end{cases}\)
              \end{proof}
        \item \(f\) --- измеримо на \(E_1, E_2, \dots \Rightarrow f\) измеримо на \(E = \bigcup E_k\)
        \item \(f\) --- измеримо на \(E, E_{\text{изм.}}'\subset E \Rightarrow f\) измеримо на \(E'\)
              \begin{proof}
                  \(E'(f < a) = E(f < a)\cap E'\)
              \end{proof}
        \item \(f \neq 0\), измеримо на \(E \Rightarrow \frac{1}{f}\) измеримо на \(E\).
        \item \(f \geq 0\), измеримо на \(E, \alpha\in\R \Rightarrow f^\alpha\) измеримо на \(E\).

              \textcolor{red}{Это неверно}, т.к. при \(f \equiv 0, \alpha = - 1\) \(\ \not\exists f^\alpha\)
    \end{enumerate}
\end{prop}

\begin{theorem}
    %<*обизмеримостипределовисупрмемумов>
    \(f_n\) --- измеримо на \(X\). Тогда:
    \begin{enumerate}
        \item \(\sup f_n, \inf f_n\) измеримо.
        \item \(\overline \lim f_n, \underline \lim f_n\) измеримо.
        \item Если \(\forall x \ \ \exists \lim\limits_{n \to +\infty} f_n(x) = h(x)\), то \(h(x)\) измеримо.
    \end{enumerate}
    %</обизмеримостипределовисупрмемумов>
\end{theorem}
%<*обизмеримостипределовисупрмемумовproof>
\begin{proof}\itemfix
    \begin{enumerate}
        \item \(g = \sup f_n \quad X(g > a) \symrefeq{два шага} \bigcup_n X(f_n > a)\) и счётное объединение измеримых множеств измеримо.

              \eqref{два шага}: \begin{itemize}
                  \item \(X(g > a) \subset \bigcup_n X(f_n > a)\), т.к. если \(x\in X(g > a)\), то \(g(x) > a\).
                        \[\sup_n f_n(x) = g(x) \neq a \Rightarrow \exists n : f_n(x) > a\]
                  \item \(X(g > a) \supset \bigcup_n X(f_n > a)\), т.к. если \(x\in X(f_n > a)\), то \(f_n(x) > a\), следовательно \(g(x) > a\).
              \end{itemize}

        \item \((\overline \lim f_n)(x) = \inf_n (s_n = \sup (f_n(x), f_{n+1}(x), \dots )) \). Т.к. \(\sup\) и \(\inf\) измерим, \(\overline \lim f_n\) тоже измерим.
        \item Очевидно, т.к. если \(\exists \lim\), то \(\lim = \overline \lim = \underline \lim\)
    \end{enumerate}
\end{proof}
%</обизмеримостипределовисупрмемумовproof>

\subsection{Меры Лебега-Стилтьеса}

\(\R, \mathcal{P}^1, g : \R \to \R\) возрастает, непрерывно.

\(\mu[a, b) : = g(b) - g(a)\) --- \(\sigma\)-конечный объем \textit{(и даже \(\sigma\)-конечная мера на \(\mathcal{P}^1\))}

Также можно определить для монотонной, но непрерывной \(g\). Тогда в точках разрыва \(\exists g(a + 0), g(a - 0)\). Пусть \(\mu[a, b) = g(b - 0) - g(a - 0)\). Такое изменение нужно, потому что исходное \(\mu\) не является объемом для разрывных функций.

Применим теорему о лебеговском продолжении меры. Получим меру \(\mu_g\) на некоторой \(\sigma\)---алгебре. Это мера Лебега-Стилтьеса.

\begin{example}
    \(g(x) = [x]\), тогда мера ячейки --- количество целых точек в этой ячейке.
\end{example}

Если \(\mu_g\) определена на Борелевской \(\sigma\)-алгебре, то она называется мерой Бореля-Стилтьеса.

\chapter{15 февраля}

\begin{theorem}[о характеризации измеримых функций с помощью ступенчатых]\itemfix
    %<*характеризацияизмеримыхфункцийспомощьюступенчатых>
    \label{характеризация измеримых функций с помощью ступенчатых}
    \begin{itemize}
        \item \(f: X \to \R\)
        \item \(f \geq 0\)
        \item \(f\) измеримо
    \end{itemize}
    Тогда \(\exists f_n\) --- ступенчатые:
    \begin{enumerate}
        \item \(0 \leq f_1 \leq f_2 \leq f_3 \leq \dots \)
        \item \(\forall x \ \ f(x) = \lim\limits_{n \to +\infty} f_n(x)\)
    \end{enumerate}
    %</характеризацияизмеримыхфункцийспомощьюступенчатых>
\end{theorem}

\begin{proof}
    %<*характеризацияизмеримыхфункцийспомощьюступенчатыхproof>
    \[e^{(n)}_k = X\left( \frac{k - 1}{n} \leq f < \frac{k}{n} \right) \quad k = 1 \dots n^2\]
    \[e^{(n)}_{n^2 + 1} : = X(n \leq f)\]
    \[g_n: = \sum_{k = 1}^{n^2 + 1} \frac{k - 1}{n} \chi_{e_k^{(n)}}\]

    \begin{figure}[h]
        \centering
        \includesvg[scale=0.7]{images/приближение_ступенчатой.svg}
    \end{figure}

    \[g_n \geq 0\]
    \[\lim\limits_{n \to +\infty} g_n(x) = f(x) : g_n(x) \leq f(x)\]

    \textcolor{red}{Не дописано.}

    % TODO: #66 дописать
    %</характеризацияизмеримыхфункцийспомощьюступенчатыхproof>
\end{proof}

\begin{corollary}\itemfix
    \begin{itemize}
        \item \(f\) --- измеримо
    \end{itemize}
    Тогда \(\exists f_n\) --- измеримые : \(f_n \xrightarrow[n \to +\infty]{} f\) всюду и \(|f_n| \leq |f|\)
\end{corollary}
\begin{proof}
    Рассмотрим срезки \(f^{+}, f^{-}\), дальше очевидно.
\end{proof}

\begin{corollary}\itemfix
    \begin{itemize}
        \item \(f, g\) --- измеримо
    \end{itemize}
    Тогда \(fg\) --- измеримо, если \(0\cdot \infty = 0\).
\end{corollary}
\begin{proof}
    \[\underbrace{f_n}_{\text{ступ.}} \to f, \underbrace{g_n}_{\text{ступ.}} \to g\]
    \[f_n g_n \text{ --- ступ.} \quad f_n g_n \to fg\]
    Измеримость выполняется в силу измеримости предела.
\end{proof}

\begin{corollary}\itemfix
    \begin{itemize}
        \item \(f, g\) --- измеримо
    \end{itemize}

    Тогда \(f + g\) измеримо.

    \begin{remark}
        Считаем, что \(\forall x\) не может быть одновременно \(f(x) = \pm \infty, g(x) = \pm \infty\).
    \end{remark}
\end{corollary}

\begin{proof}
    \[f_n + g_n \to f + g\]
\end{proof}

\begin{theorem}[об измеримости функций, непрерывных на множестве полной меры]
    %<*обизмеримостифункцийнепрерывныхнамножествеполноймеры>
    \begin{remark}
        \(A\subset X\) --- \textbf{полной меры}, если \(\mu(X\setminus A) = 0\).
    \end{remark}

    \begin{itemize}
        \item \(f: E \to \R, E\subset \R^m\)
        \item \(e\subset E\)
        \item \(\lambda_m e = 0\)
        \item \(f\) --- непрерывно на \(E' = E \setminus e\)
    \end{itemize}

    Тогда \(f\) --- измеримо.
    %</обизмеримостифункцийнепрерывныхнамножествеполноймеры>
\end{theorem}
%<*обизмеримостифункцийнепрерывныхнамножествеполноймерыproof>
\begin{proof}
    \(f\) --- измеримо на \(E'\), т.к. \(E'(f < a)\) открыто в \(E'\) по топологическому определению непрерывности.

    \(e(f < a) \subset e\), \(\lambda_m\) --- полная \( \Rightarrow e(f < a)\) --- измеримо в \(E\).

    \(E(f < a) = E'(f < a)\cup e(f < a)\), объединение измеримых множеств измеримо.
\end{proof}
%</обизмеримостифункцийнепрерывныхнамножествеполноймерыproof>

\begin{example}
    \(E = \R, f = \chi_{\text{Irr}}\), где \(\text{Irr}\) --- множество иррациональных чисел. \(f\) непр. на \(\text{Irr}\) и разрывно на \(\R\).
\end{example}

\begin{corollary}\itemfix
    \begin{itemize}
        \item \(f : E \to \R\)
        \item \(e \subset E \subset X\)
        \item \(\mu e = 0\)
        \item \(E' = E\setminus e\)
        \item \(f\) измеримо на \(E'\)
    \end{itemize}

    Тогда можно так переопределить \(f\) на \(e\), что полученная функция \(\tilde{f}\) будет измерима.
\end{corollary}
\begin{proof}
    Пусть \(\tilde{f}(x) = \begin{cases}
        f(x), x\in E' \\
        \const, x\in e
    \end{cases}\)

    \[E(\tilde{f} < a) = \underbrace{E'(\tilde{f} < a)}_{E'(f < a)} \subset \underbrace{e(\tilde{f} < a)}_{\emptyset \text{ или } e}\]
\end{proof}

\begin{corollary}
    \(f : \ev{a, b} \to \R\) --- монотонна.

    Тогда \(f\) измерима.
\end{corollary}
\begin{proof}
    \(f\) --- непрерывно на \(\ev{a, b}\) за исключением, возможно, счётного множества точек.
\end{proof}

% TODO: #67 решить задачи

\begin{exercise}
    \(f, g : \R \to \R\) --- измеримо.

    \(\varphi : \R^2 \to \R\) --- непрерывна.

    Доказать: \(x \mapsto \varphi(f(x), g(x))\) --- измеримо.
\end{exercise}

\begin{exercise}
    \(f : \R \to \R\) --- измеримо.

    Доказать: \(\R^2 \to \R : (x, y) \mapsto f(x, y)\) --- измеримо.
\end{exercise}

\begin{exercise}
    Доказать, что \(\exists \) измеримая функция \(f:\R \to \R\)

    \(\forall e\subset \R : \lambda e = 0\), если \(f\) непрерывно на \(e\), то полученная \(\tilde{f}\) разрывна всюду.
\end{exercise}

\subsection{Сходимость почти везде и по мере}

\begin{definition}\itemfix
    %<*свойствопочтивезде>
    \begin{itemize}
        \item \((X, \mathfrak{A}, \mu)\)
        \item \(E\in \mathfrak{A}\)
        \item \(W(x)\) --- высказывание \((x \in X)\)
    \end{itemize}

    \(W(x)\) --- верно при почти всех из \(E\) = почти всюду на \(E\) = почти везде на \(E\) = п.в. \(E\), если:

    \(\exists e\in E : \mu e = 0 \ \ W(x)\) --- истинно при \(x\in E\setminus e\)
    %</свойствопочтивезде>
\end{definition}

\begin{example}
    \(X =\R\), \(W\) = иррационально.
\end{example}

\begin{example}
    \(f_n(x) \xrightarrow[x \to +\infty]{} f(x)\) при п.в. \(x\in E\)
\end{example}

\begin{prop}\itemfix
    \begin{enumerate}
        \item \begin{itemize}
                  \item \(\mu\) --- полная
                  \item \(f_n, f : X \to \overline R\) п.в. \(X\)
                  \item \(f_n\) измеримо
              \end{itemize}

              Тогда \(f\) измеримо.

              \begin{proof}
                  \(f_n \to f\) на \(X'\), где \(e = X\setminus X', \mu e = 0\)

                  \(f\) --- измеримо на \(X\)

                  \(\mu\) --- полная \( \Rightarrow \) \(f\) измеримо на \(X\), т.к. \(X(f < a) = \underbrace{X'(f < a)}_{\text{изм.}} \cup \underbrace{e(f < a)}_{\subset e}\)
              \end{proof}

        \item \? % TODO #68
        \item Пусть \(\forall n \ \ W_n(x)\) истинно при почти всех \(x\).

              Тогда утверждение `` \(\forall n \ \ W_n\) истинно'' --- верно при почти всех \(X\)

              \begin{proof}
                  \(\sphericalangle e_n : \mu(e_n) = 0\). Искомое высказывание верно при \(x\in X \setminus \left( \bigcup\limits_{i = 1}^{+\infty} e_i \right), \mu(\bigcup e_i) = 0\)
              \end{proof}
    \end{enumerate}
\end{prop}

\begin{definition}
    %<*сходимостьпомере>
    \(f_n, f : X \to \overline \R\) --- почти везде конечны.

    \(f_n\) \textbf{сходится к} \(f\) \textbf{по мере} \(\mu\), обозначается \(f_n \longarrow{\mu} f : \forall \varepsilon > 0 \ \ \mu X(|f_n - f| \geq \varepsilon)\xrightarrow[n\to +\infty]{} 0\)
    %</сходимостьпомере>
\end{definition}

\begin{remark}
    \(f_n\) и \(f\) можно изменить на множестве меры \(0\), т.е. предел не задан однозначно.
\end{remark}

\begin{exercise} % TODO: #69 доказать
    \(f_n \longarrow{\mu} f; f_n \longarrow{\mu} g\). Тогда \(f\) и \(g\) эквивалентны.
\end{exercise}

\begin{example}\itemfix
    \begin{enumerate}
        \item \(f_n(x) = \frac{1}{nx}, x > 0, X = \R_{ +}, f \equiv 0\)

              \(f_n \to f\) всюду на \((0, +\infty)\)

              \(f_n \longarrow{\mu} f\)

              \[X(|f_n - f| \geq \varepsilon) = X\left(\frac{1}{nx} \geq \varepsilon\right) = X(x \leq \frac{1}{\varepsilon n})\]
              \[\lambda(\dots ) = \frac{1}{\varepsilon n} \to 0\]

        \item \(f_n(x) : = e^{ -(n - x)^2}, x\in\R\)

              \(f_n(x) \to 0\) при всех \(x\)

              \(f_n(x) \longarrow{} 0\)

              \[\mu(\R(e^{ -(n - x)^2} \geq \varepsilon)) = \const \not\to 0\]

        \item \(n = 2^k + l, 0 \leq l < 2^k, X = [0, 1], \lambda\)

              \[f_n(x) : = \chi_{[\frac{l}{2^k}, \frac{l + 1}{2^k} ]}\]

              \(\lim f_n(x)\) не существует ни при каком \(x\)!

              \[X(f_n \geq \varepsilon) = \frac{1}{2^k} \to 0 \Rightarrow f_n \longarrow{\lambda} 0\]
    \end{enumerate}
\end{example}

\begin{theorem}[Лебега]\itemfix
    %<*лебега>
    \begin{itemize}
        \item \((X, \mathfrak{A}, \mu)\)
        \item \(\mu X\) конечно
        \item \(f_n, f\) --- измеримо, п.в. конечно
        \item \(f_n \to f\) п.в.
    \end{itemize}

    Тогда \(f_n \longarrow{\mu} f\)
    %</лебега>
\end{theorem}

%<*лебегаproof>
\begin{proof}
    Переопределим \(f_n, f\) на множестве меры \(0\), чтобы сходимость была всюду.

    Рассмотрим частный случай: \(\forall x\) последовательность \(f_n(x)\) монотонно убывает к \(0\), то есть \(f\equiv 0\)
    \[X(|f_n| \geq \varepsilon) = X(f_n \geq \varepsilon) \supset X(f_{n+1} \geq \varepsilon)\]
    \[\bigcap X(f_n \geq \varepsilon)\]

    Таким образом, по теореме о непрерывности меры сверху, \(\mu X(f_n \geq \varepsilon) \to 0\)

    Рассмотрим общий случай: \(f_n \to f\), \(\varphi(x) : = \sup\limits_{k \geq n} |f_k(x) - f(x)|\)

    Тогда \(\varphi_n \to 0, \varphi_n \geq 0\) и монотонно, таким образом мы попали в частный случай.

    \[X(|f_n - f| \geq \varepsilon) \subset X(\varphi_n \geq \varepsilon)\]
    \[\mu X(|f_n - f| \geq \varepsilon) \leq \mu X(\varphi_n \geq \varepsilon) \to 0\]
\end{proof}
%</лебегаproof>

\begin{theorem}[Рисс]\itemfix
    %<*рисса>
    \begin{itemize}
        \item \((X, \mathfrak{A}, \mu)\)
        \item \(f_n \longarrow{\mu} f\).
    \end{itemize}
    Тогда \(\exists n_k : f_{n_k} \to f\) почти везде.
    %</рисса>
\end{theorem}

%<*риссаproof>
\begin{proof}
    \[\forall k \ \ \mu X\left(|f_n - f| \geq \frac{1}{k}\right) \to 0\]
    \[\exists n_k : \text{при } n \geq n_k \ \ \mu X\left( |f_n - f| \geq \frac{1}{k} \right) < \frac{1}{2^k}\]
    Можно считать, что \(n_1 < n_2 < n_3\)

    Проверим, что \(f_{n_k} \to f\) почти везде.

    \[E_k : = \bigcup_{j = k}^{+\infty} X\left( |f_{n_j} - f| \geq \frac{1}{j} \right) \quad E = \bigcap E_k\]
    \[E_k \supset E_{k+1} \quad \mu E_k \symref{счётная полуаддитивность}{\leq} \sum_{j = k}^{+\infty} \mu X\left( |f_{n_j} - f| \geq \frac{1}{j} \right) < \sum_{j = k}^{+\infty} \frac{1}{2^j} \leq \frac{2}{2^k} \to 0\]
    \[\mu E_k \to \mu E \Rightarrow \mu E = 0\]

    \blfootnote{\eqref{счётная полуаддитивность}: по счётной полуаддитивности меры.}


    Покажем, что при \(x\not\in E \ \ f_{n_k} \to f\).

    \[x\not\in E \ \ \exists N \ \ x\not\in E_k \text{ при } k > N \ \ |f_{n_k}(x) - f(x)|< \frac{1}{k}\]

    То есть \(f_{n_k}(x) \to f(a)\).

    Т.к. \(\mu E = 0\), искомое выполнено.
\end{proof}
%</риссаproof>

\begin{corollary}
    \(f_n \longarrow{\mu} f \ \ |f_n| \leq g\) почти всюду. Тогда \(|f| \leq g\) почти всюду.
\end{corollary}
\begin{proof}
    \(\exists n_k \ \ f_{n_k} \to f\) почти всюду.

    % TODO: отпарсить
\end{proof}

\[f_n \rightrightarrows f  \Rightarrow f_n(x) \to f(x) \ \  \forall x \Rightarrow f_n \longarrow{} f\]

\begin{theorem}[Егорова]\itemfix
    %<*теоремаегорова>
    \begin{itemize}
        \item \(X, \mathfrak{A}, \mu\)
        \item \(\mu X < +\infty\)
        \item \(f_n, f\) --- почти везде конечно, измеримо
    \end{itemize}

    Тогда
    \[\forall \varepsilon > 0 \ \ \exists e\subset X : \mu e < \varepsilon \quad f_n \xrightrightarrows[X\setminus e]{} f\]
    %</теоремаегорова>
\end{theorem}
\begin{proof}
    Упражнение. % TODO: #70 доказать
\end{proof}

\section{Интеграл}

\(\sphericalangle (X, \mathfrak{A}, \mu)\) --- зафиксировали.

\begin{definition}[1]\itemfix
    %<*интеграл1>
    \begin{itemize}
        \item \(f = \sum \alpha_k \chi_{E_k}\)
        \item \(E_k\) --- допустимое разбиение
        \item \(\alpha_k \geq 0\)
    \end{itemize}

    \[\int_X f d_{\mu(x)} : = \sum \alpha_k \mu E_k\]

    И пусть \(0\cdot \infty = 0\)
    %</интеграл1>
\end{definition}

\begin{prop}\itemfix
    \begin{enumerate}
        \item Не зависит от представления \(f\) в виде суммы, т.е.:

              \[f = \sum \alpha_k \chi_{E_k} = \sum \alpha'_k \chi_{E'_k} = \sum_{k, j} \alpha_k \chi_{E_k\cap E'_j}\]

              \begin{remark}
                  При \(E_k \cap E'_j \neq \emptyset\) \(\alpha_k = \alpha_j \Rightarrow \) можно писать любое из них.
              \end{remark}

              \[\int f = \sum \alpha_k \mu E_k = \sum_{k, j} \alpha_k \mu(E_k\cap E'_j) = \sum \alpha'_k \mu E'_k\] % TODO: пояснить

        \item \(\underbrace{f}_{\text{ст.}} \leq \underbrace{g}_{\text{ст.}} \Rightarrow \int_X f \leq \int_X g\)
    \end{enumerate}
\end{prop}

\begin{definition}[2]\itemfix
    %<*интеграл2>
    \begin{itemize}
        \item \(f \geq 0\)
        \item \(f\) измеримо
    \end{itemize}
    \[\int_X f d\mu : = \sup_{\substack{g \text{ --- ступ.} \\ 0 \leq g \leq f}} \int g d\mu\]
    %</интеграл2>
\end{definition}

\begin{prop}\itemfix
    \begin{itemize}
        \item Если \(f\) ступенчатая, то определение 2 = определение 1.
        \item \(0 \leq \int_X f \leq +\infty\)
        \item \(g \leq f, f\) --- измеримая, \(g\) --- измеримая \( \Rightarrow \int_X g \leq \int_X f\)
    \end{itemize}
\end{prop}

\begin{definition}[3]\itemfix
    \begin{itemize}
        \item \(f\) измеримо
        \item \(\int f^{+}\) или \(\int f^{-}\) конечен
    \end{itemize}

    \[\int_X f d\mu = \int_X f^{+} d\mu - \int_X f^{-} d\mu\]

    Требование о конечности необходимо для избегания неопределенностей.
\end{definition}

\begin{theorem}[Тонелли]\itemfix
    \begin{itemize}
        \item \(f : \R^{m + n} \to \overline \R\)
        \item \(f \geq 0\)
        \item \(f\) измерима
        \item Записывается как \(f(x, y)\), где \(x\in\R^m, y\in\R^n\)
        \item \(E\subset \R^{m + n}\)
    \end{itemize}

    \begin{obozn}
        \[\forall x\in\R^{m + n} \ \ E_x : = \{y\in\R^n : (x, y)\in E\} \]
        % TODO: #71 иллюстрация
    \end{obozn}

    Тогда:
    \begin{enumerate}
        \item При почти всех \(x\in \R^m\) функция \(y \mapsto f(x,y)\) измерима на \(\R^n\)
        \item Функция \(x \mapsto \int_{E_x} f(x, y) d\lambda_n(y) \geq 0\), измерима и корректно задана.
        \item \[\int_E f(x, y) d\mu = \int_{\R^m} \left( \int_{E_x} f(x, y) d\lambda_n(y) \right) d\lambda_m (x)\]
    \end{enumerate}
\end{theorem}

\begin{remark}
    Неформально говоря, можно разбить \(\R^{m + n}\) на \(\R^{m}\) и \(\R^n\) и интегрировать сначала по одной переменной, потом по другой.
\end{remark}

\chapter{22 февраля}

\begin{definition}
    Если оказалось, что \(\int_X f^{ +}, \int_X f^{ -}\) оба конечны, то \(f\) называется \textbf{суммируемой}.
\end{definition}

\begin{remark}\itemfix
    \begin{enumerate}
        \item Если \(f\) измеримо и \( \geq \), то интеграл определения 3 = интегралу определения 2.
    \end{enumerate}
\end{remark}

\begin{definition}[4]\itemfix
    \begin{itemize}
        \item \(E\subset X\) --- измеримо
        \item \(f\) измеримо на \(X\)
    \end{itemize}
    \[\int_E f d\mu : = \int_X f \cdot \chi_E\]
\end{definition}

\begin{remark}\itemfix
    \begin{itemize}
        \item \(f = \sum \alpha_k \chi_{E_k} \Rightarrow \int_E f = \sum \alpha_k \mu (E_k\cap E)\)
        \item \(\int_E f d\mu = \sup \{\int_E g : 0 \leq g \leq f \text{ на } E, g \text{ --- ступ.}\} \) и мы считаем, что \(g \equiv 0\) вне \(E\).
        \item \(\int_E f\) не зависит от значений \(f\) вне множества \(E\).
    \end{itemize}
\end{remark}

\begin{prop}\itemfix
    \((X, \mathfrak{A}, \mu)\) --- пространство с мерой, \(E\subset X\) --- измеримо, \(g, f\) --- измеримо.
    \begin{enumerate}
        \item Монотонность \(f \leq g : \int_E f \leq \int_E g\)

              \begin{proof}\itemfix
                  \begin{enumerate}
                      \item При \(f, g \geq 0\) --- очевидно из определения.
                      \item При произвольных \(f, g\ \) \(f^{ +} \leq g^{ +}\) и \(f^{ -} \geq g^{ -}\) \textit{(очевидно из определения)}. Из предыдущего случая \(\int_E f^{ +} \leq \int_E g^{ +}, \int_E f^{ -} \geq \int_E g^{ -}\).
                  \end{enumerate}
              \end{proof}

        \item \(\int_E 1 d\mu = \mu E, \int_E 0 d\mu = 0\)
        \item \(\mu E = 0 \Rightarrow \int_E f = 0\)

              \begin{proof}\itemfix
                  \begin{enumerate}
                      \item \(f\) --- ступ. Тривиально.
                      \item \(f\) --- измеримо, \(f \geq 0\). \(\sup 0 = 0\), поэтому искомое выполнено.
                      \item \(\int f^{ +}, \int f^{ -} = 0 \Rightarrow \int f = 0\)
                  \end{enumerate}
              \end{proof}

              \begin{remark}
                  \(f\) --- измерима. Тогда \(f\) суммируема \(\Leftrightarrow \int |f| < +\infty\)
              \end{remark}
              \begin{proof}\itemfix
                  \begin{itemize}
                      \item [ \( \Leftarrow \)] следует из \(f^{ +}, f^{ -} \leq |f|\)
                      \item [ \( \Rightarrow \)] будет доказано позже на этой лекции. \label{доказано позже}
                  \end{itemize}
              \end{proof}

        \item \(\int_E ( - f) = - \int_E f, \forall c\in\R \ \ \int_E cf = c\int_E f\)

              \begin{proof}\itemfix
                  \begin{enumerate}
                      \item \(( - f)^{ +} = f^{ -}, ( - f)^{ -} = f^{ +}\), тогда искомое очевидно.
                      \item Можно считать \(c > 0\) без потери общности, тогда для \(f \geq 0\) тривиально.
                  \end{enumerate}
              \end{proof}

        \item \(\exists \int_E f d\mu\). Тогда \(|\int_E f d\mu| \leq \int_E |f|d\mu\)

              \begin{proof}
                  \[ -|f| \leq f \leq |f|\]
                  \[ -\int |f| \leq \int f \leq \int |f|\]
                  \[ \left|\int f\right| \leq \int |f| \]
              \end{proof}

        \item \(\mu E < +\infty, a \leq f \leq b\). Тогда
              \[a\mu E \leq \int_E f \leq b\mu E\]

              \begin{corollary}
                  \(f\) --- измеримо на \(E\), \(f\) --- ограничено на \(E\), \(\mu E < +\infty\). Тогда \(f\) суммируемо на \(E\)
              \end{corollary}

        \item \(f\) суммируема на \(E\). Тогда \(f\) почти везде конечна.
              \begin{proof}\itemfix
                  \begin{enumerate}
                      \item \(f \geq 0\) и \(f = +\infty\) на \(A\subset E\). Тогда \(\int_E f \geq n \mu A \ \ \forall n\in\N\) \( \Rightarrow \mu A = 0\)
                      \item В произвольном случае аналогично со срезками.
                  \end{enumerate}
              \end{proof}
    \end{enumerate}
\end{prop}

\begin{lemma}\itemfix
    \begin{itemize}
        \item \(A = \bigsqcup\limits_{i = 1}^{+\infty} A_i\) --- измеримо
        \item \(g\) --- ступенчато
        \item \(g \geq 0\)
    \end{itemize}
    Тогда \[\int_A g d\mu = \sum_{i = 1}^{+\infty} \int_{A_i} g d\mu\]
\end{lemma}
\begin{proof}
    \begin{align*}
        \int_A g d\mu & = \sum_{\text{кон.}} \alpha_k \mu(E_k\cap A)                     \\
                      & = \sum_k \sum_i \underbrace{\alpha_k \mu(E_k\cap A_i)}_{ \geq 0} \\
                      & \symrefeq{переставлять можно} \sum_i \sum_k \dots                \\
                      & = \sum_i \int_{A_i} g d\mu
    \end{align*}

    \blfootnote{\eqref{переставлять можно}: переставлять можно, т.к. члены суммы \( \geq 0\).}
\end{proof}

\begin{theorem}\itemfix
    \begin{itemize}
        \item \(A = \bigsqcup A_i\) --- измеримо
        \item \(f : X \to \overline \R\) --- измеримо на \(A\)
        \item \(f \geq 0\)
    \end{itemize}

    Тогда \[\int_A f d\mu = \sum_{i = 1}^{+\infty} \int_{A_i} f d\mu\]
\end{theorem}
\begin{proof}\itemfix
    Докажем, что части равенства \( \leq \) и \( \geq \), тогда равенство выполнено.

    \begin{itemize}
        \item [ \( \leq \)] \(\sphericalangle g : 0 \leq g \leq f\)
              \[\int_A g \symrefeq{по лемме об интеграле} \sum\int_{A_i} g \leq \sum \int_{A_i} f\]

        \item [ \( \geq \)]
              \begin{enumerate}
                  \item \(A = A_1 \sqcup A_2\)

                        \(\sphericalangle 0 \leq g_1 \leq f \chi_{A_1}, 0 \leq g_2 \leq f \chi_{A_2}\). Пусть \(E_k\) --- совместное разбиение, у \(g_1\) коэффициенты \(\alpha_k\), у \(g_2\) : \(\beta_k\).

                        \begin{align*}
                            0 \leq g_1 + g_2                                     & \leq f \chi_A \\
                            \int_{A_1} g_1 + \int_{A_2} g_2 = \int_A (g_1 + g_2) & \leq \int_A f \\
                            \int_{A_1} f + \int_{A_2} g_2                        & \leq \int_A f \\
                            \int_{A_1} f + \int_{A_2} f                          & \leq \int_A f \\
                        \end{align*}

                  \item \(A = \bigsqcup A_i\) тривиально по индукции.
                  \item \(A = \bigsqcup_{i = 1}^n A_i \cup B_n\), где \(B_n = \bigsqcup_{i > n} A_i\)
                        \[\int_A f = \sum_{i = 1}^n \int_{A_i} f + \int_{B_n} f \geq \sum_{i = 1}^n \int A_i f\]
              \end{enumerate}
    \end{itemize}

    \blfootnote{\eqref{по лемме об интеграле}: по лемме об интеграле.}
\end{proof}

\begin{corollary}
    \(f \geq 0\) --- измеримо. Пусть \(\nu : \mathfrak{A} \to \overline \R_{ +}\) и \(\nu E : = \int_E f d\mu\). Тогда \(\nu\) --- мера.
\end{corollary}

\begin{corollary}[Счётная аддитивность интеграла]
    \(f\) суммируема на \(A = \bigsqcup A_i\) --- измеримо. Тогда
    \[\int_A f = \sum \int_{A_i} f\]
\end{corollary}
\begin{proof}
    Очевидно, если рассмотреть срезки.
\end{proof}
\begin{corollary}
    \(A\subset B, f \geq 0 \Rightarrow \int_A f \leq \int_B f\)
\end{corollary}

\subsection{Предельный переход под знаком интеграла}

Пусть \(f_n \to f\). Можно ли утверждать, что \(\int_E f_n \to \int_E f\)?

\begin{example}[контр]
    \[f_n : = \frac{1}{n} \chi_{[0, n]} \quad f \equiv 0 \quad f_n \to f \quad (\text{даже } f_n \rightrightarrows f)\]
    \[\int_\R f_n = \frac{1}{n}\lambda[0, n] = 1 \not\to 0 = \int_\R f\]
\end{example}

\begin{theorem}[Леви]\itemfix
    \label{леви}
    \begin{itemize}
        \item \((X, \mathfrak{A}, \mu)\) --- пространство с мерой
        \item \(f_n\) измеримо
        \item \(\forall n \ \ 0 \leq f_n \leq f_{n+1}\) почти везде.
        \item \(f(x) : = \lim\limits_{n \to +\infty} f_n(x)\) --- эта функция определена почти везде.
    \end{itemize}

    Тогда \[\lim_{n \to +\infty} \int_X f_n d\mu = \int_X f d\mu\]
    \begin{remark}
        \(f\) задано везде, кроме множества \(e\) меры \(0\). Считаем, что \(f = 0\) на \(e\). Тогда \(f\) измеримо на \(X\).
    \end{remark}
\end{theorem}
\begin{proof}\itemfix
    \begin{itemize}
        \item [ \( \leq \)] очевидно, т.к. \(\int f_n \leq f\) почти везде, таким образом:
              \[\int_X f_n = \int_{X\setminus e} f_n + \underbrace{\int_e f_n}_0 = \int_{X\setminus e} f_n \leq \int_{X\setminus e} f \leq \int_{X} f\]
        \item [ \( \geq \)] достаточно проверить, что \(\forall \) ступенчатой \(g : 0 \leq g < f\) выполняется следующее \(\lim \int_X f_n \geq \int_X g\)

              Сильный трюк: достаточно проверить, что \(\forall c\in(0, 1) \ \ \lim \int_X f_n \geq c \int_X g\)

              \[E_n : = X(f_n \geq cg) \quad E_1 \subset E_2 \subset \dots\]
              \(\bigcup E_n = X\), т.к. \(c < 1\)
              \[\int_X f_n \geq \int_{E_n} f_n \geq c \int_{E_n} g\]
              Тогда \(\lim \int_X f_n \geq c\cdot \lim \int_{E_n}g \symrefeq{непрерывность снизу} c\int_X g\)
    \end{itemize}

    \blfootnote{\eqref{непрерывность снизу}: по непрерывности снизу меры \(\nu : E \mapsto \int_E g\)}
\end{proof}

\begin{theorem}\itemfix
    \begin{itemize}
        \item \(f, g \geq 0\)
        \item \(f, g\) измеримо на \(E\)
    \end{itemize}
    Тогда \(\int_E f + g = \int_E f + \int_E g\)
\end{theorem}
\begin{proof}\itemfix
    \begin{enumerate}
        \item \(f, g\) --- ступенчатые, т.е. \(f = \sum \alpha_k \chi_{E_k}, g = \sum \beta_k \chi_{E_k}\)
              \[\int_E f + g = \sum (\alpha_k + \beta_k)\mu(E_k\cap E) = \sum \alpha_k \mu (E_k \cap E) + \sum \beta_k \mu (E_k \cap E) = \int_E f + \int_E g\]

        \item \(f \geq 0\), измеримо. \(\exists \text{ступ. } f_n : 0 \leq f_n \leq f_{n+1} \leq \dots \ \ \lim f_n = f\)

              \(g \geq 0\), измеримо. \(\exists \text{ступ. } g_n : 0 \leq g_n \leq g_{n+1} \leq \dots \ \ \lim g_n = g\)

              \begin{align*}
                  f_n + g_n               & \to f + g                                 \\
                  \int_E f_n + g_n        & \xrightarrow{\text{т. Леви}} \int_E f + g \\
                  \int_E f_n + \int_E g_n & \to \int_E f + \int_e g                   \\
              \end{align*}
    \end{enumerate}
\end{proof}

\begin{corollary}
    \(f, g\) суммируемы на \(E\). Тогда \(f + g\) суммируемо и \(\int_{E} f + g = \int_E f + \int_E g\). Таким образом, доказано~\ref{доказано позже}.
\end{corollary}

\begin{proof}[Доказательство суммируемости]
    \(|f + g| \leq |f| + |g|\). Пусть \(h = f + g\). Тогда
    \begin{align*}
        h^{ +} - h^{ -}                                & = f^{ +} - f^{ -} + g^{ +} - g^{ -}                               \\
        h^{ +} + f^{ -} + g^{ - }                      & = f^{ + } + g^{ +} + h^{ -}                                       \\
        \int_E h^{ +} + \int_E f^{ -} + \int_E g^{ - } & = \int_E f^{ +} + \int_E g^{ +} + \int_E h^{ - }                  \\
        \int_E h^{ +} - \int_E f^{ - }                 & = \int_E f^{ +} + \int_E g^{ +} - \int_E f^{ - } - \int_E g^{ - } \\
    \end{align*}
\end{proof}
\begin{definition}
    \(\mathcal{L}(X)\) --- множество суммируемых функций на \(X\)
\end{definition}
\begin{corollary}[следствия]
    \(\mathcal{L}(X)\) --- линейное пространство, а отображение \(f \mapsto \int_X f\) это линейный функционал\footnote{т.е. функция функций} на \(\mathcal{L}(X)\)
    , т.е. \(\forall f_1 \dots f_n \in \mathcal{L}(X) \ \ \forall \alpha_1 \dots \alpha_n \in\R\)

    \? % TODO: #76 дописать
\end{corollary}

\begin{theorem}[об интегрировании положительных рядов]\itemfix
    \begin{itemize}
        \item \((X, \mathfrak{A}, \mu)\) --- пространство с мерой
        \item \(E\in \mathfrak{A}\)
        \item \(u_n : X \to \overline\R\)
        \item \(u_n \geq 0\) почти везде
        \item \(u_n\) измеримо
    \end{itemize}

    Тогда
    \[\int_E \left( \sum_{n = 1}^{+\infty} u_n(x) \right) d\mu = \sum_{n = 1}^{+\infty} \int_E u_n d\mu\]
\end{theorem}
\begin{proof}
    По теореме Леви:

    \[S_n : = \sum_{k = 1}^n u_k \quad 0 \leq S_n \leq S_{n+1} \leq \dots \]
    Пусть \(S_n \to S\). Тогда \(\int_E S_n \to \int_E S\)

    % TODO: #77 обосновать
\end{proof}

\begin{corollary}
    \(u_n\) измеримо и
    \(\sum\limits_{n = 1}^{+\infty} \int_E |u_n| < +\infty\). Тогда ряд \(\sum u_n(x)\) абсолютно сходится при почти всех \(x\).
\end{corollary}
\begin{proof}
    \[S(x) : = \sum |u_n(x)|\]
    \[\int_E S(X) = \sum \int_E |u_n|< +\infty \Rightarrow S \text{ суммируемо} \Rightarrow S \text{ почти везде конечно}\]
\end{proof}

\begin{example}
    \(x_n\in\R\) --- произвольная последовательность, \(\sum a_n\) абсолютно сходится.

    Тогда \(\sum \cfrac{a_n}{\sqrt{|x - x_n|}}\) абсолютно сходится при почти всех \(x\).
\end{example}
\begin{proof}
    Достаточно проверить абсолютную сходимость на \([ - N, N]\) почти везде.

    \begin{align*}
        \int_{[ - N, N]} \frac{|a_n| d\lambda}{\sqrt{|x - x_n|}}
         & = \int_{ - N}^N \frac{|a_n|}{\sqrt{|x - x_n|}} dx        \\
         & = |a_n|\int_{ - N - x_n}^{N - x_n} \frac{dx}{\sqrt{|x|}} \\
         & \leq |a_n|\int_{ - N}^{N} \frac{dx}{\sqrt{|x|}}          \\
         & 4 \sqrt{N} |a_n|
    \end{align*}
\end{proof}

\chapter{1 марта}

\begin{theorem}[об абсолютной непрерывности интеграла]\itemfix
    \begin{itemize}
        \item \((X, \mathfrak{A}, \mu)\) --- пространство с мерой
        \item \(f : X \to \overline \R\)
        \item \(f\) суммируемо
    \end{itemize}

    Тогда \(\forall \varepsilon > 0 \ \ \exists \delta > 0 \ \ \forall E \text{ --- изм., } \mu E < \delta : \left|\int_E f\right|< \varepsilon\)
\end{theorem}
\begin{corollary}
    \(f\) суммируемо на \(X\), \(E_n \subset X\), тогда \(\mu E_n \to 0 \Rightarrow \int_{E_n} f \to 0\)
\end{corollary}
\begin{proof}\footnote{Теоремы, не следствия}
    \[X_n : = X(|f| \geq n)\]
    \[X_n \supset X_{n+1} \supset \dots \Rightarrow \mu\left( \bigcap X_n \right) \symrefeq{почти везде конечна} 0\]
    \blfootnote{\eqref{почти везде конечна}: Т.к. \(f\) на \(\bigcap X_n\) бесконечна и \(f\) почти везде конечна.}
    \begin{equation}
        \forall \varepsilon > 0 \ \ \exists n_\varepsilon \ \ \int_{X_{n_\varepsilon}} |f| < \frac{\varepsilon}{2} \label{непрерывность сверху меры}
    \end{equation}
    \blfootnote{\eqref{непрерывность сверху меры}: По непрерывности сверху меры \(A \mapsto \int_A |f| d\mu\)}

    Пусть \(\delta : = \frac{\varepsilon}{2n_\varepsilon} \). Тогда при \(\mu E < \delta\):
    \[\left|\int_E f\right| \leq \int_E |f| \symrefeq{оценка f} \int_{E\cap X_{n_\varepsilon}} |f| + \int_{E\cap X_{n_\varepsilon}^c} |f| \leq \int_{X_{n_\varepsilon}} |f| + \int_{E\cap X_{n_\varepsilon}^c} n_\varepsilon < \frac{\varepsilon}{2} + \underbrace{\mu E}_{\delta} \cdot n_\varepsilon \leq \varepsilon\]
    \blfootnote{\eqref{оценка f}: Т.к. \(|f|\) на \(E \cap X_{n_\varepsilon}^c\) не превосходит \(n_\varepsilon\) по построению \(X_{n_\varepsilon}\)}
\end{proof}

\begin{remark}
    Следующие два свойства не эквивалентны:
    \begin{enumerate}
        \item \(f_n \xRightarrow[\mu]{} f \xLeftrightarrow{def} \forall \varepsilon > 0 \ \ \mu X(|f_n - f| > \varepsilon) \to 0\)
        \item \(\int_X |f_n - f| d\mu \to 0\)
    \end{enumerate}

    Из 1 не следует 2: пусть \((X, \mathfrak{A}, \mu) = (\R, \mathfrak{M}, \lambda), f_n = \frac{1}{nx}\). Тогда \(f_n \xRightarrow{\lambda} 0\), но \(\int |f_n - f| = +\infty\) при всех \(n\).

    Из 2 следует 1, т.к.
    \[\mu \underbrace{X(|f_n - f|> \varepsilon)}_{X_n} = \int_{X_n} 1 \leq \int_{X_n} \frac{|f_n - f|}{\varepsilon} = \frac{1}{\varepsilon} \int_{X_n} |f_n - f| \leq \frac{1}{\varepsilon} \int_X |f_n - f| \xrightarrow{n \to +\infty} 0\]
\end{remark}

\begin{theorem}[Лебега о предельном переходе под знаком интеграла]\itemfix
    \label{лебега}
    \begin{itemize}
        \item \((X, \mathfrak{A}, \mu)\) --- пространство с мерой
        \item \(f_n, f\) --- измеримо и почти везде конечно
        \item \(f_n \xRightarrow{\mu} f\)
        \item \(\exists g\), называемое ``суммируемая мажоранта'':
              \begin{enumerate}
                  \item \(\forall n \ \ |f_n| \symref{``1''}{\leq} g\) почти везде
                  \item \(g\) --- суммируемо на \(X\)
              \end{enumerate}
    \end{itemize}

    Тогда: \(f_n, f\) --- суммируемы и \(\int_X |f_n - f| d\mu \xrightarrow{n \to +\infty} 0\), и тем более \(\int_X f_n d\mu \to \int_X f d\mu\)
\end{theorem}
\begin{remark}
    Почти везде конечность \(f_n\) и \(f\) следует из~\eqref{``1''}, поэтому в условии этого можно не требовать.
\end{remark}
\begin{proof}
    \(f_n\) --- суммируемы в силу неравенства~\eqref{``1''}, \(f\) суммируемо в силу следствия теоремы Рисса, тем более \(|\int_X f_n - \int_X f| \leq  \int_X |f_n - f| \to 0\)

    \begin{enumerate}
        \item \(\mu X < +\infty\)

              Зафиксируем \(\varepsilon\). \(X_n : = X(|f_n - f| > \varepsilon)\)

              \(f_n \Rightarrow f\), т.е. \(\mu X_n \to 0\)

              \begin{equation}
                  |f_n - f| \leq |f_n| + |f| \leq 2g \label{|f_n - f| < 2g}
              \end{equation}
              \[\int_X |f_n - f| = \int_{X_n} + \int_{X_n^c} = \underbrace{\int_{X_n} 2g}_{\xrightarrow[\text{сл. т. об абс. непр.}]{n \to +\infty} 0} + \int_{X_n^c} \varepsilon d\mu < \varepsilon + \varepsilon \mu X\]

        \item \(\mu X = +\infty\)

              Утверждение: \(\forall \varepsilon > 0 \ \ \exists A \subset X, \text{изм., конечной меры}, \mu A \text{ конечно : } \int_{X\setminus A} g < \varepsilon\). Докажем его.

              \[\int_X g = \sup \left\{\int g_n, 0 \leq g_n \leq g, g_n \text{ --- ступ.}\right\} \]
              \[A : = \{x : g_n(x) > 0\}\]
              \[0 \leq \int_X g - \int_X g_n = \int_A g - g_n + \int_{X\setminus A} g < \varepsilon\]
              \[\int_X |f_n - f| d\mu = \int_A + \int_{X\setminus A} \leq \underbrace{\int_A |f_n - f|}_{\substack{ \to 0 \\ \text{по случаю 1}}} + \underbrace{\int_{X\setminus A} 2g}_{ < 2\varepsilon} < 3 \varepsilon\]
    \end{enumerate}
\end{proof}

\begin{theorem}[Лебега]\itemfix
    \begin{itemize}
        \item \((X, \mathfrak{A}, \mu)\) --- пространство с мерой
        \item \(f_n, f\) --- измеримо
        \item \(f_n \symref{fn to f}{\to} f\) почти везде
        \item \(\exists g\), называемое ``суммируемая мажоранта'':
              \begin{enumerate}
                  \item \(\forall n \ \ |f_n| \leq g\) почти везде
                  \item \(g\) --- суммируемо на \(X\)
              \end{enumerate}
    \end{itemize}

    Тогда \(f_n, f\) --- суммируемы, \(\int_X |f_n - f| d\mu \to 0\), и тем более \(\int_X f_n \to \int_X f\)
\end{theorem}

\begin{proof}
    Суммируемость \(f_n, f\), а также утверждение ``и тем более'' доказываются так же, как в теореме \nameref{лебега}.

    \[h_n : = \sup (|f_n - f|, |f_{n+1} - f|, |f_{n + 2} - f|, \dots)\]
    \[0 \symref{тривиально неотрицательно}{ \leq } h_n \symref{по 2g}{\leq} 2g\]

    \blfootnote{\eqref{тривиально неотрицательно}: по построению}
    \blfootnote{\eqref{по 2g}: по~\eqref{|f_n - f| < 2g}}

    \(h_n\) монотонно убывает, что очевидно по определению \(\sup\).
    \blfootnote{\eqref{по fn to f}: по~\eqref{fn to f}}
    \[\lim h_n \defeq \overline\lim |f_n - f| \symrefeq{по fn to f} 0 \text{ почти везде}\]

    \(2g - h_n \geq 0\) и возрастает как последовательность функций, \(2g - h_n\to 2g\) почти везде. Тогда по теореме \nameref{леви}:
    \[\int_X 2g - h_n \to \int_X 2g \Rightarrow \int_X h_n \to 0\]
    \[\int_X |f_n - f| \leq \int_X h_n \to 0\]
\end{proof}

\begin{example}
    \(\sphericalangle x > 0, x_0 > 0\)
    \[\int_0^{+\infty} t^{x - 1}e^{ - t} dt\]
    \[\lim_{x \to x_0} \int_0^{+\infty} t^{x - 1}e^{ - t} dt \stackrel{?}{=} \int_0^{+\infty} t^{x_0 - 1}e^{ - t} dt\]
    Равенство выполнено, т.к. \(t^{x - 1}e^{ - t} \xrightarrow{x \to x_0} t^{x_0 - 1}e^{ - t}\) при \(t > 0\) и суммируемая мажоранта \(t^{\alpha - 1}e^{ - t} + t^{\beta - 1}e^{-t}\), где \(0 < \alpha < x_0, 0 < \beta\)
\end{example}

\begin{theorem}[Фату]\itemfix
    \label{фату}
    \begin{itemize}
        \item \(X, \mathfrak{A}, \mu\) --- пространство с мерой
        \item \(f_n \geq 0\)
        \item \(f_n\) измеримо
        \item \(f_n \to f\) почти везде
        \item \(\exists C > 0 \ \ \forall n \ \ \int_X f_n \leq C\)
    \end{itemize}

    Тогда \(\int_X f \leq C\)
\end{theorem}

\begin{remark}
    Странность: здесь не требуется, чтобы \(\int_X f_n \to \int_X f\) и это может быть неверно.
\end{remark}
\begin{example}
    \[f_n = \frac{1}{n}\chi_{[0, n]} \to 0 = f \text{ п.в.} \quad \int_\R f_n = 1 \leq 1\]
    По теореме \nameref{фату} \(\int_\R f \leq 1\), что верно, т.к. \(\int_\R f = 0 \leq 1\)
\end{example}
\begin{example}
    Условие \(f_n \geq 0\) важно:

    \[f_n = -\frac{1}{n}\chi_{[0, n]} \to 0 = f \text{ п.в.} \quad \int_\R f_n = - 1 \leq -1 \text{, но } \int_\R f = 0 \not \leq - 1\]
\end{example}

\begin{proof}
    \[g_n : = \inf (f_n, f_{n+1}, \dots)\]
    \[0 \leq g_n \leq g_{n+1}\]
    \[\lim g_n \defeq \underline{\lim} f_n = f \text{ п.в.}\]
    \begin{equation}
        \int_X g_n \leq \int_X f_n \leq C \label{неравенство}
    \end{equation}
    \[\int_X g_n \symref{по леви}{ \to } \int_X f\]
    \blfootnote{\eqref{по леви}: по \nameref{леви}}
    Значит \(\int_X f \leq C\) по предельному переходу в~\eqref{неравенство}
\end{proof}

\begin{corollary}\itemfix
    \begin{itemize}
        \item \(f_n, f \geq 0\)
        \item \(f_n, f\) измеримы
        \item \(f_n, f\) почти везде конечны
        \item \(f_n \Rightarrow f\)
        \item \(\exists C > 0 \ \ \forall n \ \ \int_X f_n \leq C\)
    \end{itemize}

    Тогда \(\int_X f \leq C\)
\end{corollary}
\begin{proof}
    \[f_n \Rightarrow f \implies \exists n_k : f_{n_k} \to f \text{ п.в.}\]
    По теореме \nameref{фату} получим искомое.
\end{proof}

\begin{corollary}\itemfix
    \begin{itemize}
        \item \(f_n \geq 0\)
        \item \(f_n\) измеримо
    \end{itemize}

    Тогда \(\int_X \underline\lim f_n \leq \underline\lim \int_X f_n\)
\end{corollary}
\begin{proof}
    Возьмём~\eqref{неравенство} как в теореме. Выберем \(n_k : \int_X f_{n_k} \xrightarrow{n \to +\infty} \underline\lim \int_X f_n\)
    \[
        \begin{tikzcd}[ampersand replacement=\&, column sep=small]
            \int_X g_{n_k} \arrow[swap,shift right=2em]{d}{} \leq \int_X f_{n_k} \\
            \int_X \underline\lim f_n \leq \underline\lim \int_X f_n
        \end{tikzcd}
    \]
\end{proof}

\section{Плотность одной меры по отношению к другой. Замена переменных в интеграле.}

\(\sphericalangle (X, \mathfrak{A}, \mu)\) --- пространство с мерой, \((Y, \mathfrak{B}, \text{\textvisiblespace}), \Phi : X \to Y\)

Пусть \(\Phi\) --- измеримо в следующем смысле:
\[\Phi^{ - 1}(\mathfrak{B})\subset \mathfrak{A}\]

\begin{exercise}
    Проверить, что \(\Phi^{-1}\) --- \(\sigma\)-алгебра. % TODO #89
\end{exercise}

Для \(E\in \mathfrak{B}\) положим \(\nu(E) = \mu\Phi^{-1}(E)\). Тогда \(\nu\) --- мера:
\[\nu\left(\bigsqcup E_n\right) = \mu(\Phi^{-1}\left(\bigsqcup E_n\right)) = \mu\left(\bigsqcup \Phi^{-1}(E_n)\right) = \sum \mu \Phi^{-1} E_n = \sum \nu E_n\]

Мера \(\nu\) называется \textbf{образом} \(\mu\) при отображении \(\Phi\) и \(\nu E = \int_{\Phi^{-1}(E)} 1 d\mu\)

\begin{observation}
    \label{об измеримости}
    \(f : Y \to \overline\R\) --- измеримо относительно \(\mathfrak{B}\). Тогда \(f \circ \Phi\) --- измеримо относительно \(\mathfrak{A}\).
\end{observation}

\[X(f(\Phi(x)) < a) = \Phi^{-1}(Y(f < a)) \symref{брух}{\in} \mathfrak{A}\]
\blfootnote{\eqref{брух}: т.к. \(Y(f < a)\in \mathfrak{B}\)}

\begin{definition}
    \(\omega : X \to \overline\R, \omega \geq 0\), измеримо на \(X\).
    \[\forall B\in \mathfrak{B} \ \ \nu(B) : = \int_{\Phi^{-1}(B)}\omega(x)d\mu(x)\]
    Тогда \(\nu\) называется ``\textbf{взвешенный образ меры \(\mu\)}'', \(\omega\) называется \textbf{весом}.
\end{definition}

\begin{theorem}[о вычислении интеграла по взвешенному образу меры]\itemfix
    \label{о вычислении интеграла по взвешенному образу меры}
    \begin{itemize}
        \item \((X, \mathfrak{A}, \mu)\) --- пространство с мерой
        \item \((Y, \mathfrak{B}, \nu)\) --- пространство с мерой
        \item \(\Phi : X \to Y\)
        \item \(\omega \geq 0\)
        \item \(\omega\) измеримо на \(X\)
        \item \(\nu\) взвешенный образ \(\mu\) при отображении \(\Phi\) с весом \(\omega\)
    \end{itemize}

    Тогда \(\forall\) измеримой относительно \(\mathfrak{B}\) \(f\) на \(Y, f \geq 0\) выполнено следующее:
    \begin{enumerate}
        \item \(f\circ \Phi\) измеримо на \(X\) относительно \(\mathfrak{A}\)
        \item \begin{equation}
                  \int_Y f(y) d \nu(y) = \int_X f(\Phi(x))\cdot \omega(x) d\mu(x) \label{доказываемый интеграл}
              \end{equation}
    \end{enumerate}

    То же самое верно для суммируемой \(f\).
\end{theorem}

\begin{proof}
    Измеримость \(f\circ \Phi\) выполнена по наблюдению~\ref{об измеримости}.

    \begin{enumerate}
        \setcounter{enumi}{-1}
        \item Пусть \(f = \chi_B, B \in \mathfrak{B}\)

              \[(f \circ \Phi)(x) = f(\Phi(x)) = \begin{cases}
                      1, & \Phi(x) \in B     \\
                      0, & \Phi(x) \not\in B
                  \end{cases} = \chi_{\Phi^{-1}(B)}\]

              Тогда~\eqref{доказываемый интеграл} это:
              \[\mu B \stackrel{?}{ =} \int_X \chi_{\Phi^{-1}(B)} \cdot \omega d\mu = \int_{\Phi^{-1}(B)} \omega d\mu\]
              Это выполнено по определению \(\mu B\)

        \item Пусть \(f\) --- ступенчатая

              \eqref{доказываемый интеграл} следует из линейности интеграла.

        \item Пусть \(f \geq 0\), измеримая

              По теореме \nameref{характеризация измеримых функций с помощью ступенчатых} и теореме \nameref{леви} \(\exists \{h_i\} : 0 \leq h_1 \leq h_2 \leq \dots \) --- ступенчатые, \(h_i \leq f, h_i \to f\)
              \[\int_Y h_i d\nu = \int_X h_i \circ \Phi \cdot \omega d \mu \xrightarrow{i \to +\infty}~\eqref{доказываемый интеграл}\]

        \item Пусть \(f\) измерима.

              Тогда для \(|f|\) выполнено~\eqref{доказываемый интеграл}; \(|f|\) и \(|f\circ \Phi|\cdot \omega\) суммируемы одновременно.

              \[(f \circ \Phi\cdot \omega)_+ = f_+ \circ \Phi \cdot \omega \quad (f \circ \Phi\cdot \omega)_+ = f_+ \circ \Phi \cdot \omega\]

              Таким образом, искомое выполнено для \(f_+\) и \(f_-\), а следовательно и для \(f\).
    \end{enumerate}
\end{proof}

\begin{corollary}[об интегрировании по подмножеству]
    В условиях теоремы пусть:
    \begin{itemize}
        \item \(B\in \mathfrak{B}\)
        \item \(f\) суммируемо на \(Y\)
    \end{itemize}

    Тогда
    \[\int_B f d \nu = \int_{\Phi^{-1}(B)} f(\Phi(x)) \omega d\mu\]
\end{corollary}
\begin{proof}
    В условие теоремы подставим \(f \cdot \chi_B\)
\end{proof}

\begin{definition}
    Рассмотрим частный случай: \(X = Y, \mathfrak{A} = \mathfrak{B}, \Phi = \text{id}\). Кажется, что мы убили всю содержательность, но это не так --- есть ещё \(\omega\).
    \[\nu(B) = \int_B \omega(x)d\mu\]
    В этой ситуации \(\omega\) называется \textbf{плотностью} меры \(\nu\) относительно меры \(\mu\) и тогда по теореме~\nameref{о вычислении интеграла по взвешенному образу меры}:
    \[\int_X f d\nu = \int_X f(x) \omega(x) d\mu\]
\end{definition}

\chapter{15 марта}

\begin{definition}\itemfix
    \begin{itemize}
        \item \(X, \mathfrak{A}, \mu\) --- пространство с мерой
        \item \(\nu : \mathfrak{A} \to \overline \R\) --- мера
    \end{itemize}

    \textbf{Плотность меры} \(\nu\) относительно \(\mu\) есть положительная измеримая функция \(\omega : X \to \overline \R\), такая что:
    \[\forall B\in \mathfrak{A} \ \ \nu B = \int_B \Omega d\mu\]
\end{definition}

\begin{theorem}[критерий плотности]\itemfix
    \label{критерий плотности}
    \begin{itemize}
        \item \(X, \mathfrak{A}, \mu\) --- пространство с мерой
        \item \(\nu\) --- мера
        \item \(\omega : X \to \overline\R\)
        \item \(\omega \geq 0\)
        \item \(\omega\) измеримо
    \end{itemize}

    Тогда \(\omega\) --- плотность \(\nu\) относительно \(\mu \Leftrightarrow\):
    \[\forall A\in \mathfrak{A} \ \ \mu A \cdot \inf_A \omega \leq \nu(A) \leq \mu A \sup_A \omega\]
    При этом \(0\cdot \infty\) считается \( = 0\).
\end{theorem}

\begin{example}[отсутствие плотности]
    \(X = \R, \mathfrak{A} = \mathfrak{M}^1, \mu = \lambda_1\)

    \(\nu\) --- одноточечная мера: \(\nu(A) =\begin{cases}
        1, & 0\in A     \\
        0, & 0\not\in A
    \end{cases}\)

    Необходимое условие существования плотности --- \(\mu A = 0 \Rightarrow \nu A = 0\)

    Это и достаточное условие по теореме Радона-Никодима\footnote{Возможно, мы разберём её в конце семестра.}.
\end{example}

\begin{proof}[Доказательство теоремы~\nameref{критерий плотности}]\itemfix
    \begin{itemize}
        \item [``\( \Rightarrow \)''] Очевидно.
        \item [``\( \Leftarrow \)''] Рассмотрим \(\omega > 0\). Общность не умаляется, т.к. пусть \(e = X(\omega = 0)\), тогда \(\nu(e) \defeq \int_e \omega d\mu = 0\), поэтому в случае \(A\cap e \neq \emptyset\) всё ещё только лучше.

              Фиксируем число \(q\in(0, 1)\).
              \[A_j : = A(q^j \leq \omega < q^{j - 1}), j\in\Z\]
              \[A = \bigsqcup_{j\in \Z} A_j\]
              \[\mu A_j \cdot q^j \symref{неравенство 11}{\leq} \nu A_j \symref{неравенство 12}{\leq} \mu A_j \sup_{A_j} q^{j - 1}\]
              \[\mu A_j \cdot q^j  \symref{неравенство 21}{\leq} \int_{A_j} \omega d\mu \symref{неравенство 22}{\leq} \mu A_j q^{j - 1}\]
              Тогда:
              \begin{align*}
                  q\cdot \int_A \omega d\mu & \leq q\cdot \sum \int_{A_j} \omega d\mu                            \\
                                            & \symref{по неравенство 22}{\leq} \sum q^j \mu A_j                  \\
                                            & \symref{по неравенство 11}{\leq} \underbrace{\sum \nu A_j}_{\nu A} \\
                                            & \symref{по неравенство 12}{\leq} \frac{1}{q} \sum q^j \mu A_j      \\
                                            & \symref{по неравенство 21}{\leq} \frac{1}{q}\sum \int_{A_j}        \\
                                            & = \frac{1}{q}\int_A \omega d\mu
              \end{align*}
              \blfootnote{\eqref{по неравенство 11}: по~\eqref{неравенство 11}}
              \blfootnote{\eqref{по неравенство 12}: по~\eqref{неравенство 12}}
              \blfootnote{\eqref{по неравенство 21}: по~\eqref{неравенство 21}}
              \blfootnote{\eqref{по неравенство 22}: по~\eqref{неравенство 22}}

              То есть:
              \[q \int_A \omega d\mu \leq \nu A \leq \frac{1}{q}\int_A \omega d\mu\]
              Тогда предельный переход при \(q \to 1 - 0\) дает искомое.
    \end{itemize}
\end{proof}

\begin{lemma}\itemfix
    \begin{itemize}
        \item \(f, g\) суммируемы
        \item \((X, \mathfrak{A}, \mu)\) --- пространство с мерой
        \item \(\forall A\in \mathfrak{A} \ \ \int_A f = \int_A g\)
    \end{itemize}

    Тогда \(f = g\) почти везде.
\end{lemma}

\begin{proof}
    \(h: = f - g\). Дано: \(\forall A \ \ \int_A h = 0\); доказать --- \(h = 0\) почти везде.

    \[A_{ +} : = X(h \geq 0) \quad A_{ - } : = X(h < 0) \quad X = A_{ +} \sqcup A_{ -}\]
    \[\int_{A_{ +}} |h| = \int_{A_{ +}} h = 0 \quad \int_{A_{ -}} |h| = -\int_{A_{ -}} h = 0 \implies \int_X |h| = 0 \implies h = 0 \text{ п.в.}\]
\end{proof}

\begin{remark}
    Если \(\mathcal{L}(X)\) --- линейное пространство, отображение \(l_A : f \mapsto \int_A f\) есть линейный функционал. Таким образом, множество функционалов \(\{l_A, A \in \mathfrak{A}\}\) разделяет точки, т.е. \(\forall f \neq g \in \mathcal{L}(X) \ \ \exists A : l_A(f) \neq l_A(g)\)
\end{remark}

\begin{remark}
    В \(\R^m\) \(a = (a_1 \dots a_m)\), \(l_a : x \mapsto a_1x_1 + \dots + a_n x_n\). Тогда \(\forall x, y\in\R^m \ \ \exists a : l_a(x) =\footnote{Кажется, здесь должно быть ``\(\neq\)''}\ l_a(y)\).
\end{remark}

\section{Возвращаемся в \(\R^m\)}

\begin{lemma}[о мере образа малых кубических ячеек]\itemfix
    \label{о мере образа малых кубических ячеек}
    \begin{itemize}
        \item \(\Phi : O \subset \R^m \to \R^m\)
        \item \(O\) открыто
        \item \(a\in O\)
        \item \(\Phi\in C^1\)
        \item \(c > |\det \Phi'(a)| \neq 0\)
    \end{itemize}

    Тогда \(\exists \delta > 0 \ \ \forall \text{ куба } Q\subset B(a, \delta), a\in Q\) выполняется неравенство \(\lambda \Phi(Q) < c \lambda Q\)

    \begin{remark}
        Здесь можно считать, что \(Q\) --- замкнутые кубы.
    \end{remark}
\end{lemma}
\begin{proof}
    \(L: = \Phi'(a)\) --- обратимо\footnote{т.к. \(\det \Phi'(a) \neq 0\)}
    \[\Phi(x) = \Phi(a) + L (x - a) + o(x - a)\]
    \[\underbrace{a + L^{ - 1}(\Phi(x) - \Phi(a))}_{\Psi(x)} = x + o\footnote{Это не то же самое \(o\), что строчкой выше.}(x - a)\]
    \[\forall \varepsilon > 0 \ \ \exists \text{ шар } B_{\varepsilon\footnotemark}(a) \ \ \forall x \in B_{\varepsilon}(a) \ \ |\Psi(x) - x| < \frac{\varepsilon}{\sqrt{m}}|x - a|\]
    \footnotetext{Это не радиус шара, а параметр.}

    Пусть \(Q \subset B_\varepsilon(a), a \in Q, Q\) --- куб со стороной \(h\).

    При \(x\in Q\):

    \[|x - a| \leq \sqrt{m}h\]
    \[|\Psi(x) - x| \symref{по определению шара}{<} \frac{\varepsilon}{\sqrt{m}} |x - a| \leq \varepsilon h\]
    \blfootnote{\eqref{по определению шара}: т.к. \(x\in B_\varepsilon(a)\)}

    Тогда \(\Psi(Q) \subset \) куб со стороной \((1 + 2\varepsilon)h\), т.к. при \(x,y\in Q\)
    \begin{align*}
        |\Psi(x)_i - \Psi(y)_i| & \leq |\Psi(x)_i - x_i| + |x_i - y_i| + |\Psi(y)_i - y_i| \\
                                & \leq |\Psi(x) - x| + h + |\Psi(y) - y|                   \\
                                & \leq (1 + 2\varepsilon)h
    \end{align*}

    \[\lambda(\Psi(Q)) \leq (1 + 2\varepsilon)^m \cdot \lambda Q\]

    \(\Psi\) и \(\Phi\) отличаются только сдвигом и линейным отображением.
    \[\lambda \Phi(Q) = |\det L| \cdot \lambda \Psi(Q) \leq |\det L| (1 + 2\varepsilon)^m \cdot \lambda Q\]

    Выбираем \(\varepsilon\) такое, чтобы \(|\det L| (1 + 2\varepsilon)^m < c\), потом берём \(\delta =\) радиус \(B_\varepsilon(a)\)
\end{proof}

\begin{lemma}\itemfix
    \label{лемма без имени}
    \begin{itemize}
        \item \(f : O \subset \R^m \to \R\)
        \item \(O\) открыто
        \item \(f\) непрерывна
        \item \(A\) измеримо
        \item \(A \subset Q \subset \overline Q \subset O\)
        \item \(Q\) --- кубическая ячейка
    \end{itemize}

    Тогда:
    \[\inf_{\substack{G : A\subset G \\ G \text{ откр. } \subset O}} \lambda(G) \cdot \sup_G f = \lambda A \cdot \sup_A f\]
\end{lemma}
\begin{proof}
    Упражнение. % TODO: доказать
\end{proof}

\begin{theorem}\itemfix
    \label{мера лебега при диффеоморфизме}
    \begin{itemize}
        \item \(\Phi : O\subset \R^m \to \R^m\)
        \item \(\Phi\) диффеоморфизм
    \end{itemize}

    Тогда
    \[\forall A\in \mathfrak{M}^m, A \subset O \ \ \lambda \Phi(A) = \int_A |\det \Phi'(x)| d\lambda(x)\]
\end{theorem}

\begin{proof}
    \begin{obozn}\itemfix
        \begin{itemize}
            \item \(J_\Phi(x) = |\det \Phi'(x)|\)
            \item \(\nu A : = \lambda \Phi(A)\) --- мера
        \end{itemize}
    \end{obozn}

    Надо доказать, что \(J_\Phi\) --- плотность \(\nu\) относительно \(\lambda\).

    Достаточно проверить условие теоремы~\nameref{критерий плотности}, что \(\forall \) измеримого \(A\):
    \[\inf_A J_\Phi \cdot \lambda A \leq \nu(A) \symref{искомое неравенство}{\leq} \sup_A J_\Phi \cdot \lambda A\]

    Достаточно проверить только правое неравенство, т.к. левое неравенство --- правое неравенство для \(\Phi(A)\) и отображения \(\Phi^{-1}\)
    \[\inf \frac{1}{|\det (\Phi')|} \cdot \lambda \Phi(A) \leq \lambda A \]

    \begin{enumerate}
        \item Проверяем~\eqref{искомое неравенство} для случая \(A\) --- кубическая ячейка, \(A\subset\overline A \subset O\)

              От противного: \(\lambda Q \cdot \sup_Q J_\Phi < \nu (Q)\)

              Возьмём \(C > \sup_Q J_\Phi : C \cdot \lambda Q < \nu (Q)\).

              Запускаем половинное деление: режем \(Q\) на \(2^m\) более мелких кубических ячеек. Выберем ``мелкую'' ячейку \(Q_1 \subset Q : C \cdot \lambda Q_1 < \nu Q_1\). Опять делим на \(2^m\) частей, берём \(Q_2 Ж С\cdot \lambda Q_2 < \nu Q_2\) и т.д.

              \(a \in \bigcap \overline Q_i\)

              \begin{equation}
                  Q_1 \supset Q_2 \supset \dots \quad \quad \forall n \ \ C \cdot \lambda Q_n < \nu Q_n \label{свойство кубов}
              \end{equation}
              \(C > \sup_Q J_\Phi = \sup_{\overline Q} J_\Phi\), в частности \(c > |\det \Phi'(a)|\). Мы получили противоречие с леммой~\nameref{о мере образа малых кубических ячеек}: в сколько угодно малой окрестности \(a\) имеются кубы \(\overline Q_n\), где выполнено~\eqref{свойство кубов}

        \item Проверяем~\eqref{искомое неравенство} для случая \(A\) открыто.

              Это очевидно, т.к. \(A = \bigsqcup Q_j, Q_j\) --- кубическая ячейка, \(Q_j \subset \overline Q_j \subset A\)
              \begin{equation}
                  \nu A = \sum \lambda Q_j \leq \sum \mu Q_j \sup_{Q_j} J_\Phi \leq \sup_A J_\Phi \cdot \sum \mu Q_j = \sup_A J_\Phi \cdot \lambda A \label{доказательство для открытых}
              \end{equation}

        \item По лемме~\ref{лемма без имени} неравенство~\eqref{искомое неравенство} выполнено для всех измеримых \(A\):

              \(O = \bigsqcup Q_j\) --- кубы \(Q_j \subset \overline Q_j \subset O\), \(A = \bigsqcup \underbrace{A \cap Q_j}_{A_j}\)
              \[\nu A_j \leq \nu G \leq \sup_G J_\Phi \cdot \lambda G \Rightarrow \nu A_j \leq \inf_G (\sup J_\Phi \cdot \lambda G) = \sup_{A_j} f \cdot \lambda A_j\]

              Аналогично формуле~\eqref{доказательство для открытых} получаем \(\nu A \leq \sup_A f \cdot \lambda A\)
    \end{enumerate}
\end{proof}

\begin{theorem}\itemfix
    \begin{itemize}
        \item \(\Phi : O \subset \R^m \to \R^m\)
        \item \(\Phi\) дифференцируемо
    \end{itemize}

    Тогда \(\forall\) измеримой \(f \geq 0\), заданной на \(O' = \Phi(O)\):
    \[\int_{O'} f(y)d\lambda = \int_O f(\Phi(x)) \cdot J_\Phi \cdot d \lambda, J_\Phi(x) = |\det \Phi'(x)|\]

    То же самое верно для суммируемой \(f\).
\end{theorem}
\begin{proof}
    Применяем теорему~\nameref{о вычислении интеграла по взвешенному образу меры} при \(X = Y = \R^m, \mathfrak{A} = \mathfrak{B} = \mathfrak{M}^m, \mu = \lambda, \nu(A) = \lambda(\Phi(A))\):
    \[\int_B f d\nu = \int_{\Phi^{-1}B} f(\Phi(x)) \omega(x) d\mu\]

    По теореме~\ref{мера лебега при диффеоморфизме} \(\lambda(B) = \int_{\Phi^{-1}(O)} J_\Phi d \lambda\), т.е. \(\lambda\) --- взвешенный образ исходной меры по отношению к \(\Phi\)
\end{proof}

\begin{example}\itemfix
    \begin{enumerate}
        \item Полярные координаты в \(\R^2\):
              \[\Phi = \begin{cases} x = r\cos \varphi \\ y = r\sin \varphi \end{cases} \quad \Phi : \{(r, \varphi), r > 0, \varphi\in(0, 2\pi)\} \to \R^2\]
              \[\Phi' = \begin{pmatrix} \cos \varphi & - r \sin \varphi \\ \sin \varphi & r \cos \varphi \end{pmatrix} \quad \det \Phi' = r \quad J_\Phi = r \]
              \[\iint_\Omega f(x, y)d\lambda_r = \iint_{\Phi^{ - 1}(\Omega)} f(r \cos \varphi, r \sin \varphi) \cdot r d \lambda_2\]
        \item Сферические координаты в \(\R^3\):

              \[\begin{cases}
                      x = r\cos \varphi\cos \psi \\
                      y = r\sin \varphi\cos \psi \\
                      z = r\sin \psi
                  \end{cases}\]
              \[\begin{cases}
                      r > 0                 \\
                      \varphi \in (0, 2\pi) \\
                      \psi\in \left( -\frac{\pi}{2}, \frac{\pi}{2} \right)
                  \end{cases}\]

              \[\Phi' = \begin{pmatrix}
                      \cos \varphi\cos \psi  & - r\sin \varphi \cos \psi & - r\cos \varphi \sin \psi \\
                      \sin \varphi \cos \psi & r\cos \varphi\cos \psi    & - r\sin \varphi\sin \psi  \\
                      \sin \psi              & 0                         & r\cos \psi
                  \end{pmatrix} \quad J_\Phi = r^2\cos \psi\]
              \[\det \Phi' = r^2(\sin^2\psi\cos \psi + \cos^3\psi) = r^2\cos \psi\]
    \end{enumerate}
\end{example}

\chapter{22 марта}

\subsection{Сферические координаты в \(\R^m\)}

Координаты задаются \(r, \varphi_1, \varphi_2 \dots \varphi_{m - 1}\). Зададим их по индукции:

\begin{itemize}
    \item \(\varphi_1\) --- угол между \(\overline e_1\) и \(\overline{OX}\in[0, \pi]\)
    \item \(\varphi_2\) --- угол между \(\overline e_2\) и \(P_{2_{(e_2 \dots e_n)}}(x)\in[0, \pi]\)
    \item \(\vdots\)
    \item \(\varphi_{m - 1}\) --- полярный угол в \(\R^2\)
\end{itemize}

\begin{align*}
    x_1       & = r \cos \varphi_1                                                 \\
    x_2       & = r \sin \varphi_1 \cos \varphi_2                                  \\
    x_3       & = r \sin \varphi_1 \sin \varphi_2 \cos \varphi_3                   \\
    \vdots                                                                         \\
    x_{n - 1} & = r \sin \varphi_1 \dots \sin \varphi_{m - 2} \cos \varphi_{m - 1} \\
    x_n       & = r \sin \varphi_1 \dots \sin \varphi_{m - 2} \sin \varphi_{m - 1}
\end{align*}

\[J = r^{m - 1} \sin^{m - 2}\varphi_1 \sin^{m - 3}\varphi_2 \dots \sin \varphi_{m - 2}\]

\begin{remark}
    В \(\R^3\) ``географические'' координаты имеют якобиан \(J = r^2 \cos \psi\)
\end{remark}

Поймём, почему якобиан именно такой. Можно его посчитать руками, но это трудно.

\begin{itemize}
    \item [1 шаг] \begin{align*}
              x_m       & = \rho_{m - 1} \sin \varphi_{m - 1} \\
              x_{m - 1} & = \rho_{m - 1} \cos \varphi_{m - 1} \\
          \end{align*}
          \[(x_1 \dots x_m) \rightsquigarrow (x_1 \dots x_{m - 2}, \rho_{m - 1}, \varphi_{m - 1})\]
          \[J = \begin{vmatrix} E & 0 \\ 0 & J_2 \end{vmatrix} = \rho_{m - 1}\]
    \item [2 шаг] \begin{align*}
              \rho_{m - 1} & = \rho_{m - 2} \sin \varphi_{m - 2} \\
              x_{m - 2}    & = \rho_{m - 2} \cos \varphi_{m - 2} \\
          \end{align*}
          \[(x_1 \dots x_{m - 2}, \rho_{m - 1}, \varphi_{m - 1}) \rightsquigarrow (x_1 \dots x_{m - 3}, \rho_{m - 2}, \varphi_{m - 2}, \varphi_{m - 1})\]

    \item [последний шаг]
          \[(x_1 \rho_2, \varphi_2 \dots \varphi_{m - 1}) \rightsquigarrow (r, \varphi_1 \dots \varphi_{m - 1})\]
          \begin{align*}
              \rho_2 & = r \sin \varphi_1 \\
              x_1    & = r \cos \varphi_1
          \end{align*}
\end{itemize}

\begin{align*}
    \lambda_m(\Omega) & = \int_\Omega 1 d\lambda_m                                                                                        \\
                      & \stackrel{\text{1 шаг}}{=} \int_{\Omega_1} \rho_{m - 1}                                                           \\
                      & \stackrel{\text{2 шаг}}{=} \int_{\Omega_2} \rho^2_{m - 2} \sin \varphi_{m - 2}                                    \\
                      & \stackrel{\text{3 шаг}}{=} \int_{\Omega_3} \rho^3_{m - 3} \sin^2 \varphi_{m - 3} \sin \varphi_{m - 2}             \\
                      & = \dots                                                                                                           \\
                      & = \int_{\Omega_{m - 1}} r^{m - 1} \sin^{m - 2}\varphi_1 \sin^{m - 3}\varphi_2 \dots \sin \varphi_{m - 2} d\lambda
\end{align*}

Тогда по теореме о единственности плотности искомое верно.

\section{Произведение мер}

\(\sphericalangle (X, \mathfrak{A}, \mu)\), \((Y, \mathfrak{B}, \nu)\) --- пространства с мерой

\begin{lemma}
    \(\mathfrak{A}, \mathfrak{B}\) --- полукольца \( \Rightarrow \mathfrak{A} \times \mathfrak{B} = \{A \times B \subset X \times Y : A\in \mathfrak{A}, B \in \mathfrak{B}\}\) --- полукольцо.
\end{lemma}
\begin{proof}
    Тривиально. % TODO: доказать
\end{proof}

\begin{obozn}
    \(\mathcal{P} = \mathfrak{A} \times \mathfrak{B}\) --- называем \textbf{измеримыми прямоугольниками}.

    \(m_0(A \times B) = \mu(A) \cdot \nu(B)\), при этом \(0 \cdot \infty\) принимаем за \(0\).
\end{obozn}
\begin{theorem}\itemfix
    \begin{enumerate}
        \item \(m_0\) --- мера на \(\mathcal{P}\)
        \item \(\mu, \nu\) --- \(\sigma\)-конечны \( \Rightarrow m_0\) тоже \(\sigma\)-конечно.
    \end{enumerate}
\end{theorem}
\begin{proof}\itemfix
    \begin{enumerate}
        \item Проверим счётную аддитивность \(m_0\), т.е. \(m_0 P = \sum_{k = 1}^{+\infty} m_0 P_k\)\footnote{Прочие суммы/объединения также счётны в рамках данного доказательства.}, если \(A \times B = P = \bigsqcup P_k\), где \(P_k = A_k \times B_k\)

              Заметим, что \(\chi_{A \times B}(x, y) = \chi_A(x) \cdot \chi_B(y)\).

              Тогда \(\chi_P = \sum \chi_{P_k}\), где \(\forall x\in X, y\in Y \ \ \chi_A(x) \chi_B(y) = \sum \chi_{A_k}(x) \chi_{B_k}(y)\)

              Проинтегрируем по \(y\) по мере \(\nu\) по пространству \(Y\):
              \[\chi_A(x) \nu B = \sum \chi_{A_k}(x) \cdot \nu B_k\]

              Проинтегрируем по \(x\) по мере \(\mu\) по пространству \(X\):
              \[\mu A \mu B = \sum \mu A_k \nu B_k\]
              Это и есть искомое.

        \item Очевидно, т.к.:
              \begin{itemize}
                  \item \(\mu\) \(\sigma\)-конечно \( \Rightarrow X = \bigcup X_k, \mu X_k\) --- конечно \(\forall k\)
                  \item \(\nu\) \(\sigma\)-конечно \( \Rightarrow Y = \bigcup Y_n, \nu Y_n\) --- конечно \(\forall k\)
              \end{itemize}

              Тогда \(X \times Y = \bigcup X_k \times Y_n, m_0(X_k \times Y_n) = \mu X_k \nu Y_n\). Конечное произведение конечных конечно, поэтому \(m_0\) \(\sigma\)-конечно.
    \end{enumerate}
\end{proof}

\begin{definition}\itemfix
    \begin{itemize}
        \item \(\sphericalangle (X, \mathfrak{A}, \mu)\), \((Y, \mathfrak{B}, \nu)\) --- пространства с мерой
        \item \(\mu, \nu\) \(\sigma\)-конечны
    \end{itemize}

    Пусть \(m\) --- лебеговское продолжение меры \(m_0\) на \(\sigma\)-алгебру, которую будем обозначать \(\mathfrak{A} \otimes \mathfrak{B}\)\footnote{\(\otimes\) --- не тензорное произведение}

    \begin{obozn}
        \(m = \mu \times \nu\)
    \end{obozn}

    \((X \times Y, \mathfrak{A} \otimes \mathfrak{B}, \mu \times \nu)\) --- \textbf{произведение пространств с мерой} \((X, \mathfrak{A}, \mu)\) и \((Y, \mathfrak{B}, \nu)\)
\end{definition}

\begin{remark}\itemfix
    \begin{itemize}
        \item Это произведение ассоциативно.
        \item \(\sigma\)-конечность нужна для единственности произведения.
    \end{itemize}
\end{remark}

\begin{theorem}
    \(\lambda_m \times \lambda_n = \lambda_{m + n}\)
\end{theorem}
\begin{proof}
    Не будет.
\end{proof}

\begin{definition}
    \(X, Y\) --- множества, \(C \subset X \times Y\)

    \[\forall x\in X \ \ C_x : = \{y\in Y : (x, y)\in C\}\]
    \[\forall y\in Y \ \ C^y : = \{x\in X : (x, y)\in C\}\]

    \(C_x, C^y\) называется \textbf{сечением}.
\end{definition}

\begin{remark}
    \[\left( \bigcup_\alpha C_\alpha \right)_x = \bigcup (C_\alpha)_x \quad \left( \bigcap C_\alpha \right)_x = \bigcup (C_\alpha)_x \quad (C \setminus C')_x = C_x \setminus C'_x \]
\end{remark}

\begin{theorem}[принцип Кавальери\footnote{Кавальери имеет к этой теореме косвенное отношение, т.к. он жил за пару веков до появления теории меры.}]\itemfix
    \begin{itemize}
        \item \((X, \mathfrak{A}, \mu)\)
        \item \((Y, \mathfrak{B}, \nu)\)
        \item \(\mu, \nu\) --- \(\sigma\)-конечны.
        \item \(\mu, \nu\) --- полные.
        \item \(m = \mu \times \nu\)
        \item \(C\in A \otimes B\)
    \end{itemize}

    Тогда:
    \begin{enumerate}
        \item \(C_x \in \mathfrak{B}\) при почти всех \(x\)
        \item \(x \mapsto \nu(C_x)\) --- измеримая\footnote{функция задана при почти всех \(X\); она равна п.в. некоторой измеримой функции, заданной всюду} функция на \(X\)
        \item \(mC = \int_X \nu(C_x) d \mu(x)\)
    \end{enumerate}

    Аналогичное верно для \(C^y\).
\end{theorem}

\begin{example}
    \?
\end{example}

\begin{proof}
    Пусть \(\mathfrak{D}\) --- система множеств, для которых выполнено 1.-3.

    \begin{enumerate}
        \item \(C = A \times B \Rightarrow C\in \mathfrak{D}\)

              \begin{enumerate}
                  \item \(C_x = \begin{cases}
                            \emptyset, x\not\in A \\
                            B, x\in A
                        \end{cases}\)
                  \item \(x \mapsto \nu(C_x)\) --- функция \(\nu B \cdot \chi_A\)
                  \item \(\int \nu(C_x) d\mu = \int_X \nu B \cdot \chi_A d \mu = \nu B \cdot \mu A = mC\)
              \end{enumerate}

        \item \(E_i \in \mathfrak{D}\), дизъюнктны \( \Rightarrow \bigsqcup E_i\in \mathfrak{D}\). Обозначим \(E = \bigsqcup E_i\)

              \(E_i \in \mathfrak{D} \Rightarrow (E_i)_x\) измеримо почти везде \( \Rightarrow \) при почти всех \(x\) все \((E_i)_x\) измеримы.

              Тогда при ��тих \(x\) \(E_x = \bigsqcup (E_i)_x \in \mathfrak{B}\) --- это 1.

              \(\nu E_x = \sum \underbrace{\nu(E_i)_x}_{\substack{\text{измеримая} \\ \text{функция}}} \Rightarrow\) Функция \(x \mapsto \nu E_x\) измеримо --- это 2.

              \(\int_X \nu E_x d\mu = \sum_i \int_X \nu(E_i) x = \sum_i m E_i = m E\) --- это 3.

        \item \(E_i \in \mathfrak{D}, E_1 \supset E_2 \supset \dots , E = \bigcap_i E_i, \mu E_i < +\infty\). Тогда \(E\in \mathfrak{D}\).

              \(\int_X \nu(E_i)_x d\mu = m E_i < +\infty \Rightarrow \nu(E_i)_x\) --- конечно при почти всех \(x\).

              \(\forall x\) верно \((E_1)_x \supset (E_2)_x \supset \dots , E_x = \bigcap (E_i)_x\)

              Тогда \(E_x\) измеримо \textit{(это 1.)} и \(\lim_{i \to +\infty} \nu(E_i)_x = \nu E_x\) при п.в. \(x\).

              Таким образом, \(x \mapsto \nu E_x\) измерима --- это 2.

              \(\int_x \nu E_x d\mu = \lim \int_X \nu(E_i)_x d\mu = \lim m E_i = m E\) --- это 3.

              По теореме Лебега % какой?
              \(|\nu(E_i) x| \leq \nu(E_i)x\) суммируемо.

              Итого: Если \(A_{ij}\in \mathcal{P}  = \mathfrak{A} \times \mathfrak{B}\), то \(\bigsqcap \bigcup A_{ij} \in \mathfrak{D}\)

        \item \(m E = 0 \Rightarrow E \in \mathfrak{D}\)

              \(m E = \inf \left\{\sum m_0 P_k : E \subset \bigcup P_k, P_k \in \mathcal{P}\right\} \) --- из пункта 5 теоремы о лебеговском продолжении.

              \(\exists \) множество \(H\) вида \(\bigcap_l \bigsqcup_k P_{kl}\), т.е. \(H\in \mathfrak{D}\).

              \(E\subset H, m H = m E = 0\)

              \(0 = mH = \int_X \underbrace{\nu H_x}_{ \geq 0} d\mu \Rightarrow \nu H_x = 0\) про почти всех \(x\).

              \(E_x \subset H_x, \nu\) --- полная \( \Rightarrow E_x\) --- измеримо при почти всех \(x\) --- это 1 и \(\nu E_x = 0\) почти везде, это 2.

              \(\int \nu E_x d\mu = 0 = m E\) --- это 3.

        \item \(C\) --- измеримо, \(mC < +\infty\). Тогда \(C\in \mathfrak{D}\).

              \(C = H \setminus e\), где \(H\) имеет вид \(\bigsqcap \bigcup P_{k_l}, me = 0\)

              \(mC = mH\)

              \begin{enumerate}
                  \item \(C_x = H_x \setminus e_x\) --- измеримо при почти всех \(x\)
                  \item \(\nu e_x = 0\) при почти всех \(x\) \( \Rightarrow \nu C_x = \nu H_x - \nu E_x = \nu H_x \Rightarrow \) измеримо.
                  \item \(\int_X \nu C_x d\mu = \int_X \nu H_x d\mu = mH = mC\)
              \end{enumerate}

        \item \(C\) --- произвольное измеримое множество в \(X \times Y \Rightarrow C \in \mathfrak{D}\)

              \(X = \bigsqcup X_k, \mu X_k < +\infty, Y = \bigsqcup Y_j, \nu Y_j < +\infty\)

              \(C = \bigsqcup (\underbrace{C \cap (X_k \times Y_j)}_{\? < +\infty})\) \? % TODO
    \end{enumerate}
\end{proof}

\begin{corollary}
    \(C\) измеримо в \(X \times Y\). Пусть \(P_1(C) = \{x\in X, C_x \neq \emptyset\}\) --- проекция~\(C\) на~\(X\).

    Если \(P_1(C)\) измеримо, то:
    \[mC = \int_{P_1(C)} \nu(C_x) d\mu\]
\end{corollary}
\begin{proof}
    При \(x\not\in P_1(C) \ \ \nu(C_x) = 0\)
\end{proof}

\begin{remark}\itemfix
    \begin{enumerate}
        \item \(C\) измеримо \(\nRightarrow P_1(C)\) измеримо.
        \item \(C\) измеримо \(\nRightarrow \forall x \ \ C_x\) измеримо.
        \item \(\forall x, \forall y \ \ C_x, C^y\) измеримо \(\nRightarrow C\) измеримо.

              Пример Серпинского.
    \end{enumerate}
\end{remark}

\end{document}
