\chapter{30 октября}

\section{Функторы}

\begin{example}
    Рассмотрим следующую категорию:
    \[\begin{tikzcd}
            0 \arrow[loop, distance=2em, in=305, out=235, swap]{}{\id_0}
            \arrow{r} \arrow[bend left]{rr} \arrow[bend left = 60]{rrr} \arrow[bend right]{rrrr} &
            1 \arrow[loop, distance=2em, in=305, out=235, swap]{}{\id_1}
            \arrow{r} \arrow[bend left]{rr} \arrow[bend left = 60]{rrr} \arrow[bend right]{rrrr} &
            2 \arrow[loop, distance=2em, in=305, out=235, swap]{}{\id_2}
            \arrow{r} \arrow[bend left]{rr} \arrow[bend left = 60]{rrr} \arrow[bend right]{rrrr} &
            3 \arrow[loop, distance=2em, in=305, out=235, swap]{}{\id_3}
            \arrow{r} \arrow[bend left]{rr} \arrow[bend left = 60]{rrr} &
            4 \arrow[loop, distance=2em, in=305, out=235, swap]{}{\id_4}
            \arrow{r} \arrow[bend left]{rr} &
            5 \arrow[loop, distance=2em, in=305, out=235, swap]{}{\id_5}
            \arrow{r} &
            \dots
        \end{tikzcd}\]

    Стрелки можно определить как суммирование, а можно как \( \leq \), ничего не изменится.
\end{example}

Пусть мы хотим отобразить категорию \(\mathcal{C}\) в \(\mathcal{D}\). Отобразим точки в точки и стрелки в стрелки, при этом \(\arr'(\mathcal{C}, \mathcal{D})\) задает отображение точек, а \(\arr'(\arr(\mathcal{C}), \arr(\mathcal{D}))\) задает отображение стрелок.

\begin{definition}
    \textbf{Малая категория} --- категория, объекты которой образуют множество.
\end{definition}

\begin{remark}
    При этом нет ограничения на стрелки, они могут быть не множеством.
\end{remark}

\begin{example}
    Категория из одного элемента со сложением на стрелках, где стрелки есть для всех натуральных чисел, кардиналов и пределов. Все стрелки не ``помещаются'' в множество.
\end{example}

Отображать между категориями мы можем как угодно, но хотелось бы сохранять структуру. Условие сохранения композиции:
\[\forall f_{AB}, f_{BC} \in \arr \mathcal{C} : \mathcal{F}(f_{BC}) \circ \mathcal{F}(f_{AB}) = \mathcal{F}(f_{BC} \circ f_{AB})\]
, где \(\mathcal{F}\) --- отображение из категории \(\mathcal{C}\) в категорию \(\mathcal{D}\). Из предыдущего закона следует, что петли отображаются в петли, т.к.:
\[\mathcal{F}(f_{AA} \circ f_{AA}) = \mathcal{F}(f_{AA}) \circ \mathcal{F}(f_{AA})\]
таким образом, образ \(f_{AA}\) должен иметь композицию с собой \( \Rightarrow \) это петля. Но никто не гарантирует, что эта петля будет на соответствующем объекте. Введём следующий закон:
\[\mathcal{F}(\id_X) = \id_{\mathcal{F}(X)}\]

\begin{definition}
    \(\mathcal{F}\) называется \textbf{\textit{(ковариантным\footnote{Не важно, что это значит.})} функтором}
\end{definition}

\section{Моно-- и эпиморфизмы, дуальные категории}

Рассмотрим следующую категорию (\(\id\) опущены):
\[\begin{tikzcd}
        A \arrow{r}{f} \arrow[bend left = 60]{rr}{f_{AC}} & B \arrow[bend left]{r}{h_1} \arrow[bend right = 10]{r}{h_2} \arrow[bend right = 60]{r}{h_{\dots}} & C
    \end{tikzcd}\]
Если стрелку \(A \to C\) можно получить только как \(f_{AC} = f \circ h_i\), то мы хотим:
\begin{enumerate}
    \item Ввести отношение эквивалентности, чтобы не различать разные \(f_{AC}\).
    \item Назвать стрелку \(f\) \textbf{монической} \textit{(или мономорфизмом)}.
\end{enumerate}

\begin{definition}
    \textbf{Дуальная\footnote{Или зеркальная, двойственная, сопряженная, dual, opposite.} категория} для некоторой категории \(\mathcal{C}\) это категория \(\mathcal{C}^*\), что:
    \begin{enumerate}
        \item \(\obj \mathcal{C} = \obj \mathcal{C}^*\)
        \item \(\forall f_{AB} \in \arr \mathcal{C} \ \ \exists ! f_{BA} \in \arr \mathcal{C}^*\)
        \item \(\forall f_{BA} \in \arr \mathcal{C}^* \ \ \exists ! f_{AB} \in \arr \mathcal{C}\)
    \end{enumerate}
\end{definition}

Операция сопряжения является функтором, будем обозначать его \(\tau\). Она легко строится в том смысле, что для любой категории, не зная ничего про объекты, мы можем построить её путём разворота стрелок.

\begin{example}\itemfix
    \begin{figure}[h!]
        \centering
        \begin{minipage}[b]{0.45\textwidth}
            \centering
            \[\begin{tikzcd}
                    & B \arrow[loop, distance=2em, in=125, out=55, swap]{}\arrow{dr} & \\
                    A \arrow[loop, distance=2em, in=305, out=235, swap]{}\arrow{ur}\arrow{rr} & & C\arrow[loop, distance=2em, in=305, out=235, swap]{} \\
                \end{tikzcd}\]
        \end{minipage}
        \begin{minipage}[b]{0.45\textwidth}
            \centering
            \[\begin{tikzcd}
                    & B \arrow[loop, distance=2em, in=125, out=55, swap]{}\arrow{dl} & \\
                    A \arrow[loop, distance=2em, in=305, out=235, swap]{} & & \arrow[loop, distance=2em, in=305, out=235, swap]{}\arrow{ul}\arrow{ll} C \\
                \end{tikzcd}\]
        \end{minipage}
        \caption{Категория и дуальная к ней категория.}
    \end{figure}
\end{example}

\begin{definition}\itemfix
    Если \(f\) --- мономорфизмом в категории \(\mathcal{C}\). Тогда \(f^* = \tau(f)\) является \textbf{эпиморфизмом} в категории \(\mathcal{C}^*\).
\end{definition}

\begin{definition}[альтернативное]
    Если стрелку \(C \to A\) можно получить только как \(f_{CA} = h_i \circ f\), то \(f\) называется \textbf{эпиморфизмом}.
\end{definition}

\begin{remark}
    Если стрелки соответствуют отношению предпорядка ``\( \leq \)'', то в дуальной категории стрелки соответствуют отношению ``\( \geq \)''.
\end{remark}

\begin{remark}
    Категория, где стрелки соответствуют отношению предпорядка называется Poset.
\end{remark}

\begin{definition}
    Если \(\forall f_{AB} \in \arr \mathcal{C} \ \ \exists f^{-1}_{BA} : f_{AB} \circ f^{-1}_{BA} = \id_B, f^{-1}_{BA} \circ f_{BA} = \id_A\), то \(\mathcal{C}\) --- \textbf{группоид}.
\end{definition}
