\documentclass[12pt, a4paper]{article}

%<*preamble>
% Math symbols
\usepackage{amsmath, amsthm, amsfonts, amssymb}
\usepackage{accents}
\usepackage{esvect}
\usepackage{mathrsfs}
\usepackage{mathtools}
\mathtoolsset{showonlyrefs}
\usepackage{cmll}
\usepackage{stmaryrd}
\usepackage{physics}
\usepackage[normalem]{ulem}
\usepackage{ebproof}
\usepackage{extarrows}

% Page layout
\usepackage{geometry, a4wide, parskip, fancyhdr}

% Font, encoding, russian support
\usepackage[russian]{babel}
\usepackage[sb]{libertine}
\usepackage{xltxtra}

% Listings
\usepackage{listings}
\lstset{basicstyle=\ttfamily,breaklines=true}
\setmonofont{Inconsolata}

% Miscellaneous
\usepackage{array}
\usepackage{calc}
\usepackage{caption}
\usepackage{subcaption}
\captionsetup{justification=centering,margin=2cm}
\usepackage{catchfilebetweentags}
\usepackage{enumitem}
\usepackage{etoolbox}
\usepackage{float}
\usepackage{lastpage}
\usepackage{minted}
\usepackage{svg}
\usepackage{wrapfig}
\usepackage{xcolor}
\usepackage[makeroom]{cancel}

\newcolumntype{L}{>{$}l<{$}}
    \newcolumntype{C}{>{$}c<{$}}
\newcolumntype{R}{>{$}r<{$}}

% Footnotes
\usepackage[hang]{footmisc}
\setlength{\footnotemargin}{2mm}
\makeatletter
\def\blfootnote{\gdef\@thefnmark{}\@footnotetext}
\makeatother

% References
\usepackage{hyperref}
\hypersetup{
    colorlinks,
    linkcolor={blue!80!black},
    citecolor={blue!80!black},
    urlcolor={blue!80!black},
}

% tikz
\usepackage{tikz}
\usepackage{tikz-cd}
\usetikzlibrary{arrows.meta}
\usetikzlibrary{decorations.pathmorphing}
\usetikzlibrary{calc}
\usetikzlibrary{patterns}
\usepackage{pgfplots}
\pgfplotsset{width=10cm,compat=1.9}
\newcommand\irregularcircle[2]{% radius, irregularity
    \pgfextra {\pgfmathsetmacro\len{(#1)+rand*(#2)}}
    +(0:\len pt)
    \foreach \a in {10,20,...,350}{
            \pgfextra {\pgfmathsetmacro\len{(#1)+rand*(#2)}}
            -- +(\a:\len pt)
        } -- cycle
}

\providetoggle{useproofs}
\settoggle{useproofs}{false}

\pagestyle{fancy}
\lfoot{M3137y2019}
\cfoot{}
\rhead{стр. \thepage\ из \pageref*{LastPage}}

\newcommand{\R}{\mathbb{R}}
\newcommand{\Q}{\mathbb{Q}}
\newcommand{\Z}{\mathbb{Z}}
\newcommand{\B}{\mathbb{B}}
\newcommand{\N}{\mathbb{N}}
\renewcommand{\Re}{\mathfrak{R}}
\renewcommand{\Im}{\mathfrak{I}}

\newcommand{\const}{\text{const}}
\newcommand{\cond}{\text{cond}}

\newcommand{\teormin}{\textcolor{red}{!}\ }

\DeclareMathOperator*{\xor}{\oplus}
\DeclareMathOperator*{\equ}{\sim}
\DeclareMathOperator{\sign}{\text{sign}}
\DeclareMathOperator{\Sym}{\text{Sym}}
\DeclareMathOperator{\Asym}{\text{Asym}}

\DeclarePairedDelimiter{\ceil}{\lceil}{\rceil}

% godel
\newbox\gnBoxA
\newdimen\gnCornerHgt
\setbox\gnBoxA=\hbox{$\ulcorner$}
\global\gnCornerHgt=\ht\gnBoxA
\newdimen\gnArgHgt
\def\godel #1{%
    \setbox\gnBoxA=\hbox{$#1$}%
    \gnArgHgt=\ht\gnBoxA%
    \ifnum     \gnArgHgt<\gnCornerHgt \gnArgHgt=0pt%
    \else \advance \gnArgHgt by -\gnCornerHgt%
    \fi \raise\gnArgHgt\hbox{$\ulcorner$} \box\gnBoxA %
    \raise\gnArgHgt\hbox{$\urcorner$}}

% \theoremstyle{plain}

\theoremstyle{definition}
\newtheorem{theorem}{Теорема}
\newtheorem*{definition}{Определение}
\newtheorem{axiom}{Аксиома}
\newtheorem*{axiom*}{Аксиома}
\newtheorem{lemma}{Лемма}

\theoremstyle{remark}
\newtheorem*{remark}{Примечание}
\newtheorem*{exercise}{Упражнение}
\newtheorem{corollary}{Следствие}[theorem]
\newtheorem*{statement}{Утверждение}
\newtheorem*{corollary*}{Следствие}
\newtheorem*{example}{Пример}
\newtheorem{observation}{Наблюдение}
\newtheorem*{prop}{Свойства}
\newtheorem*{obozn}{Обозначение}

% subtheorem
\makeatletter
\newenvironment{subtheorem}[1]{%
    \def\subtheoremcounter{#1}%
    \refstepcounter{#1}%
    \protected@edef\theparentnumber{\csname the#1\endcsname}%
    \setcounter{parentnumber}{\value{#1}}%
    \setcounter{#1}{0}%
    \expandafter\def\csname the#1\endcsname{\theparentnumber.\Alph{#1}}%
    \ignorespaces
}{%
    \setcounter{\subtheoremcounter}{\value{parentnumber}}%
    \ignorespacesafterend
}
\makeatother
\newcounter{parentnumber}

\newtheorem{manualtheoreminner}{Теорема}
\newenvironment{manualtheorem}[1]{%
    \renewcommand\themanualtheoreminner{#1}%
    \manualtheoreminner
}{\endmanualtheoreminner}

\newcommand{\dbltilde}[1]{\accentset{\approx}{#1}}
\newcommand{\intt}{\int\!}

% magical thing that fixes paragraphs
\makeatletter
\patchcmd{\CatchFBT@Fin@l}{\endlinechar\m@ne}{}
{}{\typeout{Unsuccessful patch!}}
\makeatother

\newcommand{\get}[2]{
    \ExecuteMetaData[#1]{#2}
}

\newcommand{\getproof}[2]{
    \iftoggle{useproofs}{\ExecuteMetaData[#1]{#2proof}}{}
}

\newcommand{\getwithproof}[2]{
    \get{#1}{#2}
    \getproof{#1}{#2}
}

\newcommand{\import}[3]{
    \subsection{#1}
    \getwithproof{#2}{#3}
}

\newcommand{\given}[1]{
    Дано выше. (\ref{#1}, стр. \pageref{#1})
}

\renewcommand{\ker}{\text{Ker }}
\newcommand{\im}{\text{Im }}
\renewcommand{\grad}{\text{grad}}
\newcommand{\rg}{\text{rg}}
\newcommand{\defeq}{\stackrel{\text{def}}{=}}
\newcommand{\defeqfor}[1]{\stackrel{\text{def } #1}{=}}
\newcommand{\itemfix}{\leavevmode\makeatletter\makeatother}
\newcommand{\?}{\textcolor{red}{???}}
\renewcommand{\emptyset}{\varnothing}
\newcommand{\longarrow}[1]{\xRightarrow[#1]{\qquad}}
\DeclareMathOperator*{\esup}{\text{ess sup}}
\newcommand\smallO{
    \mathchoice
    {{\scriptstyle\mathcal{O}}}% \displaystyle
    {{\scriptstyle\mathcal{O}}}% \textstyle
    {{\scriptscriptstyle\mathcal{O}}}% \scriptstyle
    {\scalebox{.6}{$\scriptscriptstyle\mathcal{O}$}}%\scriptscriptstyle
}
\renewcommand{\div}{\text{div}\ }
\newcommand{\rot}{\text{rot}\ }
\newcommand{\cov}{\text{cov}}

\makeatletter
\newcommand{\oplabel}[1]{\refstepcounter{equation}(\theequation\ltx@label{#1})}
\makeatother

\newcommand{\symref}[2]{\stackrel{\oplabel{#1}}{#2}}
\newcommand{\symrefeq}[1]{\symref{#1}{=}}

% xrightrightarrows
\makeatletter
\newcommand*{\relrelbarsep}{.386ex}
\newcommand*{\relrelbar}{%
    \mathrel{%
        \mathpalette\@relrelbar\relrelbarsep
    }%
}
\newcommand*{\@relrelbar}[2]{%
    \raise#2\hbox to 0pt{$\m@th#1\relbar$\hss}%
    \lower#2\hbox{$\m@th#1\relbar$}%
}
\providecommand*{\rightrightarrowsfill@}{%
    \arrowfill@\relrelbar\relrelbar\rightrightarrows
}
\providecommand*{\leftleftarrowsfill@}{%
    \arrowfill@\leftleftarrows\relrelbar\relrelbar
}
\providecommand*{\xrightrightarrows}[2][]{%
    \ext@arrow 0359\rightrightarrowsfill@{#1}{#2}%
}
\providecommand*{\xleftleftarrows}[2][]{%
    \ext@arrow 3095\leftleftarrowsfill@{#1}{#2}%
}

\allowdisplaybreaks

\newcommand{\unfinished}{\textcolor{red}{Не дописано}}

% Reproducible pdf builds 
\special{pdf:trailerid [
<00112233445566778899aabbccddeeff>
<00112233445566778899aabbccddeeff>
]}
%</preamble>


\lhead{Дифференциальные уравнения}
\cfoot{}
\rfoot{9.9.2020}

\setcounter{section}{1}
\setcounter{subsection}{3}

\begin{document}

\subsection{Задача Коши}

\begin{theorem}[Пеано (\textit{существования})]\itemfix
    \begin{itemize}
        \item $G\subset \R^2$ --- область \textit{(открытое связное множество)}
        \item $f\in C(G)$
        \item $(x_0, y_0)\in G$
    \end{itemize}
    Тогда на некоторой окрестности $(a, b) \ni x_0 \ \ $ $\exists$ решение задачи Коши.
\end{theorem}
\begin{remark}
    Теорема Пеано --- более общая, дан частный случай.
\end{remark}

\begin{theorem}[Пикара (\textit{единственности})]\itemfix
    \begin{itemize}
        \item $G\subset \R^2$
        \item $f\in C(G)$
        \item $\cfrac{\partial f}{\partial y}\in C(G)$
        \item $(x_0, y_0)\in G$
        \item $\varphi_1, \varphi_2$ --- решение задачи Коши на $(a,b)$
    \end{itemize}
    Тогда $\varphi_1 \equiv \varphi_2$ на $(a, b)$
\end{theorem}
\begin{remark}
    Это тоже частный случай теоремы.
\end{remark}
\begin{remark}
    Здесь и далее $G$ --- область.
\end{remark}
\begin{proof}
    Будет позже.
\end{proof}

\begin{definition}
    Решение $\varphi$ задачи Коши называется \textbf{особым}, если через каждую точку соответствующей интегральной кривой проходит ещё одна интегральная кривая, не совпадающая с данной в любой окрестности этой точки.
\end{definition}
\begin{example}
    В уравнении $y'=3\sqrt[3]{y^2}$ решения имеют вид $y=(x+C)^3$. Проверим это:
    $$y'=3(x+C)^2=3\sqrt[3]{((x+C)^3)^2}$$
    $$\sphericalangle f(x, y) = 3\sqrt[3]{y^2}$$
    $$\frac{\partial f}{\partial y} = 3\frac{2}{3}\frac{1}{\sqrt[3] y} = \frac{2}{\sqrt[3] y}$$
    $\cfrac{\partial f}{\partial y}$ непрерывно на $\R^2\setminus\{(x, 0)\}$

    $y=0$ --- очевидно решение, при этом в каждой точке $(x, 0)$ это решение пересекается кривой вида $y=(x+C)^3$, поэтому это решение --- особое.
\end{example}

\section{Некоторые уравнения, интегрируемые в квадратурах}

Интегрируемые в квадратурах $\Leftrightarrow$ решения представляются в виде элементарных функций.

\subsection{Неполные уравнения}

\subsubsection{$y'=f(x)$}

$\Rightarrow y=\int f(x)dx + C$

\subsubsection{$y'=f(y)$}

\begin{enumerate}
    \item $f(y)\not=0 \Rightarrow f > 0$ (для $f<0$ аналогично)

          $$\frac{dy}{dx} = f(y) \quad \Big| : f(y)$$
          $$\frac{dy}{f(y)} = dx \quad y_x' = \frac{1}{x_y'}$$
          $$x_y' = \frac{1}{f(y)}$$
          $$x(y) = \int\frac{dy}{f(y)} + C$$
    \item $\sphericalangle f(y_0) = 0 \Rightarrow y \equiv y_0$ --- решение

          Далее область поиска интегральных кривых разбивается на подобласти.

          $y'=y$

          \begin{enumerate}
              \item $y>0$
                    $$\frac{dy}{dx} = y$$
                    $$\frac{1}{y}=x_y'$$
                    $$x = \int\frac{dy}{y} + C = \ln |y| + C$$
                    $$x(y)=\ln y + C$$
                    $$e^x = e^C y = Ay,\ A>0$$
                    $$y = C e^x,\ C>0$$

              \item $y<0$
                    $$y = C e^x,\ C<0$$
          \end{enumerate}

          Ответ: $y = Ce^x, C\in\R$
\end{enumerate}

\subsection{$p_1(x)q_1(y)dx + p_2(x)q_2(y)dy = 0$}

Попробуем поделить на $q_1(y)p_2(x)$:

\begin{itemize}
    \item Если $q_1(y_0)=0 \Rightarrow y \equiv y_0$ --- решение
    \item Если $p_2(x_0)=0 \Rightarrow x \equiv x_0$ --- решение
    \item Иначе можем поделить, получаем $X(x)dx + Y(y)dy=0$ --- уравнение с разделенными переменными
\end{itemize}

\begin{theorem}\itemfix
    \begin{itemize}
        \item $X\in C(a, b)$
        \item $Y\in C(c, d)$
        \item $(x_0, y_0)\in (a,b)\times (c, d)$ --- не особая точка
    \end{itemize}
    Тогда
    $$\int_{x_0}^{x} X(s)ds + \int_{y_0}^{y} Y(s)ds = 0$$
    задает интегральную кривую уравнения $X(x)dx + Y(y)dy=0$, проходящую через $(x_0, y_0)$
\end{theorem}
\begin{proof}
    $\sphericalangle Y(y_0)\not=0 \Rightarrow y'=-\cfrac{X(x)}{Y(y)}$ --- непрерывна в некоторой окрестности $(x_0, y_0) \xRightarrow{\text{Т.1.1}} \exists \varphi$ --- решение на $(\alpha, \beta)\ni x_0$
    $$\sphericalangle U(x, y) = \int_{x_0}^{x} X(s)ds + \int_{y_0}^{y} Y(s)ds$$
    $$\frac{d U(x, \varphi(x))}{dx} = X(x) + Y(\varphi(x)) \varphi'(x) =$$
    $$= \frac{X(x) dx + Y(\varphi(x))\varphi'(x)dx}{dx}\equiv 0 \text{ на } (\alpha, \beta) \text{ т.к. $\varphi$ --- решение}$$

    $\cfrac{d U(x, \varphi(x))}{dx} \equiv 0 \Rightarrow U(x, \varphi(x)) = C \ \ \forall x\in(\alpha, \beta)$, но $U(x_0, y_0)=0 \Rightarrow C=0 \Rightarrow \forall $ решение, проходящее через $(x_0, y_0)$ удовлетворяет равенству $U(x, y) = 0$
\end{proof}

$\begin{cases}
        \int X(x)dx = \int_{x_0}^x X(s)ds + C_1 \\
        \int X(y)dy = \int_{y_0}^y Y(s)ds + C_2
    \end{cases} \Rightarrow \int X(x)dx + \int X(y)dy = C$ --- общий интеграл уравнения $X(x)dx + Y(y)dy=0$

\begin{example}
    $xdx + ydy = 0$

    $$\int xdx + \int ydy = C$$
    $$\frac{x^2}{2} + \frac{y^2}{2} = C$$
    $$x^2 + y^2 = A,\ A>0$$
\end{example}

\subsection{Однородное уравнение}

\begin{definition}
    \textbf{Однородное уравнение} - уравнение вида $P(x, y)dx + Q(x, y)dy = 0$, если $P$ и $Q$ однородные одной степени, т.е.:
    $$P(tx, ty) = t^\alpha P(x, y) \quad \forall t$$
    $$Q(tx, ty) = t^\alpha Q(x, y) \quad \forall t$$
\end{definition}

\begin{example}\itemfix
    \begin{itemize}
        \item $x^2 + y^2$
        \item$x^{\frac{3}{2}} + x\sqrt y$
        \item $\cfrac{x+y}{3\sqrt{xy}}$
    \end{itemize}
\end{example}

Замена $z=\frac{y}{x}$ сводит к уравнению с разделенными переменными.

\subsection{Линейное уравнение}

\begin{definition}\itemfix
    \begin{itemize}
        \item \textbf{Линейное уравнение} --- уравнение вида $y'=p(x)y + q(x)$
        \item \textbf{Линейное однородное уравнение} --- линейное уравнение, где $q\equiv 0$
        \item \textbf{Линейное неоднородное уравнение} --- линейное уравнение, где $q\not\equiv 0$
    \end{itemize}
\end{definition}

\begin{lemma}
    $p\in C(a, b) \Rightarrow y=Ce^{\int p(x)dx}$ --- общее решение уравнения $y'=p(x)y$
\end{lemma}
\begin{proof}
    $dy = p(x)ydx$

    \begin{enumerate}
        \item $y=0$ --- решение
        \item $y>0$
              $$\int\cfrac{dy}{y} = \int p(x)dx \Rightarrow y = Ce^{\int p(x)dx}$$
    \end{enumerate}
\end{proof}

\end{document}