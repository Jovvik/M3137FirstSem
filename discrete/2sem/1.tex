\documentclass[12pt, a4paper]{article}

%<*preamble>
% Math symbols
\usepackage{amsmath, amsthm, amsfonts, amssymb}
\usepackage{accents}
\usepackage{esvect}
\usepackage{mathrsfs}
\usepackage{mathtools}
\mathtoolsset{showonlyrefs}
\usepackage{cmll}
\usepackage{stmaryrd}
\usepackage{physics}
\usepackage[normalem]{ulem}
\usepackage{ebproof}
\usepackage{extarrows}

% Page layout
\usepackage{geometry, a4wide, parskip, fancyhdr}

% Font, encoding, russian support
\usepackage[russian]{babel}
\usepackage[sb]{libertine}
\usepackage{xltxtra}

% Listings
\usepackage{listings}
\lstset{basicstyle=\ttfamily,breaklines=true}
\setmonofont{Inconsolata}

% Miscellaneous
\usepackage{array}
\usepackage{calc}
\usepackage{caption}
\usepackage{subcaption}
\captionsetup{justification=centering,margin=2cm}
\usepackage{catchfilebetweentags}
\usepackage{enumitem}
\usepackage{etoolbox}
\usepackage{float}
\usepackage{lastpage}
\usepackage{minted}
\usepackage{svg}
\usepackage{wrapfig}
\usepackage{xcolor}
\usepackage[makeroom]{cancel}

\newcolumntype{L}{>{$}l<{$}}
    \newcolumntype{C}{>{$}c<{$}}
\newcolumntype{R}{>{$}r<{$}}

% Footnotes
\usepackage[hang]{footmisc}
\setlength{\footnotemargin}{2mm}
\makeatletter
\def\blfootnote{\gdef\@thefnmark{}\@footnotetext}
\makeatother

% References
\usepackage{hyperref}
\hypersetup{
    colorlinks,
    linkcolor={blue!80!black},
    citecolor={blue!80!black},
    urlcolor={blue!80!black},
}

% tikz
\usepackage{tikz}
\usepackage{tikz-cd}
\usetikzlibrary{arrows.meta}
\usetikzlibrary{decorations.pathmorphing}
\usetikzlibrary{calc}
\usetikzlibrary{patterns}
\usepackage{pgfplots}
\pgfplotsset{width=10cm,compat=1.9}
\newcommand\irregularcircle[2]{% radius, irregularity
    \pgfextra {\pgfmathsetmacro\len{(#1)+rand*(#2)}}
    +(0:\len pt)
    \foreach \a in {10,20,...,350}{
            \pgfextra {\pgfmathsetmacro\len{(#1)+rand*(#2)}}
            -- +(\a:\len pt)
        } -- cycle
}

\providetoggle{useproofs}
\settoggle{useproofs}{false}

\pagestyle{fancy}
\lfoot{M3137y2019}
\cfoot{}
\rhead{стр. \thepage\ из \pageref*{LastPage}}

\newcommand{\R}{\mathbb{R}}
\newcommand{\Q}{\mathbb{Q}}
\newcommand{\Z}{\mathbb{Z}}
\newcommand{\B}{\mathbb{B}}
\newcommand{\N}{\mathbb{N}}
\renewcommand{\Re}{\mathfrak{R}}
\renewcommand{\Im}{\mathfrak{I}}

\newcommand{\const}{\text{const}}
\newcommand{\cond}{\text{cond}}

\newcommand{\teormin}{\textcolor{red}{!}\ }

\DeclareMathOperator*{\xor}{\oplus}
\DeclareMathOperator*{\equ}{\sim}
\DeclareMathOperator{\sign}{\text{sign}}
\DeclareMathOperator{\Sym}{\text{Sym}}
\DeclareMathOperator{\Asym}{\text{Asym}}

\DeclarePairedDelimiter{\ceil}{\lceil}{\rceil}

% godel
\newbox\gnBoxA
\newdimen\gnCornerHgt
\setbox\gnBoxA=\hbox{$\ulcorner$}
\global\gnCornerHgt=\ht\gnBoxA
\newdimen\gnArgHgt
\def\godel #1{%
    \setbox\gnBoxA=\hbox{$#1$}%
    \gnArgHgt=\ht\gnBoxA%
    \ifnum     \gnArgHgt<\gnCornerHgt \gnArgHgt=0pt%
    \else \advance \gnArgHgt by -\gnCornerHgt%
    \fi \raise\gnArgHgt\hbox{$\ulcorner$} \box\gnBoxA %
    \raise\gnArgHgt\hbox{$\urcorner$}}

% \theoremstyle{plain}

\theoremstyle{definition}
\newtheorem{theorem}{Теорема}
\newtheorem*{definition}{Определение}
\newtheorem{axiom}{Аксиома}
\newtheorem*{axiom*}{Аксиома}
\newtheorem{lemma}{Лемма}

\theoremstyle{remark}
\newtheorem*{remark}{Примечание}
\newtheorem*{exercise}{Упражнение}
\newtheorem{corollary}{Следствие}[theorem]
\newtheorem*{statement}{Утверждение}
\newtheorem*{corollary*}{Следствие}
\newtheorem*{example}{Пример}
\newtheorem{observation}{Наблюдение}
\newtheorem*{prop}{Свойства}
\newtheorem*{obozn}{Обозначение}

% subtheorem
\makeatletter
\newenvironment{subtheorem}[1]{%
    \def\subtheoremcounter{#1}%
    \refstepcounter{#1}%
    \protected@edef\theparentnumber{\csname the#1\endcsname}%
    \setcounter{parentnumber}{\value{#1}}%
    \setcounter{#1}{0}%
    \expandafter\def\csname the#1\endcsname{\theparentnumber.\Alph{#1}}%
    \ignorespaces
}{%
    \setcounter{\subtheoremcounter}{\value{parentnumber}}%
    \ignorespacesafterend
}
\makeatother
\newcounter{parentnumber}

\newtheorem{manualtheoreminner}{Теорема}
\newenvironment{manualtheorem}[1]{%
    \renewcommand\themanualtheoreminner{#1}%
    \manualtheoreminner
}{\endmanualtheoreminner}

\newcommand{\dbltilde}[1]{\accentset{\approx}{#1}}
\newcommand{\intt}{\int\!}

% magical thing that fixes paragraphs
\makeatletter
\patchcmd{\CatchFBT@Fin@l}{\endlinechar\m@ne}{}
{}{\typeout{Unsuccessful patch!}}
\makeatother

\newcommand{\get}[2]{
    \ExecuteMetaData[#1]{#2}
}

\newcommand{\getproof}[2]{
    \iftoggle{useproofs}{\ExecuteMetaData[#1]{#2proof}}{}
}

\newcommand{\getwithproof}[2]{
    \get{#1}{#2}
    \getproof{#1}{#2}
}

\newcommand{\import}[3]{
    \subsection{#1}
    \getwithproof{#2}{#3}
}

\newcommand{\given}[1]{
    Дано выше. (\ref{#1}, стр. \pageref{#1})
}

\renewcommand{\ker}{\text{Ker }}
\newcommand{\im}{\text{Im }}
\renewcommand{\grad}{\text{grad}}
\newcommand{\rg}{\text{rg}}
\newcommand{\defeq}{\stackrel{\text{def}}{=}}
\newcommand{\defeqfor}[1]{\stackrel{\text{def } #1}{=}}
\newcommand{\itemfix}{\leavevmode\makeatletter\makeatother}
\newcommand{\?}{\textcolor{red}{???}}
\renewcommand{\emptyset}{\varnothing}
\newcommand{\longarrow}[1]{\xRightarrow[#1]{\qquad}}
\DeclareMathOperator*{\esup}{\text{ess sup}}
\newcommand\smallO{
    \mathchoice
    {{\scriptstyle\mathcal{O}}}% \displaystyle
    {{\scriptstyle\mathcal{O}}}% \textstyle
    {{\scriptscriptstyle\mathcal{O}}}% \scriptstyle
    {\scalebox{.6}{$\scriptscriptstyle\mathcal{O}$}}%\scriptscriptstyle
}
\renewcommand{\div}{\text{div}\ }
\newcommand{\rot}{\text{rot}\ }
\newcommand{\cov}{\text{cov}}

\makeatletter
\newcommand{\oplabel}[1]{\refstepcounter{equation}(\theequation\ltx@label{#1})}
\makeatother

\newcommand{\symref}[2]{\stackrel{\oplabel{#1}}{#2}}
\newcommand{\symrefeq}[1]{\symref{#1}{=}}

% xrightrightarrows
\makeatletter
\newcommand*{\relrelbarsep}{.386ex}
\newcommand*{\relrelbar}{%
    \mathrel{%
        \mathpalette\@relrelbar\relrelbarsep
    }%
}
\newcommand*{\@relrelbar}[2]{%
    \raise#2\hbox to 0pt{$\m@th#1\relbar$\hss}%
    \lower#2\hbox{$\m@th#1\relbar$}%
}
\providecommand*{\rightrightarrowsfill@}{%
    \arrowfill@\relrelbar\relrelbar\rightrightarrows
}
\providecommand*{\leftleftarrowsfill@}{%
    \arrowfill@\leftleftarrows\relrelbar\relrelbar
}
\providecommand*{\xrightrightarrows}[2][]{%
    \ext@arrow 0359\rightrightarrowsfill@{#1}{#2}%
}
\providecommand*{\xleftleftarrows}[2][]{%
    \ext@arrow 3095\leftleftarrowsfill@{#1}{#2}%
}

\allowdisplaybreaks

\newcommand{\unfinished}{\textcolor{red}{Не дописано}}

% Reproducible pdf builds 
\special{pdf:trailerid [
<00112233445566778899aabbccddeeff>
<00112233445566778899aabbccddeeff>
]}
%</preamble>


\lhead{Дискретная математика}
\cfoot{}

\begin{document}

\section{Дискретная теория вероятности}

\begin{definition}
    \textbf{Множетсво элементарных исходов} обозначается $\Omega$. Это множество не более чем счётное в дискретной теории вероятности
\end{definition}
\begin{definition}
    \textbf{Дискретная вероятностная мера \textit{(дискретная плотность вероятности)}} - отображение $p: \Omega \to \R^+; \omega\mapsto$ вероятность исхода $\omega$. $$\sum\limits_{\omega\in\Omega} p(\omega) = 1$$
\end{definition}
\begin{example}
    Честная монета
    $$\Omega = \{0, 1\} \quad p(0) = p(1) = \frac{1}{2} \quad p(\emptyset)=0 \quad p(\{0, 1\})=1 \text{ --- достоверное событие}$$
\end{example}
\begin{example}
    Нечестная монета \textit{(распределение Бернулли)}
    $$\Omega = \{0, 1\} \quad p(0) = p \quad p(1) = q \quad p + q = 1$$
\end{example}
\begin{example}
    Честная игральная кость
    $$\Omega = \{1, 2, 3, 4, 5, 6\} \quad p(1) = p(2) = \ldots = p(6) = \frac{1}{6}$$
    $$Even := \{2, 4, 6\} \quad Big := \{4, 5, 6\}$$
    $$P(Even) = \frac{1}{2} \quad P(Big) = \frac{1}{2}$$
    $$VeryBig := \{5, 6\} \quad P(VeryBig) = \frac{1}{3}$$
\end{example}
\begin{example}
    Честная колода карт
    $$\Omega = \{(r, s)\ |\ r=1\ldots13, s=1\ldots4\} \quad p((r, s))=\frac{1}{52}$$
\end{example}
\begin{example}
    Честная монета с ребром
    $$\Omega = \{0, 1, \perp\} \quad p(0) = p(1) = \frac{1}{2} \quad p(\perp) = 0$$
\end{example}
\begin{example}
    Очень нечестная монета
    $$p = 1 \quad q = 0$$
\end{example}
\begin{definition}
    \textbf{Событие} --- множество элементарных исходов
    $$A\subset\Omega$$
\end{definition}
$$P:2^\Omega \to \R^+ \quad P(A) = \sum\limits_{\omega \in A}: p(\omega)$$
\begin{definition}
    $A$ и $B$ --- \textbf{независимые}, если вероятность их пересечения равна произведению их вероятностей:
    $$P(A\cap B)=P(A)\cdot P(B)$$
\end{definition}
$$P(Even \cap Big) = P(\{4, 6\}) = \frac{1}{3}$$
$$P(Even) \cdot P(Big) = \frac{1}{2} \cdot \frac{1}{2} = \frac{1}{4} \Rightarrow Even \text{ и } Big \text{ не независимы}$$
$$P(Even \cap VeryByg) = P(\{6\}) = \frac{1}{6}$$
$$P(Even) \cdot P(VeryBig) = \frac{1}{2} \cdot \frac{1}{3} = \frac{1}{6} \Rightarrow Even \text{ и } VeryBig \text{ независимы}$$
$$\sphericalangle \Omega_1, p_1, \Omega_2, p_2 \quad \Omega := \Omega_1 \times \Omega_2 \quad P(\omega)=p_1(\omega_1)p_2(\omega_2)$$
\begin{definition}
    События $A_1\ldots A_n$ --- \textbf{независимые в совокупности}, если
    $$\forall I\subset\{1\ldots n\} \quad P\left(\bigcap\limits_{i\in I} A_i\right)=\prod\limits_{i\in I}P(A_i)$$
\end{definition}
\begin{definition}
    События $A_1\ldots A_n$ --- \textbf{независимые попарно}, если
    $$\forall i, j\in\{1\ldots n\} \quad P(A_i\cap A_j)=P(A_i)P(A_j)$$
\end{definition}
\begin{definition}
    \textbf{Условная вероятность} --- вероятность того, что произойдет $A$, если произошло $B$:
    $$P(A|B)=\frac{P(A\cap B)}{P(B)}$$
\end{definition}
$$P(Big|Even) = \frac{P(Big \cap Even)}{P(Even)} = \frac{\frac{1}{3}}{\frac{1}{2}} = \frac{2}{3}$$
$] A$ и $B$ --- независимые
$$P(A|B) = \frac{P(A \cap B)}{P(B)} = \frac{P(A)\cdot P(B)}{P(B)} = P(A)$$
$\sphericalangle A_1, A_2, \ldots A_k$ --- разбиение : $\bigcap\limits_{i=1}^k A_i=\Omega \quad A_i\cap A_j=\emptyset$
\begin{theorem}
    Формула полной вероятности:
    $$P(B)=\sum\limits_{i=1}^k P(B|A_i)\cdot P(A_i)$$
\end{theorem}
\begin{proof}
    $$P(B)=\sum\limits_{i=1}^k P(B\cap A_i)=\sum\limits_{i=1}^k P(B|A_i)\cdot P(A_i)$$
\end{proof}
\begin{theorem}
    Формула Байеса:
    $$P(A_j|B)=\frac{P(A_j\cap B)}{P(B)}=\frac{P(B|A_j)P(A_j)}{\sum\limits_{i=1}^k P(B|A_i)P(A_i)}$$
\end{theorem}
% [============================================================================]
% |Формула Байеыса:                                                            |
% |P(Aj|B) = P(Cross(Aj,B))/P(B) = P(B|Aj)*P(Aj)/Sum(i = 1 to K): P(B|Aj)*P(Ai)|
% [============================================================================]

\end{document}