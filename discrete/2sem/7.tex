\documentclass[12pt, a4paper]{article}

%<*preamble>
% Math symbols
\usepackage{amsmath, amsthm, amsfonts, amssymb}
\usepackage{accents}
\usepackage{esvect}
\usepackage{mathrsfs}
\usepackage{mathtools}
\mathtoolsset{showonlyrefs}
\usepackage{cmll}
\usepackage{stmaryrd}
\usepackage{physics}
\usepackage[normalem]{ulem}
\usepackage{ebproof}
\usepackage{extarrows}

% Page layout
\usepackage{geometry, a4wide, parskip, fancyhdr}

% Font, encoding, russian support
\usepackage[russian]{babel}
\usepackage[sb]{libertine}
\usepackage{xltxtra}

% Listings
\usepackage{listings}
\lstset{basicstyle=\ttfamily,breaklines=true}
\setmonofont{Inconsolata}

% Miscellaneous
\usepackage{array}
\usepackage{calc}
\usepackage{caption}
\usepackage{subcaption}
\captionsetup{justification=centering,margin=2cm}
\usepackage{catchfilebetweentags}
\usepackage{enumitem}
\usepackage{etoolbox}
\usepackage{float}
\usepackage{lastpage}
\usepackage{minted}
\usepackage{svg}
\usepackage{wrapfig}
\usepackage{xcolor}
\usepackage[makeroom]{cancel}

\newcolumntype{L}{>{$}l<{$}}
    \newcolumntype{C}{>{$}c<{$}}
\newcolumntype{R}{>{$}r<{$}}

% Footnotes
\usepackage[hang]{footmisc}
\setlength{\footnotemargin}{2mm}
\makeatletter
\def\blfootnote{\gdef\@thefnmark{}\@footnotetext}
\makeatother

% References
\usepackage{hyperref}
\hypersetup{
    colorlinks,
    linkcolor={blue!80!black},
    citecolor={blue!80!black},
    urlcolor={blue!80!black},
}

% tikz
\usepackage{tikz}
\usepackage{tikz-cd}
\usetikzlibrary{arrows.meta}
\usetikzlibrary{decorations.pathmorphing}
\usetikzlibrary{calc}
\usetikzlibrary{patterns}
\usepackage{pgfplots}
\pgfplotsset{width=10cm,compat=1.9}
\newcommand\irregularcircle[2]{% radius, irregularity
    \pgfextra {\pgfmathsetmacro\len{(#1)+rand*(#2)}}
    +(0:\len pt)
    \foreach \a in {10,20,...,350}{
            \pgfextra {\pgfmathsetmacro\len{(#1)+rand*(#2)}}
            -- +(\a:\len pt)
        } -- cycle
}

\providetoggle{useproofs}
\settoggle{useproofs}{false}

\pagestyle{fancy}
\lfoot{M3137y2019}
\cfoot{}
\rhead{стр. \thepage\ из \pageref*{LastPage}}

\newcommand{\R}{\mathbb{R}}
\newcommand{\Q}{\mathbb{Q}}
\newcommand{\Z}{\mathbb{Z}}
\newcommand{\B}{\mathbb{B}}
\newcommand{\N}{\mathbb{N}}
\renewcommand{\Re}{\mathfrak{R}}
\renewcommand{\Im}{\mathfrak{I}}

\newcommand{\const}{\text{const}}
\newcommand{\cond}{\text{cond}}

\newcommand{\teormin}{\textcolor{red}{!}\ }

\DeclareMathOperator*{\xor}{\oplus}
\DeclareMathOperator*{\equ}{\sim}
\DeclareMathOperator{\sign}{\text{sign}}
\DeclareMathOperator{\Sym}{\text{Sym}}
\DeclareMathOperator{\Asym}{\text{Asym}}

\DeclarePairedDelimiter{\ceil}{\lceil}{\rceil}

% godel
\newbox\gnBoxA
\newdimen\gnCornerHgt
\setbox\gnBoxA=\hbox{$\ulcorner$}
\global\gnCornerHgt=\ht\gnBoxA
\newdimen\gnArgHgt
\def\godel #1{%
    \setbox\gnBoxA=\hbox{$#1$}%
    \gnArgHgt=\ht\gnBoxA%
    \ifnum     \gnArgHgt<\gnCornerHgt \gnArgHgt=0pt%
    \else \advance \gnArgHgt by -\gnCornerHgt%
    \fi \raise\gnArgHgt\hbox{$\ulcorner$} \box\gnBoxA %
    \raise\gnArgHgt\hbox{$\urcorner$}}

% \theoremstyle{plain}

\theoremstyle{definition}
\newtheorem{theorem}{Теорема}
\newtheorem*{definition}{Определение}
\newtheorem{axiom}{Аксиома}
\newtheorem*{axiom*}{Аксиома}
\newtheorem{lemma}{Лемма}

\theoremstyle{remark}
\newtheorem*{remark}{Примечание}
\newtheorem*{exercise}{Упражнение}
\newtheorem{corollary}{Следствие}[theorem]
\newtheorem*{statement}{Утверждение}
\newtheorem*{corollary*}{Следствие}
\newtheorem*{example}{Пример}
\newtheorem{observation}{Наблюдение}
\newtheorem*{prop}{Свойства}
\newtheorem*{obozn}{Обозначение}

% subtheorem
\makeatletter
\newenvironment{subtheorem}[1]{%
    \def\subtheoremcounter{#1}%
    \refstepcounter{#1}%
    \protected@edef\theparentnumber{\csname the#1\endcsname}%
    \setcounter{parentnumber}{\value{#1}}%
    \setcounter{#1}{0}%
    \expandafter\def\csname the#1\endcsname{\theparentnumber.\Alph{#1}}%
    \ignorespaces
}{%
    \setcounter{\subtheoremcounter}{\value{parentnumber}}%
    \ignorespacesafterend
}
\makeatother
\newcounter{parentnumber}

\newtheorem{manualtheoreminner}{Теорема}
\newenvironment{manualtheorem}[1]{%
    \renewcommand\themanualtheoreminner{#1}%
    \manualtheoreminner
}{\endmanualtheoreminner}

\newcommand{\dbltilde}[1]{\accentset{\approx}{#1}}
\newcommand{\intt}{\int\!}

% magical thing that fixes paragraphs
\makeatletter
\patchcmd{\CatchFBT@Fin@l}{\endlinechar\m@ne}{}
{}{\typeout{Unsuccessful patch!}}
\makeatother

\newcommand{\get}[2]{
    \ExecuteMetaData[#1]{#2}
}

\newcommand{\getproof}[2]{
    \iftoggle{useproofs}{\ExecuteMetaData[#1]{#2proof}}{}
}

\newcommand{\getwithproof}[2]{
    \get{#1}{#2}
    \getproof{#1}{#2}
}

\newcommand{\import}[3]{
    \subsection{#1}
    \getwithproof{#2}{#3}
}

\newcommand{\given}[1]{
    Дано выше. (\ref{#1}, стр. \pageref{#1})
}

\renewcommand{\ker}{\text{Ker }}
\newcommand{\im}{\text{Im }}
\renewcommand{\grad}{\text{grad}}
\newcommand{\rg}{\text{rg}}
\newcommand{\defeq}{\stackrel{\text{def}}{=}}
\newcommand{\defeqfor}[1]{\stackrel{\text{def } #1}{=}}
\newcommand{\itemfix}{\leavevmode\makeatletter\makeatother}
\newcommand{\?}{\textcolor{red}{???}}
\renewcommand{\emptyset}{\varnothing}
\newcommand{\longarrow}[1]{\xRightarrow[#1]{\qquad}}
\DeclareMathOperator*{\esup}{\text{ess sup}}
\newcommand\smallO{
    \mathchoice
    {{\scriptstyle\mathcal{O}}}% \displaystyle
    {{\scriptstyle\mathcal{O}}}% \textstyle
    {{\scriptscriptstyle\mathcal{O}}}% \scriptstyle
    {\scalebox{.6}{$\scriptscriptstyle\mathcal{O}$}}%\scriptscriptstyle
}
\renewcommand{\div}{\text{div}\ }
\newcommand{\rot}{\text{rot}\ }
\newcommand{\cov}{\text{cov}}

\makeatletter
\newcommand{\oplabel}[1]{\refstepcounter{equation}(\theequation\ltx@label{#1})}
\makeatother

\newcommand{\symref}[2]{\stackrel{\oplabel{#1}}{#2}}
\newcommand{\symrefeq}[1]{\symref{#1}{=}}

% xrightrightarrows
\makeatletter
\newcommand*{\relrelbarsep}{.386ex}
\newcommand*{\relrelbar}{%
    \mathrel{%
        \mathpalette\@relrelbar\relrelbarsep
    }%
}
\newcommand*{\@relrelbar}[2]{%
    \raise#2\hbox to 0pt{$\m@th#1\relbar$\hss}%
    \lower#2\hbox{$\m@th#1\relbar$}%
}
\providecommand*{\rightrightarrowsfill@}{%
    \arrowfill@\relrelbar\relrelbar\rightrightarrows
}
\providecommand*{\leftleftarrowsfill@}{%
    \arrowfill@\leftleftarrows\relrelbar\relrelbar
}
\providecommand*{\xrightrightarrows}[2][]{%
    \ext@arrow 0359\rightrightarrowsfill@{#1}{#2}%
}
\providecommand*{\xleftleftarrows}[2][]{%
    \ext@arrow 3095\leftleftarrowsfill@{#1}{#2}%
}

\allowdisplaybreaks

\newcommand{\unfinished}{\textcolor{red}{Не дописано}}

% Reproducible pdf builds 
\special{pdf:trailerid [
<00112233445566778899aabbccddeeff>
<00112233445566778899aabbccddeeff>
]}
%</preamble>


\lhead{Дискретная математика}
\cfoot{}
\rfoot{Лекция 7}

\usepackage{listings}
\usepackage{courier}
\lstset{basicstyle=\footnotesize\ttfamily,breaklines=true}
\lstset{framextopmargin=50pt,frame=bottomline}

\begin{document}

\begin{definition}
    Конечное, непустое множество $\Sigma$ --- \textbf{алфавит}
\end{definition}

\begin{example}
    \begin{itemize}
        \item $\{0, 1\}$
        \item $\{a, b, c\ldots z\}$
        \item Unicode
    \end{itemize}
\end{example}

$$\Sigma^*:=\bigcup_{k=0}^\infty \Sigma^k$$

\begin{definition}
    \textbf{Конкатенация}:
    $$\Sigma^k\times \Sigma^l \to \Sigma^{k+l}$$
    Это аддитивная операция с нейтральным элементом.
\end{definition}

\begin{definition}
    \textbf{Язык} над алфавитом $\Sigma^*$ --- подмножество $\Sigma^*$
\end{definition}

$\Sigma^*$ --- бесконечное счётное множество

Множество всех языков $2^{\Sigma^*}$, т.к. каждый из элементов $\Sigma^*$ либо включен, либо нет. Это множество несчетно.

\begin{example}
    \begin{enumerate}
        \item $A=\{w | \text{ в } w \text{ четное число нулей}\}\subset \{0, 1\}^*$
        
        $01011\in A \quad 000\notin A$
        \item $Pal=\{w | w \text{ --- палиндром}\}\subset \{0, 1\}^*$
        
        $010\in Pal \quad 0000\in Pal \quad 01\notin Pal$
    \end{enumerate}
\end{example}

Языки надо задавать формально, а не что-то рода \textit{``язык палиндромов''}. Есть два способа это делать:
\begin{enumerate}
    \item \textbf{Распознавание}: есть черный ящик, который на вход получает слово и выдает булево значение --- принадлежит слово искомому языку или нет.
    \item \textbf{Конструирование}: система правил диктует то, как устроены слова в искомом языке.
\end{enumerate}

\begin{example}
    Распознавание ПСП:

    Тут код.
\end{example}

\begin{example}
    Конструирование ПСП:

    $\varepsilon$ --- ПСП
    
    $A$ --- ПСП $\Rightarrow (A)$ --- ПСП

    $A, B$ --- ПСП $\Rightarrow AB$ --- ПСП
\end{example}

Автоматы распознают принадлежность слова языку.
\begin{definition}
    \textbf{Детерменированный конечный автомат \textit{(ДКА)}}:
    \begin{enumerate}
        \item Состояния, обозначаемые кругами
        \item Переходы, обозначаемые ребрами между состояниями и помечаемые символом алфавита. Из любого состояния есть ровно один переход по каждому символу алфавита.
        \item Начальное состояние, обозначаемое входящей стрелкой из никуда
        \item Допускающие состояния, обозначаемые кругом внутри себя.
    \end{enumerate}
\end{definition}

\begin{figure}
    \centering
    \begin{minipage}{.5\textwidth}
      \centering
      \includegraphics[scale=0.7]{graphs/A.eps}
      \caption{Автомат для $\{w | \text{ в } w \text{ четное число нулей}\}$}
    %   \label{fig:test1}
    \end{minipage}%
    \begin{minipage}{.5\textwidth}
      \centering
      \includegraphics[scale=0.7]{graphs/ZZ.eps}
      \caption{Автомат $\{w | \text{ в } w \text{ содержит два нуля подряд}\}$}
    %   \label{fig:test2}
    \end{minipage}
\end{figure}

Переименуем состояния ``not $0$'' $\to$ $A$, ``last $0$'' $\to B$, ``$00$'' $\to C$.

Опишем математически ДКА:

\begin{definition}
    ДКА $AV=\langle \Sigma, Q, S\in Q, T\subset Q, \delta:Q\times\Sigma\to Q\rangle$

    \textbf{Мгновенное описание} $AV$ --- пара из состояния $A$ и оставшихся от исходного слова строки. Это объект из множества $Q\times \Sigma^*$, например $\langle A, 11011001\rangle$.

    На мгновенных описаниях можно задать отношение ``переходит за 1 шаг'', обозначим его $\vdash$. Формально оно задается следующим образом:

    $\langle p, \alpha\rangle\vdash \langle q, \beta\rangle$, если:
    \begin{enumerate}
        \item $\alpha=c\beta$
        \item $q=\delta(p, c)$
    \end{enumerate}

    Тогда для для слова $S$ верно следующее:
    $$\langle S, x\rangle\vdash \langle u_1, x_1\rangle\vdash\langle u_2, x_2\rangle\vdash\ldots\vdash \langle u_l, \varepsilon\rangle$$
    $$AV \text{ допускает } x \Leftrightarrow u_L\in T$$
    Эта запись неудобна, поэтому обозначим транзитивное замыкание $\vdash$ как $\vdash^*$, это --- отношение ``переходит $\geq 0$ шагов''. Тогда получается следующее:
    $$AV \text{ допускает } x \Leftrightarrow \langle s, x\rangle\vdash^*\langle t, \varepsilon\rangle$$
\end{definition}

Заметим, что все ДКА --- счётное множество, а языки - несчётное $\Rightarrow$ не любой язык можно описать с помощью ДКА. Назовем все языки, которые можно получить с помощью ДКА \textbf{автоматными}.

\begin{definition}
    \textbf{Базовые регулярные языки}:
    \begin{itemize}
        \item \O
        \item $\{\varepsilon\}$
        \item $\{c_i\} \quad \Sigma=\{c_1\ldots c_z\}$
    \end{itemize}
\end{definition}

Обозначим множество базовых регулярных языков над $\Sigma=\{0, 1\}$ как $R_0=\{\{\}, \{\varepsilon\}, \{0\}, \{1\}\}$

Зададим три операции над этими языками:
\begin{enumerate}
    \item Объединение: $A\cup B$
    \item Конкатенация: $\{ab\ |\ a\in A, b\in B\}$
    \item Замыкание Клини: $A^*=\bigcup\limits_{i=0}^\infty A^i$
\end{enumerate}

$$R_1:=\{M \ | \ M\in R_0 \text{ или } M=AB, A\in R_0, B\in R_0, M=A\cap B, A\in R_0, B\in R_0 \text{ или } M=A^*, A\in R_0\}$$
$$R_1=\{\{\}, \{\varepsilon\}, \{0\}, \{1\}, \{\varepsilon, 0\}, \{\varepsilon, 1\}, \{00\}, \{01\}, \{10\}, \{11\}, \{0\}^*, \{1\}^* \}$$

\begin{definition}
    \textbf{Регулярные языки}:
    $$Reg=\bigcup_{i=0}^\infty R_i$$
\end{definition}

\end{document}