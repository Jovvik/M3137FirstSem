\documentclass[12pt, a4paper]{article}

%<*preamble>
% Math symbols
\usepackage{amsmath, amsthm, amsfonts, amssymb}
\usepackage{accents}
\usepackage{esvect}
\usepackage{mathrsfs}
\usepackage{mathtools}
\mathtoolsset{showonlyrefs}
\usepackage{cmll}
\usepackage{stmaryrd}
\usepackage{physics}
\usepackage[normalem]{ulem}
\usepackage{ebproof}
\usepackage{extarrows}

% Page layout
\usepackage{geometry, a4wide, parskip, fancyhdr}

% Font, encoding, russian support
\usepackage[russian]{babel}
\usepackage[sb]{libertine}
\usepackage{xltxtra}

% Listings
\usepackage{listings}
\lstset{basicstyle=\ttfamily,breaklines=true}
\setmonofont{Inconsolata}

% Miscellaneous
\usepackage{array}
\usepackage{calc}
\usepackage{caption}
\usepackage{subcaption}
\captionsetup{justification=centering,margin=2cm}
\usepackage{catchfilebetweentags}
\usepackage{enumitem}
\usepackage{etoolbox}
\usepackage{float}
\usepackage{lastpage}
\usepackage{minted}
\usepackage{svg}
\usepackage{wrapfig}
\usepackage{xcolor}
\usepackage[makeroom]{cancel}

\newcolumntype{L}{>{$}l<{$}}
    \newcolumntype{C}{>{$}c<{$}}
\newcolumntype{R}{>{$}r<{$}}

% Footnotes
\usepackage[hang]{footmisc}
\setlength{\footnotemargin}{2mm}
\makeatletter
\def\blfootnote{\gdef\@thefnmark{}\@footnotetext}
\makeatother

% References
\usepackage{hyperref}
\hypersetup{
    colorlinks,
    linkcolor={blue!80!black},
    citecolor={blue!80!black},
    urlcolor={blue!80!black},
}

% tikz
\usepackage{tikz}
\usepackage{tikz-cd}
\usetikzlibrary{arrows.meta}
\usetikzlibrary{decorations.pathmorphing}
\usetikzlibrary{calc}
\usetikzlibrary{patterns}
\usepackage{pgfplots}
\pgfplotsset{width=10cm,compat=1.9}
\newcommand\irregularcircle[2]{% radius, irregularity
    \pgfextra {\pgfmathsetmacro\len{(#1)+rand*(#2)}}
    +(0:\len pt)
    \foreach \a in {10,20,...,350}{
            \pgfextra {\pgfmathsetmacro\len{(#1)+rand*(#2)}}
            -- +(\a:\len pt)
        } -- cycle
}

\providetoggle{useproofs}
\settoggle{useproofs}{false}

\pagestyle{fancy}
\lfoot{M3137y2019}
\cfoot{}
\rhead{стр. \thepage\ из \pageref*{LastPage}}

\newcommand{\R}{\mathbb{R}}
\newcommand{\Q}{\mathbb{Q}}
\newcommand{\Z}{\mathbb{Z}}
\newcommand{\B}{\mathbb{B}}
\newcommand{\N}{\mathbb{N}}
\renewcommand{\Re}{\mathfrak{R}}
\renewcommand{\Im}{\mathfrak{I}}

\newcommand{\const}{\text{const}}
\newcommand{\cond}{\text{cond}}

\newcommand{\teormin}{\textcolor{red}{!}\ }

\DeclareMathOperator*{\xor}{\oplus}
\DeclareMathOperator*{\equ}{\sim}
\DeclareMathOperator{\sign}{\text{sign}}
\DeclareMathOperator{\Sym}{\text{Sym}}
\DeclareMathOperator{\Asym}{\text{Asym}}

\DeclarePairedDelimiter{\ceil}{\lceil}{\rceil}

% godel
\newbox\gnBoxA
\newdimen\gnCornerHgt
\setbox\gnBoxA=\hbox{$\ulcorner$}
\global\gnCornerHgt=\ht\gnBoxA
\newdimen\gnArgHgt
\def\godel #1{%
    \setbox\gnBoxA=\hbox{$#1$}%
    \gnArgHgt=\ht\gnBoxA%
    \ifnum     \gnArgHgt<\gnCornerHgt \gnArgHgt=0pt%
    \else \advance \gnArgHgt by -\gnCornerHgt%
    \fi \raise\gnArgHgt\hbox{$\ulcorner$} \box\gnBoxA %
    \raise\gnArgHgt\hbox{$\urcorner$}}

% \theoremstyle{plain}

\theoremstyle{definition}
\newtheorem{theorem}{Теорема}
\newtheorem*{definition}{Определение}
\newtheorem{axiom}{Аксиома}
\newtheorem*{axiom*}{Аксиома}
\newtheorem{lemma}{Лемма}

\theoremstyle{remark}
\newtheorem*{remark}{Примечание}
\newtheorem*{exercise}{Упражнение}
\newtheorem{corollary}{Следствие}[theorem]
\newtheorem*{statement}{Утверждение}
\newtheorem*{corollary*}{Следствие}
\newtheorem*{example}{Пример}
\newtheorem{observation}{Наблюдение}
\newtheorem*{prop}{Свойства}
\newtheorem*{obozn}{Обозначение}

% subtheorem
\makeatletter
\newenvironment{subtheorem}[1]{%
    \def\subtheoremcounter{#1}%
    \refstepcounter{#1}%
    \protected@edef\theparentnumber{\csname the#1\endcsname}%
    \setcounter{parentnumber}{\value{#1}}%
    \setcounter{#1}{0}%
    \expandafter\def\csname the#1\endcsname{\theparentnumber.\Alph{#1}}%
    \ignorespaces
}{%
    \setcounter{\subtheoremcounter}{\value{parentnumber}}%
    \ignorespacesafterend
}
\makeatother
\newcounter{parentnumber}

\newtheorem{manualtheoreminner}{Теорема}
\newenvironment{manualtheorem}[1]{%
    \renewcommand\themanualtheoreminner{#1}%
    \manualtheoreminner
}{\endmanualtheoreminner}

\newcommand{\dbltilde}[1]{\accentset{\approx}{#1}}
\newcommand{\intt}{\int\!}

% magical thing that fixes paragraphs
\makeatletter
\patchcmd{\CatchFBT@Fin@l}{\endlinechar\m@ne}{}
{}{\typeout{Unsuccessful patch!}}
\makeatother

\newcommand{\get}[2]{
    \ExecuteMetaData[#1]{#2}
}

\newcommand{\getproof}[2]{
    \iftoggle{useproofs}{\ExecuteMetaData[#1]{#2proof}}{}
}

\newcommand{\getwithproof}[2]{
    \get{#1}{#2}
    \getproof{#1}{#2}
}

\newcommand{\import}[3]{
    \subsection{#1}
    \getwithproof{#2}{#3}
}

\newcommand{\given}[1]{
    Дано выше. (\ref{#1}, стр. \pageref{#1})
}

\renewcommand{\ker}{\text{Ker }}
\newcommand{\im}{\text{Im }}
\renewcommand{\grad}{\text{grad}}
\newcommand{\rg}{\text{rg}}
\newcommand{\defeq}{\stackrel{\text{def}}{=}}
\newcommand{\defeqfor}[1]{\stackrel{\text{def } #1}{=}}
\newcommand{\itemfix}{\leavevmode\makeatletter\makeatother}
\newcommand{\?}{\textcolor{red}{???}}
\renewcommand{\emptyset}{\varnothing}
\newcommand{\longarrow}[1]{\xRightarrow[#1]{\qquad}}
\DeclareMathOperator*{\esup}{\text{ess sup}}
\newcommand\smallO{
    \mathchoice
    {{\scriptstyle\mathcal{O}}}% \displaystyle
    {{\scriptstyle\mathcal{O}}}% \textstyle
    {{\scriptscriptstyle\mathcal{O}}}% \scriptstyle
    {\scalebox{.6}{$\scriptscriptstyle\mathcal{O}$}}%\scriptscriptstyle
}
\renewcommand{\div}{\text{div}\ }
\newcommand{\rot}{\text{rot}\ }
\newcommand{\cov}{\text{cov}}

\makeatletter
\newcommand{\oplabel}[1]{\refstepcounter{equation}(\theequation\ltx@label{#1})}
\makeatother

\newcommand{\symref}[2]{\stackrel{\oplabel{#1}}{#2}}
\newcommand{\symrefeq}[1]{\symref{#1}{=}}

% xrightrightarrows
\makeatletter
\newcommand*{\relrelbarsep}{.386ex}
\newcommand*{\relrelbar}{%
    \mathrel{%
        \mathpalette\@relrelbar\relrelbarsep
    }%
}
\newcommand*{\@relrelbar}[2]{%
    \raise#2\hbox to 0pt{$\m@th#1\relbar$\hss}%
    \lower#2\hbox{$\m@th#1\relbar$}%
}
\providecommand*{\rightrightarrowsfill@}{%
    \arrowfill@\relrelbar\relrelbar\rightrightarrows
}
\providecommand*{\leftleftarrowsfill@}{%
    \arrowfill@\leftleftarrows\relrelbar\relrelbar
}
\providecommand*{\xrightrightarrows}[2][]{%
    \ext@arrow 0359\rightrightarrowsfill@{#1}{#2}%
}
\providecommand*{\xleftleftarrows}[2][]{%
    \ext@arrow 3095\leftleftarrowsfill@{#1}{#2}%
}

\allowdisplaybreaks

\newcommand{\unfinished}{\textcolor{red}{Не дописано}}

% Reproducible pdf builds 
\special{pdf:trailerid [
<00112233445566778899aabbccddeeff>
<00112233445566778899aabbccddeeff>
]}
%</preamble>


\lhead{Конспект по дискретной математике}
\cfoot{}
\rfoot{October 22, 2019}

\begin{document}

\section{Представление информации}

Направления развития:
\begin{enumerate}
    \item Сжатие
    \item Избыточное кодирование
    \item Криптографическое кодирование
\end{enumerate}

\begin{definition}
    \textbf{Алфавитом} $\Sigma$ называется непустое конечное множество. Множество из $n$ элементов $\Sigma$ обозначается $\Sigma^n$.
\end{definition}

$$\bigcup\limits_{i=0}^\infty \Sigma^i=\Sigma^* \quad \Sigma^0=\{\varepsilon\}$$

\begin{definition}
    Конкатенация:
    $$\alpha\in\Sigma^* \quad \beta\in\Sigma^* \mapsto \alpha\beta\in\Sigma^*$$
\end{definition}

Конкатенация транзитивна $(\alpha\beta)\gamma=\alpha(\beta\gamma) \Rightarrow$ алфавит --- полугруппа.

$\alpha\varepsilon=\varepsilon\alpha=\alpha \Rightarrow$ алфавит --- моноид.

Т.к. алфавит --- полугруппа и моноид, алфавит --- свободный моноид.

\begin{definition}
    \textbf{Гомоморфизм} $\varphi: \Sigma^*\to \Pi^*$

    $\varphi(\alpha\beta)=\varphi(\alpha)\varphi(\beta)$
\end{definition}

\begin{example}
    $$0\to a, 1\to ab$$
    $$\varphi: \{0,1\}^*\to\{a,b\}^*$$
    $$\varphi(001)=aaab$$
\end{example}

$\varphi$ --- гомоморфизм $\Rightarrow \varphi(c_1,c_2\ldots c_n)=\varphi(c_1)\varphi(c_2)\ldots\varphi(c_n)$

\begin{definition}
    Отображение из произвольного $\Sigma^*$ в $\Pi^*$ называется \textbf{кодом}.

    Если $\varphi$ --- гомоморфизм, $\varphi$ --- \textbf{разделяемый}.

    Если $\Pi=\mathbb{B}$, $\varphi$ --- \textbf{бинарный/двоичный}.
\end{definition}

\begin{example}
    $$\Sigma=\{a,b,c\} \quad \varphi(a)=0, \varphi(b)=01, \varphi(c)=1$$
    $$\varphi(abc)=0011 \quad \varphi(aacc)=0011$$
\end{example}

\begin{definition}
    Код называется \textbf{однозначно декодируемым}, если $\forall x,y\in\Sigma^* \quad \varphi(x)=\varphi(y) \Rightarrow x=y$
\end{definition}

\begin{definition}
    Кодом \textbf{постоянной длины} называется код, если $\varphi: \Sigma\to\Pi^k, k=const$
\end{definition}

\begin{lemma}
    $\varphi$ --- код постоянной длины

    $\forall c\not=d\in\Sigma \quad \varphi(c)\not=\varphi(d)$

    Тогда $\varphi$ --- однозначно декодируемый.
\end{lemma}

\begin{theorem}
    $\Sigma, \Pi, |\Sigma=s|, |\Pi|=p, \Sigma\to \Pi^k$

    $k=\ceil{\log_p s}$

    $p^k<s$
\end{theorem}

\begin{theorem}
    Крафта, Мак-Милана.

    $\exists$ двоичных разделяемый однозначно декодируемый код переменной длины с длинами кодовых слов $l_1,l_2\ldots l_s\Leftrightarrow\sum_{i=1}^s 2^{-l_i}\leq 1, S\geq 2$
\end{theorem}

\begin{proof}
    Докажем ``$\Rightarrow$''.

    Пусть $ab, abb, ab$ --- все члены $\Sigma$

    $$(ab+abb+bb)^2=abab+ababb+abbb+\ldots$$

    $$(ab+abb+bb)^k - S^k \text{ слов, при этом все слова разные} $$

    $$]a=\frac{1}{2}, b=\frac{1}{2}$$

    $$ab+abb+bb=\sum 2^{-l_i}$$

    $$(\sum 2^{-l_i})^k=\sum_{j=0}^{k\max l_i}(2^{-j}+2^{-j}+2^{-j}) \text{ --- всего $\leq 2^j$ слов}$$

    $$\sum_{j=0}^{k\max l_i}(2^{-j}+2^{-j}+2^{-j})\leq k\max l_i$$

    $$\forall k: x^k\leq k\max l_i \to x\leq 1$$
\end{proof}

\begin{definition}
    \textbf{Префиксный} код: $\forall c\not=d \quad \varphi(c)$ --- не префикс $\varphi(d)$
\end{definition}

\begin{lemma}
    Префиксный код --- однозначно декодируем.
\end{lemma}

\begin{proof}
    Докажем ``$\Leftarrow$''

    $\sum 2^{-l_i}\leq 1 \Rightarrow \exists$ префиксный код с длинами $l_1\ldots l_s$

    $$l_1\leq l_2\leq \ldots\leq l_s$$

    $2^{-l_1}$

    $2^{-l_1}+2^{-l_2}$

    $\vdots$

    $2^{-l_1}+2^{-l_2}+\ldots+2^{-l_s}$

    $$S=2 \quad 2^{-l_1}+2^{-l_2}\leq 1$$
\end{proof}

Тут автор сдох.

\begin{corollary}
    $\exists$ однозначно декодируемый код с длинами $l_1\ldots l_s \Rightarrow \exists$ префиксный код с длинами $l_1\ldots l_n$
\end{corollary}

\subsection{Код Хаффмана}

Дано: $f_1,f_2\ldots f_s$ --- как часто встречаются соответствующие слова. Найти $l_1\ldots l_s$, такие что $\sum 2^{-l_i}$ и $\sum l_if_i\to \min$

$S=2\Rightarrow l_1=l_2=1$

$S>2$ Возьмём два символа $x$ и $y$, такие что $f_x$ и $f_y\to\min$ \textit{($x$, $y$ -- самые редкие)}. Заменим их на $z, f_z=f_x+f_y$.

Возьмём в качестве кодового слова для $x$ слово для $z + 0$, а для $y$ возьмём $z + 1$.

\end{document}