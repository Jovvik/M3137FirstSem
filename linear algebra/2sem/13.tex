\documentclass[12pt, a4paper]{article}

%<*preamble>
% Math symbols
\usepackage{amsmath, amsthm, amsfonts, amssymb}
\usepackage{accents}
\usepackage{esvect}
\usepackage{mathrsfs}
\usepackage{mathtools}
\mathtoolsset{showonlyrefs}
\usepackage{cmll}
\usepackage{stmaryrd}
\usepackage{physics}
\usepackage[normalem]{ulem}
\usepackage{ebproof}
\usepackage{extarrows}

% Page layout
\usepackage{geometry, a4wide, parskip, fancyhdr}

% Font, encoding, russian support
\usepackage[russian]{babel}
\usepackage[sb]{libertine}
\usepackage{xltxtra}

% Listings
\usepackage{listings}
\lstset{basicstyle=\ttfamily,breaklines=true}
\setmonofont{Inconsolata}

% Miscellaneous
\usepackage{array}
\usepackage{calc}
\usepackage{caption}
\usepackage{subcaption}
\captionsetup{justification=centering,margin=2cm}
\usepackage{catchfilebetweentags}
\usepackage{enumitem}
\usepackage{etoolbox}
\usepackage{float}
\usepackage{lastpage}
\usepackage{minted}
\usepackage{svg}
\usepackage{wrapfig}
\usepackage{xcolor}
\usepackage[makeroom]{cancel}

\newcolumntype{L}{>{$}l<{$}}
    \newcolumntype{C}{>{$}c<{$}}
\newcolumntype{R}{>{$}r<{$}}

% Footnotes
\usepackage[hang]{footmisc}
\setlength{\footnotemargin}{2mm}
\makeatletter
\def\blfootnote{\gdef\@thefnmark{}\@footnotetext}
\makeatother

% References
\usepackage{hyperref}
\hypersetup{
    colorlinks,
    linkcolor={blue!80!black},
    citecolor={blue!80!black},
    urlcolor={blue!80!black},
}

% tikz
\usepackage{tikz}
\usepackage{tikz-cd}
\usetikzlibrary{arrows.meta}
\usetikzlibrary{decorations.pathmorphing}
\usetikzlibrary{calc}
\usetikzlibrary{patterns}
\usepackage{pgfplots}
\pgfplotsset{width=10cm,compat=1.9}
\newcommand\irregularcircle[2]{% radius, irregularity
    \pgfextra {\pgfmathsetmacro\len{(#1)+rand*(#2)}}
    +(0:\len pt)
    \foreach \a in {10,20,...,350}{
            \pgfextra {\pgfmathsetmacro\len{(#1)+rand*(#2)}}
            -- +(\a:\len pt)
        } -- cycle
}

\providetoggle{useproofs}
\settoggle{useproofs}{false}

\pagestyle{fancy}
\lfoot{M3137y2019}
\cfoot{}
\rhead{стр. \thepage\ из \pageref*{LastPage}}

\newcommand{\R}{\mathbb{R}}
\newcommand{\Q}{\mathbb{Q}}
\newcommand{\Z}{\mathbb{Z}}
\newcommand{\B}{\mathbb{B}}
\newcommand{\N}{\mathbb{N}}
\renewcommand{\Re}{\mathfrak{R}}
\renewcommand{\Im}{\mathfrak{I}}

\newcommand{\const}{\text{const}}
\newcommand{\cond}{\text{cond}}

\newcommand{\teormin}{\textcolor{red}{!}\ }

\DeclareMathOperator*{\xor}{\oplus}
\DeclareMathOperator*{\equ}{\sim}
\DeclareMathOperator{\sign}{\text{sign}}
\DeclareMathOperator{\Sym}{\text{Sym}}
\DeclareMathOperator{\Asym}{\text{Asym}}

\DeclarePairedDelimiter{\ceil}{\lceil}{\rceil}

% godel
\newbox\gnBoxA
\newdimen\gnCornerHgt
\setbox\gnBoxA=\hbox{$\ulcorner$}
\global\gnCornerHgt=\ht\gnBoxA
\newdimen\gnArgHgt
\def\godel #1{%
    \setbox\gnBoxA=\hbox{$#1$}%
    \gnArgHgt=\ht\gnBoxA%
    \ifnum     \gnArgHgt<\gnCornerHgt \gnArgHgt=0pt%
    \else \advance \gnArgHgt by -\gnCornerHgt%
    \fi \raise\gnArgHgt\hbox{$\ulcorner$} \box\gnBoxA %
    \raise\gnArgHgt\hbox{$\urcorner$}}

% \theoremstyle{plain}

\theoremstyle{definition}
\newtheorem{theorem}{Теорема}
\newtheorem*{definition}{Определение}
\newtheorem{axiom}{Аксиома}
\newtheorem*{axiom*}{Аксиома}
\newtheorem{lemma}{Лемма}

\theoremstyle{remark}
\newtheorem*{remark}{Примечание}
\newtheorem*{exercise}{Упражнение}
\newtheorem{corollary}{Следствие}[theorem]
\newtheorem*{statement}{Утверждение}
\newtheorem*{corollary*}{Следствие}
\newtheorem*{example}{Пример}
\newtheorem{observation}{Наблюдение}
\newtheorem*{prop}{Свойства}
\newtheorem*{obozn}{Обозначение}

% subtheorem
\makeatletter
\newenvironment{subtheorem}[1]{%
    \def\subtheoremcounter{#1}%
    \refstepcounter{#1}%
    \protected@edef\theparentnumber{\csname the#1\endcsname}%
    \setcounter{parentnumber}{\value{#1}}%
    \setcounter{#1}{0}%
    \expandafter\def\csname the#1\endcsname{\theparentnumber.\Alph{#1}}%
    \ignorespaces
}{%
    \setcounter{\subtheoremcounter}{\value{parentnumber}}%
    \ignorespacesafterend
}
\makeatother
\newcounter{parentnumber}

\newtheorem{manualtheoreminner}{Теорема}
\newenvironment{manualtheorem}[1]{%
    \renewcommand\themanualtheoreminner{#1}%
    \manualtheoreminner
}{\endmanualtheoreminner}

\newcommand{\dbltilde}[1]{\accentset{\approx}{#1}}
\newcommand{\intt}{\int\!}

% magical thing that fixes paragraphs
\makeatletter
\patchcmd{\CatchFBT@Fin@l}{\endlinechar\m@ne}{}
{}{\typeout{Unsuccessful patch!}}
\makeatother

\newcommand{\get}[2]{
    \ExecuteMetaData[#1]{#2}
}

\newcommand{\getproof}[2]{
    \iftoggle{useproofs}{\ExecuteMetaData[#1]{#2proof}}{}
}

\newcommand{\getwithproof}[2]{
    \get{#1}{#2}
    \getproof{#1}{#2}
}

\newcommand{\import}[3]{
    \subsection{#1}
    \getwithproof{#2}{#3}
}

\newcommand{\given}[1]{
    Дано выше. (\ref{#1}, стр. \pageref{#1})
}

\renewcommand{\ker}{\text{Ker }}
\newcommand{\im}{\text{Im }}
\renewcommand{\grad}{\text{grad}}
\newcommand{\rg}{\text{rg}}
\newcommand{\defeq}{\stackrel{\text{def}}{=}}
\newcommand{\defeqfor}[1]{\stackrel{\text{def } #1}{=}}
\newcommand{\itemfix}{\leavevmode\makeatletter\makeatother}
\newcommand{\?}{\textcolor{red}{???}}
\renewcommand{\emptyset}{\varnothing}
\newcommand{\longarrow}[1]{\xRightarrow[#1]{\qquad}}
\DeclareMathOperator*{\esup}{\text{ess sup}}
\newcommand\smallO{
    \mathchoice
    {{\scriptstyle\mathcal{O}}}% \displaystyle
    {{\scriptstyle\mathcal{O}}}% \textstyle
    {{\scriptscriptstyle\mathcal{O}}}% \scriptstyle
    {\scalebox{.6}{$\scriptscriptstyle\mathcal{O}$}}%\scriptscriptstyle
}
\renewcommand{\div}{\text{div}\ }
\newcommand{\rot}{\text{rot}\ }
\newcommand{\cov}{\text{cov}}

\makeatletter
\newcommand{\oplabel}[1]{\refstepcounter{equation}(\theequation\ltx@label{#1})}
\makeatother

\newcommand{\symref}[2]{\stackrel{\oplabel{#1}}{#2}}
\newcommand{\symrefeq}[1]{\symref{#1}{=}}

% xrightrightarrows
\makeatletter
\newcommand*{\relrelbarsep}{.386ex}
\newcommand*{\relrelbar}{%
    \mathrel{%
        \mathpalette\@relrelbar\relrelbarsep
    }%
}
\newcommand*{\@relrelbar}[2]{%
    \raise#2\hbox to 0pt{$\m@th#1\relbar$\hss}%
    \lower#2\hbox{$\m@th#1\relbar$}%
}
\providecommand*{\rightrightarrowsfill@}{%
    \arrowfill@\relrelbar\relrelbar\rightrightarrows
}
\providecommand*{\leftleftarrowsfill@}{%
    \arrowfill@\leftleftarrows\relrelbar\relrelbar
}
\providecommand*{\xrightrightarrows}[2][]{%
    \ext@arrow 0359\rightrightarrowsfill@{#1}{#2}%
}
\providecommand*{\xleftleftarrows}[2][]{%
    \ext@arrow 3095\leftleftarrowsfill@{#1}{#2}%
}

\allowdisplaybreaks

\newcommand{\unfinished}{\textcolor{red}{Не дописано}}

% Reproducible pdf builds 
\special{pdf:trailerid [
<00112233445566778899aabbccddeeff>
<00112233445566778899aabbccddeeff>
]}
%</preamble>


\lhead{Линейная алгерба}
\cfoot{}
\rfoot{Лекция 13}

\renewcommand{\thesubsection}{\arabic{subsection}.}
\makeatletter
\renewcommand*{\@seccntformat}[1]{\csname the#1\endcsname\hspace{0.1cm}}
\makeatother

\begin{document}


\begin{definition}
    $U : X\to X$, такой что:
    \begin{enumerate}
        \item $\forall x,y \quad \langle Ux, Uy \rangle = \langle x, y\rangle$
        \item $\forall x \quad ||Ux||=||x||$
        \item $U^* = U^{-1} \Leftrightarrow U^*U=UU^*=I$
    \end{enumerate}
    называется \textbf{унитарным оператором}
\end{definition}

\begin{theorem}
    Свойства 1,2 и 3 эквивалентны.
\end{theorem}
\begin{proof}
    \begin{itemize}
        \item $1\Rightarrow 2$
              $$\forall x,y \quad \langle Ux, Uy \rangle = \langle x, y\rangle \stackrel{?}{\Rightarrow} ||Ux||=||x||$$
              $$||Ux||^2 = \langle Ux, Uy \rangle = \langle x, y\rangle = ||x||^2$$
        \item $2\Rightarrow 3$
              $$||Ux||=||x|| \stackrel{?}{\Rightarrow} U^*U=I$$
              $$||Ux||^2 = \langle Ux, Ux \rangle = \langle U^*Ux, x \rangle = \langle x, x\rangle$$
              $$U^*U=I$$
        \item $3\Rightarrow 1$
              $$U^*=U^{-1} \stackrel{?}{\Rightarrow} \langle Ux, Uy \rangle = \langle x, y\rangle$$
              $$\langle Ux, Uy \rangle = \langle U^*Ux, y \rangle = \langle x, y\rangle$$
    \end{itemize}
\end{proof}
\begin{lemma}
    $|\det U| = 1$
\end{lemma}
\begin{proof}
    $$1 = \det I = \det(U^*U)=\det U^* \det U \defeq \det \overline U^T \det U = \det \overline U \det U=\overline {\det U} \det U = |\det U|^2$$
\end{proof}

\begin{lemma}
    Матрица унитарного преобразования обладает свойством ортогональности по строкам и столбцам.
\end{lemma}
\begin{proof}
    Здесь $\mathcal U$ - оператор, $U$ - его матрица:
    $$\mathcal U \leftrightarrow U=||U_{ij}||$$
    $$\mathcal U^* \mathcal U=I \Rightarrow U^+U = E$$
    В следующей строке подразумевается $\forall i,k$
    $$\sum_{j=1}^n \overline U_{ji}U_{jk}=\sum_{j=1}^n (\overline U^T)_{ij} U_{jk} = \delta{ik}$$
\end{proof}
\begin{example}
    Матрица поворота - ортогональное преобразование.
\end{example}
\begin{lemma}
    Множество унитарных операторов образует мультипликативную группу $U(n)$:
    \begin{enumerate}
        \item $U_1, U_2\in U(n) \Rightarrow U_1\cdot U_2\in U(n)$
        \item $\exists I : I^*=I$
        \item $\forall U \ \ \exists U^{-1} = U^*$
        \item $U_1(U_2U_3)=(U_1U_2)U_3=U_1U_2U_3$
    \end{enumerate}
\end{lemma}
\begin{proof}
    \begin{enumerate}
        \item $U_1U_2$ - унитарный?
              $$\langle U_1U_2x, U_1U_2y \rangle = \langle U_1^*U_1U_2x, U_2y \rangle = \langle U_2 x, U_2 y\rangle = \langle x, y\rangle$$

              Остальное очевидно.
    \end{enumerate}
\end{proof}

$U(n)$ называется \textbf{унитарной группой операторов над унитарным пространством $X$, $\dim X = n$}

$\sphericalangle SU(n) \defeq \{U\in U(n) : \det U = 1\}$

\begin{lemma}
    $SU(n)$ --- подгруппа $U(n)$
\end{lemma}
\begin{lemma}
    Все собственные значения унитарного оператора по модулю равны единице.
    $$\lambda\in \sigma_U \Rightarrow |\lambda|=1 \Leftrightarrow \lambda = e^{i\varphi}$$
\end{lemma}
\begin{proof}
    $] Ux = \lambda x$
    $$||x||=||Ux||=||\lambda x||=\lambda ||x||$$
\end{proof}
\begin{lemma}
    Собственные вектора $U$, отвечающие различным собственным значениям, являются ортогональными.
\end{lemma}
\begin{proof}
\end{proof}

\begin{lemma}
    Любое инвариантное подпространство $U$ является приводящим.
    $$X = L + L^\perp \quad y\in L^\perp \Rightarrow Uy \in L^+$$
\end{lemma}
\begin{proof}
    $$\sphericalangle y\in L^\perp : 0 \langle x, y\rangle = \langle Ux, Uy\rangle=0$$
\end{proof}
\begin{theorem}
    Из собственных векторов унитарного оператора можно построить ортонормированный базис.
\end{theorem}
\begin{proof}
    Очевидно от противного, как с эрмитовым оператором.
\end{proof}

\begin{remark}
    Унитарный оператор имеет скалярный тип, ортогональный оператор \textit{(унитарный, но над $\C$)} может не иметь.
\end{remark}

\begin{theorem}
    Спектральная теорема для унитарного оператора:
    $$U=\sum_{j=1}^n \lambda_j \mathcal P_j = \sum_{j=1}^n e^{i\varphi_j}\langle e^j, \cdot \rangle e_j$$
\end{theorem}

\begin{theorem}
    Эрмитова матрица может быть приведена к диагональной формме унитарным преобразованием:
    $$\varphi^* = \varphi \Rightarrow \exists \mathcal U\in \mathcal U(n) : A_\varphi^d = U^+A_\varphi U$$
\end{theorem}
\begin{proof}
    $$A_\varphi^d = T^{-1}A_\varphi T$$
    $T$ --- состоит из собственных векторов $\varphi$, но $\varphi^*=\varphi \Rightarrow$ столбцы $T$ ортогональны $\Rightarrow T = U \leftrightarrow \mathcal U\in\mathcal U(n)$
\end{proof}

\begin{remark}
    $\varphi$ --- эрмитовский оператор $\Rightarrow e^{i\varphi}$ --- унитарный оператор.
\end{remark}
\begin{proof}
    $$(e^{i\varphi})^* = e^{-i\varphi^*}=e^{-i\varphi}$$
    $$(e^{i\varphi})^*e^{i\varphi}=I$$
\end{proof}

\section*{Квадратичные формы}

$] X$ --- линейное пространство

\begin{definition}
    Отображение $b : X\times X \to K$ --- \textbf{билинейная форма}, если выполняется следующее:
    \begin{enumerate}
        \item $K=\R:$ $b(x, y) = b(y, x) \ \ b(\alpha x + y, z) = \alpha b(x, z) + b(y, z)$
        \item $K=\C:$ $b(x, y) = \overline{b(y, x)} \ \ b(\alpha x + y, z) = \overline\alpha b(x, z) + b(y, z)$
    \end{enumerate}
\end{definition}
\begin{remark}
    $b\in \Omega_0^2$ --- тензор типа $(2, 0)$
\end{remark}

$] \{e_j\}_{j=1}^n$ --- базис $X$
$$\forall x,y\in X \quad x=\sum_{j=1}^n \xi^j e_j \quad y = \sum_{k=1}^n \eta^k e_k$$
$$b(x, y) = b(\sum_{j=1}^n \xi^j e_j, \sum_{k=1}^n \eta^k e_k)=\sum_{k,j=1}^n \overline \xi^j \eta^k \cdot \underbrace{b(e_j, e_k)}_{\text{элемент тензора }b}=\sum_{k,j=1}^n \overline \xi^j \eta^k b_{jk}$$

\begin{remark}
    В матричной форме $b(x,y) = \xi^+ B \eta$
\end{remark}

\begin{definition}
    \textbf{Квадратичной формой}, соответствующей билиненой форме $b$, называется отображение $q$:
    $$q(x)=b(x, x)$$
\end{definition}

\begin{lemma}
    $\{e_j\}_{j=1}^n \xrightarrow{T} \{\tilde e_k\}_{k=1}^n \Rightarrow \tilde Q = T^T Q T$
\end{lemma}
\begin{proof}
    Очевидно.
\end{proof}

\textcolor{red}{Скипнуто до конца лекции.}

\end{document}