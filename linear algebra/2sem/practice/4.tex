\documentclass[12pt, a4paper]{article}

%<*preamble>
% Math symbols
\usepackage{amsmath, amsthm, amsfonts, amssymb}
\usepackage{accents}
\usepackage{esvect}
\usepackage{mathrsfs}
\usepackage{mathtools}
\mathtoolsset{showonlyrefs}
\usepackage{cmll}
\usepackage{stmaryrd}
\usepackage{physics}
\usepackage[normalem]{ulem}
\usepackage{ebproof}
\usepackage{extarrows}

% Page layout
\usepackage{geometry, a4wide, parskip, fancyhdr}

% Font, encoding, russian support
\usepackage[russian]{babel}
\usepackage[sb]{libertine}
\usepackage{xltxtra}

% Listings
\usepackage{listings}
\lstset{basicstyle=\ttfamily,breaklines=true}
\setmonofont{Inconsolata}

% Miscellaneous
\usepackage{array}
\usepackage{calc}
\usepackage{caption}
\usepackage{subcaption}
\captionsetup{justification=centering,margin=2cm}
\usepackage{catchfilebetweentags}
\usepackage{enumitem}
\usepackage{etoolbox}
\usepackage{float}
\usepackage{lastpage}
\usepackage{minted}
\usepackage{svg}
\usepackage{wrapfig}
\usepackage{xcolor}
\usepackage[makeroom]{cancel}

\newcolumntype{L}{>{$}l<{$}}
    \newcolumntype{C}{>{$}c<{$}}
\newcolumntype{R}{>{$}r<{$}}

% Footnotes
\usepackage[hang]{footmisc}
\setlength{\footnotemargin}{2mm}
\makeatletter
\def\blfootnote{\gdef\@thefnmark{}\@footnotetext}
\makeatother

% References
\usepackage{hyperref}
\hypersetup{
    colorlinks,
    linkcolor={blue!80!black},
    citecolor={blue!80!black},
    urlcolor={blue!80!black},
}

% tikz
\usepackage{tikz}
\usepackage{tikz-cd}
\usetikzlibrary{arrows.meta}
\usetikzlibrary{decorations.pathmorphing}
\usetikzlibrary{calc}
\usetikzlibrary{patterns}
\usepackage{pgfplots}
\pgfplotsset{width=10cm,compat=1.9}
\newcommand\irregularcircle[2]{% radius, irregularity
    \pgfextra {\pgfmathsetmacro\len{(#1)+rand*(#2)}}
    +(0:\len pt)
    \foreach \a in {10,20,...,350}{
            \pgfextra {\pgfmathsetmacro\len{(#1)+rand*(#2)}}
            -- +(\a:\len pt)
        } -- cycle
}

\providetoggle{useproofs}
\settoggle{useproofs}{false}

\pagestyle{fancy}
\lfoot{M3137y2019}
\cfoot{}
\rhead{стр. \thepage\ из \pageref*{LastPage}}

\newcommand{\R}{\mathbb{R}}
\newcommand{\Q}{\mathbb{Q}}
\newcommand{\Z}{\mathbb{Z}}
\newcommand{\B}{\mathbb{B}}
\newcommand{\N}{\mathbb{N}}
\renewcommand{\Re}{\mathfrak{R}}
\renewcommand{\Im}{\mathfrak{I}}

\newcommand{\const}{\text{const}}
\newcommand{\cond}{\text{cond}}

\newcommand{\teormin}{\textcolor{red}{!}\ }

\DeclareMathOperator*{\xor}{\oplus}
\DeclareMathOperator*{\equ}{\sim}
\DeclareMathOperator{\sign}{\text{sign}}
\DeclareMathOperator{\Sym}{\text{Sym}}
\DeclareMathOperator{\Asym}{\text{Asym}}

\DeclarePairedDelimiter{\ceil}{\lceil}{\rceil}

% godel
\newbox\gnBoxA
\newdimen\gnCornerHgt
\setbox\gnBoxA=\hbox{$\ulcorner$}
\global\gnCornerHgt=\ht\gnBoxA
\newdimen\gnArgHgt
\def\godel #1{%
    \setbox\gnBoxA=\hbox{$#1$}%
    \gnArgHgt=\ht\gnBoxA%
    \ifnum     \gnArgHgt<\gnCornerHgt \gnArgHgt=0pt%
    \else \advance \gnArgHgt by -\gnCornerHgt%
    \fi \raise\gnArgHgt\hbox{$\ulcorner$} \box\gnBoxA %
    \raise\gnArgHgt\hbox{$\urcorner$}}

% \theoremstyle{plain}

\theoremstyle{definition}
\newtheorem{theorem}{Теорема}
\newtheorem*{definition}{Определение}
\newtheorem{axiom}{Аксиома}
\newtheorem*{axiom*}{Аксиома}
\newtheorem{lemma}{Лемма}

\theoremstyle{remark}
\newtheorem*{remark}{Примечание}
\newtheorem*{exercise}{Упражнение}
\newtheorem{corollary}{Следствие}[theorem]
\newtheorem*{statement}{Утверждение}
\newtheorem*{corollary*}{Следствие}
\newtheorem*{example}{Пример}
\newtheorem{observation}{Наблюдение}
\newtheorem*{prop}{Свойства}
\newtheorem*{obozn}{Обозначение}

% subtheorem
\makeatletter
\newenvironment{subtheorem}[1]{%
    \def\subtheoremcounter{#1}%
    \refstepcounter{#1}%
    \protected@edef\theparentnumber{\csname the#1\endcsname}%
    \setcounter{parentnumber}{\value{#1}}%
    \setcounter{#1}{0}%
    \expandafter\def\csname the#1\endcsname{\theparentnumber.\Alph{#1}}%
    \ignorespaces
}{%
    \setcounter{\subtheoremcounter}{\value{parentnumber}}%
    \ignorespacesafterend
}
\makeatother
\newcounter{parentnumber}

\newtheorem{manualtheoreminner}{Теорема}
\newenvironment{manualtheorem}[1]{%
    \renewcommand\themanualtheoreminner{#1}%
    \manualtheoreminner
}{\endmanualtheoreminner}

\newcommand{\dbltilde}[1]{\accentset{\approx}{#1}}
\newcommand{\intt}{\int\!}

% magical thing that fixes paragraphs
\makeatletter
\patchcmd{\CatchFBT@Fin@l}{\endlinechar\m@ne}{}
{}{\typeout{Unsuccessful patch!}}
\makeatother

\newcommand{\get}[2]{
    \ExecuteMetaData[#1]{#2}
}

\newcommand{\getproof}[2]{
    \iftoggle{useproofs}{\ExecuteMetaData[#1]{#2proof}}{}
}

\newcommand{\getwithproof}[2]{
    \get{#1}{#2}
    \getproof{#1}{#2}
}

\newcommand{\import}[3]{
    \subsection{#1}
    \getwithproof{#2}{#3}
}

\newcommand{\given}[1]{
    Дано выше. (\ref{#1}, стр. \pageref{#1})
}

\renewcommand{\ker}{\text{Ker }}
\newcommand{\im}{\text{Im }}
\renewcommand{\grad}{\text{grad}}
\newcommand{\rg}{\text{rg}}
\newcommand{\defeq}{\stackrel{\text{def}}{=}}
\newcommand{\defeqfor}[1]{\stackrel{\text{def } #1}{=}}
\newcommand{\itemfix}{\leavevmode\makeatletter\makeatother}
\newcommand{\?}{\textcolor{red}{???}}
\renewcommand{\emptyset}{\varnothing}
\newcommand{\longarrow}[1]{\xRightarrow[#1]{\qquad}}
\DeclareMathOperator*{\esup}{\text{ess sup}}
\newcommand\smallO{
    \mathchoice
    {{\scriptstyle\mathcal{O}}}% \displaystyle
    {{\scriptstyle\mathcal{O}}}% \textstyle
    {{\scriptscriptstyle\mathcal{O}}}% \scriptstyle
    {\scalebox{.6}{$\scriptscriptstyle\mathcal{O}$}}%\scriptscriptstyle
}
\renewcommand{\div}{\text{div}\ }
\newcommand{\rot}{\text{rot}\ }
\newcommand{\cov}{\text{cov}}

\makeatletter
\newcommand{\oplabel}[1]{\refstepcounter{equation}(\theequation\ltx@label{#1})}
\makeatother

\newcommand{\symref}[2]{\stackrel{\oplabel{#1}}{#2}}
\newcommand{\symrefeq}[1]{\symref{#1}{=}}

% xrightrightarrows
\makeatletter
\newcommand*{\relrelbarsep}{.386ex}
\newcommand*{\relrelbar}{%
    \mathrel{%
        \mathpalette\@relrelbar\relrelbarsep
    }%
}
\newcommand*{\@relrelbar}[2]{%
    \raise#2\hbox to 0pt{$\m@th#1\relbar$\hss}%
    \lower#2\hbox{$\m@th#1\relbar$}%
}
\providecommand*{\rightrightarrowsfill@}{%
    \arrowfill@\relrelbar\relrelbar\rightrightarrows
}
\providecommand*{\leftleftarrowsfill@}{%
    \arrowfill@\leftleftarrows\relrelbar\relrelbar
}
\providecommand*{\xrightrightarrows}[2][]{%
    \ext@arrow 0359\rightrightarrowsfill@{#1}{#2}%
}
\providecommand*{\xleftleftarrows}[2][]{%
    \ext@arrow 3095\leftleftarrowsfill@{#1}{#2}%
}

\allowdisplaybreaks

\newcommand{\unfinished}{\textcolor{red}{Не дописано}}

% Reproducible pdf builds 
\special{pdf:trailerid [
<00112233445566778899aabbccddeeff>
<00112233445566778899aabbccddeeff>
]}
%</preamble>


\lhead{Линейная алгерба}
\cfoot{}
\rfoot{Практика 4}

\begin{document}

$] x = \begin{pmatrix}
        \xi^1 & \xi^2 & \xi^3
    \end{pmatrix} \quad \varphi(x) = \begin{pmatrix}
        \xi^1+\xi^2, \xi^2, \xi^3
    \end{pmatrix}$ --- линеный оператор.

\begin{definition}
    \textbf{Ядро} ЛОп $\varphi$ называется множество
    $$Ker \varphi = \{ x\in X : \varphi x = O_Y\}$$
\end{definition}
\begin{remark}
    $Ker \varphi$ --- подпространство ЛП $X$
\end{remark}

\begin{definition}
    \textbf{Образом} ЛОп $\varphi$ называется множество:
    $$Im \varphi = \{y\in Y : \exists x \quad \varphi x = y\} = \varphi(X)$$
\end{definition}
\begin{remark}
    $Im \varphi$ --- подпространство ЛП $Y$
\end{remark}
\begin{proof}
    $y_1, y_2\in Im \varphi \Rightarrow \exists x_1, x_2 \in X : \varphi x_1=y_1 \ \ \varphi x_2 = y_2$

    $\sphericalangle y_1 + y_2 = \varphi(x_1) + \varphi(x_2) = \varphi(x_1+x_2) \Rightarrow y_1+y_2\in Im \varphi$
\end{proof}

\begin{example}
    $E_3$ --- евклидово пространство

    $\varphi E_3 \to E_3$

    $$\varphi(\vec x) = \vec x - \frac{(\vec x \vec n)}{(\vec n \vec n)}\vec n \quad \vec n\not=\vec 0$$

    \begin{enumerate}
        \item $Ker \varphi$
        \item $Im \varphi$
        \item Геометрический смысл
    \end{enumerate}

    $Ker \varphi:$
    $$\vec x - \frac{(\vec x \vec n)}{(\vec n \vec n)}\vec n=0$$
    $$\vec x = \frac{(\vec x \vec n)}{(\vec n \vec n)}\vec n \Rightarrow Ker \varphi = \mathcal{L}(\vec n)$$

    $Im \varphi:$
    $$\sphericalangle \varphi(\vec y) = \vec y - \frac{(\vec y \vec n)}{(\vec n \vec n)}\vec n$$
    $$y\in\mathcal{L}^\perp (\vec n) \Rightarrow \varphi(y) = y \Rightarrow Im \varphi = \mathcal{L}^\perp (\vec n)$$

    $$\mathcal{L}(\vec n)\cap\mathcal{L}^\perp(\vec n)=0 \Rightarrow E_3=\mathcal{L}(\vec n) + \mathcal{L}^\perp(\vec n)$$
\end{example}

\begin{example}
    $E_3$

    $\varphi : E_3\to E_3$
    $$\varphi(\vec x) = \vec x - \frac{(\vec x \vec n)}{(\vec a \vec n)}\vec a \quad (\vec a, \vec n)\not=0$$

    $Ker \varphi=\mathcal{L}(\vec a)$

    $\vec x \stackrel{!}{=} \vec y + \alpha \vec a$
\end{example}

$] \{e_j\}_{j=1}^n$ --- базис $X$, $\{h_k\}_{k=1}^m$ --- базис $Y$ $\Rightarrow \varphi(e_j)=\sum\limits_{k=1}^m a_j^kh_k$

$\sphericalangle A_\varphi=||a_j^k||$ --- матрица оператора $\varphi$ в паре выбранных базисов

\begin{example}
    Найти матрицу оператора $\varphi$ в стандартном базисе $E_3$

    $\vec n = \begin{pmatrix}
            1 & 2 & 3
        \end{pmatrix}\tran $

    $$\varphi(e_1)=\begin{bmatrix}
            1 \\ 0 \\ 0
        \end{bmatrix} - \frac{1}{14}\begin{bmatrix}
            1 \\
            2 \\
            3
        \end{bmatrix}=\begin{bmatrix}
            \frac{13}{14} \\
            \frac{-2}{14} \\
            \frac{-3}{14}
        \end{bmatrix}$$

    $$\varphi(e_2)=\begin{bmatrix}
            0 \\ 1 \\ 0
        \end{bmatrix} - \frac{w}{14}\begin{bmatrix}
            1 \\
            2 \\
            3
        \end{bmatrix}=\begin{bmatrix}
            \frac{-2}{14} \\
            \frac{10}{14} \\
            \frac{-6}{14}
        \end{bmatrix}$$
    То же самое для $e_3$ и собрать все в матрицу.
\end{example}

\begin{example}
    $\varphi_\theta$ --- оператор поворота вокруг $\vec S$ на $\theta$

    $] y = \beta\vec a + \gamma\vec b, \vec a\perp\vec b\perp \vec c, |\vec a|=|\vec b|=1$

    $]\vec x = \alpha\vec s + \vec y = \alpha\vec s + \beta\vec a + \gamma\vec b$

    $\varphi(x)=\varphi(\alpha\vec s + \beta\vec a + \gamma\vec b) = \alpha\vec s + (\beta\cos\theta-\gamma\sin\theta)\vec a + (\beta\sin\theta+\gamma\cos\theta)\vec b$

    $\varphi(e_1)=...$ и так далее
\end{example}

$] \varphi : X \to X$

$\{e_j\}_{j=1}^n$ --- базис $X \Rightarrow A_\varphi$

$\{\tilde e_k\}_{k=1}^n$ --- базис $X \Rightarrow \tilde A_\varphi$

$\{e\}\stackrel{T}{\to}\{\tilde e\}$

$$\sphericalangle \varphi(\tilde e_j)=\sum\limits_{k=1}^n\tilde a_j^k \tilde e_k=\sum\limits_{k=1}^n\tilde a_j^k \sum\limits_{l=1}^n t_k^le_l$$
$$\varphi(\tilde e_j) = \varphi\left(\sum\limits_{l=1}^n \tilde a_j^le_l\right)=\sum\limits_{l=1}^n \tilde a_j^l\varphi (e_l)=\sum\limits_{l=1}^n t_j^l\sum\limits_{k=1}^n a_l^k e_k$$
$$\sum\limits_{k=1}^n \tilde a_j^k t_k^l = \sum\limits_{k=1}^n t_j^k a_k^l$$
$$(T\tilde A_\varphi)_j^l=(A_\varphi T)_j^l \quad \forall l, j$$
$$T\tilde A_\varphi=A_\varphi T$$
$$\tilde A_\varphi=T^{-1}A_\varphi T$$

\end{document}
