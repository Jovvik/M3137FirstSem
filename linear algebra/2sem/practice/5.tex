\documentclass[12pt, a4paper]{article}

%<*preamble>
% Math symbols
\usepackage{amsmath, amsthm, amsfonts, amssymb}
\usepackage{accents}
\usepackage{esvect}
\usepackage{mathrsfs}
\usepackage{mathtools}
\mathtoolsset{showonlyrefs}
\usepackage{cmll}
\usepackage{stmaryrd}
\usepackage{physics}
\usepackage[normalem]{ulem}
\usepackage{ebproof}
\usepackage{extarrows}

% Page layout
\usepackage{geometry, a4wide, parskip, fancyhdr}

% Font, encoding, russian support
\usepackage[russian]{babel}
\usepackage[sb]{libertine}
\usepackage{xltxtra}

% Listings
\usepackage{listings}
\lstset{basicstyle=\ttfamily,breaklines=true}
\setmonofont{Inconsolata}

% Miscellaneous
\usepackage{array}
\usepackage{calc}
\usepackage{caption}
\usepackage{subcaption}
\captionsetup{justification=centering,margin=2cm}
\usepackage{catchfilebetweentags}
\usepackage{enumitem}
\usepackage{etoolbox}
\usepackage{float}
\usepackage{lastpage}
\usepackage{minted}
\usepackage{svg}
\usepackage{wrapfig}
\usepackage{xcolor}
\usepackage[makeroom]{cancel}

\newcolumntype{L}{>{$}l<{$}}
    \newcolumntype{C}{>{$}c<{$}}
\newcolumntype{R}{>{$}r<{$}}

% Footnotes
\usepackage[hang]{footmisc}
\setlength{\footnotemargin}{2mm}
\makeatletter
\def\blfootnote{\gdef\@thefnmark{}\@footnotetext}
\makeatother

% References
\usepackage{hyperref}
\hypersetup{
    colorlinks,
    linkcolor={blue!80!black},
    citecolor={blue!80!black},
    urlcolor={blue!80!black},
}

% tikz
\usepackage{tikz}
\usepackage{tikz-cd}
\usetikzlibrary{arrows.meta}
\usetikzlibrary{decorations.pathmorphing}
\usetikzlibrary{calc}
\usetikzlibrary{patterns}
\usepackage{pgfplots}
\pgfplotsset{width=10cm,compat=1.9}
\newcommand\irregularcircle[2]{% radius, irregularity
    \pgfextra {\pgfmathsetmacro\len{(#1)+rand*(#2)}}
    +(0:\len pt)
    \foreach \a in {10,20,...,350}{
            \pgfextra {\pgfmathsetmacro\len{(#1)+rand*(#2)}}
            -- +(\a:\len pt)
        } -- cycle
}

\providetoggle{useproofs}
\settoggle{useproofs}{false}

\pagestyle{fancy}
\lfoot{M3137y2019}
\cfoot{}
\rhead{стр. \thepage\ из \pageref*{LastPage}}

\newcommand{\R}{\mathbb{R}}
\newcommand{\Q}{\mathbb{Q}}
\newcommand{\Z}{\mathbb{Z}}
\newcommand{\B}{\mathbb{B}}
\newcommand{\N}{\mathbb{N}}
\renewcommand{\Re}{\mathfrak{R}}
\renewcommand{\Im}{\mathfrak{I}}

\newcommand{\const}{\text{const}}
\newcommand{\cond}{\text{cond}}

\newcommand{\teormin}{\textcolor{red}{!}\ }

\DeclareMathOperator*{\xor}{\oplus}
\DeclareMathOperator*{\equ}{\sim}
\DeclareMathOperator{\sign}{\text{sign}}
\DeclareMathOperator{\Sym}{\text{Sym}}
\DeclareMathOperator{\Asym}{\text{Asym}}

\DeclarePairedDelimiter{\ceil}{\lceil}{\rceil}

% godel
\newbox\gnBoxA
\newdimen\gnCornerHgt
\setbox\gnBoxA=\hbox{$\ulcorner$}
\global\gnCornerHgt=\ht\gnBoxA
\newdimen\gnArgHgt
\def\godel #1{%
    \setbox\gnBoxA=\hbox{$#1$}%
    \gnArgHgt=\ht\gnBoxA%
    \ifnum     \gnArgHgt<\gnCornerHgt \gnArgHgt=0pt%
    \else \advance \gnArgHgt by -\gnCornerHgt%
    \fi \raise\gnArgHgt\hbox{$\ulcorner$} \box\gnBoxA %
    \raise\gnArgHgt\hbox{$\urcorner$}}

% \theoremstyle{plain}

\theoremstyle{definition}
\newtheorem{theorem}{Теорема}
\newtheorem*{definition}{Определение}
\newtheorem{axiom}{Аксиома}
\newtheorem*{axiom*}{Аксиома}
\newtheorem{lemma}{Лемма}

\theoremstyle{remark}
\newtheorem*{remark}{Примечание}
\newtheorem*{exercise}{Упражнение}
\newtheorem{corollary}{Следствие}[theorem]
\newtheorem*{statement}{Утверждение}
\newtheorem*{corollary*}{Следствие}
\newtheorem*{example}{Пример}
\newtheorem{observation}{Наблюдение}
\newtheorem*{prop}{Свойства}
\newtheorem*{obozn}{Обозначение}

% subtheorem
\makeatletter
\newenvironment{subtheorem}[1]{%
    \def\subtheoremcounter{#1}%
    \refstepcounter{#1}%
    \protected@edef\theparentnumber{\csname the#1\endcsname}%
    \setcounter{parentnumber}{\value{#1}}%
    \setcounter{#1}{0}%
    \expandafter\def\csname the#1\endcsname{\theparentnumber.\Alph{#1}}%
    \ignorespaces
}{%
    \setcounter{\subtheoremcounter}{\value{parentnumber}}%
    \ignorespacesafterend
}
\makeatother
\newcounter{parentnumber}

\newtheorem{manualtheoreminner}{Теорема}
\newenvironment{manualtheorem}[1]{%
    \renewcommand\themanualtheoreminner{#1}%
    \manualtheoreminner
}{\endmanualtheoreminner}

\newcommand{\dbltilde}[1]{\accentset{\approx}{#1}}
\newcommand{\intt}{\int\!}

% magical thing that fixes paragraphs
\makeatletter
\patchcmd{\CatchFBT@Fin@l}{\endlinechar\m@ne}{}
{}{\typeout{Unsuccessful patch!}}
\makeatother

\newcommand{\get}[2]{
    \ExecuteMetaData[#1]{#2}
}

\newcommand{\getproof}[2]{
    \iftoggle{useproofs}{\ExecuteMetaData[#1]{#2proof}}{}
}

\newcommand{\getwithproof}[2]{
    \get{#1}{#2}
    \getproof{#1}{#2}
}

\newcommand{\import}[3]{
    \subsection{#1}
    \getwithproof{#2}{#3}
}

\newcommand{\given}[1]{
    Дано выше. (\ref{#1}, стр. \pageref{#1})
}

\renewcommand{\ker}{\text{Ker }}
\newcommand{\im}{\text{Im }}
\renewcommand{\grad}{\text{grad}}
\newcommand{\rg}{\text{rg}}
\newcommand{\defeq}{\stackrel{\text{def}}{=}}
\newcommand{\defeqfor}[1]{\stackrel{\text{def } #1}{=}}
\newcommand{\itemfix}{\leavevmode\makeatletter\makeatother}
\newcommand{\?}{\textcolor{red}{???}}
\renewcommand{\emptyset}{\varnothing}
\newcommand{\longarrow}[1]{\xRightarrow[#1]{\qquad}}
\DeclareMathOperator*{\esup}{\text{ess sup}}
\newcommand\smallO{
    \mathchoice
    {{\scriptstyle\mathcal{O}}}% \displaystyle
    {{\scriptstyle\mathcal{O}}}% \textstyle
    {{\scriptscriptstyle\mathcal{O}}}% \scriptstyle
    {\scalebox{.6}{$\scriptscriptstyle\mathcal{O}$}}%\scriptscriptstyle
}
\renewcommand{\div}{\text{div}\ }
\newcommand{\rot}{\text{rot}\ }
\newcommand{\cov}{\text{cov}}

\makeatletter
\newcommand{\oplabel}[1]{\refstepcounter{equation}(\theequation\ltx@label{#1})}
\makeatother

\newcommand{\symref}[2]{\stackrel{\oplabel{#1}}{#2}}
\newcommand{\symrefeq}[1]{\symref{#1}{=}}

% xrightrightarrows
\makeatletter
\newcommand*{\relrelbarsep}{.386ex}
\newcommand*{\relrelbar}{%
    \mathrel{%
        \mathpalette\@relrelbar\relrelbarsep
    }%
}
\newcommand*{\@relrelbar}[2]{%
    \raise#2\hbox to 0pt{$\m@th#1\relbar$\hss}%
    \lower#2\hbox{$\m@th#1\relbar$}%
}
\providecommand*{\rightrightarrowsfill@}{%
    \arrowfill@\relrelbar\relrelbar\rightrightarrows
}
\providecommand*{\leftleftarrowsfill@}{%
    \arrowfill@\leftleftarrows\relrelbar\relrelbar
}
\providecommand*{\xrightrightarrows}[2][]{%
    \ext@arrow 0359\rightrightarrowsfill@{#1}{#2}%
}
\providecommand*{\xleftleftarrows}[2][]{%
    \ext@arrow 3095\leftleftarrowsfill@{#1}{#2}%
}

\allowdisplaybreaks

\newcommand{\unfinished}{\textcolor{red}{Не дописано}}

% Reproducible pdf builds 
\special{pdf:trailerid [
<00112233445566778899aabbccddeeff>
<00112233445566778899aabbccddeeff>
]}
%</preamble>


\lhead{Линейная алгерба}
\cfoot{}
\rfoot{Практика 5}

\usepackage{mathrsfs}

\begin{document}

\section{Линейный оператор}

\begin{example}
    $\sphericalangle \R^2_2 \quad \varphi : A\mapsto A\tran $

    $A_\varphi?$ в стандартном базисе

    Стандартный матричный базис: $E_1=\begin{bmatrix}
            1 & 0 \\
            0 & 0
        \end{bmatrix} \quad E_2=\begin{bmatrix}
            0 & 1 \\
            0 & 0
        \end{bmatrix} \quad E_3 = \begin{bmatrix}
            0 & 0 \\
            1 & 0
        \end{bmatrix} \quad E_3 = \begin{bmatrix}
            0 & 0 \\
            0 & 1
        \end{bmatrix}$

    $$\varphi(E_1)=E_1 \Rightarrow A_{\varphi1}=\begin{pmatrix}
            1 & 0 & 0 & 0
        \end{pmatrix}\tran $$
    $$\varphi(E_2)=E_3 \Rightarrow A_{\varphi2}=\begin{pmatrix}
            0 & 0 & 1 & 0
        \end{pmatrix}\tran $$
    $$\varphi(E_3)=E_2 \Rightarrow A_{\varphi3}=\begin{pmatrix}
            0 & 1 & 0 & 0
        \end{pmatrix}\tran $$
    $$\varphi(E_4)=E_4 \Rightarrow A_{\varphi4}=\begin{pmatrix}
            0 & 0 & 0 & 1
        \end{pmatrix}\tran $$
    $$A_\varphi=\begin{bmatrix}
            1 & 0 & 0 & 0 \\
            0 & 0 & 1 & 0 \\
            0 & 1 & 0 & 0 \\
            0 & 0 & 0 & 1
        \end{bmatrix}$$
    $$Ker \varphi = \{0\}$$
    $$Im \varphi = \R_2^2$$
    $] M = \begin{bmatrix}
            \alpha & \beta  \\
            \gamma & \delta
        \end{bmatrix} \leftrightarrow \begin{pmatrix}
            \alpha & \beta & \gamma & \delta
        \end{pmatrix}\tran =m$

    $Ker\varphi : A_\varphi m=0$ --- это СЛАУ

    Заметим, что эта СЛАУ имеет только тривиальные решения $\Rightarrow \dim Ker \varphi =0 \Rightarrow \dim Im \varphi = \dim \R_2^2 - \dim Ker \varphi = 4 - 0 = 4$
\end{example}

\begin{example}
    $\sphericalangle \mathcal{P}^{x,y}_2$

    $$\varphi p := x\frac{\partial p}{\partial x} + y\frac{\partial p}{\partial y}$$

    $A_\varphi?$ в базисе $\{x^2 \ \ xy \ \ x^2\}$

    $Ker \varphi?$

    $$\varphi(x^2)=2x^2 \quad \varphi(xy) = 2xy \quad \varphi(y^2)=2y^2$$
    $$A_\varphi=\begin{bmatrix}
            2 & 0 & 0 \\
            0 & 2 & 0 \\
            0 & 0 & 2
        \end{bmatrix} \Rightarrow Ker \varphi = \{0\}$$
\end{example}

\begin{example}
    $$\varphi p := y\frac{\partial p}{\partial x} - x\frac{\partial p}{\partial y}$$
    $$\varphi(xy) = y^2-x^2$$
    $$\varphi(y^2) = -2xy$$
    $$A_\varphi = \begin{bmatrix}
            0 & -1 & 0  \\
            2 & 0  & -2 \\
            0 & 1  & 0
        \end{bmatrix}$$
    Найдём ядро:
    $$\begin{bmatrix}
            0 & -1 & 0  \\
            2 & 0  & -2 \\
            0 & 1  & 0
        \end{bmatrix} ~ \begin{bmatrix}
            1 & 0 & -1 \\
            0 & 1 & 0  \\
            0 & 0 & 0
        \end{bmatrix} \Rightarrow \begin{cases}
            x_1 - x_3 = 0 \\
            x_2 = 0
        \end{cases}$$
    ФСР: $\begin{bmatrix}
            1 \\ 0 \\ 1
        \end{bmatrix}$

    $$Ker \varphi = \mathscr{L}(x^2+y^2)$$
\end{example}

\begin{example}
    Найти ядро и образ оператора $\varphi$, заданного в некотором базисе матрицей:

    $\varphi : X\simeq \R^5 \to Y\simeq \R^5$

    $$A_\varphi = \begin{bmatrix}
            1 & 1  & 1  & 2  & -1 \\
            2 & 1  & -2 & -1 & 1  \\
            3 & -1 & 0  & -1 & 1  \\
            1 & -2 & 2  & 0  & 0  \\
            2 & -2 & -1 & -3 & 2
        \end{bmatrix} \sim \begin{bmatrix}
            1 & 0  & 0  & 0  & 0 \\
            2 & -1 & -4 & -5 & 3 \\
            3 & -4 & -3 & -7 & 4 \\
            1 & -3 & 1  & -2 & 1 \\
            2 & -4 & -3 & -7 & 4
        \end{bmatrix} \sim \begin{bmatrix}
            1 & 0 & 0  & 0  \\
            2 & 1 & 0  & 0  \\
            3 & 4 & 13 & -8 \\
            1 & 3 & 13 & -8 \\
            2 & 4 & 13 & -8 \\
        \end{bmatrix}\sim\begin{bmatrix}
            1 & 0 & 0 \\
            2 & 1 & 0 \\
            3 & 4 & 1 \\
            1 & 3 & 1 \\
            2 & 4 & 1 \\
        \end{bmatrix}$$

    Это базис $Im \varphi$
\end{example}

\begin{example}
    $\varphi : \R^5\to\R^3$

    $$A_\varphi = \begin{bmatrix}
            1 & 3  & 5  & 7  & 9  \\
            1 & -2 & 3  & -4 & 5  \\
            2 & 11 & 12 & 25 & 22
        \end{bmatrix} \quad a=\begin{bmatrix} 1 \\ 2 \\ 3\end{bmatrix}$$
    Найдём прообраз $a$
    $$A_\varphi x = a$$
    $$] x = \begin{pmatrix}
            \xi^1 & \xi^2 & \xi^3 & \xi^4 & \xi^5
        \end{pmatrix}\tran $$

    СЛАУ:
    $$\begin{bmatrix}
            1 & 3  & 5  & 7  & 9  & 1 \\
            1 & -2 & 3  & -4 & 5  & 2 \\
            2 & 11 & 12 & 25 & 22 & 1
        \end{bmatrix}$$
\end{example}

\begin{example}
    $\varphi : \R^4 \to \R^3$
    $$A_\varphi = \begin{bmatrix}
            3 & 1  & -2 & 4  \\
            2 & -3 & 6  & -5 \\
            8 & -1 & 2  & 3
        \end{bmatrix} \quad L:\begin{cases}
            x_1 + x_2 = 0 \\
            x_1 - x_3 = 0
        \end{cases}$$
    Найдем прообраз $L$ --- найдем ФСР $L$ и полный прообраз каждого элемента ФСР.
\end{example}

\begin{example}
    $\varphi : \R^3 \to \mathbb{C}_2^2$

    $$\begin{pmatrix}
            \xi^1 & \xi^2 & \xi^3
        \end{pmatrix}\tran \mapsto \begin{bmatrix}
            \xi^3          & \xi^1+i\xi^2 \\
            \xi^1 - i\xi^3 & -\xi^3
        \end{bmatrix}$$

    $Ker \varphi = \{0\}$

    $A_\varphi?$

    $$C_2^2 = \left\{
        \begin{bmatrix}
            1 & 0  \\
            0 & -1
        \end{bmatrix} \quad \begin{bmatrix}
            0 & 1 \\
            1 & 0
        \end{bmatrix} \quad \begin{bmatrix}
            0  & i \\
            -i & 0
        \end{bmatrix}
        \right\}$$

    $$\varphi(e_1)=\begin{bmatrix}
            0 & 1 \\
            1 & 0
        \end{bmatrix} \quad \varphi(e_2) = \begin{bmatrix}
            0  & i \\
            -i & 0
        \end{bmatrix}\quad \varphi(e_3) = \begin{bmatrix}
            1 & 0  \\
            0 & -1
        \end{bmatrix}$$
\end{example}

\end{document}
