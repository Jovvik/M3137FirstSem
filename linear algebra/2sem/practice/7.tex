\documentclass[12pt, a4paper]{article}

%<*preamble>
% Math symbols
\usepackage{amsmath, amsthm, amsfonts, amssymb}
\usepackage{accents}
\usepackage{esvect}
\usepackage{mathrsfs}
\usepackage{mathtools}
\mathtoolsset{showonlyrefs}
\usepackage{cmll}
\usepackage{stmaryrd}
\usepackage{physics}
\usepackage[normalem]{ulem}
\usepackage{ebproof}
\usepackage{extarrows}

% Page layout
\usepackage{geometry, a4wide, parskip, fancyhdr}

% Font, encoding, russian support
\usepackage[russian]{babel}
\usepackage[sb]{libertine}
\usepackage{xltxtra}

% Listings
\usepackage{listings}
\lstset{basicstyle=\ttfamily,breaklines=true}
\setmonofont{Inconsolata}

% Miscellaneous
\usepackage{array}
\usepackage{calc}
\usepackage{caption}
\usepackage{subcaption}
\captionsetup{justification=centering,margin=2cm}
\usepackage{catchfilebetweentags}
\usepackage{enumitem}
\usepackage{etoolbox}
\usepackage{float}
\usepackage{lastpage}
\usepackage{minted}
\usepackage{svg}
\usepackage{wrapfig}
\usepackage{xcolor}
\usepackage[makeroom]{cancel}

\newcolumntype{L}{>{$}l<{$}}
    \newcolumntype{C}{>{$}c<{$}}
\newcolumntype{R}{>{$}r<{$}}

% Footnotes
\usepackage[hang]{footmisc}
\setlength{\footnotemargin}{2mm}
\makeatletter
\def\blfootnote{\gdef\@thefnmark{}\@footnotetext}
\makeatother

% References
\usepackage{hyperref}
\hypersetup{
    colorlinks,
    linkcolor={blue!80!black},
    citecolor={blue!80!black},
    urlcolor={blue!80!black},
}

% tikz
\usepackage{tikz}
\usepackage{tikz-cd}
\usetikzlibrary{arrows.meta}
\usetikzlibrary{decorations.pathmorphing}
\usetikzlibrary{calc}
\usetikzlibrary{patterns}
\usepackage{pgfplots}
\pgfplotsset{width=10cm,compat=1.9}
\newcommand\irregularcircle[2]{% radius, irregularity
    \pgfextra {\pgfmathsetmacro\len{(#1)+rand*(#2)}}
    +(0:\len pt)
    \foreach \a in {10,20,...,350}{
            \pgfextra {\pgfmathsetmacro\len{(#1)+rand*(#2)}}
            -- +(\a:\len pt)
        } -- cycle
}

\providetoggle{useproofs}
\settoggle{useproofs}{false}

\pagestyle{fancy}
\lfoot{M3137y2019}
\cfoot{}
\rhead{стр. \thepage\ из \pageref*{LastPage}}

\newcommand{\R}{\mathbb{R}}
\newcommand{\Q}{\mathbb{Q}}
\newcommand{\Z}{\mathbb{Z}}
\newcommand{\B}{\mathbb{B}}
\newcommand{\N}{\mathbb{N}}
\renewcommand{\Re}{\mathfrak{R}}
\renewcommand{\Im}{\mathfrak{I}}

\newcommand{\const}{\text{const}}
\newcommand{\cond}{\text{cond}}

\newcommand{\teormin}{\textcolor{red}{!}\ }

\DeclareMathOperator*{\xor}{\oplus}
\DeclareMathOperator*{\equ}{\sim}
\DeclareMathOperator{\sign}{\text{sign}}
\DeclareMathOperator{\Sym}{\text{Sym}}
\DeclareMathOperator{\Asym}{\text{Asym}}

\DeclarePairedDelimiter{\ceil}{\lceil}{\rceil}

% godel
\newbox\gnBoxA
\newdimen\gnCornerHgt
\setbox\gnBoxA=\hbox{$\ulcorner$}
\global\gnCornerHgt=\ht\gnBoxA
\newdimen\gnArgHgt
\def\godel #1{%
    \setbox\gnBoxA=\hbox{$#1$}%
    \gnArgHgt=\ht\gnBoxA%
    \ifnum     \gnArgHgt<\gnCornerHgt \gnArgHgt=0pt%
    \else \advance \gnArgHgt by -\gnCornerHgt%
    \fi \raise\gnArgHgt\hbox{$\ulcorner$} \box\gnBoxA %
    \raise\gnArgHgt\hbox{$\urcorner$}}

% \theoremstyle{plain}

\theoremstyle{definition}
\newtheorem{theorem}{Теорема}
\newtheorem*{definition}{Определение}
\newtheorem{axiom}{Аксиома}
\newtheorem*{axiom*}{Аксиома}
\newtheorem{lemma}{Лемма}

\theoremstyle{remark}
\newtheorem*{remark}{Примечание}
\newtheorem*{exercise}{Упражнение}
\newtheorem{corollary}{Следствие}[theorem]
\newtheorem*{statement}{Утверждение}
\newtheorem*{corollary*}{Следствие}
\newtheorem*{example}{Пример}
\newtheorem{observation}{Наблюдение}
\newtheorem*{prop}{Свойства}
\newtheorem*{obozn}{Обозначение}

% subtheorem
\makeatletter
\newenvironment{subtheorem}[1]{%
    \def\subtheoremcounter{#1}%
    \refstepcounter{#1}%
    \protected@edef\theparentnumber{\csname the#1\endcsname}%
    \setcounter{parentnumber}{\value{#1}}%
    \setcounter{#1}{0}%
    \expandafter\def\csname the#1\endcsname{\theparentnumber.\Alph{#1}}%
    \ignorespaces
}{%
    \setcounter{\subtheoremcounter}{\value{parentnumber}}%
    \ignorespacesafterend
}
\makeatother
\newcounter{parentnumber}

\newtheorem{manualtheoreminner}{Теорема}
\newenvironment{manualtheorem}[1]{%
    \renewcommand\themanualtheoreminner{#1}%
    \manualtheoreminner
}{\endmanualtheoreminner}

\newcommand{\dbltilde}[1]{\accentset{\approx}{#1}}
\newcommand{\intt}{\int\!}

% magical thing that fixes paragraphs
\makeatletter
\patchcmd{\CatchFBT@Fin@l}{\endlinechar\m@ne}{}
{}{\typeout{Unsuccessful patch!}}
\makeatother

\newcommand{\get}[2]{
    \ExecuteMetaData[#1]{#2}
}

\newcommand{\getproof}[2]{
    \iftoggle{useproofs}{\ExecuteMetaData[#1]{#2proof}}{}
}

\newcommand{\getwithproof}[2]{
    \get{#1}{#2}
    \getproof{#1}{#2}
}

\newcommand{\import}[3]{
    \subsection{#1}
    \getwithproof{#2}{#3}
}

\newcommand{\given}[1]{
    Дано выше. (\ref{#1}, стр. \pageref{#1})
}

\renewcommand{\ker}{\text{Ker }}
\newcommand{\im}{\text{Im }}
\renewcommand{\grad}{\text{grad}}
\newcommand{\rg}{\text{rg}}
\newcommand{\defeq}{\stackrel{\text{def}}{=}}
\newcommand{\defeqfor}[1]{\stackrel{\text{def } #1}{=}}
\newcommand{\itemfix}{\leavevmode\makeatletter\makeatother}
\newcommand{\?}{\textcolor{red}{???}}
\renewcommand{\emptyset}{\varnothing}
\newcommand{\longarrow}[1]{\xRightarrow[#1]{\qquad}}
\DeclareMathOperator*{\esup}{\text{ess sup}}
\newcommand\smallO{
    \mathchoice
    {{\scriptstyle\mathcal{O}}}% \displaystyle
    {{\scriptstyle\mathcal{O}}}% \textstyle
    {{\scriptscriptstyle\mathcal{O}}}% \scriptstyle
    {\scalebox{.6}{$\scriptscriptstyle\mathcal{O}$}}%\scriptscriptstyle
}
\renewcommand{\div}{\text{div}\ }
\newcommand{\rot}{\text{rot}\ }
\newcommand{\cov}{\text{cov}}

\makeatletter
\newcommand{\oplabel}[1]{\refstepcounter{equation}(\theequation\ltx@label{#1})}
\makeatother

\newcommand{\symref}[2]{\stackrel{\oplabel{#1}}{#2}}
\newcommand{\symrefeq}[1]{\symref{#1}{=}}

% xrightrightarrows
\makeatletter
\newcommand*{\relrelbarsep}{.386ex}
\newcommand*{\relrelbar}{%
    \mathrel{%
        \mathpalette\@relrelbar\relrelbarsep
    }%
}
\newcommand*{\@relrelbar}[2]{%
    \raise#2\hbox to 0pt{$\m@th#1\relbar$\hss}%
    \lower#2\hbox{$\m@th#1\relbar$}%
}
\providecommand*{\rightrightarrowsfill@}{%
    \arrowfill@\relrelbar\relrelbar\rightrightarrows
}
\providecommand*{\leftleftarrowsfill@}{%
    \arrowfill@\leftleftarrows\relrelbar\relrelbar
}
\providecommand*{\xrightrightarrows}[2][]{%
    \ext@arrow 0359\rightrightarrowsfill@{#1}{#2}%
}
\providecommand*{\xleftleftarrows}[2][]{%
    \ext@arrow 3095\leftleftarrowsfill@{#1}{#2}%
}

\allowdisplaybreaks

\newcommand{\unfinished}{\textcolor{red}{Не дописано}}

% Reproducible pdf builds 
\special{pdf:trailerid [
<00112233445566778899aabbccddeeff>
<00112233445566778899aabbccddeeff>
]}
%</preamble>


\lhead{Линейная алгерба}
\cfoot{}
\rfoot{Практика 7}

\begin{document}

%<*собственныйвектор>
$\varphi : X\to X$
\begin{definition}
    $x\in X$ --- \textbf{собственный вектор} $\varphi$, если
    $$x\not=0 \quad \varphi x = \lambda x, \quad \lambda\in K$$
    $\lambda$ --- \textbf{собственное значение} $\varphi$, соответствующее $x$
\end{definition}

\begin{definition}
    \textbf{Спектр} $\sigma_\varphi=\{\lambda_1\ldots \lambda_n\}$ --- множество всех собственных значений вектора
\end{definition}
%</собственныйвектор>

\begin{example}
    Найти спектр и собственные вектора оператора $\varphi$, заданного матрицей:
    $$A=\begin{bmatrix}
            3  & -2 & 6  \\
            -2 & 6  & 3  \\
            6  & 3  & -2
        \end{bmatrix}$$

    Найдем спектр:
    $$\chi_\varphi(\lambda)=|A_\lambda E|=\begin{vmatrix}
            3 - \lambda & -2          & 6            \\
            -2          & 6 - \lambda & 3            \\
            6           & 3           & -2 - \lambda
        \end{vmatrix}=$$
    $$=(3-\lambda)((6-\lambda)(-2-\lambda)-9)+2(2(2+\lambda)-18)+6(-6-6(6-\lambda))=$$
    $$=(3-\lambda)(\lambda^2-4\lambda-21)+4(\lambda-7)-36(7-\lambda)=$$
    $$=(3-\lambda)(\lambda^2-4\lambda-21)+40(\lambda-7)=$$
    $$=(3-\lambda)(\lambda-7)(\lambda+3)+40(\lambda-7)=$$
    $$(49-\lambda^2)(\lambda-7)=(\lambda-7)^2(\lambda+7)$$

    $$\sigma_\varphi=\text{корни }\chi_\varphi=\{-7, 7^{(2)}\}$$

    Найдем собственные вектора:
    \begin{enumerate}
        \item $\lambda=-7$
              $$A\xi=\lambda\xi\Rightarrow (A-\lambda E)\xi=0 \text{ --- однор. СЛАУ}$$
              $$\begin{bmatrix}
                      3 + 7 & -2    & 6      \\
                      -2    & 6 + 7 & 3      \\
                      6     & 3     & -2 + 7
                  \end{bmatrix}=\begin{bmatrix}
                      10 & -2 & 6 \\
                      -2 & 13 & 3 \\
                      6  & 3  & 5
                  \end{bmatrix}\sim\begin{bmatrix}
                      -2 & 13 & 3  \\
                      0  & 42 & 14 \\
                      0  & 63 & 21
                  \end{bmatrix}\sim\begin{bmatrix}
                      -2 & 13 & 3 \\
                      0  & 3  & 1 \\
                      0  & 0  & 0
                  \end{bmatrix}$$

              $]\xi^3$ --- параметр $\Rightarrow \begin{cases}
                      -2 \xi^1 + 13 \xi^2 = -3 \xi^3 \\
                      3\xi^2 = -\xi^3
                  \end{cases}$

              $] \xi^3=3 \Rightarrow \xi^2=-1, \xi^1 = -2 \Rightarrow \xi=\begin{bmatrix}
                      2  \\
                      -1 \\
                      3
                  \end{bmatrix}$

              Собственный вектор один \textit{(см. СЛАУ)}

        \item $\lambda=7$
              $$\begin{bmatrix}
                      3 -7 & -2   & 6     \\
                      -2   & 6 -7 & 3     \\
                      6    & 3    & -2 -7
                  \end{bmatrix}\sim\begin{bmatrix}
                      -4 & -2 & 6  \\
                      -2 & -1 & 3  \\
                      6  & 3  & -9
                  \end{bmatrix}\sim\begin{bmatrix}
                      -2 & -1 & 3 \\
                      0  & 0  & 0 \\
                      0  & 0  & 0
                  \end{bmatrix} \Rightarrow \text{два собственных вектора}$$
              $] \xi^2, \xi^3$ --- параметры

              $\xi^2=2, \xi^3=0\Rightarrow \xi^1=-1$

              $\xi^2=0, \xi^3=2\Rightarrow \xi^1=3$

              $\xi_2=\begin{bmatrix}
                      -1 \\
                      2  \\
                      0
                  \end{bmatrix} \quad \begin{bmatrix}
                      3 \\
                      0 \\
                      2
                  \end{bmatrix}$

              $\{\xi_j\}_{j=1}^3$ --- базис $X$
    \end{enumerate}

    Проверка:
    $$\begin{bmatrix}
            3  & -2 & 6  \\
            -2 & 6  & 3  \\
            6  & 3  & -2
        \end{bmatrix}\begin{bmatrix}
            2  \\
            -1 \\
            3
        \end{bmatrix}=-7\begin{bmatrix}
            2  \\
            -1 \\
            3
        \end{bmatrix}$$

    Проверим, что $A$ в базисе из собственных векторов диагональна:
    $$\tilde A = T^{-1}AT$$
    $$T^{-1}=\frac{1}{\det T}\tilde T^{T}$$
    $$\det T = \begin{vmatrix}
            2  & -1 & 3 \\
            -1 & 2  & 0 \\
            3  & 0  & 2
        \end{vmatrix}=-28$$
    $$\tilde T\tran =\begin{bmatrix}
            4  & 2  & -6 \\
            2  & -5 & -3 \\
            -6 & -3 & 3
        \end{bmatrix}$$
    $$T^{-1}=\frac{-1}{28}\begin{bmatrix}
            4  & 2  & -6 \\
            2  & -5 & -3 \\
            -6 & -3 & 3
        \end{bmatrix}$$
    $$\tilde A=\frac{-1}{28}\begin{bmatrix}
            4  & 2  & -6 \\
            2  & -5 & -3 \\
            -6 & -3 & 3
        \end{bmatrix}\begin{bmatrix}
            3  & -2 & 6  \\
            -2 & 6  & 3  \\
            6  & 3  & -2
        \end{bmatrix}\begin{bmatrix}
            2  & -1 & 3 \\
            -1 & 2  & 0 \\
            3  & 0  & 2
        \end{bmatrix}=$$
    $$=\frac{-1}{28}\begin{bmatrix}
            4  & 2  & -6 \\
            2  & -5 & -3 \\
            -6 & -3 & 3
        \end{bmatrix}\begin{bmatrix}
            14 & 7   & -21 \\
            -7 & -14 & 0   \\
            21 & 0   & -14
        \end{bmatrix}=-\frac{1}{4}\begin{bmatrix}
            28 & 0   & 0   \\
            0  & -28 & 0   \\
            0  & 0   & -28
        \end{bmatrix}$$
\end{example}

\end{document}
