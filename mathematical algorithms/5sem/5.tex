\chapter{2 октября}

\begin{definition}\itemfix
    \begin{itemize}
        \item \(G\) --- группа
        \item \(S \subset G\) --- подмножество элементов \(G\)
    \end{itemize}

    \textbf{Нормализатор} \(S\): \(N_S \coloneqq \{g \in G : gS = Sg\}\)
\end{definition}

\begin{definition}\itemfix
    \begin{itemize}
        \item \(G\) --- группа
        \item \(x \in G\)
        \item \(S \subset G\)
    \end{itemize}

    \textbf{Централизатор} \(x\): \(Z_x \coloneqq \{g \in G : gx = xg\}\)

    \(Z_S \coloneqq \{g \in G : \forall y \in S \ \ gy = yg\}\)

    \(Z_G\) --- \textbf{центр} группы \(G\).
\end{definition}

\begin{example}
    В группе \(GL(n, \R)\) инвертируемых матриц \(n \times n\) центр --- единичная матрица.
\end{example}

\subsection{Цепочки гомоморфизмов}

\begin{definition}\itemfix
    \begin{itemize}
        \item \(G, G', G''\) --- группы
        \item \(\sigma \in \mathrm{hom}(G\ G')\)
        \item \(\chi \in \mathrm{hom}(G'\ G'')\)
    \end{itemize}
    Рассмотрим цепочку \begin{tikzcd}
        G \arrow{r}{\sigma} & G' \arrow{r}{\chi} & G''
    \end{tikzcd}. Такая последовательность называется \textbf{точной}, если \(\ker \chi = \im \sigma\).
\end{definition}

\begin{prop}\itemfix
    \begin{enumerate}
        \item \(\ker (\chi \circ \sigma) = G\)
        \item Если \(\sigma\) --- сюръекция, то \(\ker \chi = G'\)
        \item Если \(\chi\) --- инъекция, то \(\ker \sigma = G\)
    \end{enumerate}
\end{prop}

\begin{example}
    \(H \triangleleft G \Rightarrow \begin{tikzcd}
        H \arrow{r}{j} & G \arrow{r}{\varphi} & G / H
    \end{tikzcd}\), где \(j\) --- вложение, \(\varphi\) --- канонический гомоморфизм \(g \mapsto gH\). Тогда \(\forall h \in H \ \ (\varphi \circ j)(h) = \varphi(j(h)) = \varphi(h) = hH = 1H = 1_{G / H}\), следовательно эта последовательность точная.
\end{example}

Также рассматриваются последовательности вида \begin{tikzcd}
    0 \arrow{r} & G \arrow{r}{\sigma} & G' \arrow{r}{\chi} & G'' \arrow{r} & 0
\end{tikzcd}, где \(0\) --- группа из одного элемента. Пусть эта последовательность точная. Гомоморфизм \(0 \to G\) сопоставляет этому элементу \(G_e\), следовательно \(\im (0 \to G) = \{G_e\} \Rightarrow \ker \sigma = \{G_e\} \Rightarrow \sigma\) инъективно. Аналогичными рассуждениями \(\chi\) сюръективно.

\begin{definition}
    \(\sphericalangle\) \begin{tikzcd}
        0 \arrow{r} & G \arrow{r}{\sigma_1} & G' \arrow{r}{\sigma_2} & G'' \arrow{r}{\sigma_3} & \dots  \arrow{r}{\sigma_{n-1}} & G^{(n)} \arrow{r}{\sigma_n} & \dots
    \end{tikzcd}. Такая последовательность называется точной, если \(\ker \sigma_i = \im \sigma_{i - 1}\).
\end{definition}

\(\sphericalangle\) \begin{tikzcd}
    0 \arrow{r} & H \arrow{r}{j} & G \arrow{r}{\varphi} & G / H \arrow{r}{\tilde{\sigma}} & G'
\end{tikzcd}

\(\sphericalangle \sigma \in \mathrm{hom}(G\ G'), \sigma = \tilde{\sigma} \circ \varphi : \begin{tikzcd}
    G \arrow{rr}{\sigma} \arrow{rd}[swap]{\varphi} & & G' \\
    & G / H \arrow{ur}[swap]{\tilde{\sigma}} &
\end{tikzcd}\)

Покажем, что \(\tilde{\sigma}\) единственно. \(\tilde{\sigma} : gH \mapsto \sigma(g)\).

Рассмотрим другую цепочку \begin{tikzcd}
    G \arrow{r}{\varphi} & G / H \arrow{r}{\lambda} \arrow[bend right]{rr}{\tilde{\sigma}} & \im \sigma \arrow{r}{j} & G'
\end{tikzcd}

\(\lambda : gH \mapsto \sigma(g), \ker \lambda = \{H\}, \lambda\) --- биективно. Таким образом, \(\lambda\) --- изоморфизм и \(G / H \simeq \im \sigma\).

\begin{remark}\itemfix
    \begin{itemize}
        \item \(G\) --- группа
        \item \(H \triangleleft G, K \triangleleft G\)
        \item \(K \subset H\) --- подгруппа
    \end{itemize}

    Тогда:
    \begin{enumerate}
        \item \(K \triangleleft H\).

              \(\sphericalangle \chi : G / K \to G / H, gK \mapsto gH, \ker \chi = \{hK\}_{h \in H}\), т.к. \(hK \mapsto hH = H\).

              \begin{myemph}
                  (G / K) / (H / K) = G / H
              \end{myemph}
    \end{enumerate}
\end{remark}

\section{Действие группы}

\begin{definition}\itemfix
    \begin{itemize}
        \item \(G\) --- группа
        \item \(S\) --- множество
    \end{itemize}

    \(G\) \textbf{действует} на \(S\), если существует отображение
    \[T : G \times S \to S\]
    , при этом \((g_1g_2)s = g_1(g_2s)\)
\end{definition}

\begin{remark}
    \[T_{g_1} T_{g_2} = T_{g_1g_2} \quad T_e = \mathrm{id} \quad T_{g^{-1}} = T_g^{-1}\]
\end{remark}

\(G\) действует на \(S\) как группа перестановок.

\begin{definition}\itemfix
    \begin{itemize}
        \item \(s \in S\)
        \item \(G\) --- группа
    \end{itemize}

    \(G_s \coloneqq \{g \in G : gs = s\}\) --- \textbf{стабилизатор} элемента \(s\).
\end{definition}

\begin{example}
    \(\mathbb{Q}\) действует на \(\R^3\) через \(T\).
\end{example}

\begin{lemma}
    \(G_s \subset G\) --- подгруппа
\end{lemma}
\begin{proof}
    \(g_1, g_2 \in G_s \Rightarrow g_1 s = s, g_2 s = s\)

    \((g_1g_2) \cdot s = g_1(g_2 s) = g_1s = s\)
\end{proof}

\(\sphericalangle G / G_s\) --- фактор-множество.

\begin{lemma}
    \(s, s' \in S, s' = xs, x \in G\). Тогда \(G_{s'} = xG_sx^{-1}\) и \(G_{s'}\) вместе с \(G_s\) называются \textbf{сопряженными}
\end{lemma}
\begin{proof}
    \[g's' = s' = xs = xgs = xgx^{-1}s'\]
    \[g' = xgx^{-1}\]
\end{proof}

\begin{definition}
    Преобразование вида \(xAx^{-1}\), где \(A \subset G\) --- подгруппа \(G\), называется сопряжением.
\end{definition}

\begin{lemma}
    \(g G_s, g'G_s \in G / G_s\)

    \[gs = g's \Leftrightarrow g G_s = g' G_s\]
\end{lemma}

\subsection{Орбиты}

\begin{definition}
    \(\mathcal{O}_G(S) \coloneqq \{gs : g \in G\}\) --- \textbf{орбита}
\end{definition}

\begin{lemma}
    \(|\mathcal{O}_G(S)| = (G : G_S)\)
\end{lemma}
\begin{proof}
    Из предыдущей леммы.
\end{proof}

Остаётся на следующую лекцию:
\begin{enumerate}
    \item \(S = \bigsqcup_{S \in C} \mathcal{O}_G(S)\), где \(C\) --- непересекающиеся орбиты
    \item Действия группы на себя
\end{enumerate}
