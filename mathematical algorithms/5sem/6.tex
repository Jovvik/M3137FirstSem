\chapter{9 октября}

\begin{lemma}
    Орбиты элементов \(\mathcal{O}_G(s)\) и \(\mathcal{O}_G(s')\) или непересекаются или совпадают.
\end{lemma}
\begin{proof}
    Пусть орбиты пересекаются, т.е. \(\exists s_0 : s_0 \in \mathcal{O}_G(s)\) и \(s_0 \in \mathcal{O}_G(s')\). Тогда \(\exists g \in G : s_0 = gs, \exists g' \in G : s_0 = g's'\)

    \[\mathcal{O}_G(s') = \mathcal{O}_G(g's') = \mathcal{O}_G(s_0) = \mathcal{O}_G(gs) = \mathcal{O}_G(s)\]
    Таким образом, \(\mathcal{O}_G(s') = \mathcal{O}_G(s)\).
\end{proof}

\begin{remark}
    \[S = \bigsqcup_{i \in I} \mathcal{O}_G(S_i)\]
\end{remark}

\begin{remark}
    Если \(S\) --- конечно, то
    \[|S|= \sum_{i \in I} |\mathcal{O}_G(s_i)|\]
\end{remark}

\section{Действие группы на себя}

Пусть \(S_G = G\), т.е. группа действует сама на себя.

\subsection{Сопряжение}

Пусть \(x \in G\). \(\sigma : x \mapsto \sigma_x : \sigma_x(y) = xyx^{-1}\)

Пусть \(y, y' \in G\).
\[\sigma_x(y \cdot y') = x yy' x^{-1} = xyx^{-1}xy'x^{-1} = \sigma_x(y)\sigma_x(y')\]
\[\sigma_x(e) = e\]
Таким образом, \(\sigma_x\) --- гомоморфизм.

\(\sigma_x^{-1} = \sigma_{x^{-1}}\)

\(\sigma_x^{-1} \circ \sigma_x = \mathrm{id}_G\)

\[\sigma_x^{-1} \circ \sigma_x(y) = G_x^{-1}(xyx^{-1}) = x^{-1}xyx^{-1}x = y \ \ \forall y\]
\(\sigma_x \in \mathrm{Aut}(G) \ \ \forall x\)

\(\sphericalangle \sigma : G \to \mathrm{Aut}(G)\).
\[\sigma_x \sigma_y = \sigma_{xy} \quad \sigma_e = \mathrm{id}_G\]
Таким обрзом, \(\sigma \in \mathrm{hom}(G, \mathrm{Aut}(G))\)

\[\ker \sigma = \{x \in G : \forall y \ \ \sigma_x y = y\}\]
\[xyx^{-1} = y\]
\[xy = yx\]
Таким образом, \(\ker \sigma = Z_G\)

Рассмотрим \(G\) как множество. \(A \subset G\) --- подмножество \(G\).

\(\sphericalangle \sigma_x(A) = xAx^{-1} \subset G\)

\(\sphericalangle \sigma_x(H) = xHx^{-1} \subset G\) --- подгруппа \(G\).

Пусть \(S\) --- множество подгрупп группы \(G\), \(H\) --- подгруппа \(G\), рассмотрим \(G / H\).

Пусть \(x \in G\).
\[G_x \coloneqq \{g \in G : \sigma_g(x) = x\} = Z_x\]
\[\mathcal{O}_G(x) = \{\sigma_g(x), g \in G\}\]
\[|\mathcal{O}_G(x)| = (G : Z_x)\]
\[G = \bigsqcup_{i \in I} \mathcal{O}_G(x_i)\]
\begin{myemph}
    |G| = \sum_{i \in I}(G : Z_{x_i})
\end{myemph}
\[G_H = \{g \in G : \sigma_g H = H\} \defeq N_H\]
\[G = \bigsqcup_{i \in I} \mathcal{O}_G(H_i) \quad |G| = \sum_{i \in I} (G : N_i)\]

\subsection{Левая трансляция}

Пусть \(x \in G\). \(\tau : x \mapsto \tau_x : y \mapsto xy\).

\(\tau_x(yy') = x yy'\) --- не гомоморфизм.

Пусть \(H \subset G\) --- подгруппа \(G\). Сопряжение не определяло действие, а трансляция определяет: \(\sphericalangle G / H : [g] = gH\), тогда \(\tau_x(gH) = xgH = g'H \in G / H\).

\section{Циклические группы}

\begin{definition}
    Группа \(G\) называется \textbf{циклической}, если \(\exists g : \forall h \in G \ \ h = g^m = \underbrace{g \cdot g \cdots}_m\).
\end{definition}
\begin{notation}
    \(G = \ev{g}\)
\end{notation}

\begin{definition}
    \textbf{Показатель элемента} \(g\) в \(G = \ev{g}\) это число \(m > 0\), такое что \(g^m = e\).
\end{definition}

\begin{definition}
    \textbf{Показатель группы} \(\ev{g}\) --- число \(k > 0\), такое что \(\forall x \in G \ \ x^k = e\).
\end{definition}

\begin{example}
    \((\Z, +)\) --- бесконечная циклическая группа.

    Если \(H\) --- подгруппа \(\Z\), то \(H = \{mz\}_{m \in \Z}, z \coloneqq \min\{t \in \Z \mid t > 0\}\)
\end{example}
