\documentclass[12pt, a4paper]{article}

%<*preamble>
% Math symbols
\usepackage{amsmath, amsthm, amsfonts, amssymb}
\usepackage{accents}
\usepackage{esvect}
\usepackage{mathrsfs}
\usepackage{mathtools}
\mathtoolsset{showonlyrefs}
\usepackage{cmll}
\usepackage{stmaryrd}
\usepackage{physics}
\usepackage[normalem]{ulem}
\usepackage{ebproof}
\usepackage{extarrows}

% Page layout
\usepackage{geometry, a4wide, parskip, fancyhdr}

% Font, encoding, russian support
\usepackage[russian]{babel}
\usepackage[sb]{libertine}
\usepackage{xltxtra}

% Listings
\usepackage{listings}
\lstset{basicstyle=\ttfamily,breaklines=true}
\setmonofont{Inconsolata}

% Miscellaneous
\usepackage{array}
\usepackage{calc}
\usepackage{caption}
\usepackage{subcaption}
\captionsetup{justification=centering,margin=2cm}
\usepackage{catchfilebetweentags}
\usepackage{enumitem}
\usepackage{etoolbox}
\usepackage{float}
\usepackage{lastpage}
\usepackage{minted}
\usepackage{svg}
\usepackage{wrapfig}
\usepackage{xcolor}
\usepackage[makeroom]{cancel}

\newcolumntype{L}{>{$}l<{$}}
    \newcolumntype{C}{>{$}c<{$}}
\newcolumntype{R}{>{$}r<{$}}

% Footnotes
\usepackage[hang]{footmisc}
\setlength{\footnotemargin}{2mm}
\makeatletter
\def\blfootnote{\gdef\@thefnmark{}\@footnotetext}
\makeatother

% References
\usepackage{hyperref}
\hypersetup{
    colorlinks,
    linkcolor={blue!80!black},
    citecolor={blue!80!black},
    urlcolor={blue!80!black},
}

% tikz
\usepackage{tikz}
\usepackage{tikz-cd}
\usetikzlibrary{arrows.meta}
\usetikzlibrary{decorations.pathmorphing}
\usetikzlibrary{calc}
\usetikzlibrary{patterns}
\usepackage{pgfplots}
\pgfplotsset{width=10cm,compat=1.9}
\newcommand\irregularcircle[2]{% radius, irregularity
    \pgfextra {\pgfmathsetmacro\len{(#1)+rand*(#2)}}
    +(0:\len pt)
    \foreach \a in {10,20,...,350}{
            \pgfextra {\pgfmathsetmacro\len{(#1)+rand*(#2)}}
            -- +(\a:\len pt)
        } -- cycle
}

\providetoggle{useproofs}
\settoggle{useproofs}{false}

\pagestyle{fancy}
\lfoot{M3137y2019}
\cfoot{}
\rhead{стр. \thepage\ из \pageref*{LastPage}}

\newcommand{\R}{\mathbb{R}}
\newcommand{\Q}{\mathbb{Q}}
\newcommand{\Z}{\mathbb{Z}}
\newcommand{\B}{\mathbb{B}}
\newcommand{\N}{\mathbb{N}}
\renewcommand{\Re}{\mathfrak{R}}
\renewcommand{\Im}{\mathfrak{I}}

\newcommand{\const}{\text{const}}
\newcommand{\cond}{\text{cond}}

\newcommand{\teormin}{\textcolor{red}{!}\ }

\DeclareMathOperator*{\xor}{\oplus}
\DeclareMathOperator*{\equ}{\sim}
\DeclareMathOperator{\sign}{\text{sign}}
\DeclareMathOperator{\Sym}{\text{Sym}}
\DeclareMathOperator{\Asym}{\text{Asym}}

\DeclarePairedDelimiter{\ceil}{\lceil}{\rceil}

% godel
\newbox\gnBoxA
\newdimen\gnCornerHgt
\setbox\gnBoxA=\hbox{$\ulcorner$}
\global\gnCornerHgt=\ht\gnBoxA
\newdimen\gnArgHgt
\def\godel #1{%
    \setbox\gnBoxA=\hbox{$#1$}%
    \gnArgHgt=\ht\gnBoxA%
    \ifnum     \gnArgHgt<\gnCornerHgt \gnArgHgt=0pt%
    \else \advance \gnArgHgt by -\gnCornerHgt%
    \fi \raise\gnArgHgt\hbox{$\ulcorner$} \box\gnBoxA %
    \raise\gnArgHgt\hbox{$\urcorner$}}

% \theoremstyle{plain}

\theoremstyle{definition}
\newtheorem{theorem}{Теорема}
\newtheorem*{definition}{Определение}
\newtheorem{axiom}{Аксиома}
\newtheorem*{axiom*}{Аксиома}
\newtheorem{lemma}{Лемма}

\theoremstyle{remark}
\newtheorem*{remark}{Примечание}
\newtheorem*{exercise}{Упражнение}
\newtheorem{corollary}{Следствие}[theorem]
\newtheorem*{statement}{Утверждение}
\newtheorem*{corollary*}{Следствие}
\newtheorem*{example}{Пример}
\newtheorem{observation}{Наблюдение}
\newtheorem*{prop}{Свойства}
\newtheorem*{obozn}{Обозначение}

% subtheorem
\makeatletter
\newenvironment{subtheorem}[1]{%
    \def\subtheoremcounter{#1}%
    \refstepcounter{#1}%
    \protected@edef\theparentnumber{\csname the#1\endcsname}%
    \setcounter{parentnumber}{\value{#1}}%
    \setcounter{#1}{0}%
    \expandafter\def\csname the#1\endcsname{\theparentnumber.\Alph{#1}}%
    \ignorespaces
}{%
    \setcounter{\subtheoremcounter}{\value{parentnumber}}%
    \ignorespacesafterend
}
\makeatother
\newcounter{parentnumber}

\newtheorem{manualtheoreminner}{Теорема}
\newenvironment{manualtheorem}[1]{%
    \renewcommand\themanualtheoreminner{#1}%
    \manualtheoreminner
}{\endmanualtheoreminner}

\newcommand{\dbltilde}[1]{\accentset{\approx}{#1}}
\newcommand{\intt}{\int\!}

% magical thing that fixes paragraphs
\makeatletter
\patchcmd{\CatchFBT@Fin@l}{\endlinechar\m@ne}{}
{}{\typeout{Unsuccessful patch!}}
\makeatother

\newcommand{\get}[2]{
    \ExecuteMetaData[#1]{#2}
}

\newcommand{\getproof}[2]{
    \iftoggle{useproofs}{\ExecuteMetaData[#1]{#2proof}}{}
}

\newcommand{\getwithproof}[2]{
    \get{#1}{#2}
    \getproof{#1}{#2}
}

\newcommand{\import}[3]{
    \subsection{#1}
    \getwithproof{#2}{#3}
}

\newcommand{\given}[1]{
    Дано выше. (\ref{#1}, стр. \pageref{#1})
}

\renewcommand{\ker}{\text{Ker }}
\newcommand{\im}{\text{Im }}
\renewcommand{\grad}{\text{grad}}
\newcommand{\rg}{\text{rg}}
\newcommand{\defeq}{\stackrel{\text{def}}{=}}
\newcommand{\defeqfor}[1]{\stackrel{\text{def } #1}{=}}
\newcommand{\itemfix}{\leavevmode\makeatletter\makeatother}
\newcommand{\?}{\textcolor{red}{???}}
\renewcommand{\emptyset}{\varnothing}
\newcommand{\longarrow}[1]{\xRightarrow[#1]{\qquad}}
\DeclareMathOperator*{\esup}{\text{ess sup}}
\newcommand\smallO{
    \mathchoice
    {{\scriptstyle\mathcal{O}}}% \displaystyle
    {{\scriptstyle\mathcal{O}}}% \textstyle
    {{\scriptscriptstyle\mathcal{O}}}% \scriptstyle
    {\scalebox{.6}{$\scriptscriptstyle\mathcal{O}$}}%\scriptscriptstyle
}
\renewcommand{\div}{\text{div}\ }
\newcommand{\rot}{\text{rot}\ }
\newcommand{\cov}{\text{cov}}

\makeatletter
\newcommand{\oplabel}[1]{\refstepcounter{equation}(\theequation\ltx@label{#1})}
\makeatother

\newcommand{\symref}[2]{\stackrel{\oplabel{#1}}{#2}}
\newcommand{\symrefeq}[1]{\symref{#1}{=}}

% xrightrightarrows
\makeatletter
\newcommand*{\relrelbarsep}{.386ex}
\newcommand*{\relrelbar}{%
    \mathrel{%
        \mathpalette\@relrelbar\relrelbarsep
    }%
}
\newcommand*{\@relrelbar}[2]{%
    \raise#2\hbox to 0pt{$\m@th#1\relbar$\hss}%
    \lower#2\hbox{$\m@th#1\relbar$}%
}
\providecommand*{\rightrightarrowsfill@}{%
    \arrowfill@\relrelbar\relrelbar\rightrightarrows
}
\providecommand*{\leftleftarrowsfill@}{%
    \arrowfill@\leftleftarrows\relrelbar\relrelbar
}
\providecommand*{\xrightrightarrows}[2][]{%
    \ext@arrow 0359\rightrightarrowsfill@{#1}{#2}%
}
\providecommand*{\xleftleftarrows}[2][]{%
    \ext@arrow 3095\leftleftarrowsfill@{#1}{#2}%
}

\allowdisplaybreaks

\newcommand{\unfinished}{\textcolor{red}{Не дописано}}

% Reproducible pdf builds 
\special{pdf:trailerid [
<00112233445566778899aabbccddeeff>
<00112233445566778899aabbccddeeff>
]}
%</preamble>


\lhead{Алгоритмы в математике \textit{(практика)}}
\lfoot{Михайлов Максим}
\cfoot{}
\rfoot{18.12.2021}

\begin{document}

\begin{exercise}
    Доказать, что евклидово кольцо \(R\) есть область целостности.
\end{exercise}
\begin{solution}
    Это неверно, т.к. вырожденное кольцо \(\{0\}, 0 = 1\) является евклидовым, но не является целостным.

    Если в такое не верится, то кольцо \(\{0, 1, x, 1 + x\}, x^2 = 0\) не является целостным, т.к. \(x \cdot x = 0\), но \(x \neq 0\), при этом это кольцо является евклидовым:
    \[F(t) \coloneqq \begin{cases}
            0, & t = 0     \\
            1, & t = 1     \\
            1, & t = 1 + x \\
            2, & t = x     \\
        \end{cases}\]

    \begin{align*}
        a     & = q \cdot b + r                                                        \\
        0     & = 0 \cdot *\footnotemark + 0                                           \\
        1     & = 1 \cdot 1 + 0                                                        \\
        1     & = \underbrace{x \cdot \overbrace{x}^{F = 2}}_0 + \overbrace{1}^{F = 1} \\
        1     & = \underbrace{(1 + x) \cdot \overbrace{(1 + x)}^{F = 1}}_1 {}+ 0       \\
        x     & = x \cdot 1 + 0                                                        \\
        x     & = 1 \cdot x + 0                                                        \\
        x     & = \underbrace{x \cdot (1 + x)}_x {}+ 0                                 \\
        1 + x & = (1 + x) \cdot 1 + 0                                                  \\
        1 + x & = (1 + x) \cdot \overbrace{x}^{F = 2} + \overbrace{1}^{F = 1}          \\
        1 + x & = \underbrace{1 \cdot (1 + x)}_{1 + x} {}+ 0                           \\
    \end{align*}
    \footnotetext{Подразумевается любой элемент кольца.}
\end{solution}

\begin{exercise}
    Найти \(\gcd(9 + 12i, 5)\) в \(\Z[i]\).
\end{exercise}
\begin{solution}
    \begin{align*}
        9 + 12i  & = 5 \cdot (2 + 2i) + ( - 1 + 2i) \\
        2 + 2i   & = ( - 1 + 2i) \cdot ( -i) + i    \\
        - 1 + 2i & = i \cdot 2 - 1                  \\
        i        & = ( - 1) \cdot ( - i)            \\
    \end{align*}
    \textbf{Ответ:} \(- 1\).
\end{solution}

\begin{exercise}
    Рассмотрим кольцо \(R\):
    \[R = \{n + md \mid n,m \in \Z\}, \quad d^2 = 3\]
    Какими свойствами обладает данное кольцо? Является ли оно областью целостности? Евклидовым?
\end{exercise}
\begin{solution}\itemfix
    \begin{itemize}
        \item В этом кольце есть единица, это \(1 + 0 \cdot d\).
        \item Коммутативность тривиальна.
        \item \[\sphericalangle (a + bd) \neq 0, (x + yd) \neq 0 : (a + bd)(x + yd) = 0\]
              \begin{align*}
                  (a + bd)(x + yd)        & = 0 \\
                  ax + ayd + bdx + 3by    & = 0 \\
                  ax + 3by + ayd + bdx    & = 0 \\
                  (ax + 3by) + d(ay + bx) & = 0
              \end{align*}
              \[\begin{cases}
                      ax + 3by = 0 \\
                      ay + bx = 0
                  \end{cases}\]
              \[\begin{cases}
                      axy + 3by^2 = 0 \\
                      ayx + bx^2 = 0
                  \end{cases}\]
              \begin{align*}
                  3by^2      & = bx^2 \\
                  3y^2       & = b^2  \\
                  \sqrt{3} y & = b
              \end{align*}
              Решений нет, следовательно, это кольцо целостности.
    \end{itemize}
\end{solution}

\begin{exercise}
    Рассмотрим кольцо \(R\):
    \[R = \{n + md \mid n,m \in \Z\}, \quad d^2 = 0\]
    Какими свойствами обладает данное кольцо? Является ли оно областью целостности? Евклидовым?
\end{exercise}
\begin{solution}\itemfix
    \begin{itemize}
        \item В этом кольце есть единица, это \(1 + 0 \cdot d\).
        \item Коммутативность тривиальна.
        \item \(d \cdot d = 0, d\) --- делитель нуля.
        \item \(\sphericalangle n_1 + m_1d, n_2 + m_2d\)
              \begin{align*}
                  n_1 + m_1d & = q \cdot (n_2 + m_2d) + r                       \\
                  n_1 + m_1d & = (n_3 + m_3d) \cdot (n_2 + m_2d) + (n_4 + m_4d) \\
                  n_1 + m_1d & = n_2n_3 + n_2m_3d + n_3m_2d + n_4 + m_4d
              \end{align*}
              \[\begin{cases}
                      n_1 = n_2n_3 + n_4 \\
                      m_1 = n_2m_3 + n_3m_2 + m_4
                  \end{cases}\]
              Первое уравнение решается в \(\Z\), при этом \(|n_4| < |n_2|\) или \(n_4 = 0\). Т.к. \(n_3\) и \(m_2\) теперь фиксированы, то уравнение \(m_1 - n_3m_2 = n_2m_3 + m_4\) также решается и \(|m_4|< |m_2|\) или \(m_4 = 0\). Тогда \(N(n + md) = |n| + |m|\) подходит и кольцо евклидово.
    \end{itemize}
\end{solution}

\end{document}
