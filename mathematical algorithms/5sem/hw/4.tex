\documentclass[12pt, a4paper]{article}

%<*preamble>
% Math symbols
\usepackage{amsmath, amsthm, amsfonts, amssymb}
\usepackage{accents}
\usepackage{esvect}
\usepackage{mathrsfs}
\usepackage{mathtools}
\mathtoolsset{showonlyrefs}
\usepackage{cmll}
\usepackage{stmaryrd}
\usepackage{physics}
\usepackage[normalem]{ulem}
\usepackage{ebproof}
\usepackage{extarrows}

% Page layout
\usepackage{geometry, a4wide, parskip, fancyhdr}

% Font, encoding, russian support
\usepackage[russian]{babel}
\usepackage[sb]{libertine}
\usepackage{xltxtra}

% Listings
\usepackage{listings}
\lstset{basicstyle=\ttfamily,breaklines=true}
\setmonofont{Inconsolata}

% Miscellaneous
\usepackage{array}
\usepackage{calc}
\usepackage{caption}
\usepackage{subcaption}
\captionsetup{justification=centering,margin=2cm}
\usepackage{catchfilebetweentags}
\usepackage{enumitem}
\usepackage{etoolbox}
\usepackage{float}
\usepackage{lastpage}
\usepackage{minted}
\usepackage{svg}
\usepackage{wrapfig}
\usepackage{xcolor}
\usepackage[makeroom]{cancel}

\newcolumntype{L}{>{$}l<{$}}
    \newcolumntype{C}{>{$}c<{$}}
\newcolumntype{R}{>{$}r<{$}}

% Footnotes
\usepackage[hang]{footmisc}
\setlength{\footnotemargin}{2mm}
\makeatletter
\def\blfootnote{\gdef\@thefnmark{}\@footnotetext}
\makeatother

% References
\usepackage{hyperref}
\hypersetup{
    colorlinks,
    linkcolor={blue!80!black},
    citecolor={blue!80!black},
    urlcolor={blue!80!black},
}

% tikz
\usepackage{tikz}
\usepackage{tikz-cd}
\usetikzlibrary{arrows.meta}
\usetikzlibrary{decorations.pathmorphing}
\usetikzlibrary{calc}
\usetikzlibrary{patterns}
\usepackage{pgfplots}
\pgfplotsset{width=10cm,compat=1.9}
\newcommand\irregularcircle[2]{% radius, irregularity
    \pgfextra {\pgfmathsetmacro\len{(#1)+rand*(#2)}}
    +(0:\len pt)
    \foreach \a in {10,20,...,350}{
            \pgfextra {\pgfmathsetmacro\len{(#1)+rand*(#2)}}
            -- +(\a:\len pt)
        } -- cycle
}

\providetoggle{useproofs}
\settoggle{useproofs}{false}

\pagestyle{fancy}
\lfoot{M3137y2019}
\cfoot{}
\rhead{стр. \thepage\ из \pageref*{LastPage}}

\newcommand{\R}{\mathbb{R}}
\newcommand{\Q}{\mathbb{Q}}
\newcommand{\Z}{\mathbb{Z}}
\newcommand{\B}{\mathbb{B}}
\newcommand{\N}{\mathbb{N}}
\renewcommand{\Re}{\mathfrak{R}}
\renewcommand{\Im}{\mathfrak{I}}

\newcommand{\const}{\text{const}}
\newcommand{\cond}{\text{cond}}

\newcommand{\teormin}{\textcolor{red}{!}\ }

\DeclareMathOperator*{\xor}{\oplus}
\DeclareMathOperator*{\equ}{\sim}
\DeclareMathOperator{\sign}{\text{sign}}
\DeclareMathOperator{\Sym}{\text{Sym}}
\DeclareMathOperator{\Asym}{\text{Asym}}

\DeclarePairedDelimiter{\ceil}{\lceil}{\rceil}

% godel
\newbox\gnBoxA
\newdimen\gnCornerHgt
\setbox\gnBoxA=\hbox{$\ulcorner$}
\global\gnCornerHgt=\ht\gnBoxA
\newdimen\gnArgHgt
\def\godel #1{%
    \setbox\gnBoxA=\hbox{$#1$}%
    \gnArgHgt=\ht\gnBoxA%
    \ifnum     \gnArgHgt<\gnCornerHgt \gnArgHgt=0pt%
    \else \advance \gnArgHgt by -\gnCornerHgt%
    \fi \raise\gnArgHgt\hbox{$\ulcorner$} \box\gnBoxA %
    \raise\gnArgHgt\hbox{$\urcorner$}}

% \theoremstyle{plain}

\theoremstyle{definition}
\newtheorem{theorem}{Теорема}
\newtheorem*{definition}{Определение}
\newtheorem{axiom}{Аксиома}
\newtheorem*{axiom*}{Аксиома}
\newtheorem{lemma}{Лемма}

\theoremstyle{remark}
\newtheorem*{remark}{Примечание}
\newtheorem*{exercise}{Упражнение}
\newtheorem{corollary}{Следствие}[theorem]
\newtheorem*{statement}{Утверждение}
\newtheorem*{corollary*}{Следствие}
\newtheorem*{example}{Пример}
\newtheorem{observation}{Наблюдение}
\newtheorem*{prop}{Свойства}
\newtheorem*{obozn}{Обозначение}

% subtheorem
\makeatletter
\newenvironment{subtheorem}[1]{%
    \def\subtheoremcounter{#1}%
    \refstepcounter{#1}%
    \protected@edef\theparentnumber{\csname the#1\endcsname}%
    \setcounter{parentnumber}{\value{#1}}%
    \setcounter{#1}{0}%
    \expandafter\def\csname the#1\endcsname{\theparentnumber.\Alph{#1}}%
    \ignorespaces
}{%
    \setcounter{\subtheoremcounter}{\value{parentnumber}}%
    \ignorespacesafterend
}
\makeatother
\newcounter{parentnumber}

\newtheorem{manualtheoreminner}{Теорема}
\newenvironment{manualtheorem}[1]{%
    \renewcommand\themanualtheoreminner{#1}%
    \manualtheoreminner
}{\endmanualtheoreminner}

\newcommand{\dbltilde}[1]{\accentset{\approx}{#1}}
\newcommand{\intt}{\int\!}

% magical thing that fixes paragraphs
\makeatletter
\patchcmd{\CatchFBT@Fin@l}{\endlinechar\m@ne}{}
{}{\typeout{Unsuccessful patch!}}
\makeatother

\newcommand{\get}[2]{
    \ExecuteMetaData[#1]{#2}
}

\newcommand{\getproof}[2]{
    \iftoggle{useproofs}{\ExecuteMetaData[#1]{#2proof}}{}
}

\newcommand{\getwithproof}[2]{
    \get{#1}{#2}
    \getproof{#1}{#2}
}

\newcommand{\import}[3]{
    \subsection{#1}
    \getwithproof{#2}{#3}
}

\newcommand{\given}[1]{
    Дано выше. (\ref{#1}, стр. \pageref{#1})
}

\renewcommand{\ker}{\text{Ker }}
\newcommand{\im}{\text{Im }}
\renewcommand{\grad}{\text{grad}}
\newcommand{\rg}{\text{rg}}
\newcommand{\defeq}{\stackrel{\text{def}}{=}}
\newcommand{\defeqfor}[1]{\stackrel{\text{def } #1}{=}}
\newcommand{\itemfix}{\leavevmode\makeatletter\makeatother}
\newcommand{\?}{\textcolor{red}{???}}
\renewcommand{\emptyset}{\varnothing}
\newcommand{\longarrow}[1]{\xRightarrow[#1]{\qquad}}
\DeclareMathOperator*{\esup}{\text{ess sup}}
\newcommand\smallO{
    \mathchoice
    {{\scriptstyle\mathcal{O}}}% \displaystyle
    {{\scriptstyle\mathcal{O}}}% \textstyle
    {{\scriptscriptstyle\mathcal{O}}}% \scriptstyle
    {\scalebox{.6}{$\scriptscriptstyle\mathcal{O}$}}%\scriptscriptstyle
}
\renewcommand{\div}{\text{div}\ }
\newcommand{\rot}{\text{rot}\ }
\newcommand{\cov}{\text{cov}}

\makeatletter
\newcommand{\oplabel}[1]{\refstepcounter{equation}(\theequation\ltx@label{#1})}
\makeatother

\newcommand{\symref}[2]{\stackrel{\oplabel{#1}}{#2}}
\newcommand{\symrefeq}[1]{\symref{#1}{=}}

% xrightrightarrows
\makeatletter
\newcommand*{\relrelbarsep}{.386ex}
\newcommand*{\relrelbar}{%
    \mathrel{%
        \mathpalette\@relrelbar\relrelbarsep
    }%
}
\newcommand*{\@relrelbar}[2]{%
    \raise#2\hbox to 0pt{$\m@th#1\relbar$\hss}%
    \lower#2\hbox{$\m@th#1\relbar$}%
}
\providecommand*{\rightrightarrowsfill@}{%
    \arrowfill@\relrelbar\relrelbar\rightrightarrows
}
\providecommand*{\leftleftarrowsfill@}{%
    \arrowfill@\leftleftarrows\relrelbar\relrelbar
}
\providecommand*{\xrightrightarrows}[2][]{%
    \ext@arrow 0359\rightrightarrowsfill@{#1}{#2}%
}
\providecommand*{\xleftleftarrows}[2][]{%
    \ext@arrow 3095\leftleftarrowsfill@{#1}{#2}%
}

\allowdisplaybreaks

\newcommand{\unfinished}{\textcolor{red}{Не дописано}}

% Reproducible pdf builds 
\special{pdf:trailerid [
<00112233445566778899aabbccddeeff>
<00112233445566778899aabbccddeeff>
]}
%</preamble>


\lhead{Алгоритмы в математике \textit{(практика)}}
\cfoot{}
\rfoot{16.10.2021}

\begin{document}

\begin{exercise}
    Всякий ли элемент в \(S_3\) является простым циклом? В \(S_4\)?
\end{exercise}
\begin{solution}
    В \(S_3\) --- да, что проверяется перебором:
    \begin{center}
        \begin{tabular}{CC}
            \toprule
            \text{Перестановка}       & a_1 \dots a_k \\ \midrule
            \begin{pmatrix}
                1 & 2 & 3 \\
                1 & 2 & 3
            \end{pmatrix} & \emptyset     \\
            \begin{pmatrix}
                1 & 2 & 3 \\
                1 & 3 & 2
            \end{pmatrix} & 2, 3          \\
            \begin{pmatrix}
                1 & 2 & 3 \\
                2 & 1 & 3
            \end{pmatrix} & 1, 2          \\
            \begin{pmatrix}
                1 & 2 & 3 \\
                2 & 3 & 1
            \end{pmatrix} & 1, 2, 3       \\
            \begin{pmatrix}
                1 & 2 & 3 \\
                3 & 1 & 2
            \end{pmatrix} & 1, 2, 3       \\
            \begin{pmatrix}
                1 & 2 & 3 \\
                3 & 2 & 1
            \end{pmatrix} & 1, 2, 3       \\ \bottomrule
        \end{tabular}
    \end{center}

    В \(S_4\) --- нет, т.к. \(\begin{pmatrix}
        1 & 2 & 3 & 4 \\
        2 & 1 & 4 & 3
    \end{pmatrix}\) нельзя представить в виде простого цикла --- на все элементы она действует не тождественно (\(\forall x \ \ g(x) \neq x\)), следовательно все элементы \(\{1 \dots n\}\) в этом цикле. Пусть\footnote{Так можно говорить, т.к. \(a_1 \dots a_k\) можно циклически сдвинуть без потери общности.} \(1 = a_1\), тогда \(a_2 = 2\), тогда \(a_4\) это либо \(3\), либо \(4\), но ни для одного из них не верно \(g(x) = 1\).
\end{solution}

\begin{exercise}
    Рассмотрим все \(g \in S_4\), которые являются простыми циклами. Какие значения может принимать порядок \(g\)?
\end{exercise}
\begin{solution}\itemfix
    \begin{center}
        \begin{tabular}{CC}
            \toprule
            \text{Порядок} & \text{Пример перестановки с таким порядком} \\ \midrule
            1              & \begin{pmatrix}
                1 & 2 & 3 & 4 \\
                1 & 2 & 3 & 4
            \end{pmatrix}                  \\
            2              & \begin{pmatrix}
                1 & 2 & 3 & 4 \\
                2 & 1 & 3 & 4
            \end{pmatrix}                  \\
            3              & \begin{pmatrix}
                1 & 2 & 3 & 4 \\
                2 & 3 & 1 & 4
            \end{pmatrix}                  \\
            4              & \begin{pmatrix}
                1 & 2 & 3 & 4 \\
                2 & 3 & 4 & 1
            \end{pmatrix}                  \\
            \bottomrule
        \end{tabular}
    \end{center}

    \begin{statement}
        Элементов с порядком \( > 4\) нет.
    \end{statement}
    \begin{proof}
        Рассмотрим \(a_i\). Оно отображается в себя же за максимум последовательных 4 шага \(g\), т.к. так работает цикл.
    \end{proof}

    \textbf{Ответ}: \(\{1, 2, 3, 4\}\)
\end{solution}

\begin{exercise}
    Показать, что любой элемент \(\rho \in S_4\) представим в виде произведения независимых простых циклов:
    \[\rho = g_1 \cdots g_k\]
\end{exercise}
\begin{solution}
    Предположим обратное, что есть \(\rho \in S_4\), не представимый искомым образом. Пусть \(a_1 = 1, a_2 = \rho(a_1)\).

    Если \(a_2 = a_1\), то \(\rho\) действует тождественно на первый элемент и тогда группа \(G\) всех таких \(\rho\) изоморфна \(S_3\), а все элементы \(S_3\) --- простые циклы, следовательно \(\rho\) есть простой цикл.

    Рассмотрим случай \(a_1 \neq a_2\). Пусть \(a_3 = \rho(a_2)\). \(a_3 \neq a_2\), т.к. иначе \(\rho\) не биективно. Если \(a_3 = a_1\), то \(\rho\) действует как простой цикл длины \(2\) на элементы \(\{a_1, a_2\}\). Группа всех таких \(\rho\) изоморфна \(S_2\), т.к. осталось 2 не использованных элемента, а все элементы \(S_2\) --- простые циклы (перебор).

    Рассмотрим случай \(a_3 \neq a_1\). Пусть \(a_4 = \rho(a_3)\). \(a_4 \neq a_3\) и \(a_4 \neq a_2\) по соображениям выше. Если \(a_4 = a_1\), то \(\rho\) --- цикл длины \(3\) на элементах \(\{a_1, a_2, a_3\}\) и на оставшийся элемент не может не подействовать тривиально. Если \(a_4 \neq a_1\), то \(\sphericalangle a_5 = \rho(a_4)\). По соображениям выше \(a_5 \notin \{a_2, a_3, a_4\}\), следовательно \(a_5 = a_1\) и \(\rho\) --- цикл длины \(4\).

    Альтернативное доказательство: перебор.
    \begin{center}
        \begin{longtable}{CC}
            \toprule
            \rho                       & \text{Представление \(\rho\) в виде произведения простых циклов} \\ \midrule
            \begin{pmatrix}
                1 & 2 & 3 & 4 \\
                1 & 2 & 4 & 3
            \end{pmatrix} & \begin{pmatrix}
                1 & 2 & 3 & 4 \\
                1 & 2 & 4 & 3
            \end{pmatrix}                                       \\
            \begin{pmatrix}
                1 & 2 & 3 & 4 \\
                1 & 3 & 2 & 4
            \end{pmatrix} & \begin{pmatrix}
                1 & 2 & 3 & 4 \\
                1 & 3 & 2 & 4
            \end{pmatrix}                                       \\
            \begin{pmatrix}
                1 & 2 & 3 & 4 \\
                1 & 3 & 4 & 2
            \end{pmatrix} & \begin{pmatrix}
                1 & 2 & 3 & 4 \\
                1 & 3 & 4 & 2
            \end{pmatrix}                                       \\
            \begin{pmatrix}
                1 & 2 & 3 & 4 \\
                1 & 4 & 2 & 3
            \end{pmatrix} & \begin{pmatrix}
                1 & 2 & 3 & 4 \\
                1 & 4 & 2 & 3
            \end{pmatrix}                                       \\
            \begin{pmatrix}
                1 & 2 & 3 & 4 \\
                1 & 4 & 3 & 2
            \end{pmatrix} & \begin{pmatrix}
                1 & 2 & 3 & 4 \\
                1 & 4 & 3 & 2
            \end{pmatrix}                                       \\
            \begin{pmatrix}
                1 & 2 & 3 & 4 \\
                2 & 1 & 3 & 4
            \end{pmatrix} & \begin{pmatrix}
                1 & 2 & 3 & 4 \\
                2 & 1 & 3 & 4
            \end{pmatrix}                                       \\
            \begin{pmatrix}
                1 & 2 & 3 & 4 \\
                2 & 1 & 4 & 3
            \end{pmatrix} & \begin{pmatrix}
                1 & 2 & 3 & 4 \\
                2 & 1 & 3 & 4
            \end{pmatrix} \cdot \begin{pmatrix}
                1 & 2 & 3 & 4 \\
                1 & 2 & 4 & 3
            \end{pmatrix}      \\
            \begin{pmatrix}
                1 & 2 & 3 & 4 \\
                2 & 3 & 1 & 4
            \end{pmatrix} & \begin{pmatrix}
                1 & 2 & 3 & 4 \\
                2 & 3 & 1 & 4
            \end{pmatrix}                                       \\
            \begin{pmatrix}
                1 & 2 & 3 & 4 \\
                2 & 3 & 4 & 1
            \end{pmatrix} & \begin{pmatrix}
                1 & 2 & 3 & 4 \\
                2 & 3 & 4 & 1
            \end{pmatrix}                                       \\
            \begin{pmatrix}
                1 & 2 & 3 & 4 \\
                2 & 4 & 1 & 3
            \end{pmatrix} & \begin{pmatrix}
                1 & 2 & 3 & 4 \\
                2 & 4 & 1 & 3
            \end{pmatrix}                                       \\
            \begin{pmatrix}
                1 & 2 & 3 & 4 \\
                2 & 4 & 3 & 1
            \end{pmatrix} & \begin{pmatrix}
                1 & 2 & 3 & 4 \\
                2 & 4 & 3 & 1
            \end{pmatrix}                                       \\
            \begin{pmatrix}
                1 & 2 & 3 & 4 \\
                3 & 1 & 2 & 4
            \end{pmatrix} & \begin{pmatrix}
                1 & 2 & 3 & 4 \\
                3 & 1 & 2 & 4
            \end{pmatrix}                                       \\
            \begin{pmatrix}
                1 & 2 & 3 & 4 \\
                3 & 2 & 1 & 4
            \end{pmatrix} & \begin{pmatrix}
                1 & 2 & 3 & 4 \\
                3 & 2 & 1 & 4
            \end{pmatrix}                                       \\
            \begin{pmatrix}
                1 & 2 & 3 & 4 \\
                3 & 2 & 4 & 1
            \end{pmatrix} & \begin{pmatrix}
                1 & 2 & 3 & 4 \\
                3 & 2 & 4 & 1
            \end{pmatrix}                                       \\
            \begin{pmatrix}
                1 & 2 & 3 & 4 \\
                3 & 4 & 1 & 2
            \end{pmatrix} & \begin{pmatrix}
                1 & 2 & 3 & 4 \\
                3 & 2 & 1 & 4
            \end{pmatrix} \cdot \begin{pmatrix}
                1 & 2 & 3 & 4 \\
                1 & 4 & 3 & 2
            \end{pmatrix}      \\
            \begin{pmatrix}
                1 & 2 & 3 & 4 \\
                3 & 4 & 2 & 1
            \end{pmatrix} & \begin{pmatrix}
                1 & 2 & 3 & 4 \\
                3 & 4 & 2 & 1
            \end{pmatrix}                                       \\
            \begin{pmatrix}
                1 & 2 & 3 & 4 \\
                4 & 1 & 2 & 3
            \end{pmatrix} & \begin{pmatrix}
                1 & 2 & 3 & 4 \\
                4 & 1 & 2 & 3
            \end{pmatrix}                                       \\
            \begin{pmatrix}
                1 & 2 & 3 & 4 \\
                4 & 1 & 3 & 2
            \end{pmatrix} & \begin{pmatrix}
                1 & 2 & 3 & 4 \\
                4 & 1 & 3 & 2
            \end{pmatrix}                                       \\
            \begin{pmatrix}
                1 & 2 & 3 & 4 \\
                4 & 2 & 1 & 3
            \end{pmatrix} & \begin{pmatrix}
                1 & 2 & 3 & 4 \\
                4 & 2 & 1 & 3
            \end{pmatrix}                                       \\
            \begin{pmatrix}
                1 & 2 & 3 & 4 \\
                4 & 2 & 3 & 1
            \end{pmatrix} & \begin{pmatrix}
                1 & 2 & 3 & 4 \\
                4 & 2 & 3 & 1
            \end{pmatrix}                                       \\
            \begin{pmatrix}
                1 & 2 & 3 & 4 \\
                4 & 3 & 1 & 2
            \end{pmatrix} & \begin{pmatrix}
                1 & 2 & 3 & 4 \\
                4 & 3 & 1 & 2
            \end{pmatrix}                                       \\
            \begin{pmatrix}
                1 & 2 & 3 & 4 \\
                4 & 3 & 2 & 1
            \end{pmatrix} & \begin{pmatrix}
                1 & 2 & 3 & 4 \\
                4 & 2 & 3 & 1
            \end{pmatrix} \cdot \begin{pmatrix}
                1 & 2 & 3 & 4 \\
                1 & 3 & 2 & 4
            \end{pmatrix}      \\
            \bottomrule
        \end{longtable}
    \end{center}
\end{solution}

\begin{exercise}
    Показать, что если \(g, h \in S_n\) --- независимые простые циклы, то они коммутируют. Что можно сказать об обратном? (если \(g, h\) --- коммутирующие простые циклы, то \dots)
\end{exercise}
\begin{solution}
    \begin{statement}
        Если \(g(x) \neq x\), то \(h(g(x)) = g(x)\).
    \end{statement}
    \begin{proof}
        \(g(g(x)) \neq g(x)\) по биективности \(g\).
    \end{proof}

    \[g(h(x)) = \begin{cases}
            g(x), & g(x) \neq x     \\
            h(x), & h(x) \neq x     \\
            x ,   & h(x) = g(x) = x
        \end{cases}\]
    \[h(g(x)) = \begin{cases}
            g(x), & g(x) \neq x     \\
            h(x), & h(x) \neq x     \\
            x ,   & h(x) = g(x) = x
        \end{cases}\]
    Итого \(gh = hg\).

    Если \(g, h\) --- коммутирующие простые циклы, то они не обязательно независимы. Например, если \(g = h\): \(gg = gg\), но \(g\) зависит от себя \textit{(если это не тривиальный цикл)}. Также можно построить случай с \(g \neq h\). Пусть дано \(g\) с \(\{a_i\}_{i = 1}^n\). Тогда пусть \(h(a_i) = \begin{cases}
        a_{i-1}, & i > 1 \\
        a_n,     & i = 1
    \end{cases}\)
\end{solution}

\end{document}
