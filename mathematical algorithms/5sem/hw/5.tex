\documentclass[12pt, a4paper]{article}

%<*preamble>
% Math symbols
\usepackage{amsmath, amsthm, amsfonts, amssymb}
\usepackage{accents}
\usepackage{esvect}
\usepackage{mathrsfs}
\usepackage{mathtools}
\mathtoolsset{showonlyrefs}
\usepackage{cmll}
\usepackage{stmaryrd}
\usepackage{physics}
\usepackage[normalem]{ulem}
\usepackage{ebproof}
\usepackage{extarrows}

% Page layout
\usepackage{geometry, a4wide, parskip, fancyhdr}

% Font, encoding, russian support
\usepackage[russian]{babel}
\usepackage[sb]{libertine}
\usepackage{xltxtra}

% Listings
\usepackage{listings}
\lstset{basicstyle=\ttfamily,breaklines=true}
\setmonofont{Inconsolata}

% Miscellaneous
\usepackage{array}
\usepackage{calc}
\usepackage{caption}
\usepackage{subcaption}
\captionsetup{justification=centering,margin=2cm}
\usepackage{catchfilebetweentags}
\usepackage{enumitem}
\usepackage{etoolbox}
\usepackage{float}
\usepackage{lastpage}
\usepackage{minted}
\usepackage{svg}
\usepackage{wrapfig}
\usepackage{xcolor}
\usepackage[makeroom]{cancel}

\newcolumntype{L}{>{$}l<{$}}
    \newcolumntype{C}{>{$}c<{$}}
\newcolumntype{R}{>{$}r<{$}}

% Footnotes
\usepackage[hang]{footmisc}
\setlength{\footnotemargin}{2mm}
\makeatletter
\def\blfootnote{\gdef\@thefnmark{}\@footnotetext}
\makeatother

% References
\usepackage{hyperref}
\hypersetup{
    colorlinks,
    linkcolor={blue!80!black},
    citecolor={blue!80!black},
    urlcolor={blue!80!black},
}

% tikz
\usepackage{tikz}
\usepackage{tikz-cd}
\usetikzlibrary{arrows.meta}
\usetikzlibrary{decorations.pathmorphing}
\usetikzlibrary{calc}
\usetikzlibrary{patterns}
\usepackage{pgfplots}
\pgfplotsset{width=10cm,compat=1.9}
\newcommand\irregularcircle[2]{% radius, irregularity
    \pgfextra {\pgfmathsetmacro\len{(#1)+rand*(#2)}}
    +(0:\len pt)
    \foreach \a in {10,20,...,350}{
            \pgfextra {\pgfmathsetmacro\len{(#1)+rand*(#2)}}
            -- +(\a:\len pt)
        } -- cycle
}

\providetoggle{useproofs}
\settoggle{useproofs}{false}

\pagestyle{fancy}
\lfoot{M3137y2019}
\cfoot{}
\rhead{стр. \thepage\ из \pageref*{LastPage}}

\newcommand{\R}{\mathbb{R}}
\newcommand{\Q}{\mathbb{Q}}
\newcommand{\Z}{\mathbb{Z}}
\newcommand{\B}{\mathbb{B}}
\newcommand{\N}{\mathbb{N}}
\renewcommand{\Re}{\mathfrak{R}}
\renewcommand{\Im}{\mathfrak{I}}

\newcommand{\const}{\text{const}}
\newcommand{\cond}{\text{cond}}

\newcommand{\teormin}{\textcolor{red}{!}\ }

\DeclareMathOperator*{\xor}{\oplus}
\DeclareMathOperator*{\equ}{\sim}
\DeclareMathOperator{\sign}{\text{sign}}
\DeclareMathOperator{\Sym}{\text{Sym}}
\DeclareMathOperator{\Asym}{\text{Asym}}

\DeclarePairedDelimiter{\ceil}{\lceil}{\rceil}

% godel
\newbox\gnBoxA
\newdimen\gnCornerHgt
\setbox\gnBoxA=\hbox{$\ulcorner$}
\global\gnCornerHgt=\ht\gnBoxA
\newdimen\gnArgHgt
\def\godel #1{%
    \setbox\gnBoxA=\hbox{$#1$}%
    \gnArgHgt=\ht\gnBoxA%
    \ifnum     \gnArgHgt<\gnCornerHgt \gnArgHgt=0pt%
    \else \advance \gnArgHgt by -\gnCornerHgt%
    \fi \raise\gnArgHgt\hbox{$\ulcorner$} \box\gnBoxA %
    \raise\gnArgHgt\hbox{$\urcorner$}}

% \theoremstyle{plain}

\theoremstyle{definition}
\newtheorem{theorem}{Теорема}
\newtheorem*{definition}{Определение}
\newtheorem{axiom}{Аксиома}
\newtheorem*{axiom*}{Аксиома}
\newtheorem{lemma}{Лемма}

\theoremstyle{remark}
\newtheorem*{remark}{Примечание}
\newtheorem*{exercise}{Упражнение}
\newtheorem{corollary}{Следствие}[theorem]
\newtheorem*{statement}{Утверждение}
\newtheorem*{corollary*}{Следствие}
\newtheorem*{example}{Пример}
\newtheorem{observation}{Наблюдение}
\newtheorem*{prop}{Свойства}
\newtheorem*{obozn}{Обозначение}

% subtheorem
\makeatletter
\newenvironment{subtheorem}[1]{%
    \def\subtheoremcounter{#1}%
    \refstepcounter{#1}%
    \protected@edef\theparentnumber{\csname the#1\endcsname}%
    \setcounter{parentnumber}{\value{#1}}%
    \setcounter{#1}{0}%
    \expandafter\def\csname the#1\endcsname{\theparentnumber.\Alph{#1}}%
    \ignorespaces
}{%
    \setcounter{\subtheoremcounter}{\value{parentnumber}}%
    \ignorespacesafterend
}
\makeatother
\newcounter{parentnumber}

\newtheorem{manualtheoreminner}{Теорема}
\newenvironment{manualtheorem}[1]{%
    \renewcommand\themanualtheoreminner{#1}%
    \manualtheoreminner
}{\endmanualtheoreminner}

\newcommand{\dbltilde}[1]{\accentset{\approx}{#1}}
\newcommand{\intt}{\int\!}

% magical thing that fixes paragraphs
\makeatletter
\patchcmd{\CatchFBT@Fin@l}{\endlinechar\m@ne}{}
{}{\typeout{Unsuccessful patch!}}
\makeatother

\newcommand{\get}[2]{
    \ExecuteMetaData[#1]{#2}
}

\newcommand{\getproof}[2]{
    \iftoggle{useproofs}{\ExecuteMetaData[#1]{#2proof}}{}
}

\newcommand{\getwithproof}[2]{
    \get{#1}{#2}
    \getproof{#1}{#2}
}

\newcommand{\import}[3]{
    \subsection{#1}
    \getwithproof{#2}{#3}
}

\newcommand{\given}[1]{
    Дано выше. (\ref{#1}, стр. \pageref{#1})
}

\renewcommand{\ker}{\text{Ker }}
\newcommand{\im}{\text{Im }}
\renewcommand{\grad}{\text{grad}}
\newcommand{\rg}{\text{rg}}
\newcommand{\defeq}{\stackrel{\text{def}}{=}}
\newcommand{\defeqfor}[1]{\stackrel{\text{def } #1}{=}}
\newcommand{\itemfix}{\leavevmode\makeatletter\makeatother}
\newcommand{\?}{\textcolor{red}{???}}
\renewcommand{\emptyset}{\varnothing}
\newcommand{\longarrow}[1]{\xRightarrow[#1]{\qquad}}
\DeclareMathOperator*{\esup}{\text{ess sup}}
\newcommand\smallO{
    \mathchoice
    {{\scriptstyle\mathcal{O}}}% \displaystyle
    {{\scriptstyle\mathcal{O}}}% \textstyle
    {{\scriptscriptstyle\mathcal{O}}}% \scriptstyle
    {\scalebox{.6}{$\scriptscriptstyle\mathcal{O}$}}%\scriptscriptstyle
}
\renewcommand{\div}{\text{div}\ }
\newcommand{\rot}{\text{rot}\ }
\newcommand{\cov}{\text{cov}}

\makeatletter
\newcommand{\oplabel}[1]{\refstepcounter{equation}(\theequation\ltx@label{#1})}
\makeatother

\newcommand{\symref}[2]{\stackrel{\oplabel{#1}}{#2}}
\newcommand{\symrefeq}[1]{\symref{#1}{=}}

% xrightrightarrows
\makeatletter
\newcommand*{\relrelbarsep}{.386ex}
\newcommand*{\relrelbar}{%
    \mathrel{%
        \mathpalette\@relrelbar\relrelbarsep
    }%
}
\newcommand*{\@relrelbar}[2]{%
    \raise#2\hbox to 0pt{$\m@th#1\relbar$\hss}%
    \lower#2\hbox{$\m@th#1\relbar$}%
}
\providecommand*{\rightrightarrowsfill@}{%
    \arrowfill@\relrelbar\relrelbar\rightrightarrows
}
\providecommand*{\leftleftarrowsfill@}{%
    \arrowfill@\leftleftarrows\relrelbar\relrelbar
}
\providecommand*{\xrightrightarrows}[2][]{%
    \ext@arrow 0359\rightrightarrowsfill@{#1}{#2}%
}
\providecommand*{\xleftleftarrows}[2][]{%
    \ext@arrow 3095\leftleftarrowsfill@{#1}{#2}%
}

\allowdisplaybreaks

\newcommand{\unfinished}{\textcolor{red}{Не дописано}}

% Reproducible pdf builds 
\special{pdf:trailerid [
<00112233445566778899aabbccddeeff>
<00112233445566778899aabbccddeeff>
]}
%</preamble>


\lhead{Алгоритмы в математике \textit{(практика)}}
\cfoot{}
\rfoot{23.10.2021}

\begin{document}

\begin{exercise}
    Существуют ли некоммутативные группы порядка 4? Порядка 5?
\end{exercise}
\begin{solution}\itemfix
    \begin{enumerate}
        \item Порядка 4.

              Пусть \(G = \{e, a, b, c\}\) --- некоммутативная группа порядка \(4\), где \(a\) и \(b\) не коммутируют.\footnote{\(e\) всегда коммутирует, а разницы между \(a, b, c\) нет, поэтому общность не теряется.}

              \begin{lemma}
                  Пусть \(a\) и \(b\) не коммутируют. Тогда \(a \neq b^{-1}\) \textit{(и наоборот)}.
              \end{lemma}
              \begin{proof}
                  Пусть \(a = b^{-1}\). Тогда
                  \begin{align*}
                      a     & = b^{-1}      \\
                      ab    & = e           \\
                      bab   & = be          \\
                      (ba)b & = b           \\
                      ba    & = e = ab, !!! \\
                  \end{align*}
              \end{proof}

              По лемме \(ab \neq e\) и \(ba \neq e\). Кроме того, \(ab \neq a\) и \(ba \neq a\), т.к. иначе \(b = e\). Аналогично \(ab \neq b\) и \(ba \neq b\). Итого, \(ab, ba \notin\!\footnote{Здесь (и далее) подразумевается, что и \(ab\), и \(ba \notin \dots\)}\:\{e, a, b\}\) и при этом \(ab \neq ba\). Тогда \(ab\) и \(ba\) различные элементы и \(|G| \geq 5\), противоречие.

        \item Порядка 5.

              Пусть \(G = \{e, a, b, c, d\}\) --- некоммутативная группа порядка \(5\), где \(a\) и \(b\) не коммутируют.

              Аналогично предыдущему случаю, \(ab, ba \notin \{e, a, b\}\). Пусть \(ab = c\) и \(ba = d\) \textit{(без потери общности)}.
              \[ca = (ab)a = a(ba) = ad\]
              \[bc = b(ab) = (ba)b = db\]
              Т.к. \(c \neq e, a \neq e\), \(ca \notin \{c, a\}\). Аналогично, \(ad \notin \{a, d\}\) и по их равенству \(ca \notin \{a, c, d\}\). Кроме того, \(ca \neq e\), т.к. иначе \(c = a^{-1}\) и \(d = a^{-1}\), но доказано, что \(c \neq d\) --- противоречие. Итого, \(ca \notin \{a, c, d, e\}\), следовательно \(ca = b\). Аналогично \(bc = a\).
              \[b^2 = b(ca) = (bc)a = a^2\]

              Рассмотрим \(a^2\).
              \begin{itemize}
                  \item \(a^2 \neq a\), т.к. иначе \(a = e\).
                  \item Аналогично \(a^2 = b^2 \neq b\).
                  \item \(a^2 \neq c = ab\), т.к. иначе \(a = b\).
                  \item \(a^2 = b^2 \neq d = ba\), т.к. иначе \(b = a\).
              \end{itemize}
              Единственный оставшийся вариант --- \(a^2 = e\), но тогда:
              \[cb = ab^2 = a = db \Rightarrow c = d, !!!\]
    \end{enumerate}
\end{solution}

\begin{exercise}
    Рассмотрим группу \((\Z, +)\) по сложению. Выделим два подмножества:
    \[A = \{1337n \mid n \in \Z\} \quad B = \{n \in \Z \mid n \divided 1528\}\]
    Показать, что \(A, B\) есть подгруппы, а также \(H = A + B\) --- тоже подгруппа. Найти индекс \(H\) относительно левых смежных классов.
\end{exercise}
\begin{solution}
    \(A\) --- подгруппа:
    \begin{enumerate}
        \item \(0 \in A\)
        \item \(\forall 1337n, 1337m \in A \ \ 1337n + 1337m = 1337(n + m) \in A\)
        \item \(\forall 1337n \in A \ \ \exists 1337( - n) \in A : 1337n + 1337( - n) = 1337 \cdot 0 = 0\)
    \end{enumerate}

    Аналогичными выкладками \(B\) --- подгруппа.

    \(H\) --- подгруппа:
    \begin{enumerate}
        \item \(\underbrace{0}_{\in A} + \underbrace{0}_{\in B} \in H\)
        \item \(\forall (1337n + 1528m), (1337k + 1528l) \in H \ \ 1337n + 1528m + 1337k + 1528l = 1337(n + k) + 1528(m + l) \in H\)
        \item \(\forall 1337n + 1528m \in H \ \ \exists 1337( -n) + 1528( -m) \in H : 1337n + 1528m + 1337( - n) + 1528( - m) = 0\)
    \end{enumerate}

    Несложно посчитать, что НОД\((1337, 1528) = 191\) и тогда \(H = 191\Z\), т.к.
    \[1337n + 1528m = 191(7n + 8m)\]
    и \(7n + 8m\) пробегает всё \(\Z\).
    Кроме того, очевидно, что \([\Z : 191\Z] = 191\), т.к. левые смежные классы будут иметь вид \(191\Z + n\), два класса для \(n_1\) и \(n_2\) совпадают \(\Leftrightarrow \) \(n_1 \equiv n_2 \mod 191\).
\end{solution}

\begin{exercise}
    Рассмотрим группу \(G\) (не обязательно конечную) и некоторую её подгруппу \(H\). Показать, что условия \([G : H] = 2\) достаточно для нормальности \(H\). Найти \(G / H\) в таком случае.
\end{exercise}
\begin{solution}
    Т.к. \([G : H] = 2\), все левые смежные классы равны либо \(H\), либо \(aH\) для некоторого фиксированного \(a \in G\). Кроме того, \(aH \neq H \Rightarrow a \notin H\). Т.к. левые смежные классы делят группу на непересекающиеся множества, \(ah = G \setminus H\).

    Докажем, что \(\forall g \in G \ \ gH = Hg\). Если \(g \in H\), то искомое очевидно. Иначе \(gH = G \setminus H\), т.к. \(H \not\ni g = ge \in gH\). Аналогично \(Hg = G \setminus H\).
\end{solution}

\begin{exercise}
    Определить все подгруппы групп: \(\Z_4, \Z_2 \times \Z_2, \Z_6\)

    \textbf{Замечание:} операция ``\(\hat{\oplus}\)'' в \(\Z_2 \times \Z_2\) определяется покомпонентно:
    \[z, w \in \Z_2 \times \Z_2\]
    \[z = (a, b), \quad w = (u, v)\]
    \[z \hat{\oplus} w = (a, b) \hat{\oplus} (u, v) = (a \oplus u, b \oplus w)\]
    где ``\(\oplus\)'' есть операция в \(\Z_2\)
\end{exercise}
\begin{solution}\itemfix
    \begin{enumerate}
        \item \(\Z_4\)

              \(\Z_4, \{0\}\) --- тривиальные подгруппы.

              Здесь и далее \(H\) --- подгруппа рассматриваемой группы.

              Пусть \(1 \in H\). По замкнутости \(2 = 1 + 1 \in H, 3 = 2 + 1 \in H\), т.е. если \(1 \in H\), то \(H = \Z_4\).

              Пусть \(2 \in H\). Тогда все искомые свойства выполнены без добавления каких-либо элементов\footnote{Кроме нейтрального.}, т.к. \(2 + 2 = 0 \in H, 2^{-1} = 2\), \(\{0, 2\}\) --- подгруппа \(\Z_4\).

              Пусть \(3 \in H\). \(3^{-1} = 1 \Rightarrow 1 \in H \Rightarrow H = \Z_4\)

              \textbf{Ответ:} \(\{0\}, \Z_4, \{0, 2\}\)

        \item \(\Z_2 \times \Z_2\)

              \(\Z_2 \times \Z_2, \{(0, 0)\}\) --- тривиальные подгруппы.

              Пусть \((1, 1) \in H\). Тогда все искомые свойства выполнены без добавления элементов, т.к. \((1, 1) + (1, 1) = (0, 0) \in H, (1, 1)^{-1} = (1, 1)\), \(\{(0, 0), (1, 1)\}\) --- подгруппа \(\Z_2 \times \Z_2\).

              Пусть \((1, 0) \in H\). \((1, 0) + (1, 0) = (0, 0) \in H, (1, 0)^{-1} = (1, 0) \Rightarrow \{(0, 0), (1, 0)\}\) --- подгруппа \(\Z_2 \times \Z_2\). Аналогичное верно для \((0, 1)\).

              Пусть и \((1, 0)\), и \((0, 1) \in H\). Тогда \((1, 1) \in H\) по замкнутости и следовательно \(H = \Z_2 \times \Z_2\).

              Пусть и \((1, 1)\), и \((0, 1) \in H\). Тогда \((1, 1) + (0, 1) = (1, 0) \in H\) по замкнутости и \(H = \Z_2 \times \Z_2\). Аналогично для \((1, 1)\) и \((0, 1)\).

              \textbf{Ответ:} \(\{(0, 0)\}, \Z_2 \times \Z_2, \{(0, 0), (1, 1)\}, \{(0, 0), (1, 0)\}, \{(0, 0), (0, 1)\}\)

        \item \(\Z_6\)

              \(\Z_6, \{0\}\) --- тривиальные подгруппы.


              Пусть \(1 \in H\). Тогда \(H = \Z_6\), аналогично первому случаю.

              Пусть \(2 \in H\). Тогда \(2 + 2 = 4 \in H\). \(2^{-1} = 4, 4^{-1} = 2, 2 + 4 = 0, 4 + 4 = 2\), \(H\) --- подгруппа.

              Пусть \(3 \in H\). Тогда \(3 + 3 = 0, 3^{-1} = 3, H\) --- подгруппа.

              Пусть \(4 \in H\). Тогда \(4^{-1} = 2 \in H\), см. тот случай.

              Пусть \(5 \in H\). Тогда \(5 + 5 = 4 \in H \Rightarrow 2 \in H \Rightarrow 2 + 5 = 1 \in H \Rightarrow H = \Z_6\).

              Если \(2, 3 \in H\), то \(2 + 3 + 2 = 1 \in H \Rightarrow H = \Z_6\).

              Если \(2, 5 \in H\), то \(2 + 5 = 1 \in H \Rightarrow H = \Z_6\).

              Все случаи для \(2 \in H\) разобраны, остался случай \(3 \in H (2 \notin H)\). Если \(5 \in H\), то \(3 + 5 = 2 \in H, !!!\).

              \textbf{Ответ:} \(\{0\}, \{0, 2, 4\}, \{0, 3\}, \Z_6\).
    \end{enumerate}
\end{solution}

\begin{exercise}
    Рассмотрим циклическую группу порядка \(129\). Найти все её подгруппы.
\end{exercise}
\begin{solution}
    Рассмотрим \(H\) --- подгруппу \(C_{129}\). Пусть \(C_{129} = \ev{a}\). Тогда \(a^k \in H\). По замкнутости \(\forall i \in \Z \ \ a^{ik} \in H\). Если \(gcd(129, k) = 1\), то \(ik\) пробегает все элементы \(\Z_{129}\) и тогда \(H = C_{129}\). Если же \(gcd(129, k) \neq 1\), то \(H\) не обязательно \( = C_{129}\). Нетривиальных делителей \(129\) всего два: \(3\) и \(43\). Им соответствуют подгруппы \(\{1, g^{43}, g^{126}\}\) и \(\{1, g^3, g^6 \dots g^{126}\}\).


    \textbf{Ответ:} \(\{1, g^{43}, g^{126}\}, \{1, g^3, g^6 \dots g^{126}\}, \{e\}, C_{129}\).
\end{solution}

\end{document}
