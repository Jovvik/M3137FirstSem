\documentclass[12pt, a4paper]{article}

%<*preamble>
% Math symbols
\usepackage{amsmath, amsthm, amsfonts, amssymb}
\usepackage{accents}
\usepackage{esvect}
\usepackage{mathrsfs}
\usepackage{mathtools}
\mathtoolsset{showonlyrefs}
\usepackage{cmll}
\usepackage{stmaryrd}
\usepackage{physics}
\usepackage[normalem]{ulem}
\usepackage{ebproof}
\usepackage{extarrows}

% Page layout
\usepackage{geometry, a4wide, parskip, fancyhdr}

% Font, encoding, russian support
\usepackage[russian]{babel}
\usepackage[sb]{libertine}
\usepackage{xltxtra}

% Listings
\usepackage{listings}
\lstset{basicstyle=\ttfamily,breaklines=true}
\setmonofont{Inconsolata}

% Miscellaneous
\usepackage{array}
\usepackage{calc}
\usepackage{caption}
\usepackage{subcaption}
\captionsetup{justification=centering,margin=2cm}
\usepackage{catchfilebetweentags}
\usepackage{enumitem}
\usepackage{etoolbox}
\usepackage{float}
\usepackage{lastpage}
\usepackage{minted}
\usepackage{svg}
\usepackage{wrapfig}
\usepackage{xcolor}
\usepackage[makeroom]{cancel}

\newcolumntype{L}{>{$}l<{$}}
    \newcolumntype{C}{>{$}c<{$}}
\newcolumntype{R}{>{$}r<{$}}

% Footnotes
\usepackage[hang]{footmisc}
\setlength{\footnotemargin}{2mm}
\makeatletter
\def\blfootnote{\gdef\@thefnmark{}\@footnotetext}
\makeatother

% References
\usepackage{hyperref}
\hypersetup{
    colorlinks,
    linkcolor={blue!80!black},
    citecolor={blue!80!black},
    urlcolor={blue!80!black},
}

% tikz
\usepackage{tikz}
\usepackage{tikz-cd}
\usetikzlibrary{arrows.meta}
\usetikzlibrary{decorations.pathmorphing}
\usetikzlibrary{calc}
\usetikzlibrary{patterns}
\usepackage{pgfplots}
\pgfplotsset{width=10cm,compat=1.9}
\newcommand\irregularcircle[2]{% radius, irregularity
    \pgfextra {\pgfmathsetmacro\len{(#1)+rand*(#2)}}
    +(0:\len pt)
    \foreach \a in {10,20,...,350}{
            \pgfextra {\pgfmathsetmacro\len{(#1)+rand*(#2)}}
            -- +(\a:\len pt)
        } -- cycle
}

\providetoggle{useproofs}
\settoggle{useproofs}{false}

\pagestyle{fancy}
\lfoot{M3137y2019}
\cfoot{}
\rhead{стр. \thepage\ из \pageref*{LastPage}}

\newcommand{\R}{\mathbb{R}}
\newcommand{\Q}{\mathbb{Q}}
\newcommand{\Z}{\mathbb{Z}}
\newcommand{\B}{\mathbb{B}}
\newcommand{\N}{\mathbb{N}}
\renewcommand{\Re}{\mathfrak{R}}
\renewcommand{\Im}{\mathfrak{I}}

\newcommand{\const}{\text{const}}
\newcommand{\cond}{\text{cond}}

\newcommand{\teormin}{\textcolor{red}{!}\ }

\DeclareMathOperator*{\xor}{\oplus}
\DeclareMathOperator*{\equ}{\sim}
\DeclareMathOperator{\sign}{\text{sign}}
\DeclareMathOperator{\Sym}{\text{Sym}}
\DeclareMathOperator{\Asym}{\text{Asym}}

\DeclarePairedDelimiter{\ceil}{\lceil}{\rceil}

% godel
\newbox\gnBoxA
\newdimen\gnCornerHgt
\setbox\gnBoxA=\hbox{$\ulcorner$}
\global\gnCornerHgt=\ht\gnBoxA
\newdimen\gnArgHgt
\def\godel #1{%
    \setbox\gnBoxA=\hbox{$#1$}%
    \gnArgHgt=\ht\gnBoxA%
    \ifnum     \gnArgHgt<\gnCornerHgt \gnArgHgt=0pt%
    \else \advance \gnArgHgt by -\gnCornerHgt%
    \fi \raise\gnArgHgt\hbox{$\ulcorner$} \box\gnBoxA %
    \raise\gnArgHgt\hbox{$\urcorner$}}

% \theoremstyle{plain}

\theoremstyle{definition}
\newtheorem{theorem}{Теорема}
\newtheorem*{definition}{Определение}
\newtheorem{axiom}{Аксиома}
\newtheorem*{axiom*}{Аксиома}
\newtheorem{lemma}{Лемма}

\theoremstyle{remark}
\newtheorem*{remark}{Примечание}
\newtheorem*{exercise}{Упражнение}
\newtheorem{corollary}{Следствие}[theorem]
\newtheorem*{statement}{Утверждение}
\newtheorem*{corollary*}{Следствие}
\newtheorem*{example}{Пример}
\newtheorem{observation}{Наблюдение}
\newtheorem*{prop}{Свойства}
\newtheorem*{obozn}{Обозначение}

% subtheorem
\makeatletter
\newenvironment{subtheorem}[1]{%
    \def\subtheoremcounter{#1}%
    \refstepcounter{#1}%
    \protected@edef\theparentnumber{\csname the#1\endcsname}%
    \setcounter{parentnumber}{\value{#1}}%
    \setcounter{#1}{0}%
    \expandafter\def\csname the#1\endcsname{\theparentnumber.\Alph{#1}}%
    \ignorespaces
}{%
    \setcounter{\subtheoremcounter}{\value{parentnumber}}%
    \ignorespacesafterend
}
\makeatother
\newcounter{parentnumber}

\newtheorem{manualtheoreminner}{Теорема}
\newenvironment{manualtheorem}[1]{%
    \renewcommand\themanualtheoreminner{#1}%
    \manualtheoreminner
}{\endmanualtheoreminner}

\newcommand{\dbltilde}[1]{\accentset{\approx}{#1}}
\newcommand{\intt}{\int\!}

% magical thing that fixes paragraphs
\makeatletter
\patchcmd{\CatchFBT@Fin@l}{\endlinechar\m@ne}{}
{}{\typeout{Unsuccessful patch!}}
\makeatother

\newcommand{\get}[2]{
    \ExecuteMetaData[#1]{#2}
}

\newcommand{\getproof}[2]{
    \iftoggle{useproofs}{\ExecuteMetaData[#1]{#2proof}}{}
}

\newcommand{\getwithproof}[2]{
    \get{#1}{#2}
    \getproof{#1}{#2}
}

\newcommand{\import}[3]{
    \subsection{#1}
    \getwithproof{#2}{#3}
}

\newcommand{\given}[1]{
    Дано выше. (\ref{#1}, стр. \pageref{#1})
}

\renewcommand{\ker}{\text{Ker }}
\newcommand{\im}{\text{Im }}
\renewcommand{\grad}{\text{grad}}
\newcommand{\rg}{\text{rg}}
\newcommand{\defeq}{\stackrel{\text{def}}{=}}
\newcommand{\defeqfor}[1]{\stackrel{\text{def } #1}{=}}
\newcommand{\itemfix}{\leavevmode\makeatletter\makeatother}
\newcommand{\?}{\textcolor{red}{???}}
\renewcommand{\emptyset}{\varnothing}
\newcommand{\longarrow}[1]{\xRightarrow[#1]{\qquad}}
\DeclareMathOperator*{\esup}{\text{ess sup}}
\newcommand\smallO{
    \mathchoice
    {{\scriptstyle\mathcal{O}}}% \displaystyle
    {{\scriptstyle\mathcal{O}}}% \textstyle
    {{\scriptscriptstyle\mathcal{O}}}% \scriptstyle
    {\scalebox{.6}{$\scriptscriptstyle\mathcal{O}$}}%\scriptscriptstyle
}
\renewcommand{\div}{\text{div}\ }
\newcommand{\rot}{\text{rot}\ }
\newcommand{\cov}{\text{cov}}

\makeatletter
\newcommand{\oplabel}[1]{\refstepcounter{equation}(\theequation\ltx@label{#1})}
\makeatother

\newcommand{\symref}[2]{\stackrel{\oplabel{#1}}{#2}}
\newcommand{\symrefeq}[1]{\symref{#1}{=}}

% xrightrightarrows
\makeatletter
\newcommand*{\relrelbarsep}{.386ex}
\newcommand*{\relrelbar}{%
    \mathrel{%
        \mathpalette\@relrelbar\relrelbarsep
    }%
}
\newcommand*{\@relrelbar}[2]{%
    \raise#2\hbox to 0pt{$\m@th#1\relbar$\hss}%
    \lower#2\hbox{$\m@th#1\relbar$}%
}
\providecommand*{\rightrightarrowsfill@}{%
    \arrowfill@\relrelbar\relrelbar\rightrightarrows
}
\providecommand*{\leftleftarrowsfill@}{%
    \arrowfill@\leftleftarrows\relrelbar\relrelbar
}
\providecommand*{\xrightrightarrows}[2][]{%
    \ext@arrow 0359\rightrightarrowsfill@{#1}{#2}%
}
\providecommand*{\xleftleftarrows}[2][]{%
    \ext@arrow 3095\leftleftarrowsfill@{#1}{#2}%
}

\allowdisplaybreaks

\newcommand{\unfinished}{\textcolor{red}{Не дописано}}

% Reproducible pdf builds 
\special{pdf:trailerid [
<00112233445566778899aabbccddeeff>
<00112233445566778899aabbccddeeff>
]}
%</preamble>


\lhead{Алгоритмы в математике \textit{(практика)}}
\cfoot{}
\rfoot{30.10.2021}

\begin{document}

\begin{exercise}
    Рассмотрим группу диэдра \(D_6\). Найти в ней силовскую \(2\)-подгруппу \(\mathcal{P}_2\) и силовскую \(3\)-подгруппу \(\mathcal{P}_3\) такие, чтобы \(\mathcal{P}_3\) была нормальной. Рассмотрим множество \(S = \mathcal{P}_3 \times \mathcal{P}_2\):
    \[S = \{(\sigma, \tau) \mid \sigma \in \mathcal{P}_3, \tau \in \mathcal{P}_2\}\]

    Рассмотрим отображение \(\varphi : S \to D_6\), которое перемножает компоненты кортежа:
    \[\varphi((\sigma, \tau)) = \sigma\tau\]
    Ввести на \(S\) структуру группы, так чтобы отображение \(\varphi\) стало изоморфизмом.
\end{exercise}
\begin{solution}
    \(12 = 4 \cdot 3 \Rightarrow\) силовские \(2\)-подгруппы имеют порядок \(2^2 = 4\), силовские \(3\)-подгруппы имеют порядок \(3^1 = 3\).

    \begin{notation}
        Поворот на \(60t\) (\(60t \in [0, 360)\)) градусов это \(\rho_t\). Зафиксируем произвольную ось симметрии, тогда \(\tau_t\) --- отражение относительно оси, повернутой на \(30t\) градусов относительно фиксированной оси (\(30t \in [0, 180)\)). Все сложения и вычитания в индексах операций выполняются по модулю \(6\).
    \end{notation}

    Несложно заметить, что \(\{e, \tau_0, \rho_3, \tau_3\}\) образуют подгруппу размера \(4\), пусть она будет \(\mathcal{P}_2\).

    \(\mathcal{P}_3 = \{e, \rho_2, \rho_4\}\). Нормальность:
    \begin{enumerate}
        \item Поскольку повороты коммутируют между собой, \(\rho_i \mathcal{P}_3 \rho_i^{-1} = \rho_i\rho_i^{-1} \mathcal{P}_3 = \mathcal{P}_3\).
        \item \begin{enumerate}
                  \item \(e\) очевидно нормален
                  \item \(\tau_i \rho_2 \tau_i^{-1} = \tau_{i - 2} \tau_i = \rho_{ -2} = \rho_4\)
                  \item \(\tau_i \rho_4 \tau_i^{-1} = \tau_{i - 4} \tau_i = \rho_{ -4} = \rho_2\)
              \end{enumerate}
    \end{enumerate}

    \(e_S = (e, e)\) --- очевидно.

    Воспользуемся сопряжением:
    \[(\sigma, \tau) \circ (\sigma', \tau') \coloneqq (\sigma \tau \sigma' \tau^{-1}, \tau \tau')\]

    Интуитивное пояснение нахождения этой операции: нормальность \(\mathcal{P}_3\) явно требуется не случайно, поэтому воспользуемся тем, что \(\tau \sigma \tau^{-1} \in \mathcal{P}_3\) --- тут есть 4 варианта навешивания штрихов. Для второго элемента результата вариантов два: \(\tau \tau'\) и \(\tau' \tau\) (отбрасывать \(\tau\) или \(\tau'\) не кажется содержательным). Мы ещё забыли \(\sigma'\) \textit{(или \(\sigma\), в зависимости от штрихов в сопряжении)}, поэтому домножим на него в первом элементе результата. После небольшого перебора находится искомая операция.

    \begin{remark}
        То, что \(\varphi\) --- изоморфизм, показано в задаче 3.
    \end{remark}
\end{solution}

\begin{exercise}
    Рассмотрим группу порядка \(35\). Рассмотрим некоторую её силовскую \(5\)-подгруппу \(\mathcal{P}_5\). Показать, что она единственна.

    \begin{remark}
        Показать, что количество \(n_5\) силовских \(5\)-подгрупп равно: \(n_5 = 1\)
    \end{remark}
\end{exercise}
\begin{proof}
    \(35 = 5 \cdot 7 \Rightarrow n_5 \equiv 1 \mod 5\) и \(\frac{35}{5} \divided n_5\) по третьей теореме Силова\footnote{Второй факт, кажется, не рассматривался на лекции, но он очевиден. Я взял его с википедии, страница ``Теоремы Силова''.}. У \(7\) два делителя: \(1\) и \(7\). \(7 \not\equiv 1 \mod 5, 1 \equiv 1 \mod 5 \Rightarrow n_5 = 1\).
\end{proof}

\begin{exercise}
    Рассмотрим группу \(G\) порядка \(119\). Пусть \(\mathcal{P}_7\) --- её силовская \(7\)-подгруппа. Показать, что \(\mathcal{P}_7\) нормально. Показать, что фактор-группа \(G / \mathcal{P}_7\) --- циклическая группа. Показать, что группа \(G\) абелева.
    \begin{remark}
        см. задачу 1
    \end{remark}
\end{exercise}
\begin{solution}
    \(119 = 7 \cdot 17 \Rightarrow |\mathcal{P}_7| = 7 \Rightarrow |G / \mathcal{P}_7| = \frac{|G|}{|\mathcal{P}_7|} = \frac{119}{7} = 17\) --- простое число \( \Rightarrow G / \mathcal{P}_7\) --- циклическая группа.

    Аналогично предыдущей задаче \(n_7 \equiv 1 \mod 5, 17 \divided n_7 \Rightarrow n_7 = 1\), т.е. \(\mathcal{P}_7\) --- единственная силовская \(7\)-подгруппа. \(|g\mathcal{P}_7g^{-1}| = |\mathcal{P}_7|\), но \(g\mathcal{P}_7g^{-1}\) также является силовской \(7\)-подгруппой. В силу единственности \(g\mathcal{P}_7g^{-1} = \mathcal{P}_7 \Rightarrow \mathcal{P}_7\) нормальна.

    По первой теореме Силова \(\exists \mathcal{P}_{17}\). \(\mathcal{P}_{7} \times \mathcal{P}_{17} \cong G\) по изоморфизму \(\varphi : (a, b) \mapsto ab\), где операция на \(\mathcal{P}_7, \mathcal{P}_{17}\) есть \((a, b) \circ (c, d) = (abcb^{-1}, bd)\):
    \[abcd = \varphi(a, b) \varphi(c, d) \stackrel{?}{=} \varphi((a, b) \circ (c, d)) = \varphi((abcb^{-1}, bd)) = abcd\]

    Т.к \(7\) и \(17\) простые числа, \(\mathcal{P}_7\) и \(\mathcal{P}_{17}\) циклические, а следовательно абелевы. Пусть \(\mathcal{P}_7 = \ev{g}, \mathcal{P}_{17} = \ev{h}\). Покажем, что \(g^ih^j = h^jg^i\), тогда:
    \[(g^i, h^j) \circ (g^k, h^l) = (g^ih^jg^kh^{-j}, h^{j + l}) = (g^{i + k}, h^{j + l}) = (g^k, h^l) \circ (g^i, h^j)\]
    , то есть \(\mathcal{P}_7 \times \mathcal{P}_{17}\) абелева, тогда \(G\) абелева как изоморфная абелевой.

    Для этого покажем \(g^ih = hg^i\), тогда , тогда искомое будет верно по индукции (\(g^ih^{j + 1} = g^ih h^j = hg^i h^j = h^{j+1}g^i\))

    \[\sphericalangle a = g^i, b = hg^j \quad ab = g^ihg^j = h(h^{-1}g^ih)g^j = hg^kg^j = hg^{k + j} \quad ba = hg^{i + j}\]
    , где \(g^k = h^{-1}g^ih\), такое \(k\) существует по нормальности \(\mathcal{P}_7\).
    \begin{align*}
        g^k                                   & = h^{-1}g^ih \\
        hg^k                                  & = g^ih       \\
        g^{ - i}hg^k                          & = h          \\
        (g^{ - i}hg^k)^{17}                   & = h^{17}     \\
        (g^{ - i}hg^k)^{17}                   & = e          \\
        g^{ - i}hg^{k - i}hg^{k - i}\dots g^k & = e          \\
        (hg^{k - i})^{17}                     & = e          \\
    \end{align*}

    Таким образом, \(hg^{k - i} \in \mathcal{P}_{17}\)\footnote{Точнее, \(hg^{k - i} \in \varphi^{-1}(\mathcal{P}_{17})\)}, тогда \(g^{k - i} = 1 \Rightarrow k - i \equiv 0 \mod 7 \Rightarrow ab = hg^{k + j} = hg^{i + j} = ba\)
\end{solution}

\end{document}
