\documentclass[12pt, a4paper]{article}

%<*preamble>
% Math symbols
\usepackage{amsmath, amsthm, amsfonts, amssymb}
\usepackage{accents}
\usepackage{esvect}
\usepackage{mathrsfs}
\usepackage{mathtools}
\mathtoolsset{showonlyrefs}
\usepackage{cmll}
\usepackage{stmaryrd}
\usepackage{physics}
\usepackage[normalem]{ulem}
\usepackage{ebproof}
\usepackage{extarrows}

% Page layout
\usepackage{geometry, a4wide, parskip, fancyhdr}

% Font, encoding, russian support
\usepackage[russian]{babel}
\usepackage[sb]{libertine}
\usepackage{xltxtra}

% Listings
\usepackage{listings}
\lstset{basicstyle=\ttfamily,breaklines=true}
\setmonofont{Inconsolata}

% Miscellaneous
\usepackage{array}
\usepackage{calc}
\usepackage{caption}
\usepackage{subcaption}
\captionsetup{justification=centering,margin=2cm}
\usepackage{catchfilebetweentags}
\usepackage{enumitem}
\usepackage{etoolbox}
\usepackage{float}
\usepackage{lastpage}
\usepackage{minted}
\usepackage{svg}
\usepackage{wrapfig}
\usepackage{xcolor}
\usepackage[makeroom]{cancel}

\newcolumntype{L}{>{$}l<{$}}
    \newcolumntype{C}{>{$}c<{$}}
\newcolumntype{R}{>{$}r<{$}}

% Footnotes
\usepackage[hang]{footmisc}
\setlength{\footnotemargin}{2mm}
\makeatletter
\def\blfootnote{\gdef\@thefnmark{}\@footnotetext}
\makeatother

% References
\usepackage{hyperref}
\hypersetup{
    colorlinks,
    linkcolor={blue!80!black},
    citecolor={blue!80!black},
    urlcolor={blue!80!black},
}

% tikz
\usepackage{tikz}
\usepackage{tikz-cd}
\usetikzlibrary{arrows.meta}
\usetikzlibrary{decorations.pathmorphing}
\usetikzlibrary{calc}
\usetikzlibrary{patterns}
\usepackage{pgfplots}
\pgfplotsset{width=10cm,compat=1.9}
\newcommand\irregularcircle[2]{% radius, irregularity
    \pgfextra {\pgfmathsetmacro\len{(#1)+rand*(#2)}}
    +(0:\len pt)
    \foreach \a in {10,20,...,350}{
            \pgfextra {\pgfmathsetmacro\len{(#1)+rand*(#2)}}
            -- +(\a:\len pt)
        } -- cycle
}

\providetoggle{useproofs}
\settoggle{useproofs}{false}

\pagestyle{fancy}
\lfoot{M3137y2019}
\cfoot{}
\rhead{стр. \thepage\ из \pageref*{LastPage}}

\newcommand{\R}{\mathbb{R}}
\newcommand{\Q}{\mathbb{Q}}
\newcommand{\Z}{\mathbb{Z}}
\newcommand{\B}{\mathbb{B}}
\newcommand{\N}{\mathbb{N}}
\renewcommand{\Re}{\mathfrak{R}}
\renewcommand{\Im}{\mathfrak{I}}

\newcommand{\const}{\text{const}}
\newcommand{\cond}{\text{cond}}

\newcommand{\teormin}{\textcolor{red}{!}\ }

\DeclareMathOperator*{\xor}{\oplus}
\DeclareMathOperator*{\equ}{\sim}
\DeclareMathOperator{\sign}{\text{sign}}
\DeclareMathOperator{\Sym}{\text{Sym}}
\DeclareMathOperator{\Asym}{\text{Asym}}

\DeclarePairedDelimiter{\ceil}{\lceil}{\rceil}

% godel
\newbox\gnBoxA
\newdimen\gnCornerHgt
\setbox\gnBoxA=\hbox{$\ulcorner$}
\global\gnCornerHgt=\ht\gnBoxA
\newdimen\gnArgHgt
\def\godel #1{%
    \setbox\gnBoxA=\hbox{$#1$}%
    \gnArgHgt=\ht\gnBoxA%
    \ifnum     \gnArgHgt<\gnCornerHgt \gnArgHgt=0pt%
    \else \advance \gnArgHgt by -\gnCornerHgt%
    \fi \raise\gnArgHgt\hbox{$\ulcorner$} \box\gnBoxA %
    \raise\gnArgHgt\hbox{$\urcorner$}}

% \theoremstyle{plain}

\theoremstyle{definition}
\newtheorem{theorem}{Теорема}
\newtheorem*{definition}{Определение}
\newtheorem{axiom}{Аксиома}
\newtheorem*{axiom*}{Аксиома}
\newtheorem{lemma}{Лемма}

\theoremstyle{remark}
\newtheorem*{remark}{Примечание}
\newtheorem*{exercise}{Упражнение}
\newtheorem{corollary}{Следствие}[theorem]
\newtheorem*{statement}{Утверждение}
\newtheorem*{corollary*}{Следствие}
\newtheorem*{example}{Пример}
\newtheorem{observation}{Наблюдение}
\newtheorem*{prop}{Свойства}
\newtheorem*{obozn}{Обозначение}

% subtheorem
\makeatletter
\newenvironment{subtheorem}[1]{%
    \def\subtheoremcounter{#1}%
    \refstepcounter{#1}%
    \protected@edef\theparentnumber{\csname the#1\endcsname}%
    \setcounter{parentnumber}{\value{#1}}%
    \setcounter{#1}{0}%
    \expandafter\def\csname the#1\endcsname{\theparentnumber.\Alph{#1}}%
    \ignorespaces
}{%
    \setcounter{\subtheoremcounter}{\value{parentnumber}}%
    \ignorespacesafterend
}
\makeatother
\newcounter{parentnumber}

\newtheorem{manualtheoreminner}{Теорема}
\newenvironment{manualtheorem}[1]{%
    \renewcommand\themanualtheoreminner{#1}%
    \manualtheoreminner
}{\endmanualtheoreminner}

\newcommand{\dbltilde}[1]{\accentset{\approx}{#1}}
\newcommand{\intt}{\int\!}

% magical thing that fixes paragraphs
\makeatletter
\patchcmd{\CatchFBT@Fin@l}{\endlinechar\m@ne}{}
{}{\typeout{Unsuccessful patch!}}
\makeatother

\newcommand{\get}[2]{
    \ExecuteMetaData[#1]{#2}
}

\newcommand{\getproof}[2]{
    \iftoggle{useproofs}{\ExecuteMetaData[#1]{#2proof}}{}
}

\newcommand{\getwithproof}[2]{
    \get{#1}{#2}
    \getproof{#1}{#2}
}

\newcommand{\import}[3]{
    \subsection{#1}
    \getwithproof{#2}{#3}
}

\newcommand{\given}[1]{
    Дано выше. (\ref{#1}, стр. \pageref{#1})
}

\renewcommand{\ker}{\text{Ker }}
\newcommand{\im}{\text{Im }}
\renewcommand{\grad}{\text{grad}}
\newcommand{\rg}{\text{rg}}
\newcommand{\defeq}{\stackrel{\text{def}}{=}}
\newcommand{\defeqfor}[1]{\stackrel{\text{def } #1}{=}}
\newcommand{\itemfix}{\leavevmode\makeatletter\makeatother}
\newcommand{\?}{\textcolor{red}{???}}
\renewcommand{\emptyset}{\varnothing}
\newcommand{\longarrow}[1]{\xRightarrow[#1]{\qquad}}
\DeclareMathOperator*{\esup}{\text{ess sup}}
\newcommand\smallO{
    \mathchoice
    {{\scriptstyle\mathcal{O}}}% \displaystyle
    {{\scriptstyle\mathcal{O}}}% \textstyle
    {{\scriptscriptstyle\mathcal{O}}}% \scriptstyle
    {\scalebox{.6}{$\scriptscriptstyle\mathcal{O}$}}%\scriptscriptstyle
}
\renewcommand{\div}{\text{div}\ }
\newcommand{\rot}{\text{rot}\ }
\newcommand{\cov}{\text{cov}}

\makeatletter
\newcommand{\oplabel}[1]{\refstepcounter{equation}(\theequation\ltx@label{#1})}
\makeatother

\newcommand{\symref}[2]{\stackrel{\oplabel{#1}}{#2}}
\newcommand{\symrefeq}[1]{\symref{#1}{=}}

% xrightrightarrows
\makeatletter
\newcommand*{\relrelbarsep}{.386ex}
\newcommand*{\relrelbar}{%
    \mathrel{%
        \mathpalette\@relrelbar\relrelbarsep
    }%
}
\newcommand*{\@relrelbar}[2]{%
    \raise#2\hbox to 0pt{$\m@th#1\relbar$\hss}%
    \lower#2\hbox{$\m@th#1\relbar$}%
}
\providecommand*{\rightrightarrowsfill@}{%
    \arrowfill@\relrelbar\relrelbar\rightrightarrows
}
\providecommand*{\leftleftarrowsfill@}{%
    \arrowfill@\leftleftarrows\relrelbar\relrelbar
}
\providecommand*{\xrightrightarrows}[2][]{%
    \ext@arrow 0359\rightrightarrowsfill@{#1}{#2}%
}
\providecommand*{\xleftleftarrows}[2][]{%
    \ext@arrow 3095\leftleftarrowsfill@{#1}{#2}%
}

\allowdisplaybreaks

\newcommand{\unfinished}{\textcolor{red}{Не дописано}}

% Reproducible pdf builds 
\special{pdf:trailerid [
<00112233445566778899aabbccddeeff>
<00112233445566778899aabbccddeeff>
]}
%</preamble>


\lhead{Алгоритмы в математике \textit{(практика)}}
\cfoot{}
\rfoot{6.11.2021}

\begin{document}

\begin{exercise}
    Пусть \(H, K \vartriangleleft G\) --- две нормальные подгруппы в \(G\). Докажите, что тогда коммутатор любых двух элементов из \(H\) и \(K\) принадлежит пересечению \(H \cap K\).
\end{exercise}
\begin{solution}
    \begin{lemma}
        \((ab)^{-1} = b^{-1}a^{-1}\)
    \end{lemma}
    \begin{proof}
        \((ab)(b^{-1}a^{-1}) = e\)
    \end{proof}
    \(\sphericalangle h \in H, k \in K\)
    \[[h, k] = hk(kh)^{-1} = \overbrace{\underbrace{\lefteqn{\overbrace{\phantom{hkh^{-1}}}^{\in K}}h
        \underbrace{kh^{-1}k^{-1}}_{\in H}}_{\in H}}^{\in K}\]
    \([k, h]\) аналогично.
\end{solution}

\begin{exercise}
    Показать, что коммутант \([H, K]\) двух нормальных подгрупп \(H, K \vartriangleleft G\) есть подгруппа в пересечении \([H, K] \subset H \cap K\). Всегда ли \([H, K] = H \cap K\)?
\end{exercise}
\begin{solution}
    Т.к. коммутатор любых двух элементов \(H\) и \(K\) принадлежит и \(H\), и \(K\), то по замкнутости \(H\) и \(K\) произведение коммутаторов также принадлежит и \(H\) и \(K\). Кроме того, \(1 = [1, 1] \in [H, K]\), следовательно, \([H, K] \subset H \cap K\), т.к. это \([H, K]\) это в точности все коммутаторы вида \([hk], h \in H, k \in K\).

    \([H, K]\) не всегда \( = H \cap K\), например если \(G\) абелева, то \([H, K] = \{e\}\), но очевидно не для каждых \(H\) и \(K\) выполнено \(H \cap K = \{e\}\), например для \(H = K = G\).
\end{solution}

\begin{exercise}
    Пусть \(H, K\) --- две произвольные подгруппы. Рассмотреть отображение \(\psi : H \times K \to HK\):
    \[\psi(h, k) = hk\]
    Найти \(\psi^{-1}(x) = \{(h, k) \mid \psi(h, k) = x\}\) в явном виде. Получить из этого, что:
    \[|HK| = \frac{|H| \cdot |K|}{|H \cap K|}\]
\end{exercise}
\begin{solution}
    Пусть \(h_1k_1 = x\) и \(h_2k_2 = x\).
    \[x^{-1} = k_2^{-1}h_2^{-1}\]
    \[e = x x^{-1} = h_1k_1k_2^{-1}h_2^{-1}\]
    \[h_2h_1^{-1} = k_1k_2^{-1}\]
    Т.к. \(h_2h_1^{-1} \in H, k_1k_2^{-1} \in K, k_1k_2^{-1} = h_2h_1^{-1} \eqqcolon a \in K \cap H\). Тогда:
    \[h_2 = ih_1 \Rightarrow h_1 = i^{-1}h_2 \quad k_1 = ik_2\]
    И, следовательно, любое \((h_1, k_1) : h_1k_1 = x\) записывается в виде \((i^{-1}h_2, ik_2)\), где \(h_2, k_2\) --- произвольное решение уравнения \(hk = x\). Таким образом:
    \[\psi^{-1}(x) = \{(i^{-1}h_2 , ik_2) \mid i \in K \cap H\} \Rightarrow |\psi^{-1}(x)|= |K \cap H|\]
    \[|H| \cdot |K| = \sum_{x \in HK} |\psi^{-1}(x)| = |HK| \cdot |H \cap K|\]
\end{solution}

\begin{exercise}
    Показать, что среди 5 подгрупп порядков \(483, 1309, 3059, 2783, 3451\) есть хотя бы две абелевы.
\end{exercise}
\begin{solution}
    Разложим размеры групп на простые делители.
    \begin{center}
        \begin{tabular}{CC}\toprule
            n    & p_1^{a_1} \dots p_n^{a_n} \\ \midrule
            483  & 3 \cdot 7 \cdot 23        \\
            1309 & 7 \cdot 11 \cdot 17       \\
            3059 & 7 \cdot 19 \cdot 23       \\
            2783 & 11^2 \cdot 23             \\
            3451 & 7 \cdot 11 \cdot 29       \\
            \bottomrule
        \end{tabular}
    \end{center}

    Группу размера \(2783\) больше рассматривать не будем. Размеры всех остальных групп разбиваются на простые числа в первой степени, следовательно, по первой теореме Силова у каждой группы есть силовские \(p\)--подгруппы порядка каждого такого простого числа. Кроме того, они циклические.

    \begin{statement}
        \(G \cong H \times K \Leftrightarrow
        \begin{cases}
            G = HK           \\
            H \cap K = \{e\} \\
            H, K \vartriangleleft G
        \end{cases}\), аналогично для большего числа подгрупп.
    \end{statement}

    Найдём все такие группы, что их силовские подгруппы \(H, K, L\) нормальны, тогда
    \[|HKL| = \frac{|H| \cdot |K| \cdot |L|}{|H \cap K \cap L|} = |G|\]
    и, следовательно, каждому элементу \(g\) можно взаимно-однозначно сопоставить элемент из \(HKL\), т.е. \(G \cong HKL\) и по утверждению \(G \cong H \times K \times L\), а прямое произведение является абелевой группой.

    Из соображений предыдущего домашнего задания для нормальности силовской \(p_i\)-подгруппы \(\mathcal{P}_{p_i}\) достаточно, чтобы \(p_i \not\equiv 1 \mod p_j \ \ \forall j\).

    \begin{center}
        \begin{tabular}{L|CCC}
            483 & 3 & 7     & 23 \\ \hline
            3   & 0 & 3     & 3  \\
            7   & 1 & \dots      \\
            \dots
        \end{tabular} \quad
        \begin{tabular}{L|CCC}
            1309 & 7 & 11 & 17 \\ \hline
            7    & 0 & 7  & 7  \\
            11   & 4 & 0  & 11 \\
            17   & 3 & 6  & 0
        \end{tabular} \quad
        \begin{tabular}{L|CCC}
            3059 & 7 & 19 & 23 \\ \hline
            7    & 0 & 7  & 7  \\
            19   & 5 & 0  & 19 \\
            23   & 2 & 4  & 0
        \end{tabular}
    \end{center}

    Дальше нет смысла перебирать, т.к. для групп размера \(1309\) и \(3059\) искомое доказано.
\end{solution}

\end{document}
