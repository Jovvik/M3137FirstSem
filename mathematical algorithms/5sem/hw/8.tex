\documentclass[12pt, a4paper]{article}

%<*preamble>
% Math symbols
\usepackage{amsmath, amsthm, amsfonts, amssymb}
\usepackage{accents}
\usepackage{esvect}
\usepackage{mathrsfs}
\usepackage{mathtools}
\mathtoolsset{showonlyrefs}
\usepackage{cmll}
\usepackage{stmaryrd}
\usepackage{physics}
\usepackage[normalem]{ulem}
\usepackage{ebproof}
\usepackage{extarrows}

% Page layout
\usepackage{geometry, a4wide, parskip, fancyhdr}

% Font, encoding, russian support
\usepackage[russian]{babel}
\usepackage[sb]{libertine}
\usepackage{xltxtra}

% Listings
\usepackage{listings}
\lstset{basicstyle=\ttfamily,breaklines=true}
\setmonofont{Inconsolata}

% Miscellaneous
\usepackage{array}
\usepackage{calc}
\usepackage{caption}
\usepackage{subcaption}
\captionsetup{justification=centering,margin=2cm}
\usepackage{catchfilebetweentags}
\usepackage{enumitem}
\usepackage{etoolbox}
\usepackage{float}
\usepackage{lastpage}
\usepackage{minted}
\usepackage{svg}
\usepackage{wrapfig}
\usepackage{xcolor}
\usepackage[makeroom]{cancel}

\newcolumntype{L}{>{$}l<{$}}
    \newcolumntype{C}{>{$}c<{$}}
\newcolumntype{R}{>{$}r<{$}}

% Footnotes
\usepackage[hang]{footmisc}
\setlength{\footnotemargin}{2mm}
\makeatletter
\def\blfootnote{\gdef\@thefnmark{}\@footnotetext}
\makeatother

% References
\usepackage{hyperref}
\hypersetup{
    colorlinks,
    linkcolor={blue!80!black},
    citecolor={blue!80!black},
    urlcolor={blue!80!black},
}

% tikz
\usepackage{tikz}
\usepackage{tikz-cd}
\usetikzlibrary{arrows.meta}
\usetikzlibrary{decorations.pathmorphing}
\usetikzlibrary{calc}
\usetikzlibrary{patterns}
\usepackage{pgfplots}
\pgfplotsset{width=10cm,compat=1.9}
\newcommand\irregularcircle[2]{% radius, irregularity
    \pgfextra {\pgfmathsetmacro\len{(#1)+rand*(#2)}}
    +(0:\len pt)
    \foreach \a in {10,20,...,350}{
            \pgfextra {\pgfmathsetmacro\len{(#1)+rand*(#2)}}
            -- +(\a:\len pt)
        } -- cycle
}

\providetoggle{useproofs}
\settoggle{useproofs}{false}

\pagestyle{fancy}
\lfoot{M3137y2019}
\cfoot{}
\rhead{стр. \thepage\ из \pageref*{LastPage}}

\newcommand{\R}{\mathbb{R}}
\newcommand{\Q}{\mathbb{Q}}
\newcommand{\Z}{\mathbb{Z}}
\newcommand{\B}{\mathbb{B}}
\newcommand{\N}{\mathbb{N}}
\renewcommand{\Re}{\mathfrak{R}}
\renewcommand{\Im}{\mathfrak{I}}

\newcommand{\const}{\text{const}}
\newcommand{\cond}{\text{cond}}

\newcommand{\teormin}{\textcolor{red}{!}\ }

\DeclareMathOperator*{\xor}{\oplus}
\DeclareMathOperator*{\equ}{\sim}
\DeclareMathOperator{\sign}{\text{sign}}
\DeclareMathOperator{\Sym}{\text{Sym}}
\DeclareMathOperator{\Asym}{\text{Asym}}

\DeclarePairedDelimiter{\ceil}{\lceil}{\rceil}

% godel
\newbox\gnBoxA
\newdimen\gnCornerHgt
\setbox\gnBoxA=\hbox{$\ulcorner$}
\global\gnCornerHgt=\ht\gnBoxA
\newdimen\gnArgHgt
\def\godel #1{%
    \setbox\gnBoxA=\hbox{$#1$}%
    \gnArgHgt=\ht\gnBoxA%
    \ifnum     \gnArgHgt<\gnCornerHgt \gnArgHgt=0pt%
    \else \advance \gnArgHgt by -\gnCornerHgt%
    \fi \raise\gnArgHgt\hbox{$\ulcorner$} \box\gnBoxA %
    \raise\gnArgHgt\hbox{$\urcorner$}}

% \theoremstyle{plain}

\theoremstyle{definition}
\newtheorem{theorem}{Теорема}
\newtheorem*{definition}{Определение}
\newtheorem{axiom}{Аксиома}
\newtheorem*{axiom*}{Аксиома}
\newtheorem{lemma}{Лемма}

\theoremstyle{remark}
\newtheorem*{remark}{Примечание}
\newtheorem*{exercise}{Упражнение}
\newtheorem{corollary}{Следствие}[theorem]
\newtheorem*{statement}{Утверждение}
\newtheorem*{corollary*}{Следствие}
\newtheorem*{example}{Пример}
\newtheorem{observation}{Наблюдение}
\newtheorem*{prop}{Свойства}
\newtheorem*{obozn}{Обозначение}

% subtheorem
\makeatletter
\newenvironment{subtheorem}[1]{%
    \def\subtheoremcounter{#1}%
    \refstepcounter{#1}%
    \protected@edef\theparentnumber{\csname the#1\endcsname}%
    \setcounter{parentnumber}{\value{#1}}%
    \setcounter{#1}{0}%
    \expandafter\def\csname the#1\endcsname{\theparentnumber.\Alph{#1}}%
    \ignorespaces
}{%
    \setcounter{\subtheoremcounter}{\value{parentnumber}}%
    \ignorespacesafterend
}
\makeatother
\newcounter{parentnumber}

\newtheorem{manualtheoreminner}{Теорема}
\newenvironment{manualtheorem}[1]{%
    \renewcommand\themanualtheoreminner{#1}%
    \manualtheoreminner
}{\endmanualtheoreminner}

\newcommand{\dbltilde}[1]{\accentset{\approx}{#1}}
\newcommand{\intt}{\int\!}

% magical thing that fixes paragraphs
\makeatletter
\patchcmd{\CatchFBT@Fin@l}{\endlinechar\m@ne}{}
{}{\typeout{Unsuccessful patch!}}
\makeatother

\newcommand{\get}[2]{
    \ExecuteMetaData[#1]{#2}
}

\newcommand{\getproof}[2]{
    \iftoggle{useproofs}{\ExecuteMetaData[#1]{#2proof}}{}
}

\newcommand{\getwithproof}[2]{
    \get{#1}{#2}
    \getproof{#1}{#2}
}

\newcommand{\import}[3]{
    \subsection{#1}
    \getwithproof{#2}{#3}
}

\newcommand{\given}[1]{
    Дано выше. (\ref{#1}, стр. \pageref{#1})
}

\renewcommand{\ker}{\text{Ker }}
\newcommand{\im}{\text{Im }}
\renewcommand{\grad}{\text{grad}}
\newcommand{\rg}{\text{rg}}
\newcommand{\defeq}{\stackrel{\text{def}}{=}}
\newcommand{\defeqfor}[1]{\stackrel{\text{def } #1}{=}}
\newcommand{\itemfix}{\leavevmode\makeatletter\makeatother}
\newcommand{\?}{\textcolor{red}{???}}
\renewcommand{\emptyset}{\varnothing}
\newcommand{\longarrow}[1]{\xRightarrow[#1]{\qquad}}
\DeclareMathOperator*{\esup}{\text{ess sup}}
\newcommand\smallO{
    \mathchoice
    {{\scriptstyle\mathcal{O}}}% \displaystyle
    {{\scriptstyle\mathcal{O}}}% \textstyle
    {{\scriptscriptstyle\mathcal{O}}}% \scriptstyle
    {\scalebox{.6}{$\scriptscriptstyle\mathcal{O}$}}%\scriptscriptstyle
}
\renewcommand{\div}{\text{div}\ }
\newcommand{\rot}{\text{rot}\ }
\newcommand{\cov}{\text{cov}}

\makeatletter
\newcommand{\oplabel}[1]{\refstepcounter{equation}(\theequation\ltx@label{#1})}
\makeatother

\newcommand{\symref}[2]{\stackrel{\oplabel{#1}}{#2}}
\newcommand{\symrefeq}[1]{\symref{#1}{=}}

% xrightrightarrows
\makeatletter
\newcommand*{\relrelbarsep}{.386ex}
\newcommand*{\relrelbar}{%
    \mathrel{%
        \mathpalette\@relrelbar\relrelbarsep
    }%
}
\newcommand*{\@relrelbar}[2]{%
    \raise#2\hbox to 0pt{$\m@th#1\relbar$\hss}%
    \lower#2\hbox{$\m@th#1\relbar$}%
}
\providecommand*{\rightrightarrowsfill@}{%
    \arrowfill@\relrelbar\relrelbar\rightrightarrows
}
\providecommand*{\leftleftarrowsfill@}{%
    \arrowfill@\leftleftarrows\relrelbar\relrelbar
}
\providecommand*{\xrightrightarrows}[2][]{%
    \ext@arrow 0359\rightrightarrowsfill@{#1}{#2}%
}
\providecommand*{\xleftleftarrows}[2][]{%
    \ext@arrow 3095\leftleftarrowsfill@{#1}{#2}%
}

\allowdisplaybreaks

\newcommand{\unfinished}{\textcolor{red}{Не дописано}}

% Reproducible pdf builds 
\special{pdf:trailerid [
<00112233445566778899aabbccddeeff>
<00112233445566778899aabbccddeeff>
]}
%</preamble>


\lhead{Алгоритмы в математике \textit{(практика)}}
\lfoot{Михайлов Максим}
\cfoot{}
\rfoot{13.11.2021}

\begin{document}

\begin{exercise}
    Пусть \(H, K\) --- некоторые группы (вообще говоря неабелевы). Рассмотрим \(G = H \times K\) --- их прямое произведение. Показать, что подгруппа \(F = H \times \{e_K\}\) нормальна в \(G\):
    \[F = \{(h, e_K) \mid h \in G\} \vartriangleleft G\]
\end{exercise}
\begin{solution}
    \[(h_1, k) \circ (h_2, e_K) \circ (h_1^{-1}, k^{-1}) = (h_1 h_2 h_1^{-1}, kk^{-1}) = (h_1h_2h_1^{-1}, e_K) \in F\]
\end{solution}

\begin{exercise}
    Пусть \(G\) --- некоторая конечная абелева группа. Доказать, что существует набор циклических групп \(H_1 \dots H_k\), таких что их произведение изоморфно \(G\):
    \[H_1 \times \dots \times H_k \cong G\]
\end{exercise}
\begin{solution}
    Факторизуем \(n = |G|\):
    \[n = p_1^{q_1} \dots p_k^{q_k}\]
    По теореме Силова существуют \(p\)--подгруппы \(G\) порядков \(p_i^{q_i}\), обозначим их \(\mathcal{P}_i\). Т.к. \(G\) абелева, \(\mathcal{P}_i\) нормальны.

    Докажем по индукции по \(k\), что \(G \cong \bigtimes_{i = 1}^k \mathcal{P}_i\).

    \begin{itemize}
        \item [\textbf{База.}] \(k = 1\): очевидно.
        \item [\textbf{Переход.}] По индукционному предположению для любой абелевой \(G : |G| = n = p_1^{q_1} \dots p_k^{q_k}\) верно \(G \cong \bigtimes_{i = 1}^k \mathcal{P}_i\).

              \(\sphericalangle G' : |G'| = p_1^{q_1} \dots p_k^{q_k} \cdot p_{k+1}^{q_{k+1}} = t \cdot p_{k+1}^{q_{k+1}} = t \cdot r\).

              Покажем, что \(G' \cong H \times K\), где \(|K| = r, |H| = t\). Тогда по индукционному предположению искомое будет верно.

              Пусть \(H = \{x \in G' \mid x^r = e\}, K = \{x \in G' \mid x^t = e\}\).

              \begin{statement}
                  \(G \cong H \times K \Leftrightarrow
                  \begin{cases}
                      G = HK           \\
                      H \cap K = \{e\} \\
                      H, K \vartriangleleft G
                  \end{cases}\)
              \end{statement}
              Покажем все, что все три пункта этого утверждения выполнены для наших \(H, K, G'\):
              \begin{enumerate}
                  \item По какой-то теореме из теории чисел \(\exists a, b \in \Z : at + br = 1\).
                        \[\forall x \in G' \quad x = x^{at + br} = x^{at} x^{br}\]
                        \[(x^{at})^r = (x^{tr})^a = e^a = e \Rightarrow x^{at} \in H\]
                        \[(x^{br})^t = (x^{tr})^b = e^b = e \Rightarrow x^{br} \in K\]
                        Итого \(\forall x \in G' \ \ x = \underbrace{x^{at}}_{\in H} \underbrace{x^{br}}_{\in K} \Rightarrow G' = HK\)
                  \item \(\sphericalangle g \in H \cap K\)

                        \(g^{t} = e = g^{r}\), следовательно, порядок \(g\) есть \(\gcd(t, r) = 1 \Rightarrow H \cap K = \{e\}\)
                  \item Очевидно, т.к. \(G'\) нормальна.
              \end{enumerate}
    \end{itemize}

    Таким образом, мы доказали, что \(G\) раскладывается на прямое произведение силовских \(p\)--подгрупп. Однако они необязательно циклические. Покажем, что каждая из таких подгрупп есть прямое произведение циклических.

    \begin{statement}
        \(\sphericalangle \mathcal{P}_i, |\mathcal{P}_i| = p_i^{q_i}\).  GG to GРассмотрим произвольный элемент максимального порядка \(g\). Тогда \(\mathcal{P}_i \cong \{g\} \times K\), где \(K\) --- подгруппа \(G\).
    \end{statement}
    \begin{proof}
        Докажем по индукции по \(q_i\).
        \begin{itemize}
            \item [\textbf{База.}] \(\mathcal{P}_i = \ev{g} \cong \ev{g} \times \ev{e}\)
            \item [\textbf{Переход.}] Пусть \(g\) --- элемент максимального порядка в \(\mathcal{P}_i\) и этот порядок равен \(a\). Рассмотрим какую-нибудь подгруппу \(H\), не содержащую \(a\). Такую подгруппу можно получить как \(\ev{h}\), где \(h \notin \ev{a}, h \neq e\). Фактор--группа \(\mathcal{P}_i / H\) имеет порядок меньше \(\mathcal{P}_i\), следовательно, для неё выполняется утверждение и \(\mathcal{P}_i / H = \ev{g'} \times K'\).

                  С помощью прообразов естественного гомоморфизма фактор--группы искомое выполнено, технические детали здесь опущены.
        \end{itemize}
    \end{proof}

    Применяя это утверждение к \(K\) рекурсивно, получим искомое:
    \[\mathcal{P}_i \cong \ev{g_1^{(i)}} \times K \cong \ev{g_1^{(i)}} \times \left(\ev{g_2^{(i)}} \times K'\right) \cong \ev{g_1^{(i)}} \times \ev{g_2^{(i)}} \times \dots \times \ev{g_{r_i}^{(i)}}\]
    Очевидно, что \(\ev{g_j^{(i)}}\) есть циклическая группа и тогда:
    \[G \cong \bigtimes_{i = 1}^k \mathcal{P}_i \cong \bigtimes_{i = 1}^k \bigtimes_{j = 1}^{r_i} \ev{g_j^{(i)}}\]
\end{solution}

\begin{exercise}
    Рассмотрим аффинные преобразования плоскости. Пусть \(T\) --- множество всех трансляций, пусть \(R\) --- множество всех поворотов вокруг фиксированной точки \(O\) (одной для всех поворотов). Рассмотрим группу \(G = \ev{T \cup R}\), порождённую всеми трансляциями и поворотами вокруг \(O\). Показать, что \(T\) нормальна в \(G\). Показать, что \(G = T \cdot R\):
    \[G = \{\tau\ \rho \mid \tau \in T, \rho \in R\}\]
\end{exercise}
\begin{solution}
    Очевидно \(T\) и \(R\) замкнуты.

    \begin{notation}
        \(\tau \in T \leftrightarrow \ev{x', y'}\) --- сдвиг на \(x'\) по оси \(x\) и на \(y'\) по оси \(y\).
    \end{notation}

    Рассмотрим действие \(\tau\ \rho\):
    \[\begin{pmatrix}
            x & y
        \end{pmatrix} \ev{x', y'} \begin{pmatrix}
            \cos \theta & - \sin \theta \\
            \sin \theta & \cos \theta
        \end{pmatrix} = \begin{pmatrix}
            x + x' & y + y'
        \end{pmatrix} \begin{pmatrix}
            \cos \theta & - \sin \theta \\
            \sin \theta & \cos \theta
        \end{pmatrix}\]
    \[= \begin{pmatrix}
            (x + x') \cos \theta + (y + y') \sin \theta &
            - (x + x') \sin \theta + (y + y') \cos \theta
        \end{pmatrix}\]
    Рассмотрим действие \(\rho\ \tau\): после поворота на \(\theta\) \(\ev{x', y'}\) заменится на
    \[\ev{x' \cos \theta + y' \sin \theta, - x' \sin \theta + y' \cos \theta}\]
    и после сложения с повернутым вектором \(\begin{pmatrix}
        x \\ y
    \end{pmatrix}\) мы получим тот же самый вектор, что и при \(\tau\ \rho\). Таким образом, множество \(T \cup R\) абелево\footnote{Абелево как группа, но ещё не доказано, что это группа.}.

    Поворот на \(\theta\), сдвиг на \(\ev{x, y}\) и поворот на \(\theta'\) есть то же самое, что сдвиг на повернутое на \(\theta\) \(\ev{x, y}\) и поворот на \(\theta + \theta'\).

    Сдвиг на \(\ev{x, y}\), поворот на \(\theta\) и сдвиг на \(\ev{x', y'}\) есть то же самое, что сдвиг на повернутое на \(\ev{x, y} +\) повернутое на \(\theta\) \(\ev{x', y'}\) и поворот на \(\theta\).

    Итого, \(\ev{T \cup R} = T \cdot R \cup R \cdot T\), т.к. было показано, что нельзя поворотами и сдвигами получить что-либо кроме поворотов и сдвигов, а также тождественное действие \(e \in T, e \in R \Rightarrow T = T \cdot e \Rightarrow T \subset T \cdot R\) и аналогично \(R \subset T \cdot R\). Т.к. множество \(T \cup R\) абелево, то \(T \cdot R = R \cdot T \Rightarrow \ev{T \cup R} = T \cdot R\).

    \(\sphericalangle \rho \tau \rho^{-1}\)

    \begin{remark}
        \(\ev{\begin{array}{c}
                x' \\
                y'
            \end{array}} \Leftrightarrow \ev{x', y'}\), используется, чтобы формулы не были слишком широкими.
    \end{remark}
    \begin{align*}
        \ev{\begin{array}{c}
                x' \cos \theta + y' \sin \theta \\
                - x' \sin \theta + y' \cos \theta
            \end{array}} \begin{pmatrix}
            \cos \theta & - \sin \theta \\
            \sin \theta & \cos \theta
        \end{pmatrix}^{ - 1}
         & = \ev{\begin{array}{c}
                x' \cos \theta + y' \sin \theta \\
                - x' \sin \theta + y' \cos \theta
            \end{array}} \begin{pmatrix}
            \cos \theta  & \sin \theta \\
            -\sin \theta & \cos \theta
        \end{pmatrix} \\
         & = \ev{\begin{array}{c}
                (x' \cos \theta + y' \sin \theta) \cos \theta - (- x' \sin \theta + y' \cos \theta) \sin \theta \\
                (x' \cos \theta + y' \sin \theta) \sin \theta + (- x' \sin \theta + y' \cos \theta) \cos \theta
            \end{array}}                            \\
         & = \ev{\begin{array}{c}
                x' + y' \sin \theta \cos \theta - y' \cos \theta \sin \theta \\
                y' + x' \cos \theta \sin \theta - x' \sin \theta \cos \theta
            \end{array}}                            \\
         & = \ev{\begin{array}{c}
                x' \\
                y'
            \end{array}}                            \\
    \end{align*}

    Таким образом, \(T\) нормально в \(G\), т.к. случай \(\tau'\tau\tau'^{ - 1} \in T\) тривиален.
\end{solution}

\end{document}
