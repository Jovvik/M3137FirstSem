\documentclass[12pt, a4paper]{article}

%<*preamble>
% Math symbols
\usepackage{amsmath, amsthm, amsfonts, amssymb}
\usepackage{accents}
\usepackage{esvect}
\usepackage{mathrsfs}
\usepackage{mathtools}
\mathtoolsset{showonlyrefs}
\usepackage{cmll}
\usepackage{stmaryrd}
\usepackage{physics}
\usepackage[normalem]{ulem}
\usepackage{ebproof}
\usepackage{extarrows}

% Page layout
\usepackage{geometry, a4wide, parskip, fancyhdr}

% Font, encoding, russian support
\usepackage[russian]{babel}
\usepackage[sb]{libertine}
\usepackage{xltxtra}

% Listings
\usepackage{listings}
\lstset{basicstyle=\ttfamily,breaklines=true}
\setmonofont{Inconsolata}

% Miscellaneous
\usepackage{array}
\usepackage{calc}
\usepackage{caption}
\usepackage{subcaption}
\captionsetup{justification=centering,margin=2cm}
\usepackage{catchfilebetweentags}
\usepackage{enumitem}
\usepackage{etoolbox}
\usepackage{float}
\usepackage{lastpage}
\usepackage{minted}
\usepackage{svg}
\usepackage{wrapfig}
\usepackage{xcolor}
\usepackage[makeroom]{cancel}

\newcolumntype{L}{>{$}l<{$}}
    \newcolumntype{C}{>{$}c<{$}}
\newcolumntype{R}{>{$}r<{$}}

% Footnotes
\usepackage[hang]{footmisc}
\setlength{\footnotemargin}{2mm}
\makeatletter
\def\blfootnote{\gdef\@thefnmark{}\@footnotetext}
\makeatother

% References
\usepackage{hyperref}
\hypersetup{
    colorlinks,
    linkcolor={blue!80!black},
    citecolor={blue!80!black},
    urlcolor={blue!80!black},
}

% tikz
\usepackage{tikz}
\usepackage{tikz-cd}
\usetikzlibrary{arrows.meta}
\usetikzlibrary{decorations.pathmorphing}
\usetikzlibrary{calc}
\usetikzlibrary{patterns}
\usepackage{pgfplots}
\pgfplotsset{width=10cm,compat=1.9}
\newcommand\irregularcircle[2]{% radius, irregularity
    \pgfextra {\pgfmathsetmacro\len{(#1)+rand*(#2)}}
    +(0:\len pt)
    \foreach \a in {10,20,...,350}{
            \pgfextra {\pgfmathsetmacro\len{(#1)+rand*(#2)}}
            -- +(\a:\len pt)
        } -- cycle
}

\providetoggle{useproofs}
\settoggle{useproofs}{false}

\pagestyle{fancy}
\lfoot{M3137y2019}
\cfoot{}
\rhead{стр. \thepage\ из \pageref*{LastPage}}

\newcommand{\R}{\mathbb{R}}
\newcommand{\Q}{\mathbb{Q}}
\newcommand{\Z}{\mathbb{Z}}
\newcommand{\B}{\mathbb{B}}
\newcommand{\N}{\mathbb{N}}
\renewcommand{\Re}{\mathfrak{R}}
\renewcommand{\Im}{\mathfrak{I}}

\newcommand{\const}{\text{const}}
\newcommand{\cond}{\text{cond}}

\newcommand{\teormin}{\textcolor{red}{!}\ }

\DeclareMathOperator*{\xor}{\oplus}
\DeclareMathOperator*{\equ}{\sim}
\DeclareMathOperator{\sign}{\text{sign}}
\DeclareMathOperator{\Sym}{\text{Sym}}
\DeclareMathOperator{\Asym}{\text{Asym}}

\DeclarePairedDelimiter{\ceil}{\lceil}{\rceil}

% godel
\newbox\gnBoxA
\newdimen\gnCornerHgt
\setbox\gnBoxA=\hbox{$\ulcorner$}
\global\gnCornerHgt=\ht\gnBoxA
\newdimen\gnArgHgt
\def\godel #1{%
    \setbox\gnBoxA=\hbox{$#1$}%
    \gnArgHgt=\ht\gnBoxA%
    \ifnum     \gnArgHgt<\gnCornerHgt \gnArgHgt=0pt%
    \else \advance \gnArgHgt by -\gnCornerHgt%
    \fi \raise\gnArgHgt\hbox{$\ulcorner$} \box\gnBoxA %
    \raise\gnArgHgt\hbox{$\urcorner$}}

% \theoremstyle{plain}

\theoremstyle{definition}
\newtheorem{theorem}{Теорема}
\newtheorem*{definition}{Определение}
\newtheorem{axiom}{Аксиома}
\newtheorem*{axiom*}{Аксиома}
\newtheorem{lemma}{Лемма}

\theoremstyle{remark}
\newtheorem*{remark}{Примечание}
\newtheorem*{exercise}{Упражнение}
\newtheorem{corollary}{Следствие}[theorem]
\newtheorem*{statement}{Утверждение}
\newtheorem*{corollary*}{Следствие}
\newtheorem*{example}{Пример}
\newtheorem{observation}{Наблюдение}
\newtheorem*{prop}{Свойства}
\newtheorem*{obozn}{Обозначение}

% subtheorem
\makeatletter
\newenvironment{subtheorem}[1]{%
    \def\subtheoremcounter{#1}%
    \refstepcounter{#1}%
    \protected@edef\theparentnumber{\csname the#1\endcsname}%
    \setcounter{parentnumber}{\value{#1}}%
    \setcounter{#1}{0}%
    \expandafter\def\csname the#1\endcsname{\theparentnumber.\Alph{#1}}%
    \ignorespaces
}{%
    \setcounter{\subtheoremcounter}{\value{parentnumber}}%
    \ignorespacesafterend
}
\makeatother
\newcounter{parentnumber}

\newtheorem{manualtheoreminner}{Теорема}
\newenvironment{manualtheorem}[1]{%
    \renewcommand\themanualtheoreminner{#1}%
    \manualtheoreminner
}{\endmanualtheoreminner}

\newcommand{\dbltilde}[1]{\accentset{\approx}{#1}}
\newcommand{\intt}{\int\!}

% magical thing that fixes paragraphs
\makeatletter
\patchcmd{\CatchFBT@Fin@l}{\endlinechar\m@ne}{}
{}{\typeout{Unsuccessful patch!}}
\makeatother

\newcommand{\get}[2]{
    \ExecuteMetaData[#1]{#2}
}

\newcommand{\getproof}[2]{
    \iftoggle{useproofs}{\ExecuteMetaData[#1]{#2proof}}{}
}

\newcommand{\getwithproof}[2]{
    \get{#1}{#2}
    \getproof{#1}{#2}
}

\newcommand{\import}[3]{
    \subsection{#1}
    \getwithproof{#2}{#3}
}

\newcommand{\given}[1]{
    Дано выше. (\ref{#1}, стр. \pageref{#1})
}

\renewcommand{\ker}{\text{Ker }}
\newcommand{\im}{\text{Im }}
\renewcommand{\grad}{\text{grad}}
\newcommand{\rg}{\text{rg}}
\newcommand{\defeq}{\stackrel{\text{def}}{=}}
\newcommand{\defeqfor}[1]{\stackrel{\text{def } #1}{=}}
\newcommand{\itemfix}{\leavevmode\makeatletter\makeatother}
\newcommand{\?}{\textcolor{red}{???}}
\renewcommand{\emptyset}{\varnothing}
\newcommand{\longarrow}[1]{\xRightarrow[#1]{\qquad}}
\DeclareMathOperator*{\esup}{\text{ess sup}}
\newcommand\smallO{
    \mathchoice
    {{\scriptstyle\mathcal{O}}}% \displaystyle
    {{\scriptstyle\mathcal{O}}}% \textstyle
    {{\scriptscriptstyle\mathcal{O}}}% \scriptstyle
    {\scalebox{.6}{$\scriptscriptstyle\mathcal{O}$}}%\scriptscriptstyle
}
\renewcommand{\div}{\text{div}\ }
\newcommand{\rot}{\text{rot}\ }
\newcommand{\cov}{\text{cov}}

\makeatletter
\newcommand{\oplabel}[1]{\refstepcounter{equation}(\theequation\ltx@label{#1})}
\makeatother

\newcommand{\symref}[2]{\stackrel{\oplabel{#1}}{#2}}
\newcommand{\symrefeq}[1]{\symref{#1}{=}}

% xrightrightarrows
\makeatletter
\newcommand*{\relrelbarsep}{.386ex}
\newcommand*{\relrelbar}{%
    \mathrel{%
        \mathpalette\@relrelbar\relrelbarsep
    }%
}
\newcommand*{\@relrelbar}[2]{%
    \raise#2\hbox to 0pt{$\m@th#1\relbar$\hss}%
    \lower#2\hbox{$\m@th#1\relbar$}%
}
\providecommand*{\rightrightarrowsfill@}{%
    \arrowfill@\relrelbar\relrelbar\rightrightarrows
}
\providecommand*{\leftleftarrowsfill@}{%
    \arrowfill@\leftleftarrows\relrelbar\relrelbar
}
\providecommand*{\xrightrightarrows}[2][]{%
    \ext@arrow 0359\rightrightarrowsfill@{#1}{#2}%
}
\providecommand*{\xleftleftarrows}[2][]{%
    \ext@arrow 3095\leftleftarrowsfill@{#1}{#2}%
}

\allowdisplaybreaks

\newcommand{\unfinished}{\textcolor{red}{Не дописано}}

% Reproducible pdf builds 
\special{pdf:trailerid [
<00112233445566778899aabbccddeeff>
<00112233445566778899aabbccddeeff>
]}
%</preamble>


\lhead{Алгоритмы в математике \textit{(практика)}}
\lfoot{Михайлов Максим}
\cfoot{}
\rfoot{11.12.2021}

\begin{document}

\begin{exercise}
    Пусть \(d = \sqrt[3]{2}\). Рассмотрим кольцо порожденное элементами \(1, d\). Показать, что данное кольцо представимо в виде
    \[R = \{a + bd + cd^2 \mid a, b, c \in \Z\}.\]
    Разрешимо ли в рамках кольца \(R\) уравнение:
    \[(1 - 3d + 5d^2)x =- 18 + 10d + 20d^2\]
    Присутствуют ли в данном кольце делители нуля?
\end{exercise}
\begin{solution}
    Очевидно, что \(\ev{1} = \Z, \ev{d} = d\Z\). В \(\ev{1, d}\) лежит их сумма, т.е. \(\Z + d\Z\). \(\Z \cdot (d\Z) = d\Z\), что не добавляет новых элементов. \((d\Z) \cdot (d\Z) = d^2 \Z\) и тогда промежуточный результат это \(\Z + d\Z + d^2\Z\). \(\Z \cdot d^2\Z = d^2\Z, d\Z \cdot d^2\Z = \Z, d^2\Z \cdot d^2\Z = d\Z\), поэтому больше нечего добавлять.

    \begin{align*}
        (1 - 3d + 5d^2)(a + bd + cd^2)                        & = - 18 + 10d + 20d^2 \\
        a + bd + cd^2 - 3ad - 3bd^2 - 6c + 5ad^2 + 10b + 10cd & = - 18 + 10d + 20d^2
    \end{align*}
    \[\begin{cases}
            a - 6c + 10b = - 18 \\
            b - 3a + 10c = 10   \\
            c - 3b + 5a = 20
        \end{cases}\]
    Система не вырождена, решение есть.

    Делители нуля:
    \begin{align*}
        xy                                                                                            & = 0 \\
        (a_1 + b_1d + c_1d^2)(a_2 + b_2d + c_2d^2)                                                    & = 0 \\
        a_1a_2 + a_1b_2d + a_1c_2d^2 + b_1a_2d + b_1b_2d^2 + 2b_1c_2 + c_1a_2d^2 + 2c_1b_2 + 2c_1c_2d & = 0 \\
    \end{align*}
    \[\begin{cases}
            a_1a_2 + 2b_1c_2 + 2c_1b_2 = 0 \\
            a_1b_2 + b_1a_2 + 2c_1c_2 = 0  \\
            a_1c_2 + b_1b_2 + c_1a_2 = 0
        \end{cases}\]
    Что делать с этой системой нелинейных диофантовых уравнений не очень понятно.

    % Сведение этой системы нелинейных диофантовых уравнений к 2-SAT показывает, что решений (ненулевых) нет.
\end{solution}

\begin{exercise}
    Рассмотрим кольцо многочленов \(R = \R[x]\) и множество:
    \[J = \{p \mid p \divided x^2 + 1\}\]
    Показать, что \(J\) есть идеал. Построить \(\faktor{R}{J}\). Существуют ли в \(\faktor{R}{J}\) делители нуля?
\end{exercise}
\begin{solution}
    То, что \(J\) является подкольцом, очевидно.

    \(\sphericalangle a \in R, p \cdot (x^2 + 1) \in J\).
    \[a \cdot p \cdot (x^2 + 1) \in J\]
    Таким образом, \(J\) --- идеал.

    В каждом классе из \(\faktor{R}{J}\) есть ровно один элемент вида \(ax + b\), потому что если коэффициент при \(x^{n + 2}\) ненулевой и \(n \geq 0\), то такой многочлен можно представить как \((x^2 + 1) \cdot x^n \cdot a + p\) и тогда любой многочлен лежит в \([ax + b]\) для каких-то \(a\) и \(b\).

    Заметим, что в \(\faktor{R}{J}\) \([x^2 + 1] = [0]\), следовательно, \([x^2] = [- 1]\). Итого \(\faktor{R}{J} = \{ax + b \mid a, b \in \Z\}\) со стандартным сложением и умножением таким, что \(x^2 = - 1\). Несложно также заметить, что \(\faktor{R}{J} \cong \mathbb{C}\) по гомоморфизму \([ax + b] \mapsto b + ia\).

    В \(\mathbb{C}\) нет делителей нуля, так что и в \(\faktor{R}{J}\) их нет.
\end{solution}

\begin{exercise}
    Вычислить
    \begin{enumerate}
        \item \(\varphi(360)\)
        \item \(\varphi(125)\)
        \item \(\varphi(\varphi(12))\)
    \end{enumerate}
\end{exercise}
\begin{solution}\itemfix
    \begin{enumerate}
        \item \(\varphi(360) = \varphi(8) \cdot \varphi(9) \cdot \varphi(5) = 4 \cdot (2 - 1) \cdot 3 \cdot (3 - 1) \cdot 4 = 96\)
        \item \(\varphi(125) = \varphi(5^3) = 5^2 \cdot (5 - 1) = 100\)
        \item \(\varphi(\varphi(12)) = \varphi(|\{1,5,7,11\}|) = \varphi(4) = |\{1,3\}| = 2\)
    \end{enumerate}
\end{solution}

\begin{exercise}
    Пусть \(a, n \in \Z\) два взаимно простых числа \((a, n) = 1\). Показать, что:
    \[a^{\varphi(n)} \equiv 1 \pmod n\]
\end{exercise}
\begin{solution}
    Рассмотрим мультипликативную группу \(A\) взаимно простых с \(n\) чисел по модулю \(n\). Очевидно это действительно группа. \(|A| = \varphi(n)\). По теореме Лагранжа \(|A| \divided |\ev{a}|\), т.е. \(|\ev{a}| \cdot k = \varphi(n)\).
    \[a^{\varphi(n)} = a^{|\ev{a}| \cdot k}\]
    Из структуры \(A\) понятно, что \(\ev{a}\) это простой цикл и тогда \(a^{|\ev{a}|} \equiv 1 \pmod n\).
    \[a^{\varphi(n)} = 1 \pmod n\]
\end{solution}

\end{document}
