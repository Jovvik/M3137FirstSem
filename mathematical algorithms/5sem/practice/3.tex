\documentclass[12pt, a4paper]{article}

%<*preamble>
% Math symbols
\usepackage{amsmath, amsthm, amsfonts, amssymb}
\usepackage{accents}
\usepackage{esvect}
\usepackage{mathrsfs}
\usepackage{mathtools}
\mathtoolsset{showonlyrefs}
\usepackage{cmll}
\usepackage{stmaryrd}
\usepackage{physics}
\usepackage[normalem]{ulem}
\usepackage{ebproof}
\usepackage{extarrows}

% Page layout
\usepackage{geometry, a4wide, parskip, fancyhdr}

% Font, encoding, russian support
\usepackage[russian]{babel}
\usepackage[sb]{libertine}
\usepackage{xltxtra}

% Listings
\usepackage{listings}
\lstset{basicstyle=\ttfamily,breaklines=true}
\setmonofont{Inconsolata}

% Miscellaneous
\usepackage{array}
\usepackage{calc}
\usepackage{caption}
\usepackage{subcaption}
\captionsetup{justification=centering,margin=2cm}
\usepackage{catchfilebetweentags}
\usepackage{enumitem}
\usepackage{etoolbox}
\usepackage{float}
\usepackage{lastpage}
\usepackage{minted}
\usepackage{svg}
\usepackage{wrapfig}
\usepackage{xcolor}
\usepackage[makeroom]{cancel}

\newcolumntype{L}{>{$}l<{$}}
    \newcolumntype{C}{>{$}c<{$}}
\newcolumntype{R}{>{$}r<{$}}

% Footnotes
\usepackage[hang]{footmisc}
\setlength{\footnotemargin}{2mm}
\makeatletter
\def\blfootnote{\gdef\@thefnmark{}\@footnotetext}
\makeatother

% References
\usepackage{hyperref}
\hypersetup{
    colorlinks,
    linkcolor={blue!80!black},
    citecolor={blue!80!black},
    urlcolor={blue!80!black},
}

% tikz
\usepackage{tikz}
\usepackage{tikz-cd}
\usetikzlibrary{arrows.meta}
\usetikzlibrary{decorations.pathmorphing}
\usetikzlibrary{calc}
\usetikzlibrary{patterns}
\usepackage{pgfplots}
\pgfplotsset{width=10cm,compat=1.9}
\newcommand\irregularcircle[2]{% radius, irregularity
    \pgfextra {\pgfmathsetmacro\len{(#1)+rand*(#2)}}
    +(0:\len pt)
    \foreach \a in {10,20,...,350}{
            \pgfextra {\pgfmathsetmacro\len{(#1)+rand*(#2)}}
            -- +(\a:\len pt)
        } -- cycle
}

\providetoggle{useproofs}
\settoggle{useproofs}{false}

\pagestyle{fancy}
\lfoot{M3137y2019}
\cfoot{}
\rhead{стр. \thepage\ из \pageref*{LastPage}}

\newcommand{\R}{\mathbb{R}}
\newcommand{\Q}{\mathbb{Q}}
\newcommand{\Z}{\mathbb{Z}}
\newcommand{\B}{\mathbb{B}}
\newcommand{\N}{\mathbb{N}}
\renewcommand{\Re}{\mathfrak{R}}
\renewcommand{\Im}{\mathfrak{I}}

\newcommand{\const}{\text{const}}
\newcommand{\cond}{\text{cond}}

\newcommand{\teormin}{\textcolor{red}{!}\ }

\DeclareMathOperator*{\xor}{\oplus}
\DeclareMathOperator*{\equ}{\sim}
\DeclareMathOperator{\sign}{\text{sign}}
\DeclareMathOperator{\Sym}{\text{Sym}}
\DeclareMathOperator{\Asym}{\text{Asym}}

\DeclarePairedDelimiter{\ceil}{\lceil}{\rceil}

% godel
\newbox\gnBoxA
\newdimen\gnCornerHgt
\setbox\gnBoxA=\hbox{$\ulcorner$}
\global\gnCornerHgt=\ht\gnBoxA
\newdimen\gnArgHgt
\def\godel #1{%
    \setbox\gnBoxA=\hbox{$#1$}%
    \gnArgHgt=\ht\gnBoxA%
    \ifnum     \gnArgHgt<\gnCornerHgt \gnArgHgt=0pt%
    \else \advance \gnArgHgt by -\gnCornerHgt%
    \fi \raise\gnArgHgt\hbox{$\ulcorner$} \box\gnBoxA %
    \raise\gnArgHgt\hbox{$\urcorner$}}

% \theoremstyle{plain}

\theoremstyle{definition}
\newtheorem{theorem}{Теорема}
\newtheorem*{definition}{Определение}
\newtheorem{axiom}{Аксиома}
\newtheorem*{axiom*}{Аксиома}
\newtheorem{lemma}{Лемма}

\theoremstyle{remark}
\newtheorem*{remark}{Примечание}
\newtheorem*{exercise}{Упражнение}
\newtheorem{corollary}{Следствие}[theorem]
\newtheorem*{statement}{Утверждение}
\newtheorem*{corollary*}{Следствие}
\newtheorem*{example}{Пример}
\newtheorem{observation}{Наблюдение}
\newtheorem*{prop}{Свойства}
\newtheorem*{obozn}{Обозначение}

% subtheorem
\makeatletter
\newenvironment{subtheorem}[1]{%
    \def\subtheoremcounter{#1}%
    \refstepcounter{#1}%
    \protected@edef\theparentnumber{\csname the#1\endcsname}%
    \setcounter{parentnumber}{\value{#1}}%
    \setcounter{#1}{0}%
    \expandafter\def\csname the#1\endcsname{\theparentnumber.\Alph{#1}}%
    \ignorespaces
}{%
    \setcounter{\subtheoremcounter}{\value{parentnumber}}%
    \ignorespacesafterend
}
\makeatother
\newcounter{parentnumber}

\newtheorem{manualtheoreminner}{Теорема}
\newenvironment{manualtheorem}[1]{%
    \renewcommand\themanualtheoreminner{#1}%
    \manualtheoreminner
}{\endmanualtheoreminner}

\newcommand{\dbltilde}[1]{\accentset{\approx}{#1}}
\newcommand{\intt}{\int\!}

% magical thing that fixes paragraphs
\makeatletter
\patchcmd{\CatchFBT@Fin@l}{\endlinechar\m@ne}{}
{}{\typeout{Unsuccessful patch!}}
\makeatother

\newcommand{\get}[2]{
    \ExecuteMetaData[#1]{#2}
}

\newcommand{\getproof}[2]{
    \iftoggle{useproofs}{\ExecuteMetaData[#1]{#2proof}}{}
}

\newcommand{\getwithproof}[2]{
    \get{#1}{#2}
    \getproof{#1}{#2}
}

\newcommand{\import}[3]{
    \subsection{#1}
    \getwithproof{#2}{#3}
}

\newcommand{\given}[1]{
    Дано выше. (\ref{#1}, стр. \pageref{#1})
}

\renewcommand{\ker}{\text{Ker }}
\newcommand{\im}{\text{Im }}
\renewcommand{\grad}{\text{grad}}
\newcommand{\rg}{\text{rg}}
\newcommand{\defeq}{\stackrel{\text{def}}{=}}
\newcommand{\defeqfor}[1]{\stackrel{\text{def } #1}{=}}
\newcommand{\itemfix}{\leavevmode\makeatletter\makeatother}
\newcommand{\?}{\textcolor{red}{???}}
\renewcommand{\emptyset}{\varnothing}
\newcommand{\longarrow}[1]{\xRightarrow[#1]{\qquad}}
\DeclareMathOperator*{\esup}{\text{ess sup}}
\newcommand\smallO{
    \mathchoice
    {{\scriptstyle\mathcal{O}}}% \displaystyle
    {{\scriptstyle\mathcal{O}}}% \textstyle
    {{\scriptscriptstyle\mathcal{O}}}% \scriptstyle
    {\scalebox{.6}{$\scriptscriptstyle\mathcal{O}$}}%\scriptscriptstyle
}
\renewcommand{\div}{\text{div}\ }
\newcommand{\rot}{\text{rot}\ }
\newcommand{\cov}{\text{cov}}

\makeatletter
\newcommand{\oplabel}[1]{\refstepcounter{equation}(\theequation\ltx@label{#1})}
\makeatother

\newcommand{\symref}[2]{\stackrel{\oplabel{#1}}{#2}}
\newcommand{\symrefeq}[1]{\symref{#1}{=}}

% xrightrightarrows
\makeatletter
\newcommand*{\relrelbarsep}{.386ex}
\newcommand*{\relrelbar}{%
    \mathrel{%
        \mathpalette\@relrelbar\relrelbarsep
    }%
}
\newcommand*{\@relrelbar}[2]{%
    \raise#2\hbox to 0pt{$\m@th#1\relbar$\hss}%
    \lower#2\hbox{$\m@th#1\relbar$}%
}
\providecommand*{\rightrightarrowsfill@}{%
    \arrowfill@\relrelbar\relrelbar\rightrightarrows
}
\providecommand*{\leftleftarrowsfill@}{%
    \arrowfill@\leftleftarrows\relrelbar\relrelbar
}
\providecommand*{\xrightrightarrows}[2][]{%
    \ext@arrow 0359\rightrightarrowsfill@{#1}{#2}%
}
\providecommand*{\xleftleftarrows}[2][]{%
    \ext@arrow 3095\leftleftarrowsfill@{#1}{#2}%
}

\allowdisplaybreaks

\newcommand{\unfinished}{\textcolor{red}{Не дописано}}

% Reproducible pdf builds 
\special{pdf:trailerid [
<00112233445566778899aabbccddeeff>
<00112233445566778899aabbccddeeff>
]}
%</preamble>


\lhead{Алгоритмы в математике \textit{(практика)}}
\cfoot{}
\rfoot{25.9.2021}

\begin{document}

\section*{Теория делимости}

Будем рассматривать \(\Z\).

\begin{definition}
    \(q \in \Z\) \textbf{делит} \(n \in \Z\), если \(\exists t \in \Z : n = qt\)
\end{definition}
\begin{obozn}
    \(q \mid n, n \divided q\).
\end{obozn}

\begin{example}
    \(m^5 - m \divided 5\)
\end{example}
\begin{solution}
    \begin{caseof}
        \case{\(m = 5k\)}{Тривиально.}
        \case{\(m = 5k + 1\)}{
            \begin{align*}
                (5k + 1)^5 - (5k + 1) & = (5k + 1)((5k + 1)^2 - 1)((5k + 1)^2 + 1)                    \\
                                      & = (5k + 1)(5k + 1 - 1)(5k + 1 + 1)((5k + 1)^2 + 1) \divided 5 \\
            \end{align*}
        }
        \case{\(m = 5k + 2\)}{
            \begin{align*}
                (5k + 2)^5 - (5k + 2) & = (5k + 2)(5k + 2 - 1)(5k + 3)((5k + 2)^2 + 1) \\
                                      & = \dots (25k + 20k + 4 + 1) \divided 5         \\
            \end{align*}
        }
    \end{caseof}
    Остальные случаи опущены.
\end{solution}
\begin{definition}
    \(n, m \in Z, d : n \divided d, m \divided d\)

    \(d\) называется \textbf{общим делителем} \(n, m\).
\end{definition}
\begin{definition}
    \(n, m\) \textbf{взаимно простые}, если:
    \[n \divided d, m \divided d \Rightarrow d = \pm 1\]
\end{definition}
\begin{theorem}
    \(n \divided ab\) \(\Leftrightarrow n \divided a, n \divided b\) и \(a, b\) взаимно простые.
\end{theorem}

\begin{exercise*}
    \[m(m + 1)(2m + 1) \divided 6\]
\end{exercise*}
\begin{solution}
    \[m(m + 1) \divided 2\]
    Докажем \(m(m + 1)(2m + 1) \divided 3\)
    \begin{caseof}
        \case{\(m = 3k\)}{Тривиально.}
        \case{\(m = 3k + 1\)}{
            \(2m + 1 = 6k + 3 \divided 3\)
        }
        \case{\(m = 3k + 2\)}{Тривиально.}
    \end{caseof}
\end{solution}

\begin{exercise*}
    \(\forall n \ \ \exists k : n^2 + (n + 1)^2 = 4k + 1\)
\end{exercise*}
\begin{solution}
    \begin{align*}
        n^2 + (n + 1)^2                    & = 4k + 1 \\
        2n^2 + 2n + 1                      & = 4k + 1 \\
        2n^2 + 2n                          & = 4k     \\
        n^2 + n                            & = 2k     \\
        \underbrace{n(n + 1)}_{\divided 2} & = 2k     \\
    \end{align*}
\end{solution}

\begin{exercise*}
    \[n^3(n^2 + 3) \divided 4\]
\end{exercise*}
\begin{solution}
    Для чётных \(n\) \(n^3 \divided 4\). Для \(n = 2k + 1\) \((2k + 1)^2 + 3 = 4k^2 + 4k + 4 \divided 4\).
\end{solution}

\begin{definition}
    \(a, b \in \Z\) \textbf{сравнимы} по модулю \(n\), если \(a - b \divided n\).
\end{definition}
\begin{obozn}
    \(a \equiv b \pmod n\)
\end{obozn}
\begin{example}
    \(4 \equiv 1 \pmod 3, 8 \equiv 2 \pmod 3, 151 \equiv 11 \pmod{10}\)
\end{example}

\[\begin{rcases}
        a - c \divided n \\
        b - d \divided n
    \end{rcases} \Rightarrow \begin{cases}
        a = nk + c \\
        b = nj + d
    \end{cases}\]
\[\sphericalangle ab = \underbrace{n^2 kj + nkd + njc}_{\divided n} + cd\]

\begin{enumerate}
    \item \(a \equiv c \pmod n, b \equiv d \pmod n \Rightarrow a + b \equiv c + d \pmod n, ab \equiv cd \pmod n\)
\end{enumerate}

\begin{exercise*}
    \(a^7 - a + 56 \divided 7\)
\end{exercise*}
\begin{solution}
    \begin{caseof}
        \case{\(a \equiv 0\)}{\(0 + 0 + 56 \equiv 0\)}
        \case{\(a \equiv 1\)}{\(1 - 1 + 56 \equiv 0\)}
        \case{\(a \equiv 2\)}{\(128 - 2 + 56 \equiv 70 + 56 \equiv 0\)}
    \end{caseof}

    Остальные случаи опущены.
\end{solution}

\begin{exercise*}
    \[m^2 + n^2 \divided 7 \Rightarrow n \divided 7, m \divided 7\]
\end{exercise*}

\begin{solution}\itemfix
    \begin{center}
        \begin{tabular}{CC}
            \toprule
            m \equiv & m^2 \equiv \\ \midrule
            0        & 0          \\
            1        & 1          \\
            2        & 4          \\
            3        & 2          \\
            4        & 2          \\
            5        & 4          \\
            6        & 1          \\
            \bottomrule
        \end{tabular}
    \end{center}
\end{solution}

\begin{definition}
    \(\{a_1 \dots a_n\}\) называется \textbf{полной системой вычетов} \(\mod n\), если \(\forall a \in \Z \ \ \exists j : a \equiv a_j \mod n\)
\end{definition}
\begin{theorem}\itemfix
    \begin{itemize}
        \item \(\{a_1 \dots a_n\}\) --- полная система вычетов \(\mod n\)
        \item \(k\) взаимно просто с \(n\)
    \end{itemize}

    Тогда \(\{ka_1 \dots k a_n\}\) --- полная система вычетов \(\mod n\).
\end{theorem}

\end{document}
