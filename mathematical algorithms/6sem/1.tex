\chapter{3 марта}

\section{Алгебраическое тело}

\begin{definition}
    \textbf{Алгебраическое тело} --- множество \(T\) с бинарными операциями \(+\) и \(\cdot\), такими, что:
    \begin{enumerate}
        \item \((T, 0, +)\) --- абелева группа:
        \begin{itemize}
            \item \(\forall \alpha, \beta, \gamma \quad \alpha + (\beta + \gamma) = (\alpha + \beta) + \gamma\)
            \item \(\exists 0 : \alpha + 0 = \alpha = 0 + \alpha\)
            \item \(\forall \alpha \in T \ \ \exists ( - \alpha) : \alpha + ( - \alpha) = 0 = ( - \alpha) + \alpha\)
            \item[\(\star\)] \(\forall \alpha, \beta \in T \quad \alpha + \beta = \beta + \alpha\)
        \end{itemize}
        \item \(((T \setminus \{0\}), 1, *)\) --- группа:
        \begin{itemize}
            \item \(\alpha (\beta \gamma) = (\alpha \beta) \gamma\)
            \item \(\exists 1 : \alpha \cdot 1 = \alpha = 1 \cdot \alpha\)
            \item \(\forall \alpha \neq 0 \ \ \exists \alpha^{-1} : \alpha\alpha^{-1} = 1 = \alpha^{-1}\alpha\)
            \item[\(\star\)] Если \(\alpha\beta\neq\beta\alpha\), то \(T\) --- тело, иначе --- поле.
        \end{itemize}
        \item \(\alpha(\beta + \gamma) = \alpha\beta + \alpha\gamma, (\alpha + \beta)\gamma = \alpha \gamma + \beta\gamma\)
    \end{enumerate}
\end{definition}

\begin{example}
    \(\mathbb{F}_p\) --- поле вычетов по модулю \(p\).
    \[\mathbb{F}_p = \{0, 1, 2 \dots p - 1\}\]
    
    \begin{enumerate}
        \item \(\mathbb{F}_2 = \{0, 1\}\)
        \begin{center}
            \begin{tabular}{C|C|C}
                + & 0 & 1 \\ \hline
                0 & 0 & 1 \\ \hline
                1 & 1 & 0
            \end{tabular}
            \quad
            \begin{tabular}{C|C|C}
                \cdot & 0 & 1 \\ \hline
                0 & 0 & 0 \\ \hline
                1 & 0 & 1
            \end{tabular}
        \end{center}
    \end{enumerate}
\end{example}

\begin{definition}
    \(\mathbb{C} \cong \faktor{K[t]}{(t^2 + 1)K[t]}\)
    \begin{center}
        \begin{tabular}{C|C|C}
            \cdot & 1 & i \\ \hline
            1 & 1 & i \\ \hline
            i & i & -1
        \end{tabular}
    \end{center}
\end{definition}

\begin{theorem}[Фробениуса]
    Дано тело \(T\), такое что \(T \supset \R\). Тогда:
    \begin{enumerate}
        \item Каждый элемент \(\R\) коммутирует с каждым элементом \(T\).
        \item Каждый элемент \(T\) представим как:
        \[x = x_0 + x_1 i_1 + x_2 i_2 + \dots + x_n i_n\]
    \end{enumerate}
\end{theorem}

Из этого следует, что выполнено одно из:
\begin{enumerate}
    \item \(T\) это \(\R\)
    \item \(T\) это \(\mathbb{C}\)
    \item \(T\) это \(\mathbb{K}\)
\end{enumerate}

Если \(i_1, i_2 \dots i_n\) --- базис \(\mathbb{I}\), то \(\dim \mathbb{I} \in \{0, 1, 3\}\)
