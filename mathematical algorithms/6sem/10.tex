\chapter{7 мая}

\section{\(p\)-адические числа}

\begin{definition}
	\textbf{Модуль} --- функция \(|\cdot| : \mathbb{K} \to \R_+\) такая, что:
	\begin{enumerate}
		\item \(|x| = 0 \iff x = \mathbb{0}\)
		\item \(|xy| = |x| |y|\)
		\item \(|x + y| \le |x| + |y|\)
	\end{enumerate}
\end{definition}

\begin{definition}
	Модуль \textbf{неархимедов}, если \(|x + y| \le \max(|x|, |y|)\)
\end{definition}
\begin{remark}
    Из неархимедовости следует третье свойство модуля.
\end{remark}

\begin{definition}
	\textbf{Тривиальный} модуль: \(|x| \coloneqq \begin{cases}
		0, & x = \mathbb{0}    \\
		1, & x \neq \mathbb{0} \\
	\end{cases}\)
\end{definition}

\begin{definition}
	\textbf{\(p\)-адическая оценка\footnote{Modulation}}, где \(p\) --- простое число:
	\[v_p(x) \coloneqq \begin{cases}
			+\infty,         & x = 0                                                                 \\
			n,               & x \in \Z \land x = p^n \cdot \tilde{x} \land \tilde{x} \not\divided p \\
			v_p(m) - v_p(k), & x = \frac{m}{k}
		\end{cases}\]
\end{definition}

\begin{definition}[\(p\)-модуль]
	\(|x|_p \coloneqq p^{-v_p(x)}\)
\end{definition}

\begin{lemma}
	\begin{equation}
		v_p(xy) = v_p(x) + v_p(y) \label{eq:vplm1}
	\end{equation}
	\begin{equation}
		v_p(x + y) \ge \min(v_p(x), v_p(y)) \label{eq:vplm2}
	\end{equation}\[\]
\end{lemma}
\begin{proof}
	Пусть \(x = p^k \tilde{x}, y = p^l \tilde{y}\) и \(\tilde{x}, \tilde{y} \not\divided p\).
	\[xy = p^k \tilde{x} \cdot p^l \tilde{y} = p^{k + l} \tilde{x} \tilde{y}\]
	Пусть \(k > l\).
	\[x + y = p^k \tilde{x} + p^l \tilde{y} = p^l(p^{u - l} \tilde{x} + \tilde{y})\]
\end{proof}

\begin{lemma}
	Определение \(|\cdot|_p\) корректно, т.е. \(|\cdot|_p\) --- модуль.
\end{lemma}
\begin{proof}
	\[|xy|_p \defeq p^{-v_p(xy)} \symrefeq{eq:vplm1} p^{-v_p(x) -v_p(y)} = p^{-v_p(x)} p^{-v_p(y)} \defeq |x|_p |y|_p\]
	\[|x + y|_p \defeq p^{-v_p(x + y)} \symref{eq:vplm2}{\le} p^{-\min(v_p(x), v_p(y))}
     = \max(p^{v_p(x)}, p^{v_p(y)}) \defeq |x|_p |y|_p\]
\end{proof}

Забавный факт: \(\lim_{n \to +\infty} |p^n|_p \to 0\)

\begin{lemma}
    Свойства модуля в произвольном \(\mathbb{K}\):
    \begin{enumerate}
        \item \(|e| = 1\)
        \item \(\exists n : x^n = e \implies |x| = 1\)
        \item \(|-e| = 1\)
        \item \(|-x| = x\)
    \end{enumerate}
\end{lemma}
\begin{proof}\itemfix
    \begin{enumerate}
        \item \(|e| = |e \cdot e| = |e| \cdot |e| = 1\)  
        \item \(1 = |e| = |x^n| = |x|^n\) 
        \item \(|-e \cdot -e| = |e^2| = 1\)
        \item Из предыдущего пункта.
    \end{enumerate}
\end{proof}

\begin{lemma}
    \(x \neq |y| \implies |x + y| = \max(|x|, |y|)\)
\end{lemma}
\begin{proof}
    Пусть \(|x| > |y|\).
    \[|x + y| \le \max(|x|, |y|) = x\] 
    \[|x| = |(x + y) - y| \le \max(|x + y|, |y|) = |x + y|\] 
    \[\begin{rcases}
        |x + y| \le |x| \\
        |x| \le |x + y|
    \end{rcases} \implies |x + y| = |x|\] 
\end{proof}

\begin{lemma}
    \[|x + y| \le \max(|x|, |y|) \iff |z + 1| \le \max(|z|, 1)\] 
\end{lemma}
\begin{proof}\itemfix
    \begin{itemize}
        \item[\(\implies\)] очевидно
        \item[\(\impliedby\)] Рассмотрим случай \(y = \mathbb{0}\). Тогда \(\forall x \ \ |x| \le \max(|x|, 0) = |x|\) очевидно верно.

            Рассмотрим случай \(y \neq \mathbb{0}\). Тогда пусть \(z = \frac{x}{y}\).
            \[\left|\frac{x}{y} + 1\right| \le \max\left(\left|\frac{x}{y}\right|, 1\right)
            \implies |x + y| \le \max(|x|, |y|)\] 
    \end{itemize}
\end{proof}

\begin{statement}
    \(|x| \le 1 \implies |x - 1| \le 1\)
\end{statement}

\begin{definition}
    Метрика, порожденная модулем: \(d(x, y) \coloneqq |x - y|\)
\end{definition}

\begin{lemma}
    Если модуль неархимедов, то \(d(x, y) \le \max(d(x, z), d(z, y))\)
\end{lemma}

\begin{statement}
    Все треугольники в \(\mathbb{K}\) с неархимедовым модулем равнобедренные.
\end{statement}

\begin{example}
    \(x = p^k \tilde{x}, y = p^l \tilde{y}, \tilde{x}, \tilde{y} \ndivided p\)
    Если \(k > l\):
    \[p^k \tilde{x} + p^l \tilde{y} = p^l \underbrace{(p^{k-l} \tilde{x} + \tilde{y})}_{{}\ndivided p}\] 
    Если \(k = l\):
     \[p^k \tilde{x} + p^k \tilde{y} = p^k\underbrace{(\tilde{x} + \tilde{y})}_{\text{может } \divided p}\] 
\end{example}

\begin{example}
    \(p_1 = 5, p_2 = 3\)
    \[|50|_5 = |5^2\cdot 2|_5 = 5^{-2} = \frac{1}{25} \quad |50|_3 = 1\] 
    \[|17|_5 = 5^{-0} = 1 \quad |17|_3 = 1\] 
    \[|15|_5 = 5^{-1} = \frac{1}{5} \quad |15|_3 = 3^{-1} = \frac{1}{3}\] 
    \[\left|\frac{3}{25}\right|_5 = 5^{-(0-2)} = 25 \quad \left|\frac{3}{25}\right|_3 = 3^{-(1-0)} = \frac{1}{3}\] 
\end{example}

\begin{example}
    \(x = \frac{2}{15}, y = \frac{3}{15}, z = \frac{7}{15}\) 
    \[|x-y|_5 = \left|\frac{1}{15}\right|_5 = 5^{-(0-1)} = 5
      \quad |x-y|_3 = \left|\frac{1}{15}\right|_3 = 3\] 
    \[|x-z|_5 = \left|\frac{1}{3}\right|_5 = 1
      \quad |x-z|_3 = \left|\frac{1}{3}\right|_3 = 3\] 
    \[|y-z|_5 = \left|\frac{4}{15}\right|_5 = 5
      \quad |y-z|_3 = \left|\frac{4}{15}\right|_3 = 3\] 
    Мы получили равносторонний треугольник при \(|\cdot|_3\).
\end{example}

\begin{definition}
    \[B \coloneqq \{x \mid d(x, x_0) < r\}\]
    \[\overline{B} \coloneqq \{x \mid d(x, x_0) \le r\}\]
\end{definition}

\begin{lemma}\itemfix
    \begin{enumerate}
        \item \(b \in B(a, r) \implies B(b, r) = B(a, r)\), аналогичное верно для \(\overline{B}\)
        \item \(B(a, r)\) --- открытое и замкнутое. Если \(r \neq 0\), то \(\overline{B}(a, r)\) --- тоже открытое и замкнутое.
        \item \(r > s, B(a, r) \cap B(b, s) \neq \emptyset \implies B(b, s) \subset B(a, r)\)
    \end{enumerate}
\end{lemma}
\begin{proof}\itemfix
    \begin{enumerate}
        \item \(|b-a| < r\).
            \[\forall x \in B(a, r) \quad |x-b| \le \max(|x-a|, |b-a|) < r \implies x \in B(b, r) \implies B(a, r) \subset B(b, r)\] 
        \item Рассмотрим точку вне шара, она принадлежит с некоторой окрестностью дополнению шара, следовательно дополнение открыто, следовательно шар замкнут.
        \item \(B(a, s) = B(c, s) \subset B(c, r) = B(b, r)\) 
    \end{enumerate}
\end{proof}

