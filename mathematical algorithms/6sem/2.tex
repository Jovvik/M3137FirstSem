\chapter{11 марта}

\(\sphericalangle \mathbb{I} = \{z \mid z^2 \in \R, z^2 \leq 0\}\)

\begin{remark}
	\(\R \cap \mathbb{I} = \{0\}\)
\end{remark}

\begin{theorem}
	\(\R \oplus \mathbb{I} = T\)
\end{theorem}

\begin{lemma}
	Если \(z \in \mathbb{I}\), то \(\forall \alpha \in \R \ \ \alpha z \in \mathbb{I}\).
\end{lemma}
\begin{proof}
	\[(\alpha z)^2 = \alpha^2 z^2 \leq 0 \Rightarrow \alpha z \in \mathbb{I}\]
\end{proof}

\begin{lemma}
	Если \(z \in \mathbb{I}\) и \(z^{-1}\) существует, то \(z^{-1} \in \mathbb{I}\), где \(z^{-1}\) это такой элемент \(\mathbb{I}\), что \(zz^{-1} = 1\).
\end{lemma}
\begin{proof}
	\[z^2 (z^{-1})^2 = \underbrace{zz}_{< 0} z^{-1}z^{-1} = 1 \Rightarrow z^{-1}z^{-1} < 0 \Rightarrow z^{-1} \in \mathbb{I}\]
\end{proof}

\begin{lemma}
	Всякий элемент \(x\) из \(T\) представим единственным образом в виде:
	\[x \stackrel{!}{=} a + z, \quad a \in \R, z \in \mathbb{I}\]
\end{lemma}
\begin{proof}
	\(\sphericalangle x \in T, \{x^0, x, x^2 \dots x^{n+1}\}\) --- линейно зависимые, т.к. пространство размерности \(n + 1\), а элементов \(n + 2\). Тогда по определению линейной зависимости \(\exists \{\alpha_i\}_{i=0}^{n+1} \subset \R\), такие что:
	\[\alpha_0 + \alpha_1 x + \alpha_2 x^2 + \dots + \alpha_{n+1} x^{n+1} = 0\]
	Тогда \(x\) является корнем многочлена вида \(x-a=0\) и тогда \(x=a\), либо \(x\) является корнем многочлена вида \(x^2 + 2 \alpha x + \beta = 0\) и тогда \(x\) можно представить в виде \(a + z\).

	Покажем единственность.
	Пусть \(x = a + y\) и \(x = b + z\), где \(a, b \in \R,\ y, z \in \mathbb{I}\).
	\begin{align*}
		a + y - b - z                                                            & = 0                         \\
		a + y - b                                                                & = z                         \\
		\underbrace{(a - b)^2}_{\in \R} + 2 (a - b)y + \underbrace{y^2}_{\in \R} & = \underbrace{z^2}_{\in \R} \\
		2(a-b) y                                                                 & = 0
	\end{align*}
	Таким образом, либо \(a = b\), а следовательно \(y = z\), либо \(y = 0 \implies x \in \R \implies z = 0\)
\end{proof}

\begin{lemma}
	Пусть \(u, v \in \mathbb{I}, a, b \in \R\). Тогда \(uv + vu = \xi \in \R\) и \(au + bv = \eta \in \mathbb{I}\).
\end{lemma}
\begin{proof}
	Положим, что \(\{1, u, v\}\) линейно зависим, т.е. \(\exists \alpha, \beta, \gamma : \alpha + \beta u + \gamma v = 0\).
	\[\beta u = - \alpha - \gamma v \Rightarrow \alpha = 0 \Rightarrow u = - \frac{\gamma}{\beta} v\]
	\[\sphericalangle uv + vu = - \frac{\gamma}{\beta}v^2 - \frac{\gamma}{\beta}v^2 = \frac{-2\gamma}{\beta}v^2 \in \R\]
	\[- \frac{\alpha \gamma}{\beta} v + bv = \left(b - \frac{\alpha\gamma}{\beta}\right)v \in \mathbb{I}\]
	Положим, что \(\{1, u, v\}\) линейно независим.
	\[\eta^2 = (\beta + z)^2 = (au + bv)^2 = a^2u^2 + b^2v^2 + ab(uv + vu)\]
	\[(\beta + z)^2 = a^2u^2 + b^2v^2 + ab(\alpha + y)\]
	\[\beta^2 + 2\beta z + z^2 = a^2u^2 + b^2v^2 + ab(\alpha + y)\]
	\[2\beta z = ab(\alpha + y)\]
	Если \(z = 0\), то \(\{1, u, v\}\) линейно зависим (\(\beta = au + bv\)) --- противоречие.

	\(\sphericalangle z \neq 0, z = \frac{ab}{2\beta} y\)
	\[au + bv = \beta + \frac{ab}{2 \beta}y\]
	\[a'u + b'v = \beta' + \frac{a'b'}{2 \beta'}y\]
    \[(a-a')u + (b-b')v = (\beta - \beta') + \left(\frac{ab}{2 \beta} - \frac{a'b'}{2 \beta'}\right)y\]
    Тогда мы можем выбором \(a\) и \(b\) занулить \(\frac{ab}{2 \beta} - \frac{a'b'}{2 \beta'}\), поэтому \(\{1, u, v\}\) линейно зависимы.
	\unfinished
\end{proof}

\begin{lemma}\itemfix
	\begin{itemize}
		\item \(u, v \in \mathbb{I}\)
		\item \(u^2 = - 1\)
		\item \(v^2 = - 1\)
		\item \(w = u \cdot v\)
	\end{itemize}

	Тогда:
	\begin{align*}
		u^2 & = v^2 = w^2 = - 1 \\
		uv  & = - vu = w        \\
		vw  & = - wv = u        \\
		wu  & = - uw = v
	\end{align*}
\end{lemma}
\begin{proof}
    Дома. 
\end{proof}

