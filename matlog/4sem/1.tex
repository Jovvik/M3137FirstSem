\chapter{12 февраля}

\setcounter{section}{-1}

\section{Мотивация}

\subsection{Математикам}

\begin{axiom*}[Архимеда]
    Для любого \(k > 0\) найдётся \(n\), такое что \(kn > 1\).
\end{axiom*}

Под эту аксиому не подходят бесконечно малые числа и это является проблемой. Например, \(\lim\limits_{x \to +\infty} \frac{1}{x} = 0 = \lim\limits_{x \to +\infty} \frac{1}{x^2}\), но мы хотим уметь различать эти два числа. Ньютон предложил идею бесконечно малых чисел, откуда пошли последовательности. Возникает вопрос --- что такое последовательность и что такое число?

Общепринятое определение целых чисел \(\N\) происходит из теории множеств. Однако эта теория содержит в себе множество фундаментальных парадоксов, от которых нельзя избавиться.

Возникает вопрос --- а что такое множество? Посмотрим на некоторое множество \(A = \{x\ |\ x\notin x\}\). Содержит ли оно себя, \(A\in A\)? На этот вопрос нельзя ответить, это называется парадокс Рассела. Есть простой способ его разрешить --- запретить ставить такой вопрос. Нет вопроса --- нет парадокса. Существование такого парадокса ставит под вопрос существование любого множества --- а существует ли \(\N\)? Может быть его существование парадоксально, просто мы не нашли этот парадокс. Пришло чуть более умное решение парадокса --- запретим множества, содержащие себя. Таким образом вывели аксиоматику теории множеств \textit{(Цермело --- Френкеля)}.

\begin{example}
    Рассмотрим множество всех чисел, которые можно задать в \(\leq 1000\) слов русского языка. Фраза ``наименьшее число, которое нельзя задать в \( \leq 1000\) слов'' содержит \( \leq 1000\) слов, т.е. такое число принадлежит искомому множеству --- парадокс.
\end{example}

Возникает идея --- человеческий язык порождает парадоксы, поэтому нужно задать новый язык, который их не порождает. Этот язык и является математической логикой.

\subsection{Программистам}

Математическая логика применяется в двух областях \textit{(для программистов)}:
\begin{enumerate}
    \item Языки программирования
    \item Формальные доказательства
\end{enumerate}

Для языков программирования матлогика применима как теория типов \textit{(переменных)}.

Формальные доказательства нужны например для smart-контрактов, где корректность программы критически важна, т.к. если в нём есть ошибка, у вас злоумышленник заберет все деньги, а вы не сможете этот контракт откатить.

\section{Исчисление высказываний}

\subsection{Язык}

\begin{definition}\itemfix
    \textbf{Язык} содержит в себе:
    \begin{enumerate}
        \item \textbf{Пропозициональные переменные}

              \(A_i'\) --- большая буква начала латинского алфавита, возможно с индексом и/или штрихом.

        \item \textbf{Связки}

              Пусть \(\alpha, \beta\) --- высказывания. Тогда \((\alpha \to \beta), (\alpha \with \beta), (\alpha \lor \beta), (\neg \alpha)\) --- высказывания.

              \(\alpha,\beta\) называются \textbf{метапеременными}.
    \end{enumerate}
\end{definition}

\begin{remark}
    Математическая логика алгеброподобна \textit{(а не анализоподобна)}, т.к. в ней много определений и мало доказательств.
\end{remark}

\subsection{Метаязык и предметный язык}

У нас есть два различных языка --- \textbf{предметный язык} и \textbf{метаязык}. Метаязык --- русский, предметный язык мы определили выше.

\begin{example}
    \(\alpha \to \beta\) --- метавыражение; \(A \to (A \to A)\) --- предметное выражение.
\end{example}

\begin{notation}
    Метапеременные обозначаются различными способами в зависимости от того, что они обозначают:
    \begin{itemize}
        \item Буквы греческого алфавита (\(\alpha, \beta, \gamma, \dots, \varphi, \psi\)) --- выражения
        \item Заглавные буквы конца латинского алфавита (\(X, Y, Z\)) --- произвольные переменные
    \end{itemize}
\end{notation}

\begin{example}
    \(X \to Y \Rightarrow A \to B\) --- подстановка переменных. Этот синтаксис не формален, мы будем записывать так:
    \[(X \to Y)[X : = A, Y : = B]\equiv A \to B\]
\end{example}

\textit{Соглашение}. символы логических операций не пишутся в метаязыке.

\begin{example}
    \begin{align*}
        (\alpha \to (A \to X))[\alpha : = A, X : = B]         & \equiv A \to (A \to B)         \\
        (\alpha \to (A \to X))[\alpha : = (A \to P), X : = B] & \equiv (A \to P) \to (A \to B) \\
    \end{align*}
\end{example}

\subsection{Сокращения записи}

\begin{itemize}
    \item \(\lor, \with, \neg\) --- скобки слева направо \textit{(лево-ассоциативные операции)} \textit{(не коммутативные)}
    \item \( \to \) --- правоассоциативная.
\end{itemize}

\begin{remark}
    Здесь операторы записаны в порядке их приоритета
\end{remark}

\begin{example}
    Расставим скобки в следующем выражении:
    \[A \to B \with C \to D\]
    \[A \to ((B \with C) \to D)\]
\end{example}

\subsection{Теория моделей}

\textbf{Модель} состоит из:
\begin{notation}\itemfix
    \begin{itemize}
        \item \(P\) --- некоторое множество предметных переменных
        \item \(\tau\) --- множество высказываний предметного языка
        \item \(V\) --- множество истинностных значений. Классическое --- \(\{\text{П}, \text{Л}\}\)
        \item \(\llbracket\ \rrbracket : \tau \to V\) --- оценка высказывания \textit{(высказывание ставится в скобки)}.
    \end{itemize}
\end{notation}

\begin{enumerate}
    \item \(\llbracket x \rrbracket : P \to V\) --- задается при оценке. % TODO
    \item \(\llbracket \alpha \star \beta \rrbracket = \llbracket \alpha \rrbracket \textcolor{blue}{\star} \llbracket \beta \rrbracket\), где \(\star\) есть логическая операция ( \(\lor, \with, \neg, \to\)), а \(\textcolor{blue}{\star}\) определено естественным образом как элемент метаязыка.
\end{enumerate}

\subsection{Теория доказательств}

\begin{definition}
    \textbf{Схема высказывания} --- строка, соответствующая определению высказывания + метапеременные.
\end{definition}

\begin{example}
    \[(\alpha \to (\beta \to (A \to \alpha)))\]
\end{example}

10 схем аксиом:
\begin{enumerate}
    \item \(\alpha \to \beta \to \alpha\)
    \item \((\alpha \to \beta) \to (\alpha \to \beta \to \gamma) \to (\alpha \to \gamma)\)
    \item \(\alpha \to \beta \to \alpha \with \beta\)
    \item \(\alpha \with \beta \to \alpha\)
    \item \(\alpha \with \beta \to \beta\)
    \item \(\alpha \to \alpha \lor \beta\)
    \item \(\beta \to \alpha \lor \beta\)
    \item \((\alpha \to \gamma) \to (\beta \to \gamma) \to (\alpha \lor \beta \to \gamma)\)
    \item \((\alpha \to \beta) \to (\alpha \to \neg \beta) \to \neg \alpha\)
    \item \(\neg \neg \alpha \to \alpha\)
\end{enumerate}

\subsection{Правило Modus Ponens и доказательство}

\begin{definition}
    \textbf{Доказательство \textit{(вывод)}} есть конечная последовательность высказываний \(\alpha_1 \dots \alpha_n\), где \(\alpha_i\) --- либо аксиома, либо \(\exists k, l < i : \alpha_k \equiv \alpha_l \to \alpha_i\) \textit{(правило Modus Ponens)}
\end{definition}

\begin{example}
    \(\vdash A \to A\)

    \begin{tabular}{lll}
        1. & \(A \to A \to A\)                                               & сх. акс. 1 \\
        2. & \(A \to (A \to A) \to A\)                                       & сх. акс. 1 \\
        3. & \((A \to (A \to A)) \to (A \to (A \to A) \to A) \to (A \to A)\) & сх. акс. 2 \\
        4. & \((A \to (A \to A) \to A) \to (A \to A)\)                       & M.P. 1, 3  \\
        5. & \(A \to A\)                                                     & M.P. 2, 4
    \end{tabular}
\end{example}

\begin{definition}
    Доказательство \(\alpha_1\dots \alpha_n\) доказывает выражение \(\beta\), если \(\alpha_n \equiv \beta\)
\end{definition}
