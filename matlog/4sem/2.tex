\chapter{19 февраля}

\begin{obozn}
    Большая греческая буква середины греческого алфавита ( \(\Gamma, \Delta, \Sigma\)) --- список высказываний.
\end{obozn}

\begin{definition}[следование]
    \(\alpha\) следует из \(\Gamma\) \textit{(обозначается \(\Gamma\models\alpha\))}, если \(\Gamma = \gamma_1 \dots \gamma_n\) и всегда, когда все \(\llbracket \gamma_i \rrbracket = \text{ И}\), то \(\llbracket \alpha \rrbracket = \text{ И}\).
\end{definition}

\begin{example}
    \(\models \alpha\) --- \(\alpha\) общезначимо.
\end{example}

\begin{definition}
    \sout{Теория} Исчисление высказываний \textbf{корректно}, если при любом \(\alpha\) из \(\vdash \alpha\) следует \(\models \alpha\).
\end{definition}

\begin{definition}
    Исчисление \textbf{полно}, если при любом \(\alpha\) из \(\models \alpha\) следует \(\vdash \alpha\).
\end{definition}

\begin{theorem}[о дедукции]
    \[\Gamma, \alpha \vdash \beta \Leftrightarrow \Gamma \vdash \alpha \to \beta\]
\end{theorem}

\begin{proof}\itemfix
    \begin{itemize}
        \item [ \( \Leftarrow \)] Пусть \(\Gamma \vdash \alpha \to \beta\), т.е. существует доказательство \(\delta_1 \dots \delta_n\), где \(\delta_n \equiv \alpha \to \beta\)

              Построим новое доказательство: \(\delta_1 \dots \delta_n, \alpha \text{ (гипотеза) }, \beta \text{ (M.P.)}\). Эта новая последовательность --- доказательство \(\Gamma, \alpha \vdash \beta\)

        \item [ \( \Rightarrow \)] Рассмотрим \(\delta_1 \dots \delta_n, \Gamma, \alpha \vdash \beta\). Рассмотрим последовательность \(\sigma_1 = \alpha \to \delta_1 \dots \sigma_n = \alpha \to \delta_n\). Это не доказательство.

              Но эту последовательность можно дополнить до доказательства, так что каждый \(\sigma_i\) есть аксиома, гипотеза или получается через M.P. Докажем это.

              \begin{proof}
                  \textbf{База}: \(n = 0\) --- очевидно.

                  \textbf{Переход}: пусть \(\sigma_0 \dots \sigma_n\) --- доказательство. Покажем, что между \(\sigma_n\) и \(\sigma_{n + 1}\) можно добавить формулы так, что \(\sigma_{n + 1}\) будет доказуемо.

                  У нас есть 3 варианта обоснования \(\delta_{n + 1}\)
                  \begin{enumerate}
                      \item \(\delta_{n + 1}\) --- аксиома или гипотеза, \(\not\equiv \alpha\)

                            Будем нумеровать дробными числами, потому что нам ничто это не запрещает, т.к. нам нужна только упорядоченность.

                            \(\textcolor{blue}{n + 0.2} \quad \delta_{n + 1}\) --- верно, т.к. это аксиома или гипотеза

                            \(\textcolor{blue}{n + 0.4} \quad \delta_{n + 1} \to \alpha \to \delta_{n + 1}\) \textcolor{blue}{(аксиома 1)}

                            \(\textcolor{blue}{n + 1} \quad \alpha \to \delta_{n + 1}\) \textcolor{blue}{(M.P. \(n + 0.2, n + 0.4\))}

                      \item \(\delta_{n + 1} \equiv \alpha\)

                            \(\textcolor{blue}{n + 0.2, 0.4, 0.6, 0.8, 1}\) --- доказательство \(\alpha \to \alpha\)

                      \item \(\delta_k \equiv \delta_l \to \delta_{n + 1},\ k, l \leq n\)

                            \(\textcolor{blue}{k} \quad \alpha \to (\delta_l \to \delta_{n + 1})\)

                            \(\textcolor{blue}{l} \quad  \alpha \to \sigma_l\)

                            \(\textcolor{blue}{n + 0.2} \quad (\alpha \to \sigma_l) \to (\alpha \to (\sigma_l \to \sigma_{n + 1})) \to (\alpha \to \sigma_{n + 1})\) \textcolor{blue}{(аксиома 2)}

                            \(\textcolor{blue}{n + 0.4} \quad (\alpha \to \sigma_l \to \sigma_{n + 1}) \to (\sigma \to \sigma_{n + 1})\) \textcolor{blue}{(M.P. \(n + 2, l\))}

                            \(\textcolor{blue}{n + 1} \quad \alpha \to \sigma_{n + 1}\) \textcolor{blue}{(M.P. \(n + 0.4, k\))}
                  \end{enumerate}
              \end{proof}
    \end{itemize}
\end{proof}

\begin{theorem}
    Пусть \(\vdash \alpha\). Тогда \(\models \alpha\).
\end{theorem}
\begin{proof}
    Индукция по длине доказательства: каждая \(\llbracket \delta_i \rrbracket = \text{ И}\), если \(\delta_1 \dots \delta_n\) --- доказательство \(\alpha\)

    Рассмотрим \(n\) и пусть \(\llbracket \delta_1 \rrbracket = \text{ И}, \dots \llbracket \delta_n \rrbracket = \text{ И}\).

    Тогда рассмотрим основание \(\delta_{n + 1}\)
    \begin{enumerate}
        \item \(\delta_{n + 1}\) --- аксиома. Это упражнение.

              \begin{example}
                  \(\delta_{n + 1} \equiv \alpha \to \beta \to \alpha\)

                  \[\sphericalangle \llbracket \alpha \to \beta \to \alpha \rrbracket^{\llbracket \alpha \rrbracket : = a, \llbracket \beta \rrbracket : = b} = \text{И}\]

                  \begin{center}
                      \begin{tabular}{c|c|c|c}
                          \(a\) & \(b\) & \(\beta \to \alpha\) & \(\alpha \to \beta \to \alpha\) \\ \hline
                          Л     & Л     & И                    & И                               \\
                          Л     & И     & Л                    & И                               \\
                          И     & Л     & И                    & И                               \\
                          И     & И     & И                    & И                               \\
                      \end{tabular}
                  \end{center}
              \end{example}

              Аналогично можно доказать для остальных аксиом.

        \item \(\delta_{n + 1}\) --- M.P. \(\delta_k = \delta_l \to \delta_{n + 1}\)

              Фиксируем оценку. Тогда \(\llbracket \delta_k \rrbracket = \llbracket \delta_l \rrbracket =\) И. Тогда:

              \begin{center}
                  \begin{tabular}{c|c|c}
                      \(\llbracket \delta_k \rrbracket\) & \(\llbracket \delta_{n + 1} \rrbracket \) & \(\llbracket \delta_k \rrbracket = \llbracket \delta_l \to \delta_{n + 1} \rrbracket\) \\ \hline
                      Л                                  & Л                                         & И                                                                                      \\
                      Л                                  & И                                         & И                                                                                      \\
                      И                                  & Л                                         & Л                                                                                      \\
                      И                                  & И                                         & И                                                                                      \\
                  \end{tabular}
              \end{center}

              Первых трёх вариантов не может быть в силу \(\llbracket \delta_k \rrbracket = \llbracket \delta_l \rrbracket =\) И. Таким образом, \(\llbracket \delta_{n + 1} \rrbracket = \) И.
    \end{enumerate}
\end{proof}

\begin{theorem}[о полноте]
    Пусть \(\models \alpha\). Тогда \(\vdash \alpha\).
\end{theorem}

Фиксируем набор переменных из \(\alpha\): \(P_1 \dots P_n\).

Рассмотрим \(\llbracket \alpha \rrbracket^{P_1 : = x_1 \dots P_n: = x_n} = \text{И}\)

\begin{obozn}
    \({}_{[\beta]} \alpha \equiv \begin{cases}
        \alpha,      & \llbracket \beta \rrbracket = \text{И} \\
        \neg \alpha, & \llbracket \beta \rrbracket = \text{Л}
    \end{cases}\) и \({}_{[x]} \alpha \equiv \begin{cases}
        \alpha,      & x = \text{И} \\
        \neg \alpha, & x = \text{Л}
    \end{cases}\)
\end{obozn}

Докажем, что \(\underbrace{{}_{[x_1]} P_1, \dots {}_{[x_n]} P_n}_{\Pi} \vdash {}_{[\alpha]} \alpha\)

\begin{proof}
    По индукции по длине формулы:

    \textbf{База}: \(\alpha = P_i\) \({}_{[P_i]} P_i \vdash {}_{[P_i]} P_i\), значит \(\Pi \vdash {}_{[P_i]} P_i\)

    \textbf{Переход}: пусть \(\eta, \zeta : \Pi \vdash {}_{[\eta]}\eta, \Pi \vdash {}_{[\zeta]}\zeta\) \textit{(по индукционному предположению)}. Покажем, что \(\Pi \vdash {}_{[\eta \star \zeta]} \eta \star \zeta\), где \(\star\) --- все связки

    Это упражнение.
\end{proof}

\begin{lemma}
    \(\Gamma, \eta \vdash \zeta, \Gamma, \neg \eta \vdash \zeta\). Тогда \(\Gamma \vdash \zeta\).
\end{lemma}
\begin{proof}
    Было в ДЗ.
\end{proof}

\begin{proof}[Доказательство теоремы о полноте]
    \(\models \alpha\), т.е. \({}_{[x_1]} P_1 \dots {}_{[x_n]} P_n \vdash {}_{[\alpha]} \alpha\). Но \(\llbracket \alpha \rrbracket = \text{П}\) при любой оценке. Тогда \({}_{[x_1]} P_1 \dots {}_{[x_n]} P_n \vdash \alpha\) при все \(x_i\).

    \begin{lemma}[об исключении допущения]
        Если \({}_{[x_1]} P_1 \dots {}_{[x_n]} P_n \vdash \alpha\) и \({}_{[x_1]} P_1 \dots {}_{[x_n]} \neg P_n \vdash \alpha\), то \({}_{[x_1]} P_1 \dots {}_{[x_{n - 1}]} P_{n - 1} \vdash \alpha\)
    \end{lemma}

    \[\begin{rcases*}
            {}_{[x_1]} P_1 \dots {}_{[x_{n - 1}]} P_{n - 1}, P_n \vdash \alpha \\
            {}_{[x_1]} P_1 \dots {}_{[x_{n - 1}]} P_{n - 1}, \neg P_n \vdash \alpha
        \end{rcases*} \xRightarrow{\text{по лемме}} {}_{[x_1]} P_1 \dots {}_{[x_{n - 1}]} P_{n - 1} \vdash \alpha\]
\end{proof}

% Докажем, что \(\exists a, b \in\R\setminus Q : a^b\in \Q\).
% \begin{proof}
%     Пусть \(a = b = \sqrt{2}\). Если \(\sqrt{2}^\sqrt{2} \notin \R\setminus \Q\)
% \end{proof}

\section{Интуиционистская логика}

\subsection{BHK-интерпретация (Brouwer–Heyting–Kolmogorov)}

Определим выражения:
\begin{itemize}
    \item \(\alpha \with \beta\) --- есть \(\alpha\) и \(\beta\)
    \item \(\alpha \lor \beta\) --- есть \(\alpha\) либо \(\beta\) и мы знаем, какое
    \item \(\alpha \to \beta\) --- есть способ перестроить \(\alpha\) в \(\beta\)
    \item \(\perp\) --- конструкция без построения \textit{(bottom)}
    \item \(\neg \alpha \equiv \alpha \to \perp\)
\end{itemize}

\textbf{Теория доказательств} есть классическая логика без десятой схемы аксиомы, вместо нее \(\alpha \to \neg \alpha \to \beta\)

\textbf{Теория моделей} --- теория, в которой \(\llbracket \alpha \rrbracket\) --- открытое множество в \(\Omega\) --- топологическом пространстве.

В ней определено следующее:
\begin{align*}
    \llbracket \alpha \with \beta \rrbracket & = \llbracket \alpha \rrbracket \cap \llbracket \beta \rrbracket                       \\
    \llbracket \alpha \lor \beta \rrbracket  & = \llbracket \alpha \rrbracket \cup \llbracket \beta \rrbracket                       \\
    \llbracket \alpha \to \beta \rrbracket   & = ((X \setminus \llbracket \alpha \rrbracket) \cup \llbracket \beta \rrbracket)^\circ \\
    \llbracket \perp \rrbracket              & = \emptyset                                                                           \\
    \llbracket \neg \alpha \rrbracket        & = (X \setminus \llbracket \alpha\rrbracket)^\circ
\end{align*}
