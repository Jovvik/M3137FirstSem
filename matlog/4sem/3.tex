\chapter{26 февраля}

\subsection{Естественный \textit{(натуральный)} вывод}

Рассмотрим новый способ записи доказательств --- в виде деревьев, называемый естественным выводом.

Тогда язык будет состоять из переменных \(A\dots Z, \lor, \with, \perp, \vdash, \text{---}\)

У нас используются следующие правила вывода:
\begin{enumerate}
    \item \begin{prooftree}
              \infer0[(аксиома)]{\Gamma \vdash \gamma, \gamma\in \Gamma}
          \end{prooftree}
    \item \begin{prooftree}
              \hypo{\Gamma, \varphi \vdash \psi}
              \infer1[(введение \( \to \))]{\Gamma \vdash \varphi \to \psi}
          \end{prooftree}
    \item \begin{prooftree}
              \hypo{\Gamma \vdash \varphi}
              \hypo{\Gamma \vdash \psi}
              \infer2[(введение \(\with\))]{\Gamma \vdash \varphi \with \psi}
          \end{prooftree}
    \item \begin{prooftree}
              \hypo{\Gamma \vdash \varphi \to \psi}
              \hypo{\Gamma \vdash \varphi}
              \infer2[(удаление \( \to \))]{\Gamma \vdash \psi}
          \end{prooftree}
    \item \begin{prooftree}
              \hypo{\Gamma \vdash \varphi \with \psi}
              \infer1[(удаление \(\with\))]{\Gamma \vdash \varphi}
          \end{prooftree}
    \item \begin{prooftree}
              \hypo{\Gamma \vdash \varphi \with \psi}
              \infer1[(удаление \(\with\))]{\Gamma \vdash \psi}
          \end{prooftree}
    \item \begin{prooftree}
              \hypo{\Gamma \vdash \varphi}
              \infer1[(введение \(\lor\))]{\Gamma \vdash \psi \lor \varphi}
          \end{prooftree}
    \item \begin{prooftree}
              \hypo{\Gamma \vdash \psi}
              \infer1[(введение \(\lor\))]{\Gamma \vdash \psi \lor \varphi}
          \end{prooftree}
    \item \begin{prooftree}
              \hypo{\Gamma \vdash \perp}
              \infer1[(удаление \(\perp\))]{\Gamma \vdash \varphi}
          \end{prooftree}
    \item \begin{prooftree}
              \hypo{\Gamma, \varphi \vdash \rho}
              \hypo{\Gamma, \psi \vdash \rho}
              \hypo{\Gamma \vdash \varphi \lor \psi}
              \infer3{\Gamma \vdash \rho}
          \end{prooftree}
\end{enumerate}

\begin{example}
    \begin{prooftree}
        \infer0[(акс.)]{A\vdash A}
        \infer1[(введение \(\with\))]{\vdash A \to A}
    \end{prooftree}
\end{example}

\begin{example}
    \begin{prooftree}
        \infer0[(акс.)]{A\with B\vdash A\with B}
        \infer1{A\with B \vdash B}
        \infer0[(акс.)]{A\with B\vdash A\with B}
        \infer1{A\with B \vdash A}
        \infer2{A\with B \vdash B\with A}
        \infer1[(введение \( \to \))]{\vdash A\with B \to B\with A}
    \end{prooftree}
\end{example}

\subsection{Теория решеток}

\begin{definition}\itemfix
    \begin{itemize}
        \item \textbf{Частичный порядок} --- рефлексивное, транзитивное, антисимметричное отношение.
        \item \textbf{Линейный порядок} --- сравнимы любые два элемента.
        \item \textbf{Наименьший элемент} \(S\) --- такой \(k\in S\), что если \(x\in S\), то \(k \leq x\)
        \item \textbf{Минимальный элемент} \(S\) --- такой \(k\in S\), что нет \(x\in S\), что \(x \leq k\)
        \item \textbf{Множество верхних граней} \(a\) и \(b\) : \(\{x\ |\ a \leq x \with b \leq x\}\).
        \item \textbf{Множество нижних граней} \(a\) и \(b\) : \(\{x\ |\ x \leq a \with x \leq b\}\).
        \item \(a + b\) --- наименьший элемент множества верхних граней \textit{(может не существовать)}.
        \item \(a \cdot b\) --- наибольший элемент множества нижних граней.
        \item \textbf{Решетка} --- множество + отношение, где для каждых \(a,b\) есть как \(a + b\), так и \(a \cdot b\).
        \item \textbf{Дистрибутивная решетка} --- если всегда \(a\cdot(b + c) = a\cdot b + a\cdot c\)
    \end{itemize}
\end{definition}

\begin{lemma}
    В дистрибутивной решетке \(a + b\cdot c = (a + b)(a + c)\)
\end{lemma}

\begin{definition}\itemfix
    \begin{itemize}
        \item \textbf{Псевдодполнение} \(a\) и \(b\) обозначается \(a \to b\) и равно наибольшему элементу множества \(\{c\ |\ a\cdot c \leq b\}\)
        \item \textbf{Импликативная решетка} --- решетка, где \(\forall a, b \ \ \exists a \to b\)
        \item \(0\) --- наименьший элемент решетки.
        \item \(1\) --- наибольший элемент решетки.
        \item \textbf{Псевдобулева алгебра \textit{(алгебра Гейтинга)}} --- импликативная решетка с нулём.
        \item \textbf{Булева алгебра} --- псевдобулева алгебра, такая что \(a + (a \to 0) = 1\)
    \end{itemize}
\end{definition}

\begin{example}
    $$\begin{tikzcd}[ampersand replacement=\&]
            1 \arrow{r} \arrow[swap]{d}{} \& b \arrow{d}{} \\
            a \arrow{r} \& 0
        \end{tikzcd}$$

    \begin{align*}
        a\cdot 0 & = 0 \\
        1\cdot b & = b \\
        a\cdot b & = 0 \\
        a + b    & = 1 \\
    \end{align*}
\end{example}

\begin{lemma}
    В импликативной решетке всегда есть 1.
\end{lemma}
\begin{proof}
    Возьмём \(a \to a = 1\) для некоторого \(a\).

    \[a \to a = \text{н} \{x\ |\ a\cdot x \leq a\} = \text{н}(A)\]
    Таким образом, \(A\) имеет наибольший элемент и это \(a \to a\)
\end{proof}

\begin{theorem}\itemfix
    \begin{itemize}
        \item Любая алгебра Гейтинга --- модель интуиционистского исчисления высказываний.
        \item Любая булева алгебра --- модель классического исчисления высказываний.
    \end{itemize}
\end{theorem}

\begin{definition}[топология]
    Рассмотрим множество \(X\), называемое ``носитель'' и \(\Omega \subset \mathcal{P}(X)\) --- подмножество подмножеств \(X\), называемое ``топология'', такое что:
    \begin{enumerate}
        \item \(\bigcup_\alpha x_i \in \Omega\), где \(x_i\in\Omega\)
        \item \(\bigcap_{i = 1}^{n} x_i \in \Omega\), где \(x_i\in\Omega\)
        \item \(\emptyset\in\Omega, X\in \Omega\)
    \end{enumerate}
\end{definition}

\begin{example}
    Пусть \(X\) --- узлы дерева, \(\Omega\) --- все множества узлов, которые содержат узлы вместе со всеми потомками.
\end{example}

\begin{definition}
    \[X^\circ \defeq \text{наиб}\{w\ |\ w \subseteq X, w \text{ --- открыто}\} \]
\end{definition}

\begin{theorem}
    Пусть \((X,\Omega)\) --- топологическое пространство, \(a + b = a\cup b, a\cdot b = a \cap b, a \to b = ((X\setminus a) \cup b)^\circ\), \(a \leq b \Leftrightarrow a \subseteq b\), тогда \((\Omega, \leq)\) есть алгебра Гейтинга.
\end{theorem}

\begin{example}
    Дискретная топология --- \(\Omega = \mathcal{P}(X)\). Тогда \((\Omega, \leq )\) --- булева алгебра.
\end{example}

\begin{enumerate}
    \item \(X^\circ = X\)
    \item \(a \to 0 = (X\setminus a \cup \emptyset) = X\setminus a\)
\end{enumerate}

Таким образом, \(a + (a \to 0) = a + X\setminus a = X\)

\begin{definition}
    Пусть \(X\) --- все формулы логики. Определим отношение порядка \(\alpha \leq \beta\) это \(\alpha \vdash \beta\). Будем говорить, что \(\alpha \approx \beta\), если \(\alpha \vdash \beta\) и \(\beta \vdash \alpha\).

    \((X /_\approx, \leq )\) есть алгебра Гейтинга.
\end{definition}

\begin{definition}
    \((X /_\approx, \leq )\) --- \textbf{алгебра Линденбаума}, где \(X, \approx\) из интуиционистской логики.
\end{definition}

\begin{theorem}
    Алгебра Гейтинга --- полная модель интуиционистской логики.
\end{theorem}
\begin{proof}
    \(\vDash \alpha\) --- истинно в любой алгебре Гейтинга, в частности в \((X /_\approx, \leq )\). \(\llbracket \alpha \rrbracket = 1\), т.е. \(\llbracket \alpha \rrbracket = \llbracket A \to A \rrbracket\), т.е. \(\alpha \in [A \to A]_\approx\), т.е. \(A \to A \vdash \alpha\).
\end{proof}
