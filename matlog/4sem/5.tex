\chapter{12 марта}

\section{Изоморфизм Карри-Ховарда}

\begin{remark}
    Эта тема в нашем курсе рукомахательная.
\end{remark}

Пусть \(p\) --- программа, т.е. функция, принимающая \(\alpha\) и возвращающая \(\beta\), т.е. \(p : \alpha \to \beta\)

Можем посмотреть на это с другой стороны: \(p\) доказательство, что из \(\alpha\) следует \(\beta\), например в \texttt{Haskell} \texttt{f a = a} гласит, что \texttt{f} доказывает, что \texttt{A -> A}, где подразумевается \(\forall A\).

Такое сопоставление программам доказательств и высказываниям типов называется изоморфизмом Карри-Ховарда:

\begin{tabular}{|c|c|}
    \hline
    логическое исчисление & типизированное \(\lambda\)-исчисление   \\ \hline
    логическая формула    & тип                                     \\
    доказательство        & программа                               \\
    доказуемая формула    & обитаемый тип                           \\
    \( \to \)             & функция                                 \\
    \(\with\)             & упорядоченная пара                      \\
    \(\lor\)              & алгебраический тип \textit{(тип-сумма)} \\
    \hline
\end{tabular}

\begin{remark}
    Обитаемый тип --- тип, у которого есть хотя бы один экземпляр.
\end{remark}

Несложно заметить, что логика, соответствующая \(\lambda\)-исчислению, является интуиционистской, поэтому мы её в основном изучаем.

\subsection{Алгебраические типы}

Рассмотрим следующее определение списка в \texttt{Pascal}:

\begin{minted}{pascal}
type list : record
    nul : boolean;
    case nul of
        true: ;
        false: next ^list
    end
end;
\end{minted}

Рассмотрим то же самое в \texttt{C}, опустив \texttt{bool} и скажем, что \texttt{nul = (next == null)} \textit{(это в какой-то степени костыльно)}:

\begin{minted}{c}
struct list {
    next: *list;
}
\end{minted}

Определим таким же способом дерево:

\begin{minted}{c}
struct tree {
    tree* left;
    tree* right;
    int value;
}
\end{minted}

Это ещё более костыльно, т.к. то, является ли вершина листом, закодировано в неявном виде.

\begin{definition}
    \textbf{Отмеченное \textit{(дизъюнктное)} объединение} множеств \(A, B\) обозначается \(A\sqcup B\) или \(A \uplus B\) \footnote{или ещё десятком других символов} и равно \(\{\ev{\text{``}A\text{''}, a}\ |\ a\in A\} \cup \{\ev{\text{``}B\text{''}, b}\ |\ b\in B\}\).
\end{definition}

\begin{remark}
    Это определение интуиционистское по своей сути, т.к. если дано \(s\in A \sqcup B\), то мы знаем, из какого множества \(s\).
\end{remark}

\begin{definition}
    Тип, соответствующий такому объединению множеств, называется \textbf{алгебраическим}
\end{definition}

\begin{example}
    В \texttt{C++} такой тип реализован как \texttt{std::variant<...>}
\end{example}

\begin{example}
    Список в \texttt{Haskell}:

    \begin{minted}{haskell}
data List a = nil | Cons a (List a)
    \end{minted}
\end{example}

\subsection{Применение восьмой аксиомы интуиционистской логики}

Вспомним восьмую аксиому интуиционистской \footnote{и классической} логики и запишем её как правило натурального вывода:
\begin{center}
    \begin{prooftree}
        \hypo{\Gamma \vdash \alpha \to \gamma}
        \hypo{\Gamma \vdash \beta \to \gamma}
        \hypo{\Gamma \vdash \alpha \lor \beta}
        \infer3{\Gamma \vdash \gamma}
    \end{prooftree}
\end{center}

Рассмотрим программу в \texttt{Haskell}, которая преобразует список в строку:

\begin{minted}{haskell}
let rec string_of_list l =
    match l with
        Nil -> "Nil"
        Cons(head, tail) -> head ^ ":" ^ string_of_list tail
\end{minted}

Подставим в рассматриваемую аксиому соответствующие значения:
\begin{center}
    \begin{prooftree}
        \hypo{\Gamma \vdash Nil \to string}
        \hypo{\Gamma \vdash list \to string}
        \hypo{\Gamma \vdash Nil \lor list}
        \infer3{\Gamma \vdash string}
    \end{prooftree}
\end{center}

Несложно заметить, что эта аксиома описывает \texttt{match} в \texttt{Haskell} --- мы даем выражения после ``\texttt{->}'', т.е. правила \texttt{Nil -> string}, \texttt{list -> string} и элемент \texttt{Nil} или \texttt{list}, а \texttt{match} возвращает \texttt{string}.

\section{Исчисление предикатов}

\subsection{Язык исчисления предикатов}

Выражения в этом языке бывают двух видов:
\begin{enumerate}
    \item Логические выражения, называемые ``предикаты'' или ``формулы''
    \item Предметные выражения, называемые ``термы''
\end{enumerate}

\(\theta\) --- метапеременная для термов.

Термы бывают двух видов:
\begin{itemize}
    \item Атомы:
          \begin{itemize}
              \item Предметные переменные обозначаются буквами \(a,b,c \dots \)
              \item Метапеременные обозначаются буквами \(x,y,z\)
          \end{itemize}
    \item Применение функциональных символов:
          \begin{itemize}
              \item Функциональные символы: \(f,g,h\) и записывается \(f(\theta_1 \dots \theta_n)\)
              \item Метапеременная тоже обозначается \(f\)
          \end{itemize}
\end{itemize}

Логические выражения:
\begin{itemize}
    \item Применение предикатных символов \(P(\theta_1, \dots \theta_n)\), где \(P\) --- метапеременная для предикатных символов, а предикатный символ --- \(A, B, C \dots \)
    \item Связки \(\with, \lor, \neg, \to \) с правилами из языка классической логики.
    \item Кванторы \footnote{По записи кванторов нет общепринятого соглашения.} \(\forall x.\varphi\) или \(\exists x.\varphi\), где \(\varphi\) --- любое логическое выражение.
\end{itemize}

Мы используем жадность кванторов. \footnote{В отношении жадности кванторов также нет соглашения; встречается запись, где квантор --- унарная операция, аналогичная \(\neg\)} Это значит, что квантор берет в \(\varphi\) все, пока не встретит конец выражения или скобку, которая оканчивает этот квантор.

\begin{example}
    \(\forall x.P(x) \with \forall y.P(y) \equiv \forall x.(P(x) \with (\forall y.P(y)))\)
\end{example}

\subsection{Теория моделей}

Определим оценку формулы в исчислении предикатов:

\begin{enumerate}
    \item Фиксируем \(D\) --- предметное множество, \(V = \{\text{И}, \text{Л}\} \)
    \item Каждому \(f_i(x_1 \dots x_n)\) сопоставим функцию \(f_{f_n} : D^n \to D\)
    \item Каждому \(P_j(x_1 \dots x_n)\) сопоставим функцию \footnote{,  называемую предикат} \(f_{p_n} : D^n \to V\)
    \item Каждой \(x_i\) сопоставим \(f_{x_i}\in D\)
\end{enumerate}

\begin{itemize}
    \item \(\llbracket x \rrbracket = f_{x_i}\)
    \item \(\llbracket \alpha \star \beta \rrbracket\) --- так же, как в исчислении высказываний.
    \item \(\llbracket P_i(\theta_1 \dots \theta_n) \rrbracket = f_{p_i}(\llbracket \theta_1 \rrbracket \dots \llbracket \theta_n \rrbracket)\)
    \item \(\llbracket f_j(\theta_1 \dots \theta_n) \rrbracket = f_{f_j}(\llbracket \theta_1 \rrbracket \dots \llbracket \theta_m \rrbracket)\)
    \item \(\llbracket \forall x.\varphi \rrbracket = \begin{cases} \text{И}, & \text{если } \llbracket \varphi \rrbracket = \text{И} \text{ при всех } k\in D \\
              \text{Л}, & \text{иначе}
          \end{cases} \)
    \item \(\llbracket \exists x.\varphi \rrbracket = \begin{cases} \text{И}, & \text{если } \llbracket \varphi \rrbracket = \text{И} \text{ при некотором } k\in D \\
              \text{Л}, & \text{иначе}
          \end{cases} \)
\end{itemize}

\begin{example}
    \(\forall x.\forall y.E(x, y)\)

    Пусть \(D = \N\), \(E(x, y) = \begin{cases} \text{И}, & x = y \\ \text{Л}, & x \neq y \end{cases} \)

    \(\llbracket\forall x.\forall y.E(x, y)\rrbracket_{x: = 1, y: = 2} =\) Л, т.к. \(\llbracket E(x, y) \rrbracket =\) Л.
\end{example}

Вспомним определение предела последовательности из матанализа:

\[\forall \varepsilon > 0 \ \ \exists N \ \ \forall n > N \ \ |a_n - a| < \varepsilon\]

Перепишем это определение с богомерзкого языка матанализа на православный язык исчисления предикатов. \footnote{Это термины лектора, все претензии от адептов матанализа и других религий --- к нему.}

Пусть \(( >)(a, b) = G(a, b), |a|= m_|(a), ( -)(a,b) = m_{ -}(a, b), m_a : n \mapsto a_n, 0() = m_0\)

\begin{align*}
     & \forall \varepsilon . \varepsilon \to 0\ \exists N . \forall n. (n > N) \to (|a_n - a| < \varepsilon)      \\
     & \forall \varepsilon . \varepsilon \to 0\ \exists N . \forall n. (n > N) \to (|a_n - a| < \varepsilon)      \\
     & \forall e . G(e, m_0)\ \exists n_0 . \forall n. G(n, n_0) \to G(e, m_|(m_{ -}(m_a(n), a)))) < \varepsilon) \\
\end{align*}

\subsection{Теория доказательств}

Все аксиомы исчисления высказываний + Modus Ponens + две схемы аксиом + два правила:
\begin{enumerate}
    \item \((\forall x.\varphi) \to \varphi [x : = \theta]\)
    \item \(\varphi[x : = \theta] \to \exists x.\varphi\)
\end{enumerate}
Обе эти схемы применимы только если \(\theta\) свободен для подстановки вместо \(x\) в \(\varphi\), т.е. никакое свободное вхождение \(x\) в \(\theta\) не станет связным.

\begin{example}\itemfix
    \begin{minted}{c}
int f(int x) {
    x = y;
}
    \end{minted}

    После замены \(y : = x\) код станет следующим:
    \begin{minted}{c}
int f(int x) {
    x = x;
}
    \end{minted}

    И код потеряет свой смысл.
\end{example}

Правила следующие:
\begin{enumerate}
    \item \begin{prooftree}
              \hypo{\varphi \to \psi}
              \infer1[(правило \(\forall \))]{\varphi \to (\forall x.\psi)}
          \end{prooftree}
    \item \begin{prooftree}
              \hypo{\psi \to \varphi}
              \infer1[(правило \(\exists \))]{(\exists x.\psi) \to \varphi}
          \end{prooftree}
\end{enumerate}

