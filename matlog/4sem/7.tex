\chapter{2 апреля}

\begin{definition}
    Будем говорить, что \(\Gamma \vDash \alpha\), т.е. \(\alpha\) следует из \(\Gamma\), если при всех оценках, таких что все \(\gamma \in \Gamma \ \ \llbracket \gamma \rrbracket = \text{И}\), выполнено \(\llbracket \alpha \rrbracket = \text{И}\)
\end{definition}

\begin{example}[странный случай]
    \(x = 0 \vdash \forall x.x = 0\), но \(x = 0 \nvDash \forall x.x = 0\)
\end{example}

Условие для корректности: правила для кванторов по свободным переменным из \(\Gamma\) запрещены. Тогда \(\Gamma \vdash \alpha\) влечёт \(\Gamma \vDash \alpha\) и \(\llbracket \alpha [x: = \Theta] \rrbracket = \llbracket \alpha \rrbracket^{x: = \llbracket \Theta \rrbracket}\)

\begin{remark}
    Здесь и далее мы предполагаем условие корректности.
\end{remark}

\subsection{Полнота исчисления предикатов}

\begin{definition}
    \(\Gamma\) --- непротиворечивое. если \(\Gamma \nvdash \alpha \with \neg \alpha\) ни при каком \(\alpha\)
\end{definition}

\begin{example}\itemfix
    \begin{itemize}
        \item Непротиворечивое: \(\emptyset, A \lor \neg A\)
        \item Противоречивое: \(A \with \neg A\)
    \end{itemize}
\end{example}

Мы будем рассматривать непротиворечивое множество замкнутых бескванторных формул и обозначать \((\ldots)\).
\begin{example}\itemfix
    \begin{itemize}
        \item \(\{A\}\)
        \item \(\{0 = 0\}\)
    \end{itemize}
\end{example}

\begin{definition}
    \textbf{Моделью} для \((\ldots)\) \(\Gamma\) называется такая модель, что каждая формула из \(\Gamma\) оценивается в \(\text{И}\).
\end{definition}

\begin{definition}
    \((\ldots)\) \(\Gamma\) называется \textbf{полным}, если для каждой замкнутой бескванторной формулы \(\alpha\) либо \(\alpha \in \Gamma\), либо \(\neg \alpha \in \Gamma\).

    Аналогично определяется для не бескванторного множества.
\end{definition}

\begin{theorem}
    Если \(\Gamma\) \((\ldots)\) и \(\alpha\) --- замкнутая бескванторная формула, то либо \(\Gamma \cup \{\alpha\} \), либо \(\Gamma \cup \{\neg \alpha\}\) --- \((\ldots)\)

    Аналогичное верно для не бескванторного множества.
\end{theorem}
\begin{proof}
    Пусть и \(\Gamma \cup \{\alpha\} \), и \(\Gamma \cup \{\neg \alpha\} \) --- противоречивы, т.е.:
    \[\Gamma, \alpha \vdash \beta \with \neg \beta \quad \Gamma, \neg \alpha \vdash \beta \with \neg \beta\]

    \[\begin{cases} \Gamma \vdash \alpha \to \beta \with \neg \beta \\ \Gamma \vdash \neg \alpha \to \beta \with \neg \beta \end{cases} \Rightarrow \gamma \vdash \beta \with \neg \beta \]

    Т.е. \(\Gamma\) --- противоречиво. Это противоречие.
\end{proof}

\begin{theorem}
    Если \(\Gamma\) --- \((\ldots)\) и в языке счётное количество формул\footnote{В исчислении предикатов это верно.}, то можно построить \(\Delta\) --- полное \((\ldots)\), такое что \(\Gamma \subset \Delta\).

    Аналогичное верно для не бескванторного множества.
\end{theorem}

\begin{proof}
    Пусть \(\varphi_1, \varphi_2 \dots \) --- замкнутые бескванторные формулы исчисления предикатов.

    \(\Gamma_0: = \Gamma\)

    \(\Gamma_1 : = \Gamma_0 \cup \{\varphi_1\} \) или \(\Gamma_0 \cup \{\neg \varphi_1\} \) --- смотря что непротиворечиво

    \(\Gamma_2 : = \Gamma_1 \cup \{\varphi_2\} \) или \(\Gamma_1 \cup \{\neg \varphi_2\} \) --- смотря что непротиворечиво

    \(\vdots\)

    \(\Gamma^* : = \bigcup_i \Gamma_i\), тогда \(\Gamma^*\) --- полное и непротиворечивое. Первое очевидно, покажем второе.

    Пусть \(\Gamma^* \vdash \beta \with \neg \beta\). Это конечное доказательство \(\delta_1 \dots \delta_s\) использует конечное число гипотез, пусть они \(\gamma_1 \dots \gamma_k\) и \(\gamma_i \in \Gamma_{R_i}\). Возьмём \(\Gamma_{\max(R_i)}\). Тогда \(\Gamma_{\max(R_i)} \vdash \beta \with \neg \beta\) --- противоречие.
\end{proof}

\begin{theorem}
    Любое полное \((\ldots)\) \(\Gamma\) имеет модель, т.е. существует оценка \(\llbracket  \rrbracket\), такая что если \(\gamma \in \Gamma\), то \(\llbracket \gamma \rrbracket = \text{И}\)
\end{theorem}

\begin{proof}
    Пусть \(D\) --- все записи из функциональных символов:

    \(\llbracket f_0^n \rrbracket\)\footnote{константа} \( \Rightarrow \text{``}f_0^n\text{''}\)

    \(\llbracket f_k^n(\theta_1 \dots \theta_k) \rrbracket \Rightarrow \text{``}f_k^n(\text{''} + \llbracket \theta_1 \rrbracket \text{``},\text{''} + \cdots + \text{``},\text{''} + \llbracket \theta_n \rrbracket + \text{``})\text{''}\)

    Предикатные символы: \(\llbracket P(\theta_1 \dots \theta_n) \rrbracket = \begin{cases} \text{И}, & P(\theta_1 \dots \theta_n) \in \Gamma \\ \text{Л}, & \text{иначе} \end{cases} \)

    Свободных предметных переменных нет, поэтому для них не нужно придумывать оценку.

    Так построенная модель --- модель для \(\Gamma\). Докажем это по индукции по количеству связок: любая формула \(\alpha\), имеющая \( \leq n\) связок, истинно \(\Leftrightarrow \alpha \in \Gamma\).

    \begin{itemize}
        \item [База.] Очевидно.
        \item [Переход.] Рассмотрим случай \(\alpha \with \beta\).

              \begin{enumerate}
                  \item Если \(\llbracket \alpha \rrbracket = \text{И}\) и \(\llbracket \beta \rrbracket = \text{И}\), то \(\alpha \with \beta \in \Gamma\)
                  \item Если \(\llbracket \alpha \rrbracket \neq \text{И}\) или \(\llbracket \beta \rrbracket \neq \text{И}\), то \(\alpha \with \beta \notin \Gamma\)
              \end{enumerate}
    \end{itemize}
\end{proof}

\begin{definition}
    \textbf{Предварённая нормальная форма} --- форма, где \(\forall \exists \forall \dots (\tau)\), где \(\tau\) --- формула без кванторов.
\end{definition}
\begin{theorem}
    Если \(\varphi\) --- формула, то существует \(\psi\) в предварённой нормальной форме и при этом \(\varphi \to \psi\) и \(\psi \to \varphi\).
\end{theorem}

\begin{theorem}[Гёделя о полноте исчисления предикатов]
    Если \(\Gamma\) --- полное непротиворечивое множество замкнутых формул, то оно имеет модель.
\end{theorem}
\begin{proof}
    План таков: рассмотрим \(\Gamma\) --- полное непротиворечивое множество замкнутых формул. Построим по нему \(\Gamma^\Delta\) --- п.н.м. бескванторных з.ф. Построим по нему по теореме о существовании модели модель \(M^\Delta\) и покажем, что \(M^\Delta\) --- модель для \(\Gamma\):
    \[\begin{tikzcd}
            \Gamma \arrow{d}[swap]{\text{без кванторов}} & M \\
            \Gamma^\Delta \arrow{r}{\text{теорема}}      & M^\Delta \arrow{u}{id}
        \end{tikzcd}\]

    Рассмотрим \(\Gamma_0 \subset \Gamma_1 \dots \Gamma_i \dots \subset \Gamma^*\) и \(\Gamma^* = \bigcup_i \Gamma_i\), а также \(\Gamma_0 = \Gamma\), где все формулы в предварённой нормальной форме. Определим переход \(\Gamma_i \to \Gamma_{i + 1}\).

    Построим семейство функциональных символов \(d^i_j\), которые нигде ранее не использовались.

    Рассмотрим случаи того, чем является \(\varphi_j \in \Gamma_i\).
    \begin{enumerate}
        \item \(\varphi_j\) без кванторов --- не трогаем.
        \item \(\varphi_j \equiv \forall x.\psi\) --- добавим все формулы вида \(\psi[x : = \theta]\), где \(\theta\) --- терм, составленный из \(f, d_0^l, d_1^{l'}, \dots d_{i - 1}^{l^{\prime \dots \prime}}\)
        \item \(\varphi_j \equiv \exists x.\psi\) --- добавим формулу \(\psi[x: = d_i^j]\)
    \end{enumerate}

    Таким образом, мы получим \(\Gamma_{i + 1} = \Gamma_i \cup \{\text{все добавленные формулы}\}\).
\end{proof}

\begin{corollary}
    Пусть \(\vDash \alpha\) и \(\alpha\) замкнута, тогда \(\vdash \alpha\).
\end{corollary}
\begin{proof}
    Пусть \(\vDash \alpha\), но не \(\nvdash \alpha\). Значит, \(\{\neg \alpha\} \) --- непротиворечивое множество замкнутых формул.

    Почему непротиворечиво? \(\neg \alpha \vdash \beta \with \neg \beta, \beta \with \neg \beta \vdash \alpha\), следовательно \(\neg \alpha \vdash \alpha\), но ещё и \(\alpha \vdash \alpha\). Таким образом, \(\vdash \alpha\).

    Значит, у \(\neg \alpha\) есть модель \(M\), \(\llbracket \neg \alpha \rrbracket_M = \text{И}\). Значит, \(\nVdash \alpha\)
\end{proof}

\begin{theorem}
    Если \(\Gamma_i\) непротиворечиво, то \(\Gamma_{i + 1}\) непротиворечиво.
\end{theorem}
\begin{theorem}
    \(\Gamma^*\) непротиворечиво.
\end{theorem}
\(\Gamma^\Delta = \Gamma^*\) без формул с \(\forall , \exists \)
