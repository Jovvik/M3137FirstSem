\documentclass[12pt, a4paper, oneside]{book}

\usepackage{catchfilebetweentags}
\ExecuteMetaData[../../preamble.sty]{preamble}

\addto\captionsrussian{\renewcommand{\chaptername}{Лекция}}

\let\MakeUppercase\relax

\lfoot{}
\cfoot{}
\rfoot{}
\lhead{\leftmark}

\usepackage{tocloft}
\advance\cftsecnumwidth -1em\relax
\advance\cftsubsecindent -1em\relax
\advance\cftsubsecnumwidth -1em\relax

\usepackage{titletoc}
\titlecontents*{chapter}% <section-type>
  [0pt]% <left>
  {}% <above-code>
  {\bfseries\chaptername\ \thecontentslabel\quad}% <numbered-entry-format>
  {}% <numberless-entry-format>
  {\bfseries\hfill\contentspage}% <filler-page-format>

\let\endtitlepage\relax

\patchcmd{\chapter}{\thispagestyle{plain}}{\thispagestyle{fancy}}{}{}

\usepackage{remreset}
\makeatletter
  \@removefromreset{section}{chapter}
\makeatother

\renewcommand{\thesection}{\arabic{section}.}
\renewcommand{\thesubsection}{\thesection\arabic{subsection}.}

\begin{document}

\title{Математическая логика}
\author{Михайлов Максим}

\maketitle

\tableofcontents

\chapter{12 февраля}

\setcounter{section}{-1}

\section{Мотивация}

\subsection{Математикам}

\begin{axiom}[Архимеда]
    Для любого \(k > 0\) найдётся \(n\), такое что \(kn > 1\).
\end{axiom}

Под эту аксиому не подходят бесконечно малые числа и это является проблемой. Например, \(\lim\limits_{x \to +\infty} \frac{1}{x} = 0 = \lim\limits_{x \to +\infty} \frac{1}{x^2}\), но мы хотим уметь различать эти два числа. Ньютон предложил идею бесконечно малых чисел, откуда пошли последовательности. Возникает вопрос --- что такое последовательность и что такое число?

Общепринятое определение целых чисел \(\N\) происходит из теории множеств. Однако эта теория содержит в себе множество фундаментальных парадоксов, от которых нельзя избавиться.

Возникает вопрос --- а что такое множество? Посмотрим на некоторое множество \(A = \{x\ |\ x\not\in x\}\). Содержит ли оно себя, \(A\in A\)? На этот вопрос нельзя ответить, это называется парадокс Рассела. Есть простой способ его разрешить --- запретить ставить такой вопрос. Нет вопроса --- нет парадокса. Существование такого парадокса ставит под вопрос существование любого множества --- а существует ли \(\N\)? Может быть его существование парадоксально, просто мы не нашли этот парадокс. Пришло чуть более умное решение парадокса --- запретим множества, содержащие себя. Таким образом вывели аксиоматику теории множеств \textit{(Цермело --- Френкеля)}.

\begin{example}
    Рассмотрим множество всех чисел, которые можно задать в \(\leq 1000\) слов русского языка. Фраза ``наименьшее число, которое нельзя задать в \( \leq 1000\) слов'' содержит \( \leq 1000\) слов, т.е. такое число принадлежит искомому множеству --- парадокс.
\end{example}

Возникает идея --- человеческий язык порождает парадоксы, поэтому нужно задать новый язык, который их не порождает. Этот язык и является математической логикой.

\subsection{Программистам}

Математическая логика применяется в двух областях \textit{(для программистов)}:
\begin{enumerate}
    \item Языки программирования
    \item Формальные доказательства
\end{enumerate}

Для языков программирования матлогика применима как теория типов \textit{(переменных)}.

Формальные доказательства нужны например для smart-контрактов, где корректность программы критически важна, т.к. если в нём есть ошибка, у вас злоумышленник заберет все деньги, а вы не сможете этот контракт откатить.

\section{Исчисление высказываний}

\subsection{Язык}

\begin{definition}\itemfix
    \textbf{Язык} содержит в себе:
    \begin{enumerate}
        \item \textbf{Пропозициональные переменные}

              \(A_i'\) --- большая буква начала латинского алфавита, возможно с индексом и/или штрихом.

        \item \textbf{Связки}

              Пусть \(\alpha, \beta\) --- высказывания. Тогда \((\alpha \to \beta), (\alpha \with \beta), (\alpha \lor \beta), (\neg \alpha)\) --- высказывания.

              \(\alpha,\beta\) называются \textbf{метапеременными}.
    \end{enumerate}
\end{definition}

\begin{remark}
    Математическая логика алгеброподобна \textit{(а не анализоподобна)}, т.к. в ней много определений и мало доказательств.
\end{remark}

\subsection{Метаязык и предметный язык}

У нас есть два различных языка --- \textbf{предметный язык} и \textbf{метаязык}. Метаязык --- русский, предметный язык мы определили выше.

\begin{example}
    \(\alpha \to \beta\) --- метавыражение; \(A \to (A \to A)\) --- предметное выражение.
\end{example}

\begin{obozn}
    Метапеременные обозначаются различными способами в зависимости от того, что они обозначают:
    \begin{itemize}
        \item Буквы греческого алфавита (\(\alpha, \beta, \gamma, \dots, \varphi, \psi\)) --- выражения
        \item Заглавные буквы конца латинского алфавита (\(X, Y, Z\)) --- произвольные переменные
    \end{itemize}
\end{obozn}

\begin{example}
    \(X \to Y \Rightarrow A \to B\) --- подстановка переменных. Этот синтаксис не формален, мы будем записывать так:
    \[(X \to Y)[X : = A, Y : = B]\equiv A \to B\]
\end{example}

\textit{Соглашение}. символы логических операций не пишутся в метаязыке.

\begin{example}
    \begin{align*}
        (\alpha \to (A \to X))[\alpha : = A, X : = B]         & \equiv A \to (A \to B)         \\
        (\alpha \to (A \to X))[\alpha : = (A \to P), X : = B] & \equiv (A \to P) \to (A \to B) \\
    \end{align*}
\end{example}

\subsection{Сокращения записи}

\begin{itemize}
    \item \(\lor, \with, \neg\) --- скобки слева направо \textit{(лево-ассоциативные операции)} \textit{(не коммутативные)}
    \item \( \to \) --- правоассоциативная.
\end{itemize}

\begin{remark}
    Здесь операторы записаны в порядке их приоритета
\end{remark}

\begin{example}
    Расставим скобки в следующем выражении:
    \[A \to B \with C \to D\]
    \[A \to ((B \with C) \to D)\]
\end{example}

\subsection{Теория моделей}

\textbf{Модель} состоит из:
\begin{obozn}\itemfix
    \begin{itemize}
        \item \(P\) --- некоторое множество предметных переменных
        \item \(\tau\) --- множество высказываний предметного языка
        \item \(V\) --- множество истинных значений. Классическое --- \(\{\text{П}, \text{Л}\}\)
        \item \(\llbracket\ \rrbracket : \tau \to V\) --- оценка высказывания \textit{(высказывание ставится в скобки)}.
    \end{itemize}
\end{obozn}

\begin{enumerate}
    \item \(\llbracket x \rrbracket : P \to V\) --- задается при оценке. % TODO
    \item \(\llbracket \alpha \star \beta \rrbracket = \llbracket \alpha \rrbracket \textcolor{blue}{\star} \llbracket \beta \rrbracket\), где \(\star\) есть логическая операция ( \(\lor, \with, \neg, \to\)), а \(\textcolor{blue}{\star}\) определено естественным образом как элемент метаязыка.
\end{enumerate}

\subsection{Теория доказательств}

\begin{definition}
    \textbf{Схема высказывания} --- строка, соответствующая определению высказывания + метапеременные.
\end{definition}

\begin{example}
    \[(\alpha \to (\beta \to (A \to \alpha)))\]
\end{example}

10 схем аксиом:
\begin{enumerate}
    \item \(\alpha \to \beta \to \alpha\)
    \item \((\alpha \to \beta) \to (\alpha \to \beta \to \gamma) \to (\alpha \to \gamma)\)
    \item \(\alpha \to \beta \to \alpha \with \beta\)
    \item \(\alpha \with \beta \to \alpha\)
    \item \(\alpha \with \beta \to \beta\)
    \item \(\alpha \to \alpha \lor \beta\)
    \item \(\beta \to \alpha \lor \beta\)
    \item \((\alpha \to \gamma) \to (\beta \to \gamma) \to (\alpha \lor \beta \to \gamma)\)
    \item \((\alpha \to \beta) \to (\alpha \to \neg \beta) \to \neg \alpha\)
    \item \(\neg \neg \alpha \to \alpha\)
\end{enumerate}

\subsection{Правило Modus Ponens и доказательство}

\begin{definition}
    \textbf{Доказательство \textit{(вывод)}} есть конечная последовательность высказываний \(\alpha_1 \dots \alpha_n\), где \(\alpha_i\) --- либо аксиома, либо \(\exists k, l < i : \alpha_k \equiv \alpha_l \to \alpha_i\) \textit{(правило Modus Ponens)}
\end{definition}

\begin{example}
    \(\vdash A \to A\)

    \begin{tabular}{lll}
        1. & \(A \to A \to A\)                                               & сх. акс. 1 \\
        2. & \(A \to (A \to A) \to A\)                                       & сх. акс. 1 \\
        3. & \((A \to (A \to A)) \to (A \to (A \to A) \to A) \to (A \to A)\) & сх. акс. 2 \\
        4. & \((A \to (A \to A) \to A) \to (A \to A)\)                       & M.P. 1, 3  \\
        5. & \(A \to A\)                                                     & M.P. 2, 4
    \end{tabular}
\end{example}

\begin{definition}
    Доказательство \(\alpha_1\dots \alpha_n\) доказывает выражение \(\beta\), если \(\alpha_n \equiv \beta\)
\end{definition}

\chapter{19 февраля}

\begin{obozn}
    Большая греческая буква середины греческого алфавита ( \(\Gamma, \Delta, \Sigma\)) --- список высказываний.
\end{obozn}

\begin{definition}[следование]
    \(\alpha\) следует из \(\Gamma\) \textit{(обозначается \(\Gamma\models\alpha\))}, если \(\Gamma = \gamma_1 \dots \gamma_n\) и всегда, когда все \(\llbracket \gamma_i \rrbracket = \text{ И}\), то \(\llbracket \alpha \rrbracket = \text{ И}\).
\end{definition}

\begin{example}
    \(\models \alpha\) --- \(\alpha\) общезначимо.
\end{example}

\begin{definition}
    \sout{Теория} Исчисление высказываний \textbf{корректно}, если при любом \(\alpha\) из \(\vdash \alpha\) следует \(\models \alpha\).
\end{definition}

\begin{definition}
    Исчисление \textbf{полно}, если при любом \(\alpha\) из \(\models \alpha\) следует \(\vdash \alpha\).
\end{definition}

\begin{theorem}[о дедукции]
    \[\Gamma, \alpha \vdash \beta \Leftrightarrow \Gamma \vdash \alpha \to \beta\]
\end{theorem}

\begin{proof}\itemfix
    \begin{itemize}
        \item [ \( \Leftarrow \)] Пусть \(\Gamma \vdash \alpha \to \beta\), т.е. существует доказательство \(\delta_1 \dots \delta_n\), где \(\delta_n \equiv \alpha \to \beta\)

              Построим новое доказательство: \(\delta_1 \dots \delta_n, \alpha \text{ (гипотеза) }, \beta \text{ (M.P.)}\). Эта новая последовательность --- доказательство \(\Gamma, \alpha \vdash \beta\)

        \item [ \( \Rightarrow \)] Рассмотрим \(\delta_1 \dots \delta_n, \Gamma, \alpha \vdash \beta\). Рассмотрим последовательность \(\sigma_1 = \alpha \to \delta_1 \dots \sigma_n = \alpha \to \delta_n\). Это не доказательство.

              Но эту последовательность можно дополнить до доказательства, так что каждый \(\sigma_i\) есть аксиома, гипотеза или получается через M.P. Докажем это.

              \begin{proof}
                  \textbf{База}: \(n = 0\) --- очевидно.

                  \textbf{Переход}: пусть \(\sigma_0 \dots \sigma_n\) --- доказательство. Покажем, что между \(\sigma_n\) и \(\sigma_{n + 1}\) можно добавить формулы так, что \(\sigma_{n + 1}\) будет доказуемо.

                  У нас есть 3 варианта обоснования \(\delta_{n + 1}\)
                  \begin{enumerate}
                      \item \(\delta_{n + 1}\) --- аксиома или гипотеза, \(\not\equiv \alpha\)

                            Будем нумеровать дробными числами, потому что нам ничто это не запрещает, т.к. нам нужна только упорядоченность.

                            \(\textcolor{blue}{n + 0.2} \quad \delta_{n + 1}\) --- верно, т.к. это аксиома или гипотеза

                            \(\textcolor{blue}{n + 0.4} \quad \delta_{n + 1} \to \alpha \to \delta_{n + 1}\) \textcolor{blue}{(аксиома 1)}

                            \(\textcolor{blue}{n + 1} \quad \alpha \to \delta_{n + 1}\) \textcolor{blue}{(M.P. \(n + 0.2, n + 0.4\))}

                      \item \(\delta_{n + 1} \equiv \alpha\)

                            \(\textcolor{blue}{n + 0.2, 0.4, 0.6, 0.8, 1}\) --- доказательство \(\alpha \to \alpha\)

                      \item \(\delta_k \equiv \delta_l \to \delta_{n + 1},\ k, l \leq n\)

                            \(\textcolor{blue}{k} \quad \alpha \to (\delta_l \to \delta_{n + 1})\)

                            \(\textcolor{blue}{l} \quad  \alpha \to \sigma_l\)

                            \(\textcolor{blue}{n + 0.2} \quad (\alpha \to \sigma_l) \to (\alpha \to (\sigma_l \to \sigma_{n + 1})) \to (\alpha \to \sigma_{n + 1})\) \textcolor{blue}{(аксиома 2)}

                            \(\textcolor{blue}{n + 0.4} \quad (\alpha \to \sigma_l \to \sigma_{n + 1}) \to (\sigma \to \sigma_{n + 1})\) \textcolor{blue}{(M.P. \(n + 2, l\))}

                            \(\textcolor{blue}{n + 1} \quad \alpha \to \sigma_{n + 1}\) \textcolor{blue}{(M.P. \(n + 0.4, k\))}
                  \end{enumerate}
              \end{proof}
    \end{itemize}
\end{proof}

\begin{theorem}
    Пусть \(\vdash \alpha\). Тогда \(\models \alpha\).
\end{theorem}
\begin{proof}
    Индукция по длине доказательства: каждая \(\llbracket \delta_i \rrbracket = \text{ И}\), если \(\delta_1 \dots \delta_n\) --- доказательство \(\alpha\)

    Рассмотрим \(n\) и пусть \(\llbracket \delta_1 \rrbracket = \text{ И}, \dots \llbracket \delta_n \rrbracket = \text{ И}\).

    Тогда рассмотрим основание \(\delta_{n + 1}\)
    \begin{enumerate}
        \item \(\delta_{n + 1}\) --- аксиома. Это упражнение.

              \begin{example}
                  \(\delta_{n + 1} \equiv \alpha \to \beta \to \alpha\)

                  \[\sphericalangle \llbracket \alpha \to \beta \to \alpha \rrbracket^{\llbracket \alpha \rrbracket : = a, \llbracket \beta \rrbracket : = b} = \text{И}\]

                  \begin{center}
                      \begin{tabular}{c|c|c|c}
                          \(a\) & \(b\) & \(\beta \to \alpha\) & \(\alpha \to \beta \to \alpha\) \\ \hline
                          Л     & Л     & И                    & И                               \\
                          Л     & И     & Л                    & И                               \\
                          И     & Л     & И                    & И                               \\
                          И     & И     & И                    & И                               \\
                      \end{tabular}
                  \end{center}
              \end{example}

              Аналогично можно доказать для остальных аксиом.

        \item \(\delta_{n + 1}\) --- M.P. \(\delta_k = \delta_l \to \delta_{n + 1}\)

              Фиксируем оценку. Тогда \(\llbracket \delta_k \rrbracket = \llbracket \delta_l \rrbracket =\) И. Тогда:

              \begin{center}
                  \begin{tabular}{c|c|c}
                      \(\llbracket \delta_k \rrbracket\) & \(\llbracket \delta_{n + 1} \rrbracket \) & \(\llbracket \delta_k \rrbracket = \llbracket \delta_l \to \delta_{n + 1} \rrbracket\) \\ \hline
                      Л                                  & Л                                         & И                                                                                      \\
                      Л                                  & И                                         & И                                                                                      \\
                      И                                  & Л                                         & Л                                                                                      \\
                      И                                  & И                                         & И                                                                                      \\
                  \end{tabular}
              \end{center}

              Первых трёх вариантов не может быть в силу \(\llbracket \delta_k \rrbracket = \llbracket \delta_l \rrbracket =\) И. Таким образом, \(\llbracket \delta_{n + 1} \rrbracket = \) И.
    \end{enumerate}
\end{proof}

\begin{theorem}[о полноте]
    Пусть \(\models \alpha\). Тогда \(\vdash \alpha\).
\end{theorem}

Фиксируем набор переменных из \(\alpha\): \(P_1 \dots P_n\).

Рассмотрим \(\llbracket \alpha \rrbracket^{P_1 : = x_1 \dots P_n: = x_n} = \text{И}\)

\begin{obozn}
    \({}_{[\beta]} \alpha \equiv \begin{cases}
        \alpha,      & \llbracket \beta \rrbracket = \text{И} \\
        \neg \alpha, & \llbracket \beta \rrbracket = \text{Л}
    \end{cases}\) и \({}_{[x]} \alpha \equiv \begin{cases}
        \alpha,      & x = \text{И} \\
        \neg \alpha, & x = \text{Л}
    \end{cases}\)
\end{obozn}

Докажем, что \(\underbrace{{}_{[x_1]} P_1, \dots {}_{[x_n]} P_n}_{\Pi} \vdash {}_{[\alpha]} \alpha\)

\begin{proof}
    По индукции по длине формулы:

    \textbf{База}: \(\alpha = P_i\) \({}_{[P_i]} P_i \vdash {}_{[P_i]} P_i\), значит \(\Pi \vdash {}_{[P_i]} P_i\)

    \textbf{Переход}: пусть \(\eta, \zeta : \Pi \vdash {}_{[\eta]}\eta, \Pi \vdash {}_{[\zeta]}\zeta\) \textit{(по индукционному предположению)}. Покажем, что \(\Pi \vdash {}_{[\eta \star \zeta]} \eta \star \zeta\), где \(\star\) --- все связки

    Это упражнение.
\end{proof}

\begin{lemma}
    \(\Gamma, \eta \vdash \zeta, \Gamma, \neg \eta \vdash \zeta\). Тогда \(\Gamma \vdash \zeta\).
\end{lemma}
\begin{proof}
    Было в ДЗ.
\end{proof}

\begin{proof}[Доказательство теоремы о полноте]
    \(\models \alpha\), т.е. \({}_{[x_1]} P_1 \dots {}_{[x_n]} P_n \vdash {}_{[\alpha]} \alpha\). Но \(\llbracket \alpha \rrbracket = \text{П}\) при любой оценке. Тогда \({}_{[x_1]} P_1 \dots {}_{[x_n]} P_n \vdash \alpha\) при все \(x_i\).

    \begin{lemma}[об исключении допущения]
        Если \({}_{[x_1]} P_1 \dots {}_{[x_n]} P_n \vdash \alpha\) и \({}_{[x_1]} P_1 \dots {}_{[x_n]} \neg P_n \vdash \alpha\), то \({}_{[x_1]} P_1 \dots {}_{[x_{n - 1}]} P_{n - 1} \vdash \alpha\)
    \end{lemma}

    \[\begin{rcases*}
            {}_{[x_1]} P_1 \dots {}_{[x_{n - 1}]} P_{n - 1}, P_n \vdash \alpha \\
            {}_{[x_1]} P_1 \dots {}_{[x_{n - 1}]} P_{n - 1}, \neg P_n \vdash \alpha
        \end{rcases*} \xRightarrow{\text{по лемме}} {}_{[x_1]} P_1 \dots {}_{[x_{n - 1}]} P_{n - 1} \vdash \alpha\]
\end{proof}

% Докажем, что \(\exists a, b \in\R\setminus Q : a^b\in \Q\).
% \begin{proof}
%     Пусть \(a = b = \sqrt{2}\). Если \(\sqrt{2}^\sqrt{2} \not\in \R\setminus \Q\)
% \end{proof}

\section{Интуиционистская логика}

\subsection{ВНК-интерпретация}

Определим выражения:
\begin{itemize}
    \item \(\alpha \with \beta\) --- есть \(\alpha\) и \(\beta\)
    \item \(\alpha \lor \beta\) --- есть \(\alpha\) либо \(\beta\) и мы знаем, какое
    \item \(\alpha \to \beta\) --- есть способ перестроить \(\alpha\) в \(\beta\)
    \item \(\perp\) --- конструкция без построения \textit{(bottom)}
    \item \(\neg \alpha \equiv \alpha \to \perp\)
\end{itemize}

\textbf{Теория доказательств} есть классическая логика без десятой схемы аксиомы, вместо нее \(\alpha \to \neg \alpha \to \beta\)

\textbf{Теория моделей} --- теория, в которой \(\llbracket \alpha \rrbracket\) --- открытое множество в \(\Omega\) --- топологическом пространстве.

В ней определено следующее:
\begin{align*}
    \llbracket \alpha \with \beta \rrbracket & = \llbracket \alpha \rrbracket \cap \llbracket \beta \rrbracket                       \\
    \llbracket \alpha \lor \beta \rrbracket  & = \llbracket \alpha \rrbracket \cup \llbracket \beta \rrbracket                       \\
    \llbracket \alpha \to \beta \rrbracket   & = ((X \setminus \llbracket \alpha \rrbracket) \cup \llbracket \beta \rrbracket)^\circ \\
    \llbracket \perp \rrbracket              & = \emptyset                                                                           \\
    \llbracket \neg \alpha \rrbracket        & = (X \setminus \llbracket \alpha\rrbracket)^\circ
\end{align*}

\chapter{26 февраля}

Рассмотрим новый способ записи доказательств --- в виде деревьев.

Тогда язык будет состоять из переменных \(A\dots Z, \lor, \with, \perp, \vdash, \text{---}\)

У нас используются следующие правила вывода:
\begin{enumerate}
    \item \begin{prooftree}
              \infer0[(аксиома)]{\Gamma \vdash \gamma, \gamma\in \Gamma}
          \end{prooftree}
    \item \begin{prooftree}
              \hypo{\Gamma, \varphi \vdash \psi}
              \infer1[(введение \( \to \))]{\Gamma \vdash \varphi \to \psi}
          \end{prooftree}
    \item \begin{prooftree}
              \hypo{\Gamma \vdash \varphi}
              \hypo{\Gamma \vdash \psi}
              \infer2[(введение \(\with\))]{\Gamma \vdash \varphi \with \psi}
          \end{prooftree}
    \item \begin{prooftree}
              \hypo{\Gamma \vdash \varphi \to \psi}
              \hypo{\Gamma \vdash \varphi}
              \infer2[(удаление \( \to \))]{\Gamma \vdash \psi}
          \end{prooftree}
    \item \begin{prooftree}
              \hypo{\Gamma \vdash \varphi \with \psi}
              \infer1[(удаление \(\with\))]{\Gamma \vdash \varphi}
          \end{prooftree}
    \item \begin{prooftree}
              \hypo{\Gamma \vdash \varphi \with \psi}
              \infer1[(удаление \(\with\))]{\Gamma \vdash \psi}
          \end{prooftree}
    \item \begin{prooftree}
              \hypo{\Gamma \vdash \varphi}
              \infer1[(введение \(\lor\))]{\Gamma \vdash \psi \lor \varphi}
          \end{prooftree}
    \item \begin{prooftree}
              \hypo{\Gamma \vdash \psi}
              \infer1[(введение \(\lor\))]{\Gamma \vdash \psi \lor \varphi}
          \end{prooftree}
    \item \begin{prooftree}
              \hypo{\Gamma \vdash \perp}
              \infer1[(удаление \(\perp\))]{\Gamma \vdash \varphi}
          \end{prooftree}
    \item \begin{prooftree}
              \hypo{\Gamma, \varphi \vdash \rho}
              \hypo{\Gamma, \psi \vdash \rho}
              \hypo{\Gamma \vdash \varphi \lor \psi}
              \infer3{\Gamma \vdash \rho}
          \end{prooftree}
\end{enumerate}

\begin{example}
    \begin{prooftree}
        \infer0[(акс.)]{A\vdash A}
        \infer1[(введение \(\with\))]{\vdash A \to A}
    \end{prooftree}
\end{example}

\begin{example}
    \begin{prooftree}
        \infer0[(акс.)]{A\with B\vdash A\with B}
        \infer1{A\with B \vdash B}
        \infer0[(акс.)]{A\with B\vdash A\with B}
        \infer1{A\with B \vdash A}
        \infer2{A\with B \vdash B\with A}
        \infer1[(введение \( \to \))]{\vdash A\with B \to B\with A}
    \end{prooftree}
\end{example}

\begin{definition}\itemfix
    \begin{itemize}
        \item \textbf{Частичный порядок} --- рефлексивное, транзитивное, антисимметричное отношение.
        \item \textbf{Линейный порядок} --- сравнимы любые два элемента.
        \item \textbf{Наименьший элемент} \(S\) --- такой \(k\in S\), что если \(x\in S\), то \(k \leq x\)
        \item \textbf{Минимальный элемент} \(S\) --- такой \(k\in S\), что нет \(x\in S\), что \(x \leq k\)
        \item \textbf{Множество верхних граней} \(a\) и \(b\) : \(\{x\ |\ a \leq x \with b \leq x\}\).
        \item \textbf{Множество нижних граней} \(a\) и \(b\) : \(\{x\ |\ x \leq a \with x \leq b\}\).
        \item \(a + b\) --- наименьший элемент множества верхних граней \textit{(может не существовать)}.
        \item \(a \cdot b\) --- наибольший элемент множества нижних граней.
        \item \textbf{Решетка} --- множество + отношение, где для каждых \(a,b\) есть как \(a + b\), так и \(a \cdot b\).
        \item \textbf{Дистрибутивная решетка} --- если всегда \(a\cdot(b + c) = a\cdot b + a\cdot c\)
    \end{itemize}
\end{definition}

\begin{lemma}
    В дистрибутивной решетке \(a + b\cdot c = (a + b)(a + c)\)
\end{lemma}

\begin{definition}\itemfix
    \begin{itemize}
        \item \textbf{Псевдодполнение} \(a\) и \(b\) обозначается \(a \to b\) и равно наибольшему элементу множества \(\{c\ |\ a\cdot c \leq b\}\)
        \item \textbf{Импликативная решетка} --- решетка, где \(\forall a, b \ \ \exists a \to b\)
        \item \(0\) --- наименьший элемент решетки.
        \item \(1\) --- наибольший элемент решетки.
        \item \textbf{Псевдобулева алгебра \textit{(алгебра Гейтинга)}} --- импликативная решетка с нулём.
        \item \textbf{Булева алгебра} --- псевдобулева алгебра, такая что \(a + (a \to 0) = 1\)
    \end{itemize}
\end{definition}

\begin{example}
    $$\begin{tikzcd}[ampersand replacement=\&]
            1 \arrow{r} \arrow[swap]{d}{} \& b \arrow{d}{} \\
            a \arrow{r} \& 0
        \end{tikzcd}$$

    \begin{align*}
        a\cdot 0 & = 0 \\
        1\cdot b & = b \\
        a\cdot b & = 0 \\
        a + b    & = 1 \\
    \end{align*}
\end{example}

\begin{lemma}
    В импликативной решетке всегда есть 1.
\end{lemma}
\begin{proof}
    Возьмём \(a \to a = 1\) для некоторого \(a\).

    \[a \to a = \text{н} \{x\ |\ a\cdot x \leq a\} = \text{н}(A)\]
    Таким образом, \(A\) имеет наибольший элемент и это \(a \to a\)
\end{proof}

\begin{theorem}\itemfix
    \begin{itemize}
        \item Любая алгебра Гейтинга --- модель интуиционистского исчисления высказываний.
        \item Любая булева алгебра --- модель классического исчисления высказываний.
    \end{itemize}
\end{theorem}

\begin{definition}[топология]
    Рассмотрим множество \(X\), называемое ``носитель'' и \(\Omega \subset \mathcal{P}(X)\) --- подмножество подмножеств \(X\), называемое ``топология'', такое что:
    \begin{enumerate}
        \item \(\bigcup_\alpha x_\alpha \in \Omega\), где \(x_i\in\Omega\)
        \item \(\bigcap_{i = 1}^{n} x_i \in \Omega\), где \(x_i\in\Omega\)
        \item \(\emptyset\in\Omega, X\in \Omega\)
    \end{enumerate}
\end{definition}

\begin{example}
    Пусть \(X\) --- узлы дерева, \(\Omega\) --- все множества узлов, которые содержат узлы вместе со всеми потомками.
\end{example}

\begin{theorem}
    Пусть \((X,\Omega)\) --- топологическое пространство, \(a + b = a\cup b, a\cdot b = a \cap b, a \to b = ((X\setminus a)\subset b)^\circ\), \(a \leq b \Leftrightarrow a \leq b\), тогда \((\Omega, \leq)\) есть алгебра Гейтинга.
\end{theorem}

\begin{example}
    Дискретная топология --- \(\Omega = \mathcal{P}(X)\). Тогда \((\Omega, \leq )\) --- булева алгебра.
\end{example}

\begin{enumerate}
    \item \(X^0 = X\)
    \item \(a \to 0 = (X\setminus a \cup \emptyset) = X\setminus a\)
\end{enumerate}

Таким образом, \(a + (a \to 0) = a + X\setminus a = X\)

\begin{definition}
    Пусть \(X\) --- все формулы логики. Определим отношение порядка \(\alpha \leq \beta\) это \(\alpha \vdash \beta\). Будем говорить, что \(\alpha \approx \beta\), если \(\alpha \vdash \beta\) и \(\beta \vdash \alpha\).

    \((X /_\approx, \leq )\) есть алгебра Гейтинга.
\end{definition}

\begin{definition}
    \((X /_\approx, \leq )\) --- \textbf{алгебра Линденбаума}, где \(X, \approx\) из интуиционистской логики.
\end{definition}

\begin{theorem}
    Алгебра Гейтинга --- полная модель интуиционистской логики.
\end{theorem}
\begin{proof}
    \(\models \alpha\) --- истинно в любой алгебре Гейтинга, в частности в \((X /_\approx, \leq )\). \(\llbracket \alpha \rrbracket = 1\), т.е. \(\llbracket \alpha \rrbracket = \llbracket A \to A \rrbracket\), т.е. \(\alpha \in [A \to A]_\approx\), т.е. \(A \to A \vdash \alpha\).
\end{proof}

\end{document}
