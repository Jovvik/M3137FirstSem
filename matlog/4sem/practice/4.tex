\documentclass[12pt, a4paper]{article}

%<*preamble>
% Math symbols
\usepackage{amsmath, amsthm, amsfonts, amssymb}
\usepackage{accents}
\usepackage{esvect}
\usepackage{mathrsfs}
\usepackage{mathtools}
\mathtoolsset{showonlyrefs}
\usepackage{cmll}
\usepackage{stmaryrd}
\usepackage{physics}
\usepackage[normalem]{ulem}
\usepackage{ebproof}
\usepackage{extarrows}

% Page layout
\usepackage{geometry, a4wide, parskip, fancyhdr}

% Font, encoding, russian support
\usepackage[russian]{babel}
\usepackage[sb]{libertine}
\usepackage{xltxtra}

% Listings
\usepackage{listings}
\lstset{basicstyle=\ttfamily,breaklines=true}
\setmonofont{Inconsolata}

% Miscellaneous
\usepackage{array}
\usepackage{calc}
\usepackage{caption}
\usepackage{subcaption}
\captionsetup{justification=centering,margin=2cm}
\usepackage{catchfilebetweentags}
\usepackage{enumitem}
\usepackage{etoolbox}
\usepackage{float}
\usepackage{lastpage}
\usepackage{minted}
\usepackage{svg}
\usepackage{wrapfig}
\usepackage{xcolor}
\usepackage[makeroom]{cancel}

\newcolumntype{L}{>{$}l<{$}}
    \newcolumntype{C}{>{$}c<{$}}
\newcolumntype{R}{>{$}r<{$}}

% Footnotes
\usepackage[hang]{footmisc}
\setlength{\footnotemargin}{2mm}
\makeatletter
\def\blfootnote{\gdef\@thefnmark{}\@footnotetext}
\makeatother

% References
\usepackage{hyperref}
\hypersetup{
    colorlinks,
    linkcolor={blue!80!black},
    citecolor={blue!80!black},
    urlcolor={blue!80!black},
}

% tikz
\usepackage{tikz}
\usepackage{tikz-cd}
\usetikzlibrary{arrows.meta}
\usetikzlibrary{decorations.pathmorphing}
\usetikzlibrary{calc}
\usetikzlibrary{patterns}
\usepackage{pgfplots}
\pgfplotsset{width=10cm,compat=1.9}
\newcommand\irregularcircle[2]{% radius, irregularity
    \pgfextra {\pgfmathsetmacro\len{(#1)+rand*(#2)}}
    +(0:\len pt)
    \foreach \a in {10,20,...,350}{
            \pgfextra {\pgfmathsetmacro\len{(#1)+rand*(#2)}}
            -- +(\a:\len pt)
        } -- cycle
}

\providetoggle{useproofs}
\settoggle{useproofs}{false}

\pagestyle{fancy}
\lfoot{M3137y2019}
\cfoot{}
\rhead{стр. \thepage\ из \pageref*{LastPage}}

\newcommand{\R}{\mathbb{R}}
\newcommand{\Q}{\mathbb{Q}}
\newcommand{\Z}{\mathbb{Z}}
\newcommand{\B}{\mathbb{B}}
\newcommand{\N}{\mathbb{N}}
\renewcommand{\Re}{\mathfrak{R}}
\renewcommand{\Im}{\mathfrak{I}}

\newcommand{\const}{\text{const}}
\newcommand{\cond}{\text{cond}}

\newcommand{\teormin}{\textcolor{red}{!}\ }

\DeclareMathOperator*{\xor}{\oplus}
\DeclareMathOperator*{\equ}{\sim}
\DeclareMathOperator{\sign}{\text{sign}}
\DeclareMathOperator{\Sym}{\text{Sym}}
\DeclareMathOperator{\Asym}{\text{Asym}}

\DeclarePairedDelimiter{\ceil}{\lceil}{\rceil}

% godel
\newbox\gnBoxA
\newdimen\gnCornerHgt
\setbox\gnBoxA=\hbox{$\ulcorner$}
\global\gnCornerHgt=\ht\gnBoxA
\newdimen\gnArgHgt
\def\godel #1{%
    \setbox\gnBoxA=\hbox{$#1$}%
    \gnArgHgt=\ht\gnBoxA%
    \ifnum     \gnArgHgt<\gnCornerHgt \gnArgHgt=0pt%
    \else \advance \gnArgHgt by -\gnCornerHgt%
    \fi \raise\gnArgHgt\hbox{$\ulcorner$} \box\gnBoxA %
    \raise\gnArgHgt\hbox{$\urcorner$}}

% \theoremstyle{plain}

\theoremstyle{definition}
\newtheorem{theorem}{Теорема}
\newtheorem*{definition}{Определение}
\newtheorem{axiom}{Аксиома}
\newtheorem*{axiom*}{Аксиома}
\newtheorem{lemma}{Лемма}

\theoremstyle{remark}
\newtheorem*{remark}{Примечание}
\newtheorem*{exercise}{Упражнение}
\newtheorem{corollary}{Следствие}[theorem]
\newtheorem*{statement}{Утверждение}
\newtheorem*{corollary*}{Следствие}
\newtheorem*{example}{Пример}
\newtheorem{observation}{Наблюдение}
\newtheorem*{prop}{Свойства}
\newtheorem*{obozn}{Обозначение}

% subtheorem
\makeatletter
\newenvironment{subtheorem}[1]{%
    \def\subtheoremcounter{#1}%
    \refstepcounter{#1}%
    \protected@edef\theparentnumber{\csname the#1\endcsname}%
    \setcounter{parentnumber}{\value{#1}}%
    \setcounter{#1}{0}%
    \expandafter\def\csname the#1\endcsname{\theparentnumber.\Alph{#1}}%
    \ignorespaces
}{%
    \setcounter{\subtheoremcounter}{\value{parentnumber}}%
    \ignorespacesafterend
}
\makeatother
\newcounter{parentnumber}

\newtheorem{manualtheoreminner}{Теорема}
\newenvironment{manualtheorem}[1]{%
    \renewcommand\themanualtheoreminner{#1}%
    \manualtheoreminner
}{\endmanualtheoreminner}

\newcommand{\dbltilde}[1]{\accentset{\approx}{#1}}
\newcommand{\intt}{\int\!}

% magical thing that fixes paragraphs
\makeatletter
\patchcmd{\CatchFBT@Fin@l}{\endlinechar\m@ne}{}
{}{\typeout{Unsuccessful patch!}}
\makeatother

\newcommand{\get}[2]{
    \ExecuteMetaData[#1]{#2}
}

\newcommand{\getproof}[2]{
    \iftoggle{useproofs}{\ExecuteMetaData[#1]{#2proof}}{}
}

\newcommand{\getwithproof}[2]{
    \get{#1}{#2}
    \getproof{#1}{#2}
}

\newcommand{\import}[3]{
    \subsection{#1}
    \getwithproof{#2}{#3}
}

\newcommand{\given}[1]{
    Дано выше. (\ref{#1}, стр. \pageref{#1})
}

\renewcommand{\ker}{\text{Ker }}
\newcommand{\im}{\text{Im }}
\renewcommand{\grad}{\text{grad}}
\newcommand{\rg}{\text{rg}}
\newcommand{\defeq}{\stackrel{\text{def}}{=}}
\newcommand{\defeqfor}[1]{\stackrel{\text{def } #1}{=}}
\newcommand{\itemfix}{\leavevmode\makeatletter\makeatother}
\newcommand{\?}{\textcolor{red}{???}}
\renewcommand{\emptyset}{\varnothing}
\newcommand{\longarrow}[1]{\xRightarrow[#1]{\qquad}}
\DeclareMathOperator*{\esup}{\text{ess sup}}
\newcommand\smallO{
    \mathchoice
    {{\scriptstyle\mathcal{O}}}% \displaystyle
    {{\scriptstyle\mathcal{O}}}% \textstyle
    {{\scriptscriptstyle\mathcal{O}}}% \scriptstyle
    {\scalebox{.6}{$\scriptscriptstyle\mathcal{O}$}}%\scriptscriptstyle
}
\renewcommand{\div}{\text{div}\ }
\newcommand{\rot}{\text{rot}\ }
\newcommand{\cov}{\text{cov}}

\makeatletter
\newcommand{\oplabel}[1]{\refstepcounter{equation}(\theequation\ltx@label{#1})}
\makeatother

\newcommand{\symref}[2]{\stackrel{\oplabel{#1}}{#2}}
\newcommand{\symrefeq}[1]{\symref{#1}{=}}

% xrightrightarrows
\makeatletter
\newcommand*{\relrelbarsep}{.386ex}
\newcommand*{\relrelbar}{%
    \mathrel{%
        \mathpalette\@relrelbar\relrelbarsep
    }%
}
\newcommand*{\@relrelbar}[2]{%
    \raise#2\hbox to 0pt{$\m@th#1\relbar$\hss}%
    \lower#2\hbox{$\m@th#1\relbar$}%
}
\providecommand*{\rightrightarrowsfill@}{%
    \arrowfill@\relrelbar\relrelbar\rightrightarrows
}
\providecommand*{\leftleftarrowsfill@}{%
    \arrowfill@\leftleftarrows\relrelbar\relrelbar
}
\providecommand*{\xrightrightarrows}[2][]{%
    \ext@arrow 0359\rightrightarrowsfill@{#1}{#2}%
}
\providecommand*{\xleftleftarrows}[2][]{%
    \ext@arrow 3095\leftleftarrowsfill@{#1}{#2}%
}

\allowdisplaybreaks

\newcommand{\unfinished}{\textcolor{red}{Не дописано}}

% Reproducible pdf builds 
\special{pdf:trailerid [
<00112233445566778899aabbccddeeff>
<00112233445566778899aabbccddeeff>
]}
%</preamble>


\lhead{Математическая логика \textit{(практика)}}
\cfoot{}
\rfoot{4.3.2021}

\mathtoolsset{showonlyrefs=false}

\begin{document}

\begin{enumerate}
      \item Обозначим выводимость в ИИВ <<гильбертовского стиля>> как $\vdash_\text{и}$,
            а выводимость в ИИВ <<системы натурального (естественного) вывода>> как $\vdash_\text{е}$.

            Заметим, что хоть языки этих исчислений и отличаются, мы можем построить преобразование
            высказываний этих исчислений друг в друга: приняв $\bot \Rightarrow A\with\neg A$ и $\neg \alpha \Rightarrow (\alpha\rightarrow\bot)$.
            Будем обозначать высказывания в гильбертовском ИИВ обычными греческими буквами,
            а соответствующие им высказывания в ИИВ натурального вывода ---
            буквами с апострофами: $\alpha', \beta', \dots$.

            \begin{enumerate}
                  \item Пусть $\Gamma\vdash_\text{и}\alpha$. Покажите, что $\Gamma\vdash_\text{е}\alpha'$: предложите общую схему
                        перестроения доказательства, постройте доказательства для одного случая базы и одного случая перехода индукции.

                        Общая схема: перебираем шаги доказательства в гильбертовом стиле и добавляем соответствующие шаги в естественном выводе в зависимости от происхождения этого шага.

                        \textbf{База}: рассмотрим случай \(\gamma_1 \in \Gamma\). Тогда дерево вывода \(\gamma_1\) : \begin{prooftree}
                              \infer0{\Gamma \vdash \gamma_1, \gamma_1\in \Gamma}
                        \end{prooftree}

                        \textbf{Переход}: рассмотрим случай MP. Тогда дерево вывода \(\gamma_n\) : \begin{prooftree}
                              \hypo{\dots}
                              \infer1{\Gamma \vdash \gamma_i \to \gamma_k}
                              \hypo{\dots}
                              \infer1{\Gamma \vdash \gamma_i}
                              \infer2{\Gamma \vdash \gamma_k}
                        \end{prooftree}

                  \item Пусть $\Gamma\vdash_\text{е}\alpha'$. Покажите, что $\Gamma\vdash_\text{и}\alpha$.

                        Докажем по индукции по высоте дерева.
                        \begin{proof}\itemfix
                              \textbf{База}: \(n = 1\). Единственный случай --- аксиома. Он очевиден.

                              \textbf{Переход}: \begin{enumerate}
                                    \item \begin{prooftree}
                                                \hypo{\Gamma, \varphi \vdash \psi}
                                                \infer1{\Gamma \vdash \varphi \to \psi}
                                          \end{prooftree}. По индукционному предположению \(\Gamma, \varphi \vdash \psi\) и по теореме об индукции \(\Gamma \vdash \varphi \to \psi\)

                                    \item \begin{prooftree}
                                                \hypo{\Gamma \vdash \varphi}
                                                \hypo{\Gamma \vdash \psi}
                                                \infer2{\Gamma \vdash \varphi \with \psi}
                                          \end{prooftree}. По индукционному предположению \(\Gamma \vdash \varphi, \Gamma \vdash \psi\). Объединим два доказательства, припишем аксиому 3 \(\varphi \to \psi \to \varphi \with \psi\) и применим дважды MP.

                                    \item \begin{prooftree}
                                                \hypo{\Gamma \vdash \varphi \to \psi}
                                                \hypo{\Gamma \vdash \varphi}
                                                \infer2{\Gamma \vdash \psi}
                                          \end{prooftree}. По индукционному предположению \(\Gamma \vdash \varphi \to \psi, \Gamma \vdash \varphi\). Объединим два доказательства, используем MP.
                              \end{enumerate}

                              Прочие случаи аналогичны.
                        \end{proof}
            \end{enumerate}

      \item Рассмотрим $\mathbb{N}_0$ (натуральные числа с нулём) с традиционным отношением порядка как решётку.
            Каков будет смысл операций $(+)$ и $(\cdot)$ в данной решётке, есть ли в ней псевдодополнение,
            определены ли 0 или 1? Приведите несколько свойств традиционных определений $(+)$ и $(\cdot)$,
            которые будут всё равно выполнены при таком переопределении, и несколько свойств, которые перестанут выполняться.

            \[a + b = \max(a, b) \quad a \cdot b = \min(a, b)\]

            Псевдополнения нет для произвольных элементов, т.к. \(\min(a, c) \leq b\) не ограничивает сверху \(c\) для \(a \leq b\). Для \(a \not \leq b\) \(a \to b = b\).

            \(\mathbf 0\) это \(0\), т.к. \(\forall x\in\N_0 \ \ 0 \leq x\)

            \(\mathbf 1\) нет, т.к. \(\mathbf 1 + 1 \not \leq \mathbf 1\)

            Выполнены:
            \begin{enumerate}
                  \item \(a \cdot 0 = 0\)
                  \item \(a + 0 = a\)
                  \item \(a + b + c = a + (b + c)\)
                  \item \(a \cdot b \cdot c = a \cdot (b \cdot c)\)
                  \item \(a + b = b + a\)
                  \item \(a\cdot b = b\cdot a\)
            \end{enumerate}

            Не выполнены:
            \begin{enumerate}
                  \item \((a + b) \cdot c = a\cdot c + b\cdot c\)
            \end{enumerate}

      \item Постройте следующие примеры:
            \begin{enumerate}
                  \item \begin{itemize}
                              \item непустого частично-упорядоченного множества, не имеющего операций $(+)$ и $(\cdot)$ ни для каких элементов;

                                    Такого не существует, т.к. \(\forall a \ \ a \leq a\), следовательно в \(a + a\) все элементы сравнимы с \(a\) и при этом \(a\in(a + a)\). Таким образом, наименьший --- \(a\). Аналогично можно сказать про \(\cdot\).

                              \item имеющего операцию $(+)$ для всех элементов, но не имеющего $(\cdot)$ для некоторых;

                                    Следующий номер, но наоборот.

                              \item имеющего операцию $(\cdot)$ для всех элементов, но не имеющего $(+)$ для некоторых.

                                    \(\{1, 2, 3\}\), упорядоченное по делимости.

                        \end{itemize}


                  \item \begin{itemize}
                              \item решётки, не являющейся дистрибутивной решёткой;

                                    \(\N_0\) со стандартным порядком.

                              \item дистрибутивной, но не импликативной решётки;

                                    \(\Z\) и его конечные подмножества с отношением \(\subset\), т.е. \(\{X\ |\ X\subset \Z, |X|\in\N_0\} \). Дистрибутивность тривиальна из теории множеств, как и то, что это решетка. Нет \(\{0\} \to \Z\), т.к. \(\{c\ |\ \{0\} \cdot c \leq \Z\} \) есть все конечные подмножества, а среди них нет наибольшего.

                              \item импликативной решётки без 0.

                                    \( - \N_0, \leq \)
                        \end{itemize}
            \end{enumerate}

      \item Покажите следующие тождества и свойства для импликативных решёток:
            \begin{enumerate}
                  \item ассоциативность: $a + (b + c) = (a + b) + c$ и $a \cdot (b \cdot c) = (a \cdot b) \cdot c$;

                        Тривиально из теории множеств.

                  \item монотонность: пусть $a \preceq b$ и $c \preceq d$, тогда $a + c \preceq b + d$ и $a \cdot c \preceq b \cdot d$;

                        \(a \leq b \Rightarrow a \leq b + d, c \leq d \Rightarrow c \leq b + d\). Таким образом, \(a + c \leq b + d\).

                        Вторая часть аналогично.

                  \item \emph{Законы поглощения:} $a \cdot (a + b) = a$; $a + (a \cdot b) = a$;

                        \begin{enumerate}
                              \item \(a\cdot (a + b) = a\)

                                    \(a + b\) либо \( = a\), либо \( \leq a\). В обоих случаях \(a\cdot (a + b) = a\)

                              \item \(a + (a \cdot b) = a\)

                                    \(a + b\) либо \( = a\), либо \( \geq a\). В обоих случаях \(a + (a \cdot b) = a\)
                        \end{enumerate}

                  \item $a \preceq b$ выполнено тогда и только тогда, когда $a \rightarrow b = 1$;

                        \[a \to b = 1 \Leftrightarrow 1 \in \{c\ |\ a\cdot c \leq b\} \Leftrightarrow a\cdot 1 \leq b \Leftrightarrow a \leq b\]

                  \item из $a \preceq b$ следует $b\rightarrow c \preceq a\rightarrow c$ и $c\rightarrow a \preceq c \rightarrow b$;

                        \begin{align*}
                              a \cdot (b \to c) & \leq b \cdot (b \to c) \leq c \\
                              b \to c           & \leq a \to c
                        \end{align*}
                        \begin{align*}
                              c \cdot (c \to a) & \leq a \leq b \\
                              c \to a           & \leq c \to b
                        \end{align*}

                  \item из $a \preceq b \rightarrow c$ следует $a \cdot b \preceq c$;

                        \[a \leq b \to c \Rightarrow \exists d : \begin{cases}
                                    d \geq a \\
                                    b\cdot d \leq c
                              \end{cases} \Rightarrow b\cdot a \leq c\]

                        Так как множество, из которого берется \(b\cdot a\) есть подмножество ``\(b\cdot d\)''

                  \item $b \preceq a \rightarrow b$ и $a \rightarrow (b \rightarrow a) = 1$;

                        \begin{align*}
                              b\cdot a & \leq b       \\
                              a\cdot b & \leq b       \\
                              b        & \leq a \to b
                        \end{align*}

                        \(a \leq b \to a\) по пункту d.

                  \item $a \rightarrow b \preceq ((a \rightarrow (b \rightarrow c)) \rightarrow (a \rightarrow c))$;
                  \item $a \preceq b \rightarrow a \cdot b$ и $a \rightarrow (b \rightarrow (a \cdot b)) = 1$
                  \item $a \rightarrow c \preceq (b \rightarrow c) \rightarrow (a + b \rightarrow c)$
            \end{enumerate}

      \item Покажите, что импликативная решётка дистрибутивна.

            Пусть \(d = a \cdot b + a \cdot c\). Рассмотрим \(a \to d\).
            \begin{align}
                  a \cdot b      & \leq d \label{по построению}            \\
                  b              & \leq a \to d \label{по определению to}  \\
                  a \cdot c      & \leq d \label{по построению2}           \\
                  c              & \leq a \to d \label{по определению to2} \\
                  b + c          & \leq a \to d \label{из предыдущих}      \\
                  a\cdot (b + c) & \leq a \cdot (a \to d)                  \\
                                 & \leq d                                  \\
                                 & = a\cdot b + a\cdot c
            \end{align}

            \begin{itemize}
                  \item \eqref{по построению} и \eqref{по построению2}: по построению \(d\)
                  \item \eqref{по определению to} и \eqref{по определению to2}: по определению \( \to \)
                  \item \eqref{из предыдущих}: из \eqref{по определению to} и \eqref{по определению to2}
            \end{itemize}

            Итого \(a\cdot (b + c) \leq a\cdot b + a\cdot c\), покажем, что \(a\cdot (b + c) \geq a\cdot b + a\cdot c\)
            \begin{align}
                  a \cdot b             & \leq a                                  \\
                  a \cdot b             & \leq b \leq b + c                       \\
                  a \cdot b             & \leq a \cdot (b + c) \label{этому}      \\
                  a \cdot c             & \leq a \cdot (b + c) \label{аналогично} \\
                  a \cdot b + a \cdot c & \leq a \cdot (b + c)
            \end{align}
            \begin{itemize}
                  \item \eqref{аналогично}: аналогично \eqref{этому}
            \end{itemize}

      \item Покажите, что в дистрибутивной решётке (всегда $(a + b)\cdot c = (a \cdot c) + (b \cdot c)$) также выполнено
            и $a + (b \cdot c) = (a + b) \cdot (a + c)$.

            \begin{align}
                  (a + b) \cdot (a + c) & = (a + b) \cdot a + (a + b) \cdot c \label{дистр} \\
                                        & = a + (a + b) \cdot c                             \\
                                        & = a + a \cdot c + b \cdot c                       \\
                                        & = a + b \cdot c
            \end{align}

            \begin{itemize}
                  \item \eqref{дистр}: по дистрибутивности
            \end{itemize}

      \item Рассмотрим топологическое пространство $\langle X, \Omega \rangle$, упорядочим его топологию $\Omega$ отношением $\subseteq$.
            Покажите, что такая конструкция является псевдобулевой алгеброй, а если топология --- дискретная (любое подмножество $X$ открыто),
            то булевой алгеброй.

      \item Докажите, что ИИВ корректно, если в качестве модели выбрать псевдобулеву алгебру, а функции оценок определить так:
            $$\begin{array}{ccc}
                        \llbracket\alpha \with \beta\rrbracket       & = & \llbracket\alpha\rrbracket \cdot \llbracket\beta\rrbracket       \\
                        \llbracket\alpha \vee \beta\rrbracket        & = & \llbracket\alpha\rrbracket + \llbracket\beta\rrbracket           \\
                        \llbracket\alpha \rightarrow \beta\rrbracket & = & \llbracket\alpha\rrbracket \rightarrow \llbracket\beta\rrbracket \\
                        \llbracket\neg\alpha\rrbracket               & = & \llbracket\alpha\rrbracket \rightarrow 0                         \\
                        \llbracket\bot\rrbracket                     & = & 0                                                                \\
                  \end{array}$$

      \item Пусть задано отношение \emph{предпорядка} $R$ (транзитивное и рефлексивное, но необязательно антисимметричное) на множестве $A$.
            Напомним несколько определений:
            \begin{itemize}
                  \item определим отношение $R^= := \{ \langle x,y \rangle\ |\ xRy \text{ и } yRx \}$;
                  \item $[a]_{R^=} := \{ x\ |\ aR^=x \}$ --- класс эквивалентности, порождённый элементом $a$;
                  \item фактор-множество $A/{R^=} := \{ [a]_{R^=}\ |\ a \in A \}$;
                  \item на $A/{R^=}$ можно перенести отношение $R^* := \{ \langle [a],[b] \rangle\ |\ aRb \}$.
            \end{itemize}

            Покажите, что: отношение $R^=$ --- отношение эквивалентности; если $x \in [a]_{R^=}$, $y \in [b]_{R^=}$ и $aRb$, то $xRy$; отношение $R^*$ ---
            отношение порядка на $A/{R^=}$.

      \item Покажем, что конструкция из определения алгебры Линденбаума действительно является решёткой:
            \begin{enumerate}
                  \item Покажите, что отношение $(\approx)$ --- отношение эквивалентности (напомним, что $\alpha\preceq\beta$, если $\alpha\vdash\beta$, а $\alpha\approx\beta$, если
                        $\alpha\vdash\beta$ и $\beta\vdash\alpha$). \emph{Подсказка:} воспользуйтесь предыдущим заданием.
                  \item Покажите, что $[\alpha]_\approx\cdot[\beta]_\approx = [\alpha\with\beta]_\approx$. Для этого, например, можно показать:
                        \begin{itemize}
                              \item $\alpha\with\beta \preceq \alpha$;
                              \item если $\gamma \preceq \alpha$ и $\gamma\preceq\beta$, то $\gamma\preceq\alpha\with\beta$;
                              \item операция однозначно определена для всех элементов решётки (то есть определена для всех классов
                                    эквивалентности и не зависит от выбора представителей). \emph{Подсказка:} воспользуйтесь предыдущим заданием.
                        \end{itemize}
                  \item Покажите, что $[\alpha]+[\beta]=[\alpha\vee\beta]$.
                  \item Покажите, что $[\alpha]\rightarrow[\beta]=[\alpha\rightarrow\beta]$.
                  \item Найдите классы эквивалентности для 0 и 1.
            \end{enumerate}
\end{enumerate}

\end{document}