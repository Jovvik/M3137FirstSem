\documentclass[12pt, a4paper]{article}

%<*preamble>
% Math symbols
\usepackage{amsmath, amsthm, amsfonts, amssymb}
\usepackage{accents}
\usepackage{esvect}
\usepackage{mathrsfs}
\usepackage{mathtools}
\mathtoolsset{showonlyrefs}
\usepackage{cmll}
\usepackage{stmaryrd}
\usepackage{physics}
\usepackage[normalem]{ulem}
\usepackage{ebproof}
\usepackage{extarrows}

% Page layout
\usepackage{geometry, a4wide, parskip, fancyhdr}

% Font, encoding, russian support
\usepackage[russian]{babel}
\usepackage[sb]{libertine}
\usepackage{xltxtra}

% Listings
\usepackage{listings}
\lstset{basicstyle=\ttfamily,breaklines=true}
\setmonofont{Inconsolata}

% Miscellaneous
\usepackage{array}
\usepackage{calc}
\usepackage{caption}
\usepackage{subcaption}
\captionsetup{justification=centering,margin=2cm}
\usepackage{catchfilebetweentags}
\usepackage{enumitem}
\usepackage{etoolbox}
\usepackage{float}
\usepackage{lastpage}
\usepackage{minted}
\usepackage{svg}
\usepackage{wrapfig}
\usepackage{xcolor}
\usepackage[makeroom]{cancel}

\newcolumntype{L}{>{$}l<{$}}
    \newcolumntype{C}{>{$}c<{$}}
\newcolumntype{R}{>{$}r<{$}}

% Footnotes
\usepackage[hang]{footmisc}
\setlength{\footnotemargin}{2mm}
\makeatletter
\def\blfootnote{\gdef\@thefnmark{}\@footnotetext}
\makeatother

% References
\usepackage{hyperref}
\hypersetup{
    colorlinks,
    linkcolor={blue!80!black},
    citecolor={blue!80!black},
    urlcolor={blue!80!black},
}

% tikz
\usepackage{tikz}
\usepackage{tikz-cd}
\usetikzlibrary{arrows.meta}
\usetikzlibrary{decorations.pathmorphing}
\usetikzlibrary{calc}
\usetikzlibrary{patterns}
\usepackage{pgfplots}
\pgfplotsset{width=10cm,compat=1.9}
\newcommand\irregularcircle[2]{% radius, irregularity
    \pgfextra {\pgfmathsetmacro\len{(#1)+rand*(#2)}}
    +(0:\len pt)
    \foreach \a in {10,20,...,350}{
            \pgfextra {\pgfmathsetmacro\len{(#1)+rand*(#2)}}
            -- +(\a:\len pt)
        } -- cycle
}

\providetoggle{useproofs}
\settoggle{useproofs}{false}

\pagestyle{fancy}
\lfoot{M3137y2019}
\cfoot{}
\rhead{стр. \thepage\ из \pageref*{LastPage}}

\newcommand{\R}{\mathbb{R}}
\newcommand{\Q}{\mathbb{Q}}
\newcommand{\Z}{\mathbb{Z}}
\newcommand{\B}{\mathbb{B}}
\newcommand{\N}{\mathbb{N}}
\renewcommand{\Re}{\mathfrak{R}}
\renewcommand{\Im}{\mathfrak{I}}

\newcommand{\const}{\text{const}}
\newcommand{\cond}{\text{cond}}

\newcommand{\teormin}{\textcolor{red}{!}\ }

\DeclareMathOperator*{\xor}{\oplus}
\DeclareMathOperator*{\equ}{\sim}
\DeclareMathOperator{\sign}{\text{sign}}
\DeclareMathOperator{\Sym}{\text{Sym}}
\DeclareMathOperator{\Asym}{\text{Asym}}

\DeclarePairedDelimiter{\ceil}{\lceil}{\rceil}

% godel
\newbox\gnBoxA
\newdimen\gnCornerHgt
\setbox\gnBoxA=\hbox{$\ulcorner$}
\global\gnCornerHgt=\ht\gnBoxA
\newdimen\gnArgHgt
\def\godel #1{%
    \setbox\gnBoxA=\hbox{$#1$}%
    \gnArgHgt=\ht\gnBoxA%
    \ifnum     \gnArgHgt<\gnCornerHgt \gnArgHgt=0pt%
    \else \advance \gnArgHgt by -\gnCornerHgt%
    \fi \raise\gnArgHgt\hbox{$\ulcorner$} \box\gnBoxA %
    \raise\gnArgHgt\hbox{$\urcorner$}}

% \theoremstyle{plain}

\theoremstyle{definition}
\newtheorem{theorem}{Теорема}
\newtheorem*{definition}{Определение}
\newtheorem{axiom}{Аксиома}
\newtheorem*{axiom*}{Аксиома}
\newtheorem{lemma}{Лемма}

\theoremstyle{remark}
\newtheorem*{remark}{Примечание}
\newtheorem*{exercise}{Упражнение}
\newtheorem{corollary}{Следствие}[theorem]
\newtheorem*{statement}{Утверждение}
\newtheorem*{corollary*}{Следствие}
\newtheorem*{example}{Пример}
\newtheorem{observation}{Наблюдение}
\newtheorem*{prop}{Свойства}
\newtheorem*{obozn}{Обозначение}

% subtheorem
\makeatletter
\newenvironment{subtheorem}[1]{%
    \def\subtheoremcounter{#1}%
    \refstepcounter{#1}%
    \protected@edef\theparentnumber{\csname the#1\endcsname}%
    \setcounter{parentnumber}{\value{#1}}%
    \setcounter{#1}{0}%
    \expandafter\def\csname the#1\endcsname{\theparentnumber.\Alph{#1}}%
    \ignorespaces
}{%
    \setcounter{\subtheoremcounter}{\value{parentnumber}}%
    \ignorespacesafterend
}
\makeatother
\newcounter{parentnumber}

\newtheorem{manualtheoreminner}{Теорема}
\newenvironment{manualtheorem}[1]{%
    \renewcommand\themanualtheoreminner{#1}%
    \manualtheoreminner
}{\endmanualtheoreminner}

\newcommand{\dbltilde}[1]{\accentset{\approx}{#1}}
\newcommand{\intt}{\int\!}

% magical thing that fixes paragraphs
\makeatletter
\patchcmd{\CatchFBT@Fin@l}{\endlinechar\m@ne}{}
{}{\typeout{Unsuccessful patch!}}
\makeatother

\newcommand{\get}[2]{
    \ExecuteMetaData[#1]{#2}
}

\newcommand{\getproof}[2]{
    \iftoggle{useproofs}{\ExecuteMetaData[#1]{#2proof}}{}
}

\newcommand{\getwithproof}[2]{
    \get{#1}{#2}
    \getproof{#1}{#2}
}

\newcommand{\import}[3]{
    \subsection{#1}
    \getwithproof{#2}{#3}
}

\newcommand{\given}[1]{
    Дано выше. (\ref{#1}, стр. \pageref{#1})
}

\renewcommand{\ker}{\text{Ker }}
\newcommand{\im}{\text{Im }}
\renewcommand{\grad}{\text{grad}}
\newcommand{\rg}{\text{rg}}
\newcommand{\defeq}{\stackrel{\text{def}}{=}}
\newcommand{\defeqfor}[1]{\stackrel{\text{def } #1}{=}}
\newcommand{\itemfix}{\leavevmode\makeatletter\makeatother}
\newcommand{\?}{\textcolor{red}{???}}
\renewcommand{\emptyset}{\varnothing}
\newcommand{\longarrow}[1]{\xRightarrow[#1]{\qquad}}
\DeclareMathOperator*{\esup}{\text{ess sup}}
\newcommand\smallO{
    \mathchoice
    {{\scriptstyle\mathcal{O}}}% \displaystyle
    {{\scriptstyle\mathcal{O}}}% \textstyle
    {{\scriptscriptstyle\mathcal{O}}}% \scriptstyle
    {\scalebox{.6}{$\scriptscriptstyle\mathcal{O}$}}%\scriptscriptstyle
}
\renewcommand{\div}{\text{div}\ }
\newcommand{\rot}{\text{rot}\ }
\newcommand{\cov}{\text{cov}}

\makeatletter
\newcommand{\oplabel}[1]{\refstepcounter{equation}(\theequation\ltx@label{#1})}
\makeatother

\newcommand{\symref}[2]{\stackrel{\oplabel{#1}}{#2}}
\newcommand{\symrefeq}[1]{\symref{#1}{=}}

% xrightrightarrows
\makeatletter
\newcommand*{\relrelbarsep}{.386ex}
\newcommand*{\relrelbar}{%
    \mathrel{%
        \mathpalette\@relrelbar\relrelbarsep
    }%
}
\newcommand*{\@relrelbar}[2]{%
    \raise#2\hbox to 0pt{$\m@th#1\relbar$\hss}%
    \lower#2\hbox{$\m@th#1\relbar$}%
}
\providecommand*{\rightrightarrowsfill@}{%
    \arrowfill@\relrelbar\relrelbar\rightrightarrows
}
\providecommand*{\leftleftarrowsfill@}{%
    \arrowfill@\leftleftarrows\relrelbar\relrelbar
}
\providecommand*{\xrightrightarrows}[2][]{%
    \ext@arrow 0359\rightrightarrowsfill@{#1}{#2}%
}
\providecommand*{\xleftleftarrows}[2][]{%
    \ext@arrow 3095\leftleftarrowsfill@{#1}{#2}%
}

\allowdisplaybreaks

\newcommand{\unfinished}{\textcolor{red}{Не дописано}}

% Reproducible pdf builds 
\special{pdf:trailerid [
<00112233445566778899aabbccddeeff>
<00112233445566778899aabbccddeeff>
]}
%</preamble>


\lhead{Математическая логика \textit{(практика)}}
\cfoot{}
\rfoot{12.3.2021}

\begin{document}

\begin{enumerate}[wide, labelwidth=!, labelindent=0pt]
      \item Рассмотрим табличную модель $\mathfrak{V}$ с $n$ истинностными значениями,
            и с выделенным значением $T$ для истины.
            Покажите, что $$\models\bigvee_{1 \le i\ne j \le n+1} (P_i \rightarrow P_j)\with(P_j\rightarrow P_i)$$
            В частности, покажите, что в любой корректной модели если $\llbracket\alpha\rrbracket = \llbracket\beta\rrbracket$, то
            $\llbracket\alpha\rightarrow\beta\rrbracket = T$; если $\llbracket\gamma\rrbracket = \llbracket\delta\rrbracket = T$, то
            $\llbracket\gamma\with\delta\rrbracket = T$;
            если $\llbracket\gamma\rrbracket = T$, то $\llbracket\gamma\vee\eta\rrbracket = \llbracket\eta\vee\gamma\rrbracket = T$.

            \begin{enumerate}
                  \item \(\llbracket\alpha\rrbracket = \llbracket\beta\rrbracket \Rightarrow \llbracket\alpha\rightarrow\beta\rrbracket = T\)

                        Предпололжим обратное, т.е. \(\exists \alpha, \beta : \llbracket \alpha \rrbracket = \llbracket \beta \rrbracket = A, \llbracket \alpha \to \beta \rrbracket \neq T\).

                        \(\llbracket \alpha \to \beta \rrbracket = f_\to (\llbracket \alpha \rrbracket, \llbracket \beta \rrbracket) = f_{\to}(A, A) \neq T\). Но \(\vdash \alpha \to \alpha\), следовательно по корректности \(\models \alpha \to \alpha\), следовательно \(f_\to (A, A) = T\). Противоречие.

                  \item \(\llbracket\gamma\rrbracket = \llbracket\delta\rrbracket = T \Rightarrow \llbracket\gamma\with\delta\rrbracket = T\)

                        \(f_{\with}(T, T) = T\), т.к. \(\vdash (\alpha \to \alpha) \with (\alpha \to \alpha) \Rightarrow \models (\alpha \to \alpha) \with (\alpha \to \alpha) \Rightarrow T = f_{\with}(\llbracket \alpha \to \alpha \rrbracket, \llbracket \alpha \to \alpha \rrbracket) = f_\with (T, T)\)

                  \item \(T = f_\lor(\llbracket \alpha \to \alpha \rrbracket, \llbracket \beta \rrbracket) = f_\lor(T, A)\), т.к. \(\vdash (\alpha \to \alpha) \lor \beta\). Аналогично для симметричного случая.
            \end{enumerate}

            По принципу Дирихле \(\exists i \neq j : \llbracket P_i \rrbracket = \llbracket P_j \rrbracket \Rightarrow \llbracket P_i \to P_j \rrbracket = T \Rightarrow \llbracket \varphi \rrbracket = T\)

      \item Покажите, что какая бы ни была формула $\alpha$ и модель Крипке,
            если $W_i \Vdash \alpha$ и $W_i \preceq W_j$, то $W_j \Vdash \alpha$.

            Докажем это по индукции по количеству операторов в \(\alpha\)

            \begin{itemize}
                  \item [База:] \(n = 0\), т.е. \(\alpha = P_i\) --- переменная. Искомое верно по определению \(W\).
                  \item [Переход:] 4 случая:
                        \begin{enumerate}
                              \item \(\alpha = \neg \beta\)

                                    \(W_i \Vdash \alpha \Rightarrow \forall W_k: W_i \leq W_k \ \ W_k \nVdash \beta \Rightarrow \forall W_k: W_j \leq W_k \ \ W_k \nVdash \beta \Rightarrow W_j \Vdash \alpha\)

                              \item \(\alpha = \beta \with \gamma\)

                                    \(W_i \Vdash \alpha \Rightarrow W_i \Vdash \beta, W_i \Vdash \gamma\), по индукционному предположению \(W_j \Vdash \beta, W_j \Vdash \gamma \Rightarrow W_j \Vdash \alpha\)

                              \item \(\alpha = \beta \lor \gamma\)

                                    \(W_i \Vdash \alpha \Rightarrow W_i \Vdash \beta\) или \(W_i \Vdash \gamma\), пусть это \(\beta\) \textit{(иначе переименуем)}. Тогда \(W_j \Vdash \beta\) по индукционному предположению \(W_j \Vdash \beta \Rightarrow W_j \Vdash \alpha\)

                              \item \(\alpha = \beta \to \gamma\)

                                    \(W_i \Vdash \alpha \Rightarrow \forall W_j : W_i \leq W_j \ \ W_j \Vdash \beta \Rightarrow W_j \Vdash \gamma\), по транзитивности \( \leq \) выполняется \(W_j \Vdash \alpha\)
                        \end{enumerate}
            \end{itemize}

      \item Общезначимы ли следующие высказывания в ИИВ? Опровергните, построив модель Крипке, или докажите, построив натуральный вывод.
            \begin{enumerate}[wide, labelwidth=!, labelindent=0pt]
                  \item $P \vee \neg P$;

                        Пусть \(P\) --- одна переменная и модель Крипке \(W_1 \nVdash P\), а \(W_2 \Vdash P\) и \(W_1 \leq W_2\). Тогда несложно заметить, что \(W_1 \nVdash \neg P\), т.к. \(\exists W_2 : W_1 \leq W_2, W_2 \Vdash P\). Т.к. \(W_1 \nVdash P, W_1 \nVdash \neg P\), получаем, что \(W_1 \nVdash P \lor \neg P\)

                  \item $\neg\neg P \rightarrow P$;

                        \[W_1 \nVdash \neg\neg P \rightarrow P \Leftarrow \begin{cases}
                                    W_1 \Vdash \neg \neg P \\
                                    W_1 \nVdash P
                              \end{cases} \Leftarrow \begin{cases}
                                    W_1 \nVdash \neg P \\
                                    W_1 \nVdash P
                              \end{cases}\]

                        Пусть \(W_2 \Vdash P\) и \(W_1 \leq W_2\), тогда искомое выполнено.

                  \item $P \vee \neg P \vee \neg\neg P \vee \neg\neg\neg P$;

                        \(W_1 \nVdash P, W_2 \Vdash P, W_3 \Vdash \neg P, W_4 \neg \neg P\), все упорядочены

                  \item $((P \rightarrow Q) \rightarrow P) \rightarrow P$;

                        \(W_1 \Vdash (P \to Q) \to P, W_1 \nVdash P\). Второе выполнено по построению; придумаем, как выполнить первое.

                        \(W_1 \nVdash P \to Q \Leftarrow \begin{cases}
                              W_2 \Vdash P \\
                              W_2 \nVdash Q
                        \end{cases}\). Противоречий не возникло.

                        Ответ: \(W = \{W_1, W_2\}, \leq = \{(W_1, W_2)\}\) \textit{(плюс рефлексивность)}, \(W_2 \Vdash P\)

                  \item $(A \rightarrow B) \vee (B \rightarrow C) \vee (C \rightarrow A)$;

                        \(W_1 \nVdash (A \to B), W_1 \nVdash (B \to C), W_1 \nVdash (C \to A) \Rightarrow \exists W_2, W_3, W_4 : W_2 \Vdash A, W_2 \nVdash B, W_3 \Vdash B, W_3 \nVdash C, W_4 \Vdash C, W_4 \nVdash A\). Если \( \leq = \{(W_1, W_2), (W_1, W_3), (W_1, W_4)\} \) \textit{(плюс рефлексивность)}, то противоречий нет.

                  \item $\neg(\neg A \with \neg B) \rightarrow A \vee B$;

                        \(W_1 \Vdash \neg (\neg A \with \neg B), W_1 \nVdash A\lor B \Rightarrow W_1 \nVdash A, W_1 \nVdash B\)

                        Попробуем выполнить первое утверждение. \(W_1 \nVdash \neg A \with \neg B \Rightarrow W_1 \nVdash \neg A\) или \(W_1 \nVdash \neg B\). Пусть \(W_1 \nVdash \neg A\) без потери общности. \(W_1 \nVdash \neg A \Leftrightarrow \exists W_2 : W_2 \Vdash A\) и \(W_1 \leq W_2\)

                        Ответ: \(W = \{W_1, W_2\}, \leq = \{(W_1, W_2)\}\) \textit{(плюс рефлексивность)}, \(W_1 \nVdash A, W_1 \nVdash B, W_2 \Vdash A\).

                  \item $(\neg A \vee B) \rightarrow (A \rightarrow B)$;

                        \begin{prooftree}
                              \infer0{\neg A \lor B, A, \neg A \vdash A \to \perp}
                              \infer0{\neg A \lor B, A, \neg A \vdash A}
                              \infer2{\neg A \lor B, A, \neg A \vdash \perp}
                              \infer1[(удаление \(\perp\))]{\neg A \lor B, A, \neg A \vdash B}
                              \infer0{\neg A \lor B, A, B \vdash B}
                              \infer0{\neg A \lor B, A \vdash \neg A \lor B}
                              \infer3{\neg A \lor B, A \vdash B}
                              \infer1[(введение \( \to \))]{\neg A \lor B \vdash A \to B}
                              \infer1[(введение \( \to \))]{\neg A \lor B \to A \to B}
                        \end{prooftree}

                  \item $(A \rightarrow B) \rightarrow (\neg A \vee B)$;

                        \[W_1 \nVdash (A \rightarrow B) \rightarrow (\neg A \vee B) \Leftarrow \begin{cases}
                                    W_1 \Vdash A \to B \\
                                    W_1 \nVdash \neg A \lor B
                              \end{cases} \Leftarrow \begin{cases}
                                    W_1 \nVdash A      \\
                                    W_1 \nVdash \neg A \\
                                    W_1 \nVdash B
                              \end{cases}\]

                        Пусть \(W_2 \Vdash A\), \(W_2 \Vdash B\) и \(W_1 \leq W_2\), тогда \(W_2 \Vdash A \to B, W_1 \Vdash A \to B\) пустотно и искомое выполнено.

                  \item $\neg\bot$.

                        \begin{prooftree}
                              \infer0{\perp \vdash \perp}
                              \infer1{\vdash \perp \to \perp}
                              \infer1{\vdash \neg \perp}
                        \end{prooftree}
            \end{enumerate}

      \item Рассмотрим некоторую модель Крипке $\langle\mathfrak{W},\preceq,\Vdash\rangle$.
            Пусть $\Omega = \{ \mathcal{W} \subseteq \mathfrak{W}\ |\ \text{если }W_i \in \mathcal{W}\text{ и }W_i \preceq W_j\text{, то } W_j \in \mathcal{W}\}$.
            Пусть $\mathcal{W}_\alpha := \{ W_i \in \mathfrak{W}\ |\ W_i \Vdash \alpha \}$ (множество миров, где вынуждена формула $\alpha$).
            \begin{enumerate}
                  \item На лекции формулировалась теорема без доказательства, что пара $\langle\mathfrak{W}, \Omega\rangle$ --- топологическое пространство. Докажите её.

                        Покажем, что выполняются три аксиомы топологии:
                        \begin{enumerate}
                              \item \(\mathcal{W} = \bigcup_\alpha \mathcal{W}_\alpha \in \Omega\), где \(\mathcal{W}_\alpha\in\Omega\)

                                    \(\sphericalangle W_i \in \mathcal{W}\) и при этом \(W_i\in \mathcal{W}_\alpha\). Если \(W_i \leq W_j\), то т.к. \(W_i \in \mathcal{W}_\alpha\), то \(W_j \in \mathcal{W}_\alpha\), а следовательно \(W_j \in \mathcal{W}\).

                              \item \(\bigcap_{i = 1}^{n} \mathcal{W}_i \in \Omega\), где \(\mathcal{W}_i \in\Omega\)

                                    \(\sphericalangle W_i \in \mathcal{W} \Rightarrow W_i \in \mathcal{W}_\alpha \ \ \forall \alpha\). Если \(W_i \leq W_j\), то \(W_j \in \mathcal{W}_\alpha \ \ \forall \alpha\) и следовательно \(W_j \in \mathcal{W}\).

                              \item \(\emptyset\in\Omega, \mathfrak{W}\in \Omega\)

                                    Первое выполнено в силу пустотности утверждения \textit{(vacuous, не знаю, как по-русски)}. Второе очевидно выполнено.
                        \end{enumerate}

                  \item Пусть $\mathcal{W}_\alpha$ и $\mathcal{W}_\beta$ --- открытые множества. Выразите $\mathcal{W}_{\alpha\with\beta}$ и $\mathcal{W}_{\alpha\vee\beta}$
                        через $\mathcal{W}_\alpha$ и $\mathcal{W}_\beta$ и покажите, что они также открыты.

                        \(\mathcal{W}_{\alpha \with \beta} = \{W_i \in \mathfrak{W} : W_i \Vdash \alpha \with \beta\} = \{W_i \in \mathfrak{W} : W_i \Vdash \alpha\} \cap \{W_i \in \mathfrak{W} : W_i \Vdash \beta\} = \mathcal{W}_\alpha \cap \mathcal{W}_\beta\).

                        \(\mathcal{W}_{\alpha \lor \beta} = \mathcal{W}_\alpha \cup \mathcal{W}_\beta\)

                        Открытость тривиальна из того, что \((\mathfrak{W}, \Omega)\) --- топологическое пространство.

                  \item Пусть $\mathcal{W}_\alpha$ и $\mathcal{W}_\beta$ --- открытые множества. Выразите $\mathcal{W}_{\alpha\rightarrow\beta}$ через
                        них и покажите, что оно также открыто.

                        \(\mathcal{W}_{\alpha \to \beta} = (\mathfrak{W} \setminus (\mathcal{W}_\alpha \setminus \mathcal{W}_\beta))^\circ\)

                  \item Покажите, что $\Omega$ --- в точности множество всех множеств миров, на которых может быть вынуждена какая-либо формула.
                        А именно, покажите, что для любой формулы $\alpha$ множество миров $\mathcal{W}_\alpha$, где она вынуждена, всегда открыто
                        ($\mathcal{W}_\alpha \in \Omega$) --- и что для любого открытого множества найдётся формула, которая вынуждена ровно на нём
                        (для $Q \in \Omega$ существует формула $\alpha$, что $\mathcal{W}_\alpha = Q$).

                        Покажем, что \(\mathcal{W}_\alpha \in \Omega \ \ \forall \alpha\) по индукции.

                        \begin{itemize}
                              \item [База:] \(\alpha\) есть одна переменная. Искомое выполнено по монотонности вынужденности.
                              \item [Переход:] 4 случая, разобранных в пунктах a,b,c.
                        \end{itemize}

                        Покажем, что \(\forall Q\in \Omega \ \ Q = \mathcal{W}_\alpha\)

                        \textcolor{red}{Не покажем :(}
            \end{enumerate}

      \item Постройте топологическое пространство, соответствующее (в смысле предыдущего задания) модели Крипке, опровергающей
            высказывание $\neg\neg P\rightarrow P$.
            Постройте соответствующую ему табличную модель.

            \(\mathfrak{W} = \{W_1\}, \Omega = \{\emptyset, \{W_1\}\} \)

      \item Назовём \emph{древовидной} моделью Крипке модель, в которой множество
            миров $\mathfrak{W}$ упорядочено как дерево: (a) существует наименьший мир
            $W_0$; (b) для любого $W_i \ne W_0$ существует единственный предшествующий мир
            $W_k: W_k \prec W_i$.
            \begin{enumerate}
                  \item Докажите, что любое высказывание, опровергаемое моделью Крипке, может
                        быть опровергнуто древовидной моделью Крипке.

                  \item Найдите высказывание, которое не может быть опровергнуто древовидной моделью Крипке
                        высотой менее 2.
                  \item Покажите, что для любого натурального $n$ найдётся опровержимое в моделях Крипке высказывание,
                        неопровергаемое никакой моделью с $n$ мирами.
            \end{enumerate}

      \item Будем говорить, что топологическое пространство $\langle X, \Omega\rangle$ \emph{связно}, если нет таких
            открытых множеств $A$ и $B$, что $X = A \cup B$, но $A \cap B = \varnothing$. Пусть задана некоторая модель Крипке.
            Докажите, что соответствующее модели Крипке топологическое пространство связно тогда и только тогда, когда её граф
            миров связен в смысле теории графов.

            Если граф не связен, то выберем одну КС, все миры в ней будут \(A\), а остальные миры \(B\). Несложно заметить, что \(X = A\cup B, A\cap B = \emptyset\).

            Если пространство не связно, то рассмотрим \(A\) и \(B\) из условия. Предположим, что граф связен, в частности есть ребро \(a \to b\), где \(a\in A, b\in B\) \textit{(или наоборот)}. Т.к. \(a \leq b\), то \(b\in A\), что противоречит \(A\cap B =\emptyset\).

      \item Покажите, что модель Крипке с единственным миром задаёт классическую модель (в ней выполнены
            все доказуемые в КИВ высказывания).

      \item Пусть заданы алгебры Гейтинга $\mathcal{A},\mathcal{B}$, гомоморфизм $\varphi: \mathcal{A} \rightarrow \mathcal{B}$
            и согласованные оценки $\llbracket\rrbracket_\mathcal{A}$ и $\llbracket\rrbracket_\mathcal{B}$:
            $\varphi(\llbracket\alpha\rrbracket_\mathcal{A}) = \llbracket\alpha\rrbracket_\mathcal{B}$.
            \begin{enumerate}
                  \item Покажите, что гомоморфизм сохраняет порядок: если $a_1\preceq a_2$, то $\varphi(a_1) \preceq \varphi(a_2)$.
                  \item Покажите, что если $\llbracket \alpha \rrbracket_\mathcal{A} = 1_\mathcal{A}$, то $\llbracket\alpha\rrbracket_\mathcal{B} = 1_\mathcal{B}$.

                        \(\llbracket \alpha \rrbracket_{\mathcal{B}} = \varphi(\llbracket \alpha \rrbracket_{\mathcal{A}}) = \varphi(1_{\mathcal{A}}) = 1_{\mathcal{B}}\)
            \end{enumerate}

      \item Пусть заданы алгебры Гейтинга $\mathcal{A},\mathcal{B}$. Всегда ли можно построить гомоморфизм $\varphi: \mathcal{A}\rightarrow\mathcal{B}$?

      \item Пусть $\mathcal{A}$ --- алгебра Гейтинга. Покажите, что $\Gamma(\mathcal{A})$ --- алгебра Гейтинга и гёделева алгебра.

      \item Пусть $\mathcal{A}$ --- булева алгебра. Всегда ли (возможно ли, что) $\Gamma(\mathcal{A})$ будет булевой алгеброй?

            Не всегда, например в алгебре \(\Gamma(1 \to 0)\) выполняется следующее: \(1 + (1 \to 0) = 1\), а должно быть \(1_\Gamma\)
\end{enumerate}

\end{document}