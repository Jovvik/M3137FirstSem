\chapter{17 февраля}

\subsection{Метод золотого сечения}

Рассмотрим отрезок \([0, 1]\). Пусть \(x_2 = \tau\), тогда симметрично расположенная \(x_1 = 1 - \tau\). Пусть дальше был выбран отрезок \([0, \tau]\), тогда пусть \(x_2' = 1 - \tau\). Чтобы новые точки делили отрезок в таком же соотношении, необходимо, чтобы \(\frac{1}{\tau} = \frac{\tau}{1 - \tau} \Rightarrow \tau^2 = 1 - \tau \Rightarrow \tau = \frac{\sqrt{5} - 1}{2} \approx 0.61803\). Таким образом, \(x_1 = 1 - \tau = \frac{3 - \sqrt{5}}{2}, x_2 = \tau = \frac{\sqrt{5} - 1}{2}\)

В общем случае для отрезка \([a, b]\):
\begin{equation}
    x_1 = a + \frac{3 - \sqrt{5}}{2}(b - a), x_2 = a + \frac{\sqrt{5} - 1}{2} (b - a) \label{x_1, x_2, метод золотого сечения}
\end{equation}

Вычислим погрешность:
\[\Delta_n = \tau^n (b - a) \quad \varepsilon_n = \frac{\Delta_n}{2} = \frac{1}{2}\left( \frac{\sqrt{5} - 1}{2} \right)^n (b - a)\]

Для заданного \(\varepsilon\) условия окончания \(\varepsilon_n \leq \varepsilon\).

Результат метода:
\[x^* = \frac{a_{(n)} + b_{(n)}}{2}\]

Оценка числа шагов для достижения искомой точности:
\[n \geq \ln\left( \frac{\frac{2\varepsilon}{b - a}}{\ln \tau} \right) \approx 2 \cdot 1 \cdot \ln\left( \frac{b - a}{2\varepsilon} \right)\]

\begin{itemize}
    \item [Шаг 1:] Находим \(x_1\) и \(x_2\) по формуле \eqref{x_1, x_2, метод золотого сечения}, вычисляем \(f(x_1)\) и \(f(x_2)\). \(\varepsilon_n = \frac{b - a}{2}, \tau = \frac{\sqrt{5} - 1}{2} \).
    \item [Шаг 2:]
          \begin{itemize}
              \item Если \(\varepsilon_n > \varepsilon\), переходим к шагу 3.
              \item Если \(\varepsilon_n \leq \varepsilon\), переходим к шагу 4.
          \end{itemize}
    \item [Шаг 3:] Сравниваем \(f(x_1)\) и \(f(x_2)\).
          \begin{itemize}
              \item Если \(f(x_1) \leq f(x_2)\), то \(b = x_2, x_2 = x_1, x_1 = b - \tau (b - a)\). Мы запоминаем \(f(x_2)\) для следующего шага, т.к. оно равно \(f(x_1)\) на этом шаге.
              \item Иначе \(a = x_1, x_1 = x_2, f(x_1) = f(x_2)\). Мы запоминаем \(f(x_1)\) для следующего шага, т.к. оно равно \(f(x_2)\) на этом шаге.
          \end{itemize}
    \item [Шаг 4:] \(X^* \approx \overline X = \frac{a_{(n)} + b_{(n)}}{2}\)
\end{itemize}

\subsection{Метод Фибоначчи}

Мы знаем, что \(F_n = \frac{\left( \frac{1 + \sqrt{5}}{2} \right)^n - \left( \frac{1 - \sqrt{5}}{2} \right)^n}{\sqrt{5}} \), а также при \(n \to +\infty\ \) \(F_n \approx \frac{\left( \frac{1 + \sqrt{5}}{2} \right)^n}{\sqrt{5}}\)

Рассмотрим нулевую итерацию:
\[x_1 = a + \frac{F_n}{F_{n + 2}} (b - a) \quad x_2 = a + \frac{F_{n+1}}{F_{n + 2}} (b - a)\]

Рассмотрим \(k\)-тую итерацию:
\[x_1 = a_{(k)} + \frac{F_{n - k + 1}}{F_{n - k + 3}} (b_k - a_k) = a_k + \frac{F_{n - k + 1}}{F_{n + 2}} (b_0 - a_0)\]
\[x_2 = a_{(k)} + \frac{F_{n - k + 2}}{F_{n - k + 3}} (b_k - a_k) = a_k + \frac{F_{n - k + 2}}{F_{n + 2}} (b_0 - a_0)\]

Пусть \(k = n\), тогда:
\[x_1 = a_n + \frac{F_1}{F_{n + 2}} (b_0 - a_0) \quad x_2 = a_n + \frac{F_2}{F_{n + 2}} (b_0 - a_0)\]

Условие на погрешность:
\[\frac{b_n - a_n}{2} = \frac{b_0 - a_0}{F_{n + 2}} < \varepsilon\]
Какое брать \(n?\) Такое, что \(\frac{b_0 - a_0}{\varepsilon} < F_{n + 2}\)

Есть проблема, при большом \(n\) \(\frac{F_n}{F_{n + 2}}\) есть бесконечная десятичная дробь, вследствие чего образуется погрешность.

\subsection{Метод парабол}

\begin{figure}[h]
    \centering
    \includesvg{images/метод_параболы.svg}
    \caption{Функция \(f(x)\) и её приближение параболой.}
\end{figure}

Пусть \(\exists x_1, x_2, x_3\in[a, b]\), такие что \(\begin{cases}
    x_1 < x_2 < x_3 \\
    f(x_1) \geq f(x_2) \leq f(x_3)
\end{cases}\)

Тогда приближающая парабола имеет вид \(q(x) = a_0 + a_1(x - x_1) + a_2(x - x_1)(x - x_2)\). Мы имеем условия на коэффициенты этой параболы: \(\begin{cases}
    q(x_1) = f(x_1) = f_1 \\
    q(x_2) = f(x_2) = f_2 \\
    q(x_3) = f(x_3) = f_3
\end{cases}\)

Коэффициенты можно найти следующим образом:
\[a_0 = f_1 \quad a_1 = \frac{f_2 - f_1}{x_2 - x_1} \quad a_2 = \frac{1}{x_3 - x_2} \left( \frac{f_3 - f_1}{x_3 - x_1} - \frac{f_2 - f_1}{x_2 - x_1} \right) \]

Тогда результат итерации есть \(\overline x = \frac{1}{2} \left( x_1 + x_2 - \frac{a_1}{a_2} \right)\), на следующей лекции будет рассказан переход к следующей итерации.

Точки \(x_1, x_2, x_3\) для новой итерации выбираются следующим образом:
\begin{enumerate}
    \item \begin{enumerate}
              \item Если \(x_1 < \overline x < x_2 < x_3\) и \(f(\overline x) \geq f(x_2)\), то \(x^* \in [\overline x, x_3], x_1 = \overline x\), точки \(x_2\) и \(x_3\) не меняются.
              \item Если \(x_1 < \overline x < x_2 < x_3\) и \(f(\overline x) < f(x_2)\), то \(x^* \in [x_1, x_2], x_3 = x_2, x_2 = \overline x\), точка \(x_1\) не меняется.
          \end{enumerate}
    \item \begin{enumerate}
              \item Если \(x_1 < x_2 < \overline x < x_3\) и \(f(\overline x) \leq f(x_2)\), то \(x^* \in [x_2, x_3], x_1 = x_2, x_2 = \overline x\), точка \(x_3\) не меняется.
              \item Если \(x_1 < x_2 < \overline x < x_3\) и \(f(\overline x) > f(x_2)\), то \(x^* \in [x_1, \overline x], x_3 = \overline x\), точки \(x_1\) и \(x_2\) не меняются.
          \end{enumerate}
\end{enumerate}

\begin{remark}
    Метод парабол имеет квадратичную сходимость.
\end{remark}

\begin{remark}
    Метод парабол требует гладкость функции, что неверно для предыдущих методов.
\end{remark}

\subsection{Комбинированный метод Брента}

Для собственного изучения.
