\chapter{3 марта}

Пусть \(x_k\) --- текущая оценка решения \(x^*\)

Рассмотрим ряд Тейлора:
\[f(x_k + p) = f(x_k) + pf'(x_k) + \frac{1}{2}p^2  f''(x_k) + \dots \]
\begin{align*}
    f(x*) & = \min_x f(x)                                                               \\
          & = \min_p f(x_k + p)                                                         \\
          & = \min_p \left( f(x_k) + pf'(x_k) + \frac{1}{2}p^2 f''(x_k) + \dots \right) \\
          & \approx \min_p \left( f(x_k) + pf'(x_k) + \frac{1}{2}p^2 f''(x_k) \right)   \\
\end{align*}

Приравняем производную выражения под \(\min\) к нулю:
\[f'(x_k) + pf''(x_k) = 0\]
\[p = -\frac{f'(x_k)}{f''(x_k)} \]

Тогда \(x^* \approx x_k + p\) и \(x_{k+1} = x_k + p = x_k - \frac{f'(x_k)}{f''(x_k)} \)

Главное преимущество метода Ньютона --- квадратичная скорость сходимости, т.е. если \(x_k\) достаточно близка к \(x^*\) и \(f''(x^*) > 0\), то \(|x_{k+1} - x^*| \leq \beta|x_k - x^*|^2\)

Метод Ньютона может потерпеть неудачу в следующих случаях:
\begin{enumerate}
    \item \(f(x)\) плохо аппроксимируется первыми тремя членами в ряде Тейлора. Тогда \(x_{k+1}\) может быть хуже \textit{(как аппроксимация)} \(x_k\).
    \item \(f''(x_k) = 0\), тогда \(p\) не определен.
    \item Кроме \(f\) нужно вычислять \(f'\) и \(f''\), что затруднительно в реальных задачах.
\end{enumerate}

Мы можем аппроксимировать производную по определению:
\[f'(x_k) \approx \frac{f(x_k + h) - f(x_k)}{h} \]
Эта формула называется правой разностной схемой, у нее есть улучшение, называемое центральной разностной схемой:
\[f'(x_k) \approx \frac{f(x_k + h) - f(x_k - h)}{2h} \]

Если \(f(x)\) --- квадратичная функция, то метод Ньютона сходится за один шаг при любом выборе \(x_0\).

\subsubsection{Достаточное условие монотонной сходимости метода Ньютона}

Пусть \(x^* \in [a, b]\) и \(f(x)\) трижды непрерывно дифференцируемая и выпуклая на \([a, b]\) функция. Тогда \(\{x_k\}\) будет сходиться к пределу \(x^*\) монотонно, если \(0 < \frac{x^* - x_{k+1}}{x^* - x_k} < 1\)

\[f'(x^*) = 0 = f'(x_k) + f''(x_k)(x^* - x_k) + \frac{f'''(x)}{2} (x^* - x_k)^2\]
\[\frac{x^K - x_{k+1}}{x^* - x_k} = \frac{x^* - x_k + \frac{f'(x_k)}{f''(x_k)}}{x^* - x_k} = 1 - \frac{2}{2 + \frac{f'''(x) (x^* - x_k)^2}{f'(x_k)}}\]
Последовательность итераций \(\{x_k\}\) монотонна, если \(\frac{f'''(x)}{f'(x_k)} > 0\), таким образом условие монотонной сходимости метода Ньютона --- постоянство на \(x\in [x^*, x_0]\) знака \(f'''(x)\) и его совпадение с \(f'(x_0)\).

\begin{example}
    \(f(x) = x \cdot \arctg (x) - \frac{1}{2} \?\)

    \[f'(x) = \arctg x \quad f''(x) = \frac{1}{1 + x^2} > 0 \quad f'''(x) = - \frac{2x}{(1 + x^2)^2}\]

    \(f'(x) \cdot f'''(x) < 0\), таким образом не будет монотонной сходимости.

    Пусть \(x_0 = 1\).

    \begin{tabular}{c|c|c|c}
        \(k\) & \(x_k\)              & \(f'(x_k)\)  & \(f''(x_k)\)    \\ \hline
        \(0\) & \(1\)                & \(0.785\)    & \(\frac{1}{2}\) \\
        \(1\) & \( - 0.57\)          & \( - 0.518\) & \(a\)           \\
        \(2\) & \(0.117\)            & \( 0.\)      &                 \\
              &                      &              &                 \\
        \(4\) & \(9\cdot 10^{ - 8}\) &
    \end{tabular}
\end{example}

\subsection{Модификации метода Ньютона}

\subsubsection{Метод Ньютона-Рафсона}

\[x_{k+1} = x_k - \tau_k \frac{f'(x_k)}{f''(x_k)}, 0 < \tau_k \leq 1\]

\(\tau_k\) --- константы. Если \(\tau = 1\), то метод Ньютона-Рафсона вырождается в метод Ньютона.

Для нахождения \(\tau_k\) зададим \(\varphi(\tau)\):

\[\varphi(\tau) = f(x_k - \tau \frac{f'(x_k)}{f''(x_k)} ) \to \min\]

Тогда \[\tau_k = \frac{(f'(x_k))^2}{(f'(x_k))^2 + (f'(\tilde{x}))^2} \text{ , где } \tilde{x} = x_k - \frac{f'(x_k)}{f''(x_k)}  \]

\subsubsection{Метод Марквардта}

\[x_{k+1} = x_k - \frac{f'(x_k)}{f''(x_k) + \mu_k} \]
, где \(\mu_k > 0\)

\(\mu_0\) выбирают на порядок выше значения \(f''(x_0)\), \(\mu_{k + 1} = \begin{cases} \frac{m_k}{2} & , \text{ если } f(x_{k+1}) < f(x_k) \\ \mu_{k+1} = 2 \mu_k & , \text{ если } f(x_{k+1}) \geq f(x_k) \end{cases}\)

\section{Метод минимизации многомодальных функций \textit{(метод ломаных)}}

\begin{definition}
    \(f(x), x\in[a, b]\) \textbf{удовлетворяет условию Липшица}, если \(\forall x_1, x_2\in[a, b] \ \ |f(x_1) - f(x_2)| \leq L |x_1 - x_2|\)
\end{definition}

\begin{itemize}
    \item [Шаг 1] Возьмём \(x_1^* = \frac{1}{2L}(f(a) - f(b) + L(a + b))\) и \(p_1^* = \frac{1}{2}(f(a) + f(b) + L(a - b))\). Добавим в рассматриваемое множество \(x_1' = x_1^* - \Delta_1\) и \(x_1'' = x_1^* + \Delta_1\), где \(\Delta_1 = \frac{1}{2L} (f(x_1^*) - p_1)\)
    \item [Шаг 2] Ииз пар \((x_1', p_1)\) и \((x_1'', p_1)\) выберем пару с минимальной \(p\) : \((x_2^*, p_2^*)\) и исключим из рассматриваемого множества.
    \item [Шаг \(n\)] В результате мы получим множество из \(n\) пар \((x, p)\). Исключаем пару с минимальной \(p\) и вместо неё
\end{itemize}

\begin{example}
    \(f(x) = \frac{\sin x}{x} \), \([a, b] = [10, 15]\), \(\varepsilon = 0.01\)

    Проверим условие Липшица:
    \[|f'(x)| = \left|\frac{x\cos x - \sin x}{x^2} \right| < \frac{x |\cos x| + \sin |x|}{x^2} < \frac{x + 1}{x^2} \leq 0.11\]

    \begin{tabular}{c|c|c|c|c|c|c}
        \(n\)  & \(x_n^*\)  & \(p_n^*\)    & \(2L \Delta_n\)         & \(x_n'\)   & \(x_n''\)  & \(p_n\)      \\ \hline
        \(1\)  & \(12.056\) & \( - 0.281\) & \(0.240\)               & \(10.963\) & \(13.149\) & \( - 0.161\) \\
        \(2\)  & \(10.963\) & \( - 0.161\) & \(0.070\)               & \(10.646\) & \(11.280\) & \( - 0.126\) \\
        \(3\)  & \(13.149\) & \( - 0.161\) & \(0.203\)               & \(12.227\) & \(14.701\) & \( - 0.096\) \\
        \(4\)  & \(10.646\) & \( - 0.126\) & \(0.038\)               & \(10.474\) & \(10.818\) & \( - 0.107\) \\
        \(5\)  & \(11.280\) & \( - 0.126\) & \(0.041\)               & \(11.094\) & \(11.466\) & \( - 0.106\) \\
        \(6\)  & \(10.474\) & \( - 0.107\) & \(0.024\)               & \(10.364\) & \(10.584\) & \( - 0.095\) \\
        \(7\)  & \(10.818\) & \( - 0.107\) & \(0.160\)               & \(10.745\) & \(10.891\) & \( - 0.099\) \\
        \(8\)  & \(11.094\) & \( - 0.106\) & \(0.016\)               & \(11.020\) & \(11.168\) & \( - 0.098\) \\
        \(9\)  & \(11.466\) & \( - 0.106\) & \(0.028\)               & \(11.338\) & \(11.594\) & \( - 0.092\) \\
        \(10\) & \(10.891\) & \( - 0.099\) & \(0.008 < \varepsilon\) &            &            &              \\
    \end{tabular}
\end{example}

