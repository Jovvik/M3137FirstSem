\chapter{13 февраля}

\section{События}

%<*2.1>
\subsection{Статистическое определение вероятности}

\begin{definition}
    Пусть проводится \(n\) реальных экспериментов, событие \(A\) произошло в \(n_A\) экспериментах. Отношение \(\cfrac{n_A}{n}\) называется \textbf{частотой события} \(A\). Эксперименты показывают, что при увеличении числа \(n\) эта частота ``стабилизируется'' около некоторого числа, под которым понимаем \textbf{статистическую вероятность}.
    \[P(A)\approx \frac{n_A}{n}\]
\end{definition}

Очевидно это определение не формально, поэтому мы им пользоваться не будем.
%</2.1>

%<*1>
\subsection{Пространство элементарных исходов. Случайные события.}

\begin{definition}\itemfix
    \begin{itemize}
        \item \textbf{Пространством элементарных исходов} \(\Omega\) называется множество, содержащее все возможные результаты данного эксперимента, из которых при испытании происходит ровно один.
        \item Элементы данного множества называются \textbf{элементарными исходами} и обозначаются \(w\in \Omega\).
        \item \textbf{Случайными событиями} называются подмножества \(A\subset \Omega\).
        \item Событие \(A\) \textbf{наступило}, если в ходе эксперимента произошёл один из элементарных исходов, входящих в \(A\).
        \item Такие исходы называются \textbf{благоприятными} к \(A\).
    \end{itemize}
\end{definition}

\begin{example}\itemfix
    \begin{enumerate}
        \item Бросают монетку. \(\Omega = \{\text{Г}, \text{Р}\} \) \textit{(герб, решка)}.
        \item Бросают кубик. \(\Omega = \{1,2,3,4,5,6\} \). \(A\) --- выпало четное число очков. Тогда \(A = \{2,4,6\} \).
        \item Монета бросается дважды:
              \begin{enumerate}
                  \item Учитываем порядок: \(\Omega = \{\text{ГГ, РР, ГР, РГ}\} \)
                  \item Не учитываем порядок: \(\Omega = \{\text{ГГ, РР, ГР}\} \)
              \end{enumerate}
        \item Бросается дважды кубик, порядок учитывается. \(A\) --- разность очков делится на 3, т.е. \(A = \{(1,4), (4,1), (3,3), (5,2), (2,5), (3, 6), (6, 3), (1, 1), (2, 2), (4, 4), (5, 5), (6, 6)\} \)
        \item Монета бросается до выпадения герба. \(\Omega = \{\text{Г, РГ, РРГ, \dots}\} \) --- счётное число исходов.
        \item Монета бросается на плоскость. \(\Omega = \{(x, y) \ | \ x, y\in\R\} \) --- несчётное число исходов.
    \end{enumerate}
\end{example}

\subsection{Операции над событиями}

\(\Omega\) --- универсальное \textit{(достоверное)} событие, т.к. содержит все элементарные исходы.

\(\emptyset\) --- невозможное событие.

\begin{definition}
    \(A + B\) это \(A\cup B\)

    \begin{figure}[h]
        \centering
        \includesvg[scale=0.5]{images/union.svg}
    \end{figure}
\end{definition}

\begin{definition}
    \(A\cdot B\) это \(A\cap B\)

    \begin{figure}[h]
        \centering
        \includesvg[scale=0.5]{images/intersection.svg}
    \end{figure}
\end{definition}

\begin{definition}
    \textbf{Противоположным} к \(A\) называется событие \(\overline A\), соответствующее тому, что \(A\) не произошло, т.е. \(\Omega\setminus A\)
\end{definition}

\begin{definition}
    Дополнение \(A\setminus B\) это \(A\cdot \overline B\)
\end{definition}

\begin{definition}
    События \(A\) и \(B\) называются \textbf{несовместными}, если \(A\cdot B = \emptyset\)
\end{definition}

\begin{definition}
    Событие \(A\) влечет событие \(B\), если \(A\subset B\).
\end{definition}
%</1>

\section{Вероятность}

\begin{definition}
    \(0 \leq P(A) \leq 1\) --- вероятность наступления события \(A\).
\end{definition}

%<*2.2>
\subsection{Классическое определение вероятности}

Пусть \(\Omega\) содержит конечное число исходов, причем их можно считать равновозможными. Тогда применимо классическое определение вероятности.

\(P(A) = \cfrac{|A|}{|\Omega|} = \cfrac{m}{n}\), где \(n\) --- число всех возможных элементарных исходов, \(m\) --- число элементарных исходов, благоприятных \(A\).

В частности, если \(|\Omega|= n\), а \(A\) --- элементарный исход, то \(P(A) = \frac{1}{n}\).

\begin{prop}\itemfix
    \begin{enumerate}
        \item \(0 \leq P(A) \leq 1\)
        \item \(P(\emptyset) = 0\)
        \item \(P(\Omega) = 1\)
        \item Если \(A\) и \(B\) несовместны, то \(P(A + B) = P(A) + P(B)\)
    \end{enumerate}
\end{prop}

\begin{proof}
    \(|A| : = m_1, |B| : = m_2, |A\cup B|= m_1 + m_2\)

    \[P(A + B) = \frac{m_1 + m_2}{n} = \frac{m_1}{n} + \frac{m_2}{n} = P(A) + P(B)\]
\end{proof}

\begin{example}
    Найти вероятность, что при бросании кости выпадет чётное число очков.

    \[n = 6, m = 3, \frac{m}{n} = \frac{1}{2}\]
\end{example}

\begin{example}
    В ящике лежат 3 белых и 2 чёрных шара. Вынули 3 шара. Найти вероятность того, что из них две белых и один чёрный.

    \[n = \binom{5}{3} = 10\]
    \[m = \binom{3}{2} \binom{2}{1} = 6\]
    \[P(A) = \frac{6}{10}\]
\end{example}

Однако, это определение редко применимо.
%</2.2>

\subsection{Геометрическое определение вероятности}

%<*3>
\begin{definition}\itemfix
    \begin{itemize}
        \item \(\Omega \subset \R^n\) --- замкнутая ограниченная область.
        \item \(\mu\) --- конечная мера множества \(\Omega\), например мера Лебега
    \end{itemize}

    Пусть выбирают точку наугад, т.е. вероятность попадания точки в область \(A\) зависит от меры \(A\), но не от её положения.

    Тогда
    \(P(A) = \cfrac{\mu(A)}{\mu(\Omega)}\)
\end{definition}

\begin{remark}
    По этому определению мера точки равна \(0\) и вероятность попадания в конкретную точку тоже равна \(0\).
\end{remark}

\begin{example}
    Монета диаметром 6 сантиметров бросается на пол, вымощенный квадратной плиткой со стороной 20 сантиметров. Найти вероятность того, что монета целиком окажется на одной плитке.

    Без ущерба для общности можно рассматривать, что монета бросается на одну плитку и положение монеты определяется положением её центра.

    Чтобы монета лежала полностью на одной плитке, необходимо, чтобы её центр лежал на расстоянии \( \geq 3\) сантиметра от каждой стороны:

    \begin{figure}[h]
        \centering
        \includesvg[scale=0.8]{\detokenize{images/задача-1.svg}}
    \end{figure}

    \[S(\Omega) = 20^2 = 400\]
    \[S(A) = 14^2 = 196\]
    \[P(A) = \frac{196}{400} = 0.49\]
\end{example}

\begin{example}[задача Бюффона]
    Пол вымощен ламинатом. На пол бросается игла длинной, равной ширине доски. Найти вероятность того, что она пересечёт стык.

    Пусть \(l\) --- длина иглы, \(x\) --- расстояние от центра иглы до ближайшего края. Положение иглы определяется центром и углом поворота \(\varphi\).

    \[A : x \leq l \sin \varphi\]
    \[x \in [0, l] \quad \varphi \in [0, \pi]\]
    \[S(\Omega) = \pi l\]
    \[S(A) = \int_0^\pi l \sin \varphi d\varphi = - l \cos \varphi \Big|_0^\pi = - l(\cos \pi - \cos 0) = 2l\]
    \[P(A) = \frac{S(A)}{S(\Omega)} = \frac{2}{\pi}\]
\end{example}

Это определение кажется хорошим --- оно согласовано с классическим. Но и это определение редко применимо на практике, т.к. обычно вероятность зависит от положения в пространстве или событие неизмеримо.
%</3>
