\chapter{20 февраля}

\subsection{Аксиоматическое определение вероятности}

%<*4>
Пусть \(\Omega\) --- пространство элементарных исходов.

\begin{definition}
    Систему \(\mathcal{F}\) подмножеств \(\Omega\) называют \(\sigma\)-алгеброй событий, если:
    \begin{enumerate}
        \item \(\Omega\in\mathcal{F}\)
        \item \(A\in \mathcal{F} \Rightarrow \overline A \in \mathcal{F}\)
        \item \(A_1 \dots A_n \dots \in \mathcal{F} \Rightarrow \bigcup\limits_{i = 1}^{+\infty} A_i \in \mathcal{F}\)
    \end{enumerate}
\end{definition}

\begin{remark}
    Из 2 и 3 следует 1.
\end{remark}

\begin{prop}\itemfix
    \begin{enumerate}
        \item \(\emptyset \in \mathcal{F}\), т.к. \(\overline \Omega = \emptyset\)
        \item \(A_1 \dots \in \mathcal{F} \Rightarrow \bigcap A_i \in \mathcal{F}\)

              \begin{proof}
                  \(A_1 \dots \in \mathcal{F} \Rightarrow \overline A_1 \dots \in \mathcal{F} \Rightarrow \bigcup \overline A_i \in \mathcal{F} \Rightarrow \overline{\bigcup \overline A_i} = \bigcap A_i \in \mathcal{F}\)
              \end{proof}

        \item \(A, B \in \mathcal{F} \Rightarrow A\setminus B \in \mathcal{F}\)
    \end{enumerate}
\end{prop}

\begin{example}\itemfix
    \begin{enumerate}
        \item \(\mathcal{F} = \{\Omega, \emptyset\}\)
        \item \(\mathcal{F} = \{\Omega, \emptyset, A, \overline{A}\}\)
    \end{enumerate}
\end{example}

\begin{definition}
    Пусть \(\Omega\) --- множество элементарных исходов, \(\mathcal{F}\) --- \(\sigma\)-алгебра над ним.

    \textbf{Вероятностью} на \((\Omega, \mathcal{F})\) называется функция \(P(A) : \mathcal{F} \to \R\) со свойствами:
    \begin{enumerate}
        \item \(P(A) \geq 0\) --- свойство неотрицательности
        \item Если события \(A_1 \dots A_n \dots \) --- попарно несовместны, т.е. \(\forall i \neq j \ \ A_i \cap A_j = \emptyset\), то \(P(\bigcup A_i) = \sum P(A_i)\) --- свойство счётной аддитивности
        \item \(P(\Omega) = 1\) --- свойство нормированности
    \end{enumerate}
\end{definition}

\begin{remark}
    Вероятность есть нормированная мера.
\end{remark}

\begin{definition}
    Тройка \((\Omega, \mathcal{F}, P)\) называется вероятностным пространством.
\end{definition}

\begin{prop}\itemfix
    \begin{enumerate}
        \item \(P(\emptyset) = 0\)

              \begin{proof}
                  \(\underbrace{P(\emptyset + \Omega)}_{1} = P(\emptyset) + \underbrace{P(\Omega)}_1 \Rightarrow P(\emptyset) = 0\)
              \end{proof}

        \item Формула обратной вероятности: \(P(A) = 1 - P(\overline A)\)

              \begin{proof}
                  \(A\) и \(\overline A\) --- несовместны, \(A + \overline A = \Omega\).

                  \(P(A + \overline A) = P(A) + P(\overline A) = 1 \Rightarrow P(A) = 1 - P(\overline A)\)
              \end{proof}

        \item \(0 \leq P(A) \leq 1\)

              \begin{proof}
                  \begin{enumerate}
                      \item \(P(A) \geq 0\)
                      \item \(P(A) = 1 - P(\overline A) \leq 1\)
                  \end{enumerate}
              \end{proof}
    \end{enumerate}
\end{prop}
%</4>

\subsection{Аксиома непрерывности}

%<*5>
Пусть имеется убывающая цепочка событий \(A_1 \supset A_2 \supset \dots \) и \(\bigcap A_i = \emptyset\). Тогда \(P(A_n) \xrightarrow{n \to +\infty} 0\).

Смысл: при непрерывном изменении области \(A \subset \R^n\) соответствующая вероятность также должна изменяться непрерывно.

\begin{theorem}
    Аксиома непрерывности следует из аксиомы счётной аддитивности.
\end{theorem}
\begin{proof}
    \[A_n = \sum_{i = n}^{+\infty} A_i \overline A_{i+1} \cup \bigcap_{i = n}^{+\infty} A_i\]

    Т.к. эти события несовместны, то по аксиоме счётной аддитивности
    \[P(A_n) = \sum_{i = n}^{+\infty} P(A_i \overline{A_{i+1}}) + P\left(\bigcap_{i = n}^{+\infty} A_i\right)\]

    Т.к. по условию \(P(\bigcap_{i = 1}^{+\infty} A_i) = \emptyset\) и \(\bigcap_{i = n}^{+\infty} A_i = \bigcap_{i = 1}^{+\infty} A_i\), то \(P(\bigcap_{i = n}^{+\infty} A_i) = 0\).

    Таким образом, \(P(A_n) = \sum_{i = n}^{+\infty} P(A_i \overline A_{i + 1})\) --- остаточный член сходящегося ряда \(\sum_{i = 1}^{+\infty} P(A_i \overline A_{i+1})\), который сходится, т.к. равен \(P(A_1)\). Тогда \(P(A_n) \to 0\)
\end{proof}

\begin{remark}
    Аксиома счётной аддитивности следует из аксиомы непрерывности и свойства конечной аддитивности.
\end{remark}
%</5>

%<*6>
\begin{remark}
    \[A(B + C) = AB + AC\]
\end{remark}

\subsection{Формула сложения}

Если \(A\) и \(B\) несовместны, то \(P(A + B) = P(A) + P(B)\)

\begin{theorem}
    \(P(A + B) = P(A) + P(B) - P(A B)\)
\end{theorem}
\begin{proof}
    \[A + B = A\overline B + AB + \overline A B\]
    \begin{align*}
        P(A + B) & = P(A\overline B) + P(AB) + P(\overline A B)                     \\
                 & = (P(A\overline B) + P(AB)) + (P(\overline A B) + P(AB)) - P(AB) \\
                 & = P(A) + P(B) - P(AB)
    \end{align*}
\end{proof}

\begin{example}
    Пусть \(D\) --- дама, \(\Pi\) --- пика. Найти вероятность \(P(D + \Pi)\).
    \[P(D + \Pi) = P(D) + P(\Pi) = \frac{4}{36} + \frac{9}{36} - \frac{1}{36} = \frac{1}{3}\]
\end{example}

Аналогично можно доказать формулу включения-исключения:
\[P\left(\sum A_i\right) = \sum P(A_i) - \sum_{i < j} P(A_i A_j) + \sum P(A_i A_j A_k) + \dots + ( - 1)^{n - 1} P(A_1 A_2 \dots A_n)\]

\begin{example}
    \(n\) писем раскладываются в \(n\) конвертов. Найти вероятность того, что хотя бы одно письмо попадёт в свой конверт. Чему равна эта вероятность при \(n \to +\infty\).

    Пусть \(A_i\) --- \(i\)-тое письмо попало в свой конверт. \(A\) --- хотя бы одно письмо попало в свой конверт.
    \[A = \sum A_i\]
    \[P(A_i) = \frac{1}{n} \quad P(A_i A_j) = \frac{1}{A_n^2} \quad P(A_i A_j A_k) = \frac{1}{A_n^3} \quad \dots \quad P(A_1 \dots A_n) = \frac{1}{n!}\]
    \[P(A) = n \cdot \frac{1}{n} - C_n^2 \frac{1}{A_n^2} + \dots + ( -1)^{n - 1} \frac{1}{n!} = \sum_i ( - 1)^{i + 1} \frac{1}{i!} \to 1 - e^{ - 1} \approx 0.62\]
\end{example}
%</6>

\section{Независимые события}

%<*7>
\begin{definition}
    События \(A\) и \(B\) называются \textbf{независимыми}, если \(P(AB) = P(A)P(B)\)
\end{definition}

\begin{exercise}
    Пусть \(P(A), P(B) \neq 0\). Доказать, что если \(A\) и \(B\) несовместны, то они зависимы.
\end{exercise}
\begin{proof}
    \[0 = P(AB) = P(A)P(B)\]
    Таким образом, либо \(P(A) = 0\), либо \(P(B) = 0\), что является противоречием.
\end{proof}

\begin{prop}
    Если \(A\) и \(B\) независимы, то \(A\) и \(\overline{B}\) --- независимы.
\end{prop}
\begin{proof}
    \[P(A) = P(A(B + \overline B)) = P(AB + A\overline B) = P(AB) + P(A\overline B)\]
    \[P(A \overline B) = P(A) - P(AB) = P(A) - P(A)P(B) = P(A)(1 - P(B)) = P(AB)\]
    Таким образом, \(A\) и \(\overline{B}\) --- независимы.
\end{proof}

\begin{definition}
    События \(A_1 \dots A_n\) называются \textbf{независимыми в совокупности}, если для любого набора \(1 \leq i_1 \dots i_k \leq n\) \(P(A_{i_1}, A_{i_2}, \dots A_{i_k}) = P(A_{i_1})P(A_{i_2}) \dots P(A_{i_k})\)
\end{definition}

\begin{example}[Бернштейн]
    Три грани правильного тетраедра выкрашены в красный, синий, зеленый цвета, а четвертая грань --- во все эти три цвета.

    Бросаем тетраэдр и смотрим на грань, на которую он упал. События:
    \begin{itemize}
        \item \(A\) --- красный цвет
        \item \(B\) --- синий цвет
        \item \(C\) --- зеленый цвет
    \end{itemize}

    \[P(A) = P(B) = P(C) = \frac{1}{2}\]
    \[P(AB) = P(AC) = P(BC) = \frac{1}{4}\]
    Таким образом, все события попарно независимы.
    \[P(ABC) = \frac{1}{4} \neq P(A)P(B)P(C) = \frac{1}{8}\]
    Таким образом, события не независимы в совокупности.
\end{example}
%</7>

\begin{remark}
    Если в условии есть ``хотя бы'', т.е. требуется найти вероятность суммы совместных независимых событий, то применима формула обратной вероятности.
\end{remark}

\begin{example}
    Найти вероятность того, что при четырёх бросаниях кости хотя бы один раз выпадет шестерка.

    \(A_i\) --- при \(i\)-том броске хотя бы один раз выпала шестерка.

    \[P(\overline A_1) = P(\overline A_2) = P(\overline A_3) = P(\overline A_4) = \frac{5}{6}\]

    \[\overline A = \overline A_1 \dots \overline A_4\]
    \[P(\overline A) = \left( \frac{5}{6} \right)^4\]
    \[P(A) = 1 - \left( \frac{5}{6} \right)^4\]
\end{example}

\begin{example}
    Два стрелка стреляют по мишени. Вероятность попадания первого стрелка \(0.6\), второго --- \(0.8\). Найти вероятность того, что попадет ровно один стрелок.

    \(A_1\) --- первый стрелок попал, \(A_2\) --- второй стрелок попал, \(A\) --- ровно один попал.

    \[P(A_1) = 0.8, P(\overline A_1) = 0.2, P(A_2) = 0.6, P(\overline A_2) = 0.4\]

    \[A = A_1 \overline A_2 + \overline A_1 A_2\]
    \[P(A) = P(A_1) P(\overline A_2) + P(\overline A_1) P(A_2) = 0.8 \cdot 0.4 + 0.6 \cdot 0.2 = 0.44\]
\end{example}
