\documentclass[12pt, a4paper]{article}

%<*preamble>
% Math symbols
\usepackage{amsmath, amsthm, amsfonts, amssymb}
\usepackage{accents}
\usepackage{esvect}
\usepackage{mathrsfs}
\usepackage{mathtools}
\mathtoolsset{showonlyrefs}
\usepackage{cmll}
\usepackage{stmaryrd}
\usepackage{physics}
\usepackage[normalem]{ulem}
\usepackage{ebproof}
\usepackage{extarrows}

% Page layout
\usepackage{geometry, a4wide, parskip, fancyhdr}

% Font, encoding, russian support
\usepackage[russian]{babel}
\usepackage[sb]{libertine}
\usepackage{xltxtra}

% Listings
\usepackage{listings}
\lstset{basicstyle=\ttfamily,breaklines=true}
\setmonofont{Inconsolata}

% Miscellaneous
\usepackage{array}
\usepackage{calc}
\usepackage{caption}
\usepackage{subcaption}
\captionsetup{justification=centering,margin=2cm}
\usepackage{catchfilebetweentags}
\usepackage{enumitem}
\usepackage{etoolbox}
\usepackage{float}
\usepackage{lastpage}
\usepackage{minted}
\usepackage{svg}
\usepackage{wrapfig}
\usepackage{xcolor}
\usepackage[makeroom]{cancel}

\newcolumntype{L}{>{$}l<{$}}
    \newcolumntype{C}{>{$}c<{$}}
\newcolumntype{R}{>{$}r<{$}}

% Footnotes
\usepackage[hang]{footmisc}
\setlength{\footnotemargin}{2mm}
\makeatletter
\def\blfootnote{\gdef\@thefnmark{}\@footnotetext}
\makeatother

% References
\usepackage{hyperref}
\hypersetup{
    colorlinks,
    linkcolor={blue!80!black},
    citecolor={blue!80!black},
    urlcolor={blue!80!black},
}

% tikz
\usepackage{tikz}
\usepackage{tikz-cd}
\usetikzlibrary{arrows.meta}
\usetikzlibrary{decorations.pathmorphing}
\usetikzlibrary{calc}
\usetikzlibrary{patterns}
\usepackage{pgfplots}
\pgfplotsset{width=10cm,compat=1.9}
\newcommand\irregularcircle[2]{% radius, irregularity
    \pgfextra {\pgfmathsetmacro\len{(#1)+rand*(#2)}}
    +(0:\len pt)
    \foreach \a in {10,20,...,350}{
            \pgfextra {\pgfmathsetmacro\len{(#1)+rand*(#2)}}
            -- +(\a:\len pt)
        } -- cycle
}

\providetoggle{useproofs}
\settoggle{useproofs}{false}

\pagestyle{fancy}
\lfoot{M3137y2019}
\cfoot{}
\rhead{стр. \thepage\ из \pageref*{LastPage}}

\newcommand{\R}{\mathbb{R}}
\newcommand{\Q}{\mathbb{Q}}
\newcommand{\Z}{\mathbb{Z}}
\newcommand{\B}{\mathbb{B}}
\newcommand{\N}{\mathbb{N}}
\renewcommand{\Re}{\mathfrak{R}}
\renewcommand{\Im}{\mathfrak{I}}

\newcommand{\const}{\text{const}}
\newcommand{\cond}{\text{cond}}

\newcommand{\teormin}{\textcolor{red}{!}\ }

\DeclareMathOperator*{\xor}{\oplus}
\DeclareMathOperator*{\equ}{\sim}
\DeclareMathOperator{\sign}{\text{sign}}
\DeclareMathOperator{\Sym}{\text{Sym}}
\DeclareMathOperator{\Asym}{\text{Asym}}

\DeclarePairedDelimiter{\ceil}{\lceil}{\rceil}

% godel
\newbox\gnBoxA
\newdimen\gnCornerHgt
\setbox\gnBoxA=\hbox{$\ulcorner$}
\global\gnCornerHgt=\ht\gnBoxA
\newdimen\gnArgHgt
\def\godel #1{%
    \setbox\gnBoxA=\hbox{$#1$}%
    \gnArgHgt=\ht\gnBoxA%
    \ifnum     \gnArgHgt<\gnCornerHgt \gnArgHgt=0pt%
    \else \advance \gnArgHgt by -\gnCornerHgt%
    \fi \raise\gnArgHgt\hbox{$\ulcorner$} \box\gnBoxA %
    \raise\gnArgHgt\hbox{$\urcorner$}}

% \theoremstyle{plain}

\theoremstyle{definition}
\newtheorem{theorem}{Теорема}
\newtheorem*{definition}{Определение}
\newtheorem{axiom}{Аксиома}
\newtheorem*{axiom*}{Аксиома}
\newtheorem{lemma}{Лемма}

\theoremstyle{remark}
\newtheorem*{remark}{Примечание}
\newtheorem*{exercise}{Упражнение}
\newtheorem{corollary}{Следствие}[theorem]
\newtheorem*{statement}{Утверждение}
\newtheorem*{corollary*}{Следствие}
\newtheorem*{example}{Пример}
\newtheorem{observation}{Наблюдение}
\newtheorem*{prop}{Свойства}
\newtheorem*{obozn}{Обозначение}

% subtheorem
\makeatletter
\newenvironment{subtheorem}[1]{%
    \def\subtheoremcounter{#1}%
    \refstepcounter{#1}%
    \protected@edef\theparentnumber{\csname the#1\endcsname}%
    \setcounter{parentnumber}{\value{#1}}%
    \setcounter{#1}{0}%
    \expandafter\def\csname the#1\endcsname{\theparentnumber.\Alph{#1}}%
    \ignorespaces
}{%
    \setcounter{\subtheoremcounter}{\value{parentnumber}}%
    \ignorespacesafterend
}
\makeatother
\newcounter{parentnumber}

\newtheorem{manualtheoreminner}{Теорема}
\newenvironment{manualtheorem}[1]{%
    \renewcommand\themanualtheoreminner{#1}%
    \manualtheoreminner
}{\endmanualtheoreminner}

\newcommand{\dbltilde}[1]{\accentset{\approx}{#1}}
\newcommand{\intt}{\int\!}

% magical thing that fixes paragraphs
\makeatletter
\patchcmd{\CatchFBT@Fin@l}{\endlinechar\m@ne}{}
{}{\typeout{Unsuccessful patch!}}
\makeatother

\newcommand{\get}[2]{
    \ExecuteMetaData[#1]{#2}
}

\newcommand{\getproof}[2]{
    \iftoggle{useproofs}{\ExecuteMetaData[#1]{#2proof}}{}
}

\newcommand{\getwithproof}[2]{
    \get{#1}{#2}
    \getproof{#1}{#2}
}

\newcommand{\import}[3]{
    \subsection{#1}
    \getwithproof{#2}{#3}
}

\newcommand{\given}[1]{
    Дано выше. (\ref{#1}, стр. \pageref{#1})
}

\renewcommand{\ker}{\text{Ker }}
\newcommand{\im}{\text{Im }}
\renewcommand{\grad}{\text{grad}}
\newcommand{\rg}{\text{rg}}
\newcommand{\defeq}{\stackrel{\text{def}}{=}}
\newcommand{\defeqfor}[1]{\stackrel{\text{def } #1}{=}}
\newcommand{\itemfix}{\leavevmode\makeatletter\makeatother}
\newcommand{\?}{\textcolor{red}{???}}
\renewcommand{\emptyset}{\varnothing}
\newcommand{\longarrow}[1]{\xRightarrow[#1]{\qquad}}
\DeclareMathOperator*{\esup}{\text{ess sup}}
\newcommand\smallO{
    \mathchoice
    {{\scriptstyle\mathcal{O}}}% \displaystyle
    {{\scriptstyle\mathcal{O}}}% \textstyle
    {{\scriptscriptstyle\mathcal{O}}}% \scriptstyle
    {\scalebox{.6}{$\scriptscriptstyle\mathcal{O}$}}%\scriptscriptstyle
}
\renewcommand{\div}{\text{div}\ }
\newcommand{\rot}{\text{rot}\ }
\newcommand{\cov}{\text{cov}}

\makeatletter
\newcommand{\oplabel}[1]{\refstepcounter{equation}(\theequation\ltx@label{#1})}
\makeatother

\newcommand{\symref}[2]{\stackrel{\oplabel{#1}}{#2}}
\newcommand{\symrefeq}[1]{\symref{#1}{=}}

% xrightrightarrows
\makeatletter
\newcommand*{\relrelbarsep}{.386ex}
\newcommand*{\relrelbar}{%
    \mathrel{%
        \mathpalette\@relrelbar\relrelbarsep
    }%
}
\newcommand*{\@relrelbar}[2]{%
    \raise#2\hbox to 0pt{$\m@th#1\relbar$\hss}%
    \lower#2\hbox{$\m@th#1\relbar$}%
}
\providecommand*{\rightrightarrowsfill@}{%
    \arrowfill@\relrelbar\relrelbar\rightrightarrows
}
\providecommand*{\leftleftarrowsfill@}{%
    \arrowfill@\leftleftarrows\relrelbar\relrelbar
}
\providecommand*{\xrightrightarrows}[2][]{%
    \ext@arrow 0359\rightrightarrowsfill@{#1}{#2}%
}
\providecommand*{\xleftleftarrows}[2][]{%
    \ext@arrow 3095\leftleftarrowsfill@{#1}{#2}%
}

\allowdisplaybreaks

\newcommand{\unfinished}{\textcolor{red}{Не дописано}}

% Reproducible pdf builds 
\special{pdf:trailerid [
<00112233445566778899aabbccddeeff>
<00112233445566778899aabbccddeeff>
]}
%</preamble>


\usepackage{sectsty}

\allsectionsfont{\raggedright}
\sectionfont{\fontsize{14}{15}\selectfont}

\lhead{Билеты}
\rfoot{}

\settoggle{useproofs}{true}

\renewcommand{\import}[3]{
    \section{#1}
    \getwithproof{#2}{#3}
}

\begin{document}

\import{Пространство элементарных исходов. Случайные события. Операции над  событиями.}{1}{1}

\import{Статистическое определение вероятности. Классическое определение вероятности.}{1}{2.1}
\get{1}{2.2}

\import{Геометрическое определение вероятности. Задача Бюффона об игле.}{1}{3}

\import{Аксиоматическое определение вероятности. Вероятностное пространство. Свойства вероятности.}{2}{4}

\import{Аксиома непрерывности. Ее смысл и вывод.}{2}{5}

\import{Свойства операций сложения и умножения. Формула сложения вероятностей.}{2}{6}

\import{Независимость событий. Независимые события в совокупности и попарно. Пример Бернштейна.}{2}{7}

\import{Условная вероятность. Формула умножения событий.}{3}{8}

\import{Полная группа событий. Формула полной вероятности. Формула Байеса.}{3}{9}

\import{Последовательность независимых испытаний. Формула Бернулли. Наиболее вероятное число успехов в схеме Бернулли.}{4}{10}

\import{Локальная и интегральная формулы Муавра-Лапласа (без док-ва).}{4}{11}

\import{Вероятность отклонения относительной частоты от вероятности события. Закон больших чисел Бернулли.}{4}{12}

\import{Схемы испытаний: Бернулли, до первого успеха. Биномиальное и геометрическое распределения. Свойство отсутствия последействия.}{5}{}

\subsection{Схема Бернулли}

См. билет 10.

\get{5}{13.1}
\get{5}{13.2}
\get{5}{13.3}
\get{5}{13.4}

\import{Урновая схема с возвратом и без возврата. Гипергеометрическое распределение. Теорема об его асимптотическом приближении к биномиальному.}{5}{14}

\import{Схема Пуассона. Формула Пуассона. Оценка погрешности в формуле Пуассона.}{5}{15}

Схема Пуассона: см. формулу Пуассона, т.е. \(n \to +\infty, p \to 0\) в схеме Бернулли.

\import{Случайные величины, определение. Измеримость функции, ее смысл. Вероятностное пространство (R, B, P). Распределение случайной величины.}{1}{}

\import{Дискретные случайные величины. Определение, закон распределения, числовые характеристики.}{1}{}

\import{Свойства математического ожидания и дисперсии дискретной случайной величины.}{1}{}

\import{Стандартные дискретные распределения и их числовые характеристики (Бернулли, биномиальное, геометрическое, Пуассона).}{1}{}

\import{Функция распределения и ее свойства (в свойствах 4, 5, 6 достаточно привести одно из доказательств).}{1}{}

% \import{Абсолютно непрерывные случайные величины. Плотность и ее свойства.}{1}{}

% \import{Числовые характеристики абсолютно непрерывной случайной величины, их свойства.}{1}{}

% \import{Равномерное распределение. }{1}{}

% \import{Показательное распределение. Свойство нестарения.}{1}{}

% \import{Нормальное распределение. Стандартное нормальное распределение, его числовые характеристики.}{1}{}

% \import{Связь между стандартным нормальным и нормальным распределениями. Следствия.}{1}{}

% \import{Гамма-функция и гамма-распределение, его свойства.}{1}{}

% \import{Сингулярные распределения. Теорема Лебега (без док-ва).}{1}{}

% \import{Преобразования случайных величин. Борелевские функции. Стандартизация случайной величины. }{1}{}

% \import{Линейное преобразование случайной величины. Теорема о монотонном преобразовании (без док-ва).}{1}{}

% \import{Квантильное преобразование. Моделирование случайной величины с помощью датчика случайных чисел.}{1}{}

% \import{Виды сходимостей случайных величин, связь между ними. Теорема об эквивалентности сходимостей к константе (все без док-ва).}{1}{}

% \import{Математическое ожидание преобразованной случайной величины. Свойства моментов.}{1}{}

% \import{Неравенство Йенсена, следствие.}{1}{}

% \import{Неравенства Маркова, Чебышева, правило трех сигм.}{1}{}

% \import{Среднее арифметическое одинаковых независимых случайных величин. Закон больших чисел Чебышева.}{1}{}

% \import{Вывод закона больших чисел Бернулли из закона больших чисел Чебышева. Законы больших чисел Хинчина и Колмогорова (только формулировки), закон больших чисел Маркова (с док-м).}{1}{}

% \import{Совместные распределения случайных величин. Функция совместного распределения, ее свойства. Независимость случайных величин.}{1}{}

% \import{Дискретная система двух случайных величин. Закон совместного распределения. Маргинальные распределения.}{1}{}

% \import{Абсолютно непрерывная система двух случайных величин. Плотность совместного распределения, ее свойства.}{1}{}

% \import{Функции от двух случайных величин. Теорема о функции распределения. Формула свертки.}{1}{}

% \import{Суммы стандартных распределений, устойчивость по суммированию (биномиальное, Пуассона, стандартное нормальное).}{1}{}

% \import{Условные распределения и условные математические ожидания. Случаи дискретной и абсолютно непрерывной систем двух случайных величин.}{1}{}

% \import{Пространство случайных величин. Скалярное произведение, неравенство Коши-Буняковского-Шварца. }{1}{}

% \import{Условное математическое ожидание как случайная величина, его свойства. Обобщенная формула полной вероятности.}{1}{}

% \import{Числовые характеристики зависимости случайных величин. Ковариация, ее свойства. Коэффициент корреляции, его свойства. Корреляция случайных величин.}{1}{}

% \import{Характеристическая функция случайной величины, ее свойства. Теорема о непрерывном соответствии (формулировка).}{1}{}

% \import{Характеристические функции стандартных распределений (Бернулли. биномиальное, Пуассона, нормальное). Следствия.}{1}{}

% \import{Доказательство закона больших чисел Хинчина.}{1}{}

% \import{Центральная предельная теорема. Вывод из нее предельной теоремы Муавра-Лапласа. Неравенство Берри-Ессеена (формулировка). }{1}{}

\unfinished

\end{document}
