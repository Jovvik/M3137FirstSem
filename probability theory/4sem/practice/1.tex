\documentclass[12pt, a4paper]{article}

%<*preamble>
% Math symbols
\usepackage{amsmath, amsthm, amsfonts, amssymb}
\usepackage{accents}
\usepackage{esvect}
\usepackage{mathrsfs}
\usepackage{mathtools}
\mathtoolsset{showonlyrefs}
\usepackage{cmll}
\usepackage{stmaryrd}
\usepackage{physics}
\usepackage[normalem]{ulem}
\usepackage{ebproof}
\usepackage{extarrows}

% Page layout
\usepackage{geometry, a4wide, parskip, fancyhdr}

% Font, encoding, russian support
\usepackage[russian]{babel}
\usepackage[sb]{libertine}
\usepackage{xltxtra}

% Listings
\usepackage{listings}
\lstset{basicstyle=\ttfamily,breaklines=true}
\setmonofont{Inconsolata}

% Miscellaneous
\usepackage{array}
\usepackage{calc}
\usepackage{caption}
\usepackage{subcaption}
\captionsetup{justification=centering,margin=2cm}
\usepackage{catchfilebetweentags}
\usepackage{enumitem}
\usepackage{etoolbox}
\usepackage{float}
\usepackage{lastpage}
\usepackage{minted}
\usepackage{svg}
\usepackage{wrapfig}
\usepackage{xcolor}
\usepackage[makeroom]{cancel}

\newcolumntype{L}{>{$}l<{$}}
    \newcolumntype{C}{>{$}c<{$}}
\newcolumntype{R}{>{$}r<{$}}

% Footnotes
\usepackage[hang]{footmisc}
\setlength{\footnotemargin}{2mm}
\makeatletter
\def\blfootnote{\gdef\@thefnmark{}\@footnotetext}
\makeatother

% References
\usepackage{hyperref}
\hypersetup{
    colorlinks,
    linkcolor={blue!80!black},
    citecolor={blue!80!black},
    urlcolor={blue!80!black},
}

% tikz
\usepackage{tikz}
\usepackage{tikz-cd}
\usetikzlibrary{arrows.meta}
\usetikzlibrary{decorations.pathmorphing}
\usetikzlibrary{calc}
\usetikzlibrary{patterns}
\usepackage{pgfplots}
\pgfplotsset{width=10cm,compat=1.9}
\newcommand\irregularcircle[2]{% radius, irregularity
    \pgfextra {\pgfmathsetmacro\len{(#1)+rand*(#2)}}
    +(0:\len pt)
    \foreach \a in {10,20,...,350}{
            \pgfextra {\pgfmathsetmacro\len{(#1)+rand*(#2)}}
            -- +(\a:\len pt)
        } -- cycle
}

\providetoggle{useproofs}
\settoggle{useproofs}{false}

\pagestyle{fancy}
\lfoot{M3137y2019}
\cfoot{}
\rhead{стр. \thepage\ из \pageref*{LastPage}}

\newcommand{\R}{\mathbb{R}}
\newcommand{\Q}{\mathbb{Q}}
\newcommand{\Z}{\mathbb{Z}}
\newcommand{\B}{\mathbb{B}}
\newcommand{\N}{\mathbb{N}}
\renewcommand{\Re}{\mathfrak{R}}
\renewcommand{\Im}{\mathfrak{I}}

\newcommand{\const}{\text{const}}
\newcommand{\cond}{\text{cond}}

\newcommand{\teormin}{\textcolor{red}{!}\ }

\DeclareMathOperator*{\xor}{\oplus}
\DeclareMathOperator*{\equ}{\sim}
\DeclareMathOperator{\sign}{\text{sign}}
\DeclareMathOperator{\Sym}{\text{Sym}}
\DeclareMathOperator{\Asym}{\text{Asym}}

\DeclarePairedDelimiter{\ceil}{\lceil}{\rceil}

% godel
\newbox\gnBoxA
\newdimen\gnCornerHgt
\setbox\gnBoxA=\hbox{$\ulcorner$}
\global\gnCornerHgt=\ht\gnBoxA
\newdimen\gnArgHgt
\def\godel #1{%
    \setbox\gnBoxA=\hbox{$#1$}%
    \gnArgHgt=\ht\gnBoxA%
    \ifnum     \gnArgHgt<\gnCornerHgt \gnArgHgt=0pt%
    \else \advance \gnArgHgt by -\gnCornerHgt%
    \fi \raise\gnArgHgt\hbox{$\ulcorner$} \box\gnBoxA %
    \raise\gnArgHgt\hbox{$\urcorner$}}

% \theoremstyle{plain}

\theoremstyle{definition}
\newtheorem{theorem}{Теорема}
\newtheorem*{definition}{Определение}
\newtheorem{axiom}{Аксиома}
\newtheorem*{axiom*}{Аксиома}
\newtheorem{lemma}{Лемма}

\theoremstyle{remark}
\newtheorem*{remark}{Примечание}
\newtheorem*{exercise}{Упражнение}
\newtheorem{corollary}{Следствие}[theorem]
\newtheorem*{statement}{Утверждение}
\newtheorem*{corollary*}{Следствие}
\newtheorem*{example}{Пример}
\newtheorem{observation}{Наблюдение}
\newtheorem*{prop}{Свойства}
\newtheorem*{obozn}{Обозначение}

% subtheorem
\makeatletter
\newenvironment{subtheorem}[1]{%
    \def\subtheoremcounter{#1}%
    \refstepcounter{#1}%
    \protected@edef\theparentnumber{\csname the#1\endcsname}%
    \setcounter{parentnumber}{\value{#1}}%
    \setcounter{#1}{0}%
    \expandafter\def\csname the#1\endcsname{\theparentnumber.\Alph{#1}}%
    \ignorespaces
}{%
    \setcounter{\subtheoremcounter}{\value{parentnumber}}%
    \ignorespacesafterend
}
\makeatother
\newcounter{parentnumber}

\newtheorem{manualtheoreminner}{Теорема}
\newenvironment{manualtheorem}[1]{%
    \renewcommand\themanualtheoreminner{#1}%
    \manualtheoreminner
}{\endmanualtheoreminner}

\newcommand{\dbltilde}[1]{\accentset{\approx}{#1}}
\newcommand{\intt}{\int\!}

% magical thing that fixes paragraphs
\makeatletter
\patchcmd{\CatchFBT@Fin@l}{\endlinechar\m@ne}{}
{}{\typeout{Unsuccessful patch!}}
\makeatother

\newcommand{\get}[2]{
    \ExecuteMetaData[#1]{#2}
}

\newcommand{\getproof}[2]{
    \iftoggle{useproofs}{\ExecuteMetaData[#1]{#2proof}}{}
}

\newcommand{\getwithproof}[2]{
    \get{#1}{#2}
    \getproof{#1}{#2}
}

\newcommand{\import}[3]{
    \subsection{#1}
    \getwithproof{#2}{#3}
}

\newcommand{\given}[1]{
    Дано выше. (\ref{#1}, стр. \pageref{#1})
}

\renewcommand{\ker}{\text{Ker }}
\newcommand{\im}{\text{Im }}
\renewcommand{\grad}{\text{grad}}
\newcommand{\rg}{\text{rg}}
\newcommand{\defeq}{\stackrel{\text{def}}{=}}
\newcommand{\defeqfor}[1]{\stackrel{\text{def } #1}{=}}
\newcommand{\itemfix}{\leavevmode\makeatletter\makeatother}
\newcommand{\?}{\textcolor{red}{???}}
\renewcommand{\emptyset}{\varnothing}
\newcommand{\longarrow}[1]{\xRightarrow[#1]{\qquad}}
\DeclareMathOperator*{\esup}{\text{ess sup}}
\newcommand\smallO{
    \mathchoice
    {{\scriptstyle\mathcal{O}}}% \displaystyle
    {{\scriptstyle\mathcal{O}}}% \textstyle
    {{\scriptscriptstyle\mathcal{O}}}% \scriptstyle
    {\scalebox{.6}{$\scriptscriptstyle\mathcal{O}$}}%\scriptscriptstyle
}
\renewcommand{\div}{\text{div}\ }
\newcommand{\rot}{\text{rot}\ }
\newcommand{\cov}{\text{cov}}

\makeatletter
\newcommand{\oplabel}[1]{\refstepcounter{equation}(\theequation\ltx@label{#1})}
\makeatother

\newcommand{\symref}[2]{\stackrel{\oplabel{#1}}{#2}}
\newcommand{\symrefeq}[1]{\symref{#1}{=}}

% xrightrightarrows
\makeatletter
\newcommand*{\relrelbarsep}{.386ex}
\newcommand*{\relrelbar}{%
    \mathrel{%
        \mathpalette\@relrelbar\relrelbarsep
    }%
}
\newcommand*{\@relrelbar}[2]{%
    \raise#2\hbox to 0pt{$\m@th#1\relbar$\hss}%
    \lower#2\hbox{$\m@th#1\relbar$}%
}
\providecommand*{\rightrightarrowsfill@}{%
    \arrowfill@\relrelbar\relrelbar\rightrightarrows
}
\providecommand*{\leftleftarrowsfill@}{%
    \arrowfill@\leftleftarrows\relrelbar\relrelbar
}
\providecommand*{\xrightrightarrows}[2][]{%
    \ext@arrow 0359\rightrightarrowsfill@{#1}{#2}%
}
\providecommand*{\xleftleftarrows}[2][]{%
    \ext@arrow 3095\leftleftarrowsfill@{#1}{#2}%
}

\allowdisplaybreaks

\newcommand{\unfinished}{\textcolor{red}{Не дописано}}

% Reproducible pdf builds 
\special{pdf:trailerid [
<00112233445566778899aabbccddeeff>
<00112233445566778899aabbccddeeff>
]}
%</preamble>


\lhead{Теория вероятности \textit{(практика)}}
\cfoot{}
\rfoot{9.2.2021}

\begin{document}

\subsection*{Разбалловка}

Будут две \textit{(возможно три)} контрольные по 30 баллов, 10 баллов за посещения и 10 баллов за домашние задания. Будет возможность набрать дополнительные баллы. Максимальный автомат без экзамена --- 4.

\section*{Базовая теория вероятности}

Случайные события обозначаются буквами \(A, B, C\), каждому событию соответствует числовая харакетристика \(P(A)\) --- вероятность наступления события \(A\), при этом \(0 \leq P(A) \leq 1\).

Формула вероятности \(P(A) = \frac{m}{n}\), где \(n\) --- число всех возможных исходов, \(m\) --- число исходов, соответствующих событию \(A\).

\begin{exercise}
    Бросается кубик. Какова вероятность, что выпадет чётное число?

    \textbf{Ответ}: \(\frac{3}{6} = \frac{1}{2}\)
\end{exercise}

\begin{exercise}
    В коробке 4 красных и 3 синих карандаша. Вынули 3 из них. Найти вероятность того, что из них будут два красных, а один --- синий.

    Число всех возможных элементарных исходов:
    \[n = \binom{7}{3} = \frac{7!}{3!4!} = 35\]
    Число искомых исходов:
    \[m = \binom{4}{2} \binom{3}{1} = \frac{4!}{2!2!} \cdot \frac{3!}{1!2!} = 18\]

    \textbf{Ответ}: \(\frac{18}{35}\)
\end{exercise}

\begin{exercise}
    В ящике 5 чёрных шаров, 3 белых и 2 красных. Вынули половину из них. Найти вероятность того, что из них будут два белых и два чёрных.

    Число всех возможных элементарных исходов:
    \[n = \binom{10}{5} = \frac{10!}{5!5!} = 252\]
    Число искомых исходов:
    \[m = \binom{3}{2} \binom{5}{2} \binom{2}{1} = \frac{3!}{2!1!} \cdot \frac{5!}{3!2!} \cdot \frac{2!}{1!1!} = 60\]

    \textbf{Ответ}: \(\frac{60}{252} = \frac{10}{43} \)
\end{exercise}

\begin{exercise}
    На первом этаже шестиэтажного дома в лифт зашли трое человек. Какова вероятность того, что они выйдут на разных этажах? \textit{(Выйти на первом этаже нельзя)}

    Первый человек выходит где угодно, второй --- на любом из оставшихся четырех этажей, третий --- на любом из трех.

    \[1 \cdot \frac{4}{5} \cdot \frac{3}{5} = \frac{12}{25}\]

    Формальное решение:
    \[n = 5^3 = 125\]
    \[m = A_5^3 = 5\cdot 4\cdot 3 = 60\]
    \[P(A) = \frac{60}{125} = \frac{12}{25}\]

    \textbf{Ответ}: \(\frac{12}{25}\)
\end{exercise}

\begin{exercise}
    На полке расставляется 8 книг. Найти вероятность того, что 3 конкретные книги будут стоять рядом.

    \[n = 8! = 40320\]
    \[m = 6 \cdot 3! \cdot 5! = 4320\]
    \[P(A) = \frac{4320}{40320} = \frac{3}{28}\]

    \textbf{Ответ}: \(\frac{3}{28}\)
\end{exercise}

\begin{exercise}
    За круглым столом сидят 5 человек. Найти вероятность того, что два конкретных человека окажутся рядом.

    \[n = 5! = 120\]
    \[m = 5\cdot 2! \cdot 3! = 60\]
    \[P(A) = \frac{1}{2}\]

    \textbf{Ответ}: \(\frac{1}{2}\)
\end{exercise}

\begin{exercise}
    На доске поставили белую и черную ладьи. Найти вероятность того, что они не будут бить друг друга.

    \[n = 64 \cdot 63\]
    \[m = 64 \cdot 49\]
    \[P(A) = \frac{49}{63} = \frac{7}{9}\]

    \textbf{Ответ}: \(\frac{7}{9}\)
\end{exercise}

\begin{exercise}
    8 команд разбивают на две группы. Найти вероятность того, что две сильнейших команды окажутся в разных группах.

    \textbf{Ответ}: \(\frac{3}{7}\)
\end{exercise}

\begin{exercise}
    Бросаются два кубика. Найти вероятность того, что в сумме выпало не менее девяти очков.

    \[n = 36\]

    Искомых случаев \(5\cdot 2\): \((6, 6), (6, 5), (6, 4), (6, 3), (5, 4)\) и их перестановки.

    \[m = 5\cdot 2\]
    \[P(A) = \frac{10}{36} = \frac{5}{18}\]

    \textbf{Ответ}: \(\frac{5}{18}\)
\end{exercise}

\begin{exercise}
    Имеются карточки с буквами ``П'', ``Л'', ``И'', ``А''. Карточки выкладываются в ряд. Найти вероятность того, что получится осмысленное слово.

    \[n = 4! = 12\]
    Слов два: ``пила'' и ``липа''.

    \[m = 2\]

    \textbf{Ответ}: \(\frac{2}{12} = \frac{1}{6}\)
\end{exercise}

\subsection*{Домашнее задание}

\begin{exercise}
    В коробке 4 красных, 3 синих и 2 жёлтых карандаша. Из коробки вынули 6 карандашей. Найти вероятность того, что среди них будет поровну каждого цвета.
\end{exercise}

\begin{exercise}
    Известно, что у двух человек четыре козыря. Найти вероятность того, что они у них распределились в соотношении:
    \begin{enumerate}
        \item 2-2
        \item 3-1 или 1-3
        \item 4-0 или 0-4
    \end{enumerate}
\end{exercise}

\end{document}