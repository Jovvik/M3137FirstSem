\chapter{15 ноября}

\section{Общая модель линейной регрессии}

%<*41>
Пусть признак--результат \(X\) зависит от \(k\) факторов \(Z_1 \dots Z_k\). Рассматривается теоретическая модель линейной регрессии \(\E(X \mid \vec{Z}) = \beta_1Z_1 + \dots + \beta_kZ_k\)

\begin{notation}
    \[\vec{Z} = \begin{pmatrix}
            Z_1    \\
            \vdots \\
            Z_k
        \end{pmatrix} \quad \vec{\beta} = \begin{pmatrix}
            \beta_1 \\
            \vdots  \\
            \beta_k
        \end{pmatrix}\]
\end{notation}

Пусть проведено \(n \geq k\) экспериментов, \(\vec{Z}^{(i)} = (Z_1^{(i)} \dots Z_k^{(i)})\) --- значение фактора при \(i\)-том эксперименте (возможно заранее заданные). \(\vec{X} = (X_1 \dots X_n)\) --- полученные экспериментальные данные признака--результата \(X\). Согласно модели:
\[\begin{cases}
        X_1 = \beta_1 Z_1^{(1)} + \dots + \beta_k Z_k^{(1)} + \varepsilon_1 \\
        X_2 = \beta_1 Z_1^{(2)} + \dots + \beta_k Z_k^{(2)} + \varepsilon_2 \\
        \vdots                                                              \\
        X_n = \beta_1 Z_1^{(n)} + \dots + \beta_k Z_k^{(n)} + \varepsilon_n
    \end{cases}\]
, где \(\varepsilon_i\) --- случайная теоретическая ошибка при \(i\)-том эксперименте.
\begin{notation}[вектор случайных ошибок]
    \[\vec{\varepsilon} \coloneqq \begin{pmatrix}
            \varepsilon_1 \\
            \vdots        \\
            \varepsilon_n
        \end{pmatrix}\]
\end{notation}
\begin{notation}[матрица плана]
    \[Z_{k \times n} \coloneqq \begin{pmatrix}
            Z_1^{(1)} & Z_1^{(2)} & \dots  & Z_1^{(n)} \\
            Z_2^{(1)} & Z_2^{(2)} & \dots  & Z_2^{(n)} \\
            \vdots    & \vdots    & \ddots & \vdots    \\
            Z_k^{(1)} & Z_k^{(2)} & \dots  & Z_k^{(n)}
        \end{pmatrix}\]
\end{notation}

Тогда теоретическую модель можно записать в матричной форме:
\[\vec{X} = Z\tran \vec{\beta} + \vec{\varepsilon}\]

Требуется по данной матрице плана \(Z\) и вектору результатов \(\vec{X}\) найти оценки \(\vec{B} = (b_1 \dots b_k)\) для параметров регрессии \(\vec{\beta} = (\beta_1 \dots \beta_k)\) и параметров распределения ошибок \(\varepsilon_i\)

\begin{remark}
    Заметим, что в данной модели мы не теряем свободный член \(a\), т.к. можно считать, что \(Z_1 = \mathbbm{1}\) и ей соответствует строка из единиц.
\end{remark}

\subsection{Метод наименьших квадратов и нормальные уравнения}

Будем считать, что выполнено два условия:
\begin{enumerate}
    \item \(\rank Z = k\), т.е. все строки матрицы плана линейно независимы.\footnote{Это условие тривиально выполнить путём отброса линейно зависимых экспериментов.}\label{cond1}
    \item Случайные ошибки \(\varepsilon_i\) независимы и имеет одинаковое нормальное распределение с параметром \(a = 0\).\footnote{Это условие уже было в теореме~\nameref{Гаусса--Маркова}}\label{cond2}
\end{enumerate}

\begin{notation}
    \(A_{k \times k} \coloneqq Z Z\tran\)
\end{notation}
\begin{prop}\itemfix
    \begin{enumerate}
        \item \(A\) --- симметричная.
        \item \(A\) --- положительно определенная.
        \item Существует вещественная симметрическая матрица \(\sqrt{A}\), такая что \((\sqrt{A})^2 = A\).
    \end{enumerate}
\end{prop}

Найдём оценку \(\vec{B} = (b_1 \dots b_k)\), которая минимизирует функцию \[L(\vec{B}) = \sum_{i=1}^{n} \hat{\varepsilon}_i^2 = \norm{\hat{\varepsilon}}^2 = \norm{\vec{X} - Z\tran \vec{B}}^2\]

Заметим, что \(\norm{\vec{X} - Z\tran \vec{B}}^2\) --- квадрат расстояния от точки \(\vec{X}\) до точки \(Z\tran \vec{B}\), которая является точкой подпространства \(\ev{Z\tran \vec{t}}\), где \(\vec{t} \in \R^k\). Таким образом, искомое минимальное расстояние это расстояние до данного подпространства \(\ev{Z\tran \vec{t}}\). Это расстояние получаем при условии, что вектор \(\vec{X} - Z\tran \vec{B}\) будет ортогонален всем векторам подпространства, т.е. скалярное произведение \(\ev{Z\tran \vec{t}, \vec{X} - Z\tran \vec{B}} = 0 \ \ \forall \vec{t}\)
\begin{align*}
    \ev{Z\tran \vec{t}, \vec{X} - Z\tran \vec{B}}
     & = (Z\tran \vec{t})\tran \cdot (\vec{X} - Z\tran \vec{B})\ \\
     & = \vec{t}\:\tran \cdot Z \cdot (\vec{X} - Z\tran \vec{B}) \\
     & = \vec{t}\:\tran \cdot (Z\vec{X} - ZZ\tran \vec{B})
\end{align*}
Т.к. скалярное произведение вектора со всеми другими векторами равно нулю тогда и только тогда, когда он является нулевым, то:
\begin{align*}
    Z\vec{X} - ZZ\tran \vec{B} & = 0        \\
    ZZ\tran \vec{B}            & = Z\vec{X} \\
    A \vec{B}                  & = Z\vec{X}
\end{align*}
Это система из \(k\) линейных нормальных уравнений, из которой можно найти оценки \(\vec{B}\) неизвестных параметров. По свойству 2 матрица невырожденная, поэтому эта система имеет единственное решение:
\[\vec{B} = A^{-1}Z \vec{X}\]
%</41>

\begin{definition}
    \textbf{Матрицей ковариации} случайного вектора \(\vec{X}\) называется матрица \(\D \vec{X} = \E(\vec{X} - \E \vec{X})(\vec{X} - \E \vec{X})\tran\), состоящая из элементов \(\cov(X_i, X_j)\)
\end{definition}

\begin{prop}[оценок метода наименьших квадратов]\itemfix
    %<*42>
    \begin{enumerate}
        \item \(\vec{B} - \vec{\beta} = A^{-1} Z \vec{\varepsilon}\)
              \begin{proof}
                  \begin{align*}
                      \vec{B} - \vec{\beta}
                       & = A^{-1}Z \vec{X} - \vec{\beta}                                                           \\
                       & = A^{-1}Z (Z\tran \vec{\beta} + \vec{\varepsilon}) - \vec{\beta}                          \\
                       & = A^{-1} \underbrace{Z Z\tran}_{A} \vec{\beta} + A^{-1} Z \vec{\varepsilon} - \vec{\beta} \\
                       & = A^{-1} Z \vec{\varepsilon}                                                              \\
                  \end{align*}
              \end{proof}
        \item \(B\) --- несмещённая оценка параметров \(\vec{\beta}\)
              \begin{proof}
                  \begin{align*}
                      \E \vec{B}
                       & = \E(A^{-1}Z \vec{\varepsilon} + \vec{\beta})              \\
                       & = \E(A^{-1}Z \vec{\varepsilon}) + \E\vec{\beta}            \\
                       & = A^{-1}Z \underbrace{\E\vec{\varepsilon}}_0 + \vec{\beta} \\
                       & = \vec{\beta}
                  \end{align*}
              \end{proof}

              Кроме того, если \(\E \vec{\varepsilon} \neq 0\), то оценка смещённая.
        \item \(\D \vec{\varepsilon} =
              \mqty(\dmat{\sigma^2,\ddots,\sigma^2}) = \sigma^2 E_n\)\footnote{\(E_n\) --- единичная матрица \(n \times n\)}
              \begin{proof}
                  По условию \ref{cond2} \(\varepsilon_i \in N(0, \sigma^2)\) и независимы, следовательно, \(d_{ij} = 0\) при \(i \neq j\) и \(d_{ii} = \sigma^2\)
              \end{proof}
        \item \(\D (\sqrt{A} \cdot \vec{B}) = \mqty(\dmat{\sigma^2,\ddots,\sigma^2}) = \sigma^2 E_k\) \label{prop4}
              \begin{proof}
                  \begin{align*}
                      \D (\sqrt{A} \cdot \vec{B})
                       & = \E(\sqrt{A} \cdot \vec{B} - \E \sqrt{A} \cdot \vec{B})(\sqrt{A} \cdot \vec{B} - \E \sqrt{A} \cdot \vec{B})\tran \\
                       & = \E (\sqrt{A} (\vec{B} - \E \vec{B})) \cdot (\sqrt{A} (\vec{B} - \E \vec{B}))\tran                               \\
                       & = \E (\sqrt{A} (\vec{B} - \vec{\beta})) \cdot (\sqrt{A} (\vec{B} - \vec{\beta}))\tran                             \\
                       & = \E (\sqrt{A} A^{-1}Z \vec{\varepsilon}) \cdot (\sqrt{A} A^{-1}Z \vec{\varepsilon})\tran                         \\
                       & = \sqrt{A} A^{-1}Z \E(\vec{\varepsilon} \cdot \vec{\varepsilon}\,\tran) Z\tran (A^{-1})\tran (\sqrt{A})\tran      \\
                       & = \sqrt{A} A^{-1}Z \sigma^2 E_n Z\tran A^{-1} \sqrt{A}                                                            \\
                       & = \sigma^2 \sqrt{A} A^{-1}\underbrace{Z Z\tran}_{A} A^{-1} \sqrt{A}                                               \\
                       & = \sigma^2 \sqrt{A} A^{-1} \sqrt{A}                                                                               \\
                       & = \sigma^2 \sqrt{A} \sqrt{A^{-1}} \sqrt{A^{-1}} \sqrt{A}                                                          \\
                       & = \sigma^2 E_k                                                                                                    \\
                  \end{align*}
              \end{proof}

              \begin{corollary}
                  Координаты вектора \(\sqrt{A} \cdot \vec{B}\) некоррелированны.
              \end{corollary}
        \item \(\D \vec{B} = \sigma^2 A^{-1}\)\label{prop:ОМНК5}
              \begin{proof}
                  По свойству 4 \(\D(\sqrt{A} \cdot \vec{B}) = \sigma^2 E_k\). В силу билинейности ковариации:
                  \[\sigma^2 E_k = \D(\sqrt{A} \cdot \vec{B}) = (\sqrt{A})^2 \D \vec{B} = A \D \vec{B}\]
              \end{proof}

              Отсюда можем получить дисперсии оценок \(b_i\) по формулам:
              \[\D b_i = \sigma^2 (A^{-1})_{ii}\]
              Обозначим через \(\hat{\sigma}^2 \coloneqq \frac{1}{n} \sum_{i=1}^{n} \hat{\varepsilon}_i^2 = \frac{1}{n} \sum_{i=1}^{n} (\vec{X} - Z\tran \vec{B})^2\). Заметим, что \(\hat{\sigma}^2\) --- точная оценка неизвестного параметра \(\sigma^2\).
    \end{enumerate}
    %</42>
\end{prop}

%<*43>
\begin{theorem}\label{th:основная теорема множественной линейной регрессии}
    Пусть выполнены условия \ref{cond1} и \ref{cond2}. Тогда:
    \begin{enumerate}
        \item Вектор \(\frac{1}{\sigma}\sqrt{A}(\vec{B} - \vec{\beta})\) состоит из \(k\) независимых случайных величин со стандартным нормальным распределением.
        \item \(\frac{n \hat{\sigma}^2}{\sigma^2} \in H_{n - k}\) и не зависит от \(\vec{B}\).
        \item Исправленная оценка \(S^2 = \frac{n \hat{\sigma}^2}{n - k} = \frac{1}{n - k} \sum_{i=1}^{k} \hat{\varepsilon}_i^2\) --- несмещённая оценка для \(\sigma^2\). \label{ОТМЛР3}
    \end{enumerate}
\end{theorem}
\begin{proof}\itemfix
    \begin{enumerate}
        \item Вектор \(\sqrt{A}(\vec{B} - \vec{\beta}) = \sqrt{A} - A^{-1} Z \vec{\varepsilon} = (\sqrt{A})^{-1} Z \vec{\varepsilon}\) является линейным преобразованием нормального вектора \(\vec{\varepsilon}\), поэтому имеет нормальное совместное распределение, т.е. его координаты --- нормальные случайные величины.

              По свойству~\ref{cond2} \(\E(\sqrt{A} (\vec{B} - \vec{\beta})) = 0\), а его матрица ковариаций по свойству~\ref{prop4}:
              \[\D(\sqrt{A} (\vec{B} - \vec{\beta})) = \D (\sqrt{A} \vec{B}) = \sigma^2 E_k\]
              Отсюда видим, что координаты данного вектора не коррелированны и, следовательно, независимы. Тогда \(\D\left(\frac{1}{\sigma}\sqrt{A} \vec{B}\right) = E_k\)
        \item Нудно.
        \item Т.к. \(\E(\chi^2_{n - k}) = n - k\), то:
              \[\E \hat{\sigma}^2 = \frac{\sigma^2}{n} \E \left(\frac{n \hat{\sigma}^2}{\sigma^2}\right) = \frac{\sigma^2}{n} \cdot (n - k) = \frac{n - k}{n}\sigma^2\]
              Это смещённая оценка, следовательно, \(S^2 = \cfrac{n \hat{\sigma}^2}{n - k}\)
    \end{enumerate}
\end{proof}
%</43>

\begin{exercise}
    Используя результаты этой лекции, получить в качестве частного случая уравнение парной линейной регрессии, доказать теорему Гаусса--Маркова и критерии проверки гипотез.
\end{exercise}
