\chapter{29 ноября}

\textcolor{red}{Начало лекции возможно записано с ошибками, т.к. в записи не было звука.}

%<*48>
Пусть имеется уравнение множественной регрессии в векторной форме: \(\vec{X} = Z\tran \vec{\beta} + \vec{\varepsilon}\), где \(\vec{X} = (X_1 \dots X_k)\tran\) --- наблюдаемые значения переменной, \(Z = \left(Z_i^{(j)}\right)\) --- матрица плана, \(\vec{\beta} = (\beta_1 \dots \beta_k)\tran\) --- \? коэффициенты, \(\vec{\varepsilon} = (\varepsilon_1 \dots \varepsilon_n)\) --- \? ошибки (случайные).

\?, если\footnote{Моя вольная интерпретация: это наши предположения.}:
\begin{enumerate}
    \item Строки \(Z\) --- \? независимые
    \item \(\varepsilon_i \in N(0, \sigma^2)\) --- независимы
\end{enumerate}

\subsection{Некоррелированные наблюдения}

Пусть \(\E \varepsilon_i = 0, \D \varepsilon_i = \sigma^2 v_i\), но ошибки \(\varepsilon_i\) --- некоррелированные, т.е. \(\cov(\varepsilon_i, \varepsilon_j) = 0\) при \(i \neq j\).

\(\D \vec{\varepsilon} = \sigma^2 V\), где \(V\) --- диагональная матрица с элементами \(v_i\) на главной диагонали.

\begin{remark}
    Логично придать наблюдениям с меньшим разбросом больший ``вес''. Делается это изменением масштаба следующим образом: положим \(w_i = \frac{1}{v_i}, \tilde{X}_i = \sqrt{w_i} X_i\). Тогда получим новую модель \(\vec{\tilde{X}} = \tilde{Z}\tran \vec{\beta} + \vec{\tilde{\varepsilon}}\), где \(\tilde{Z}^{(j)}_i = \sqrt{w_j} Z_i^{(j)}, \tilde{\varepsilon}_i = \sqrt{w_i} \varepsilon_i\). При этом:
    \begin{enumerate}
        \item \(\E \tilde{\varepsilon}_i = 0\)
        \item \(\D \tilde{\varepsilon}_i = \D \sqrt{w_i} \varepsilon_i = \frac{1}{v_i} \D \varepsilon_i = \frac{1}{v_i} v_i \sigma^2 = \sigma^2\)
    \end{enumerate}
    Таким образом, получили прежнюю (классическую) ситуацию. Далее находим оценки \(\vec{B} = (b_1 \dots b_k)\tran\) параметров \(\beta = (\beta_1 \dots \beta_k)\tran\), как и раньше.
\end{remark}

\begin{example}\itemfix
    \begin{enumerate}
        \item Модель: \(X = \beta_0 + \varepsilon\).
              \[\hat{\beta}_0 = \frac{\sum w_i Z_i X_i}{\sum w_i Z_i^2}\]
              В частности, при \(Z_i = 1\), то \(\hat{\beta}_0 = \frac{\sum w_i X_i}{\sum w_i}\) --- взвешенное среднее.
        \item Пропорция (прямая проходит через начало координат).

              Пусть \(X\) --- потери тепла в квартире, \(Z\) --- разность внутренней и наружней температуры.
              \(X = \beta Z + \varepsilon\)
              \begin{enumerate}
                  \item Допустим, что \(\D \varepsilon_i = c Z_i\). Тогда \(w_i = \frac{\sigma^2}{c Z_i}\) и после вычислений получим
                        \[\hat{\beta} = \frac{\sum X_i}{\sum Z_i} = \frac{\overline{X}}{\overline{Z}}\]
                  \item Допустим, что \(\D \varepsilon_i = c Z_i^2\). Тогда
                        \[\hat{\beta} = \frac{1}{n} \sum_{i=1}^{n} \frac{X_i}{Z_i}\]
              \end{enumerate}
        \item Повторные наблюдения.

              Пусть проводилось \(n\) серий из \(k_i\) экспериментов, \(1 \leq i \leq n\). Пусть \(X_i\) --- средний результат экспериментов этой серии, \(\E X_{ij} = a, \D \varepsilon_{ij} = \sigma^2\). Тогда \(\D \varepsilon_i = \frac{\sigma^2}{k_i}\). Берём \(w_i = \frac{1}{k_i}\) и далее как раньше. То есть мы присваиваем больший вес сериям из большего числа экспериментов.
    \end{enumerate}
\end{example}

\subsection{Коррелированные наблюдения}

Пусть дисперсия ошибок \(\D \varepsilon_i\) не только различны, но и зависимы между собой, т.е. \(\cov(\varepsilon_i, \varepsilon_j) = v_{ij}\). Пусть \(V = (v_{ij})\). Тогда матрица ковариаций \(\D \vec{\varepsilon} = \sigma^2 V\). Известно, что такая матрица симметричная и положительно определенная, следовательно, существует матрица \(\sqrt{V}\). Итак, имеем модель \(\vec{X} = Z\tran \vec{\beta} + \vec{\varepsilon}\). Умножим обе части слева на \(\sqrt{V}^{-1}\), после чего получим новую модель:
\[\vec{\tilde{X}} = \tilde{Z}\tran \vec{\beta} + \vec{\tilde{\varepsilon}}\]
, где \(\vec{\tilde{X}} = \sqrt{V}^{-1} \vec{X}, \tilde{Z}\tran = \sqrt{V}^{-1} Z, \vec{\tilde{\varepsilon}} = \sqrt{V}^{-1} \vec{\varepsilon}\).
Заметим, что:
\begin{enumerate}
    \item \(\E \vec{\varepsilon} = \E \sqrt{V}^{-1} \vec{\varepsilon} = \sqrt{V}^{-1} \E \vec{\varepsilon} = 0\)
    \item \(\D \vec{\tilde{\varepsilon}} = \E \left(\left(\vec{\tilde{\varepsilon}} - \E \vec{\tilde{\varepsilon}}\right)\left(\vec{\tilde{\varepsilon}} - \E \vec{\tilde{\varepsilon}}\right)\tran\right) = \E \left(\vec{\tilde{\varepsilon}} \vec{\tilde{\varepsilon}}\tran\right) = \E \left(\sqrt{V}^{-1} \vec{\varepsilon} \left(\sqrt{V}^{-1} \vec{\varepsilon}\right)\tran\right) = \sigma^2 I\)
\end{enumerate}
Таким образом, мы получили стандартную ситуацию.
%</48>

\section{Нелинейная парная регрессия}

%<*49>
\subsection{Модель очень быстрого роста}

\begin{align*}
    X     & = ae^{bZ} \cdot \varepsilon    \\
    \ln X & = \ln a + bZ + \ln \varepsilon \\
    X'    & = a' + bZ + \varepsilon'
\end{align*}

\subsection{Модель быстрого роста}

\(b > 1\)
\begin{align*}
    X     & = aZ^b \cdot \varepsilon            \\
    \ln X & = \ln a + b \ln Z + \ln \varepsilon \\
    X'    & = a' + bZ' + \varepsilon'
\end{align*}

\subsection{Модель среднего роста}

\(0 < b < 1\)
\begin{align*}
    X & = a Z ^b \cdot \varepsilon
\end{align*}
Аналогично.

\subsection{Модель медленнего роста}

\begin{align*}
    X & = a + b \ln Z + \varepsilon \\
    X & = a + b Z' + \varepsilon
\end{align*}

\subsection{Модель медленной стабилизации}

\begin{align*}
    X & = a + \frac{b}{Z} + \varepsilon \\
    X & = a + Z' + \varepsilon
\end{align*}

\subsection{Модель быстрой стабилизации}

\begin{align*}
    X & = a + \frac{b}{e^Z} + \varepsilon \\
    X & = a + Z' + \varepsilon
\end{align*}

\subsection{Зависимость в виде полинома}

\begin{align*}
    X & = \beta_0 + \beta_1 Z + \beta_2 Z^2 + \dots + \beta_k Z^k   \\
    X & = \beta_0 + \beta_1 Z_1 + \beta_2 Z_2 + \dots + \beta_k Z_k
\end{align*}

\begin{remark}
    Т.к. \(Z_i\) функционально зависимы, то обычно берут \(k \leq 4\). Модель неустойчива.
\end{remark}
%</49>

\section{Метод главных компонент (приведение к главным осям)}

\underline{Идея}: \(A = ZZ\tran\) --- симметричная, положительно определенная, то есть это матрица квадратичной формы, причём все собственные числа \(\lambda_i > 0\). Ранжируем собственные числа \(\lambda_1 > \lambda_2 > \dots > \lambda_k\). Находим соответствующие собственные вектора \(X_1, X_2 \dots X_k\).

\(X_1\) показывает направление главной оси эллипсоида, дисперсия проекций результатов экспериментов на эту ось будет наибольшей. Прямую регрессии проводим через эту ось.

\begin{remark}
    Эта прямая не обязана совпадать с прямой МНК.
\end{remark}

\textcolor{red}{Иллюстрация МНК против метода главных компонент не сделана.}

\section{Метод Тейла}

Иногда применяется, если в экспериментальных данных наблюдаются выбросы.

Для регрессии \(X = a + bZ + \varepsilon\):
\[\tilde{b} \coloneqq \operatorname{med}\left\{\frac{X_i - X_j}{Z_i - Z_j}\right\}, 1 \leq i, j \leq n\]
\[r = \operatorname{med}\left\{X_i - \tilde{b} Z_i\right\}, 1 \leq i \leq n\]

\section{Ортогональная матрица плана \(Z\)}

Пусть эксперимент управляем, то есть можем выбирать значения факторов \(Z_1 \dots Z_k\). Тогда желательно составить матрицу плана \(Z\) составить так, чтобы она была ортогональной. Тогда матрица \(A = ZZ\tran\) будет диагональной, \(A^{-1}\) элементарно находится и все формулы упрощаются. Кроме того, при добавлении новых факторов в модель коэффициенты при старых не меняются. Дисперсии оценок при этом будут наименьшие из возможных, следовательно, оценки будут эффективными.
