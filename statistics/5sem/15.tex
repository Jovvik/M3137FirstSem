\chapter{13 декабря}

\section{Метод Монте-Карло}

\begin{example}[Бюффона]
    Вероятность того, что игла длины \(l\) пересечёт стык досок ширины \(l\) равна \(\frac{2}{\pi}\).
    \[\frac{n_A}{n} \xrightarrow[n \to \infty]{P} p = \frac{2}{\pi}\]
    Таким образом, бросая иголки, можно посчитать \(\frac{n_A}{n} \approx \frac{2}{\pi}\) и примерно посчитать \(\pi\).
\end{example}

%<*55>
Общая постановка метода: пусть требуется найти \(a\) и имеется случайная величина \(\xi\), такая что \(\E \xi = a\). Тогда согласно сильному закону больших чисел
\[\frac{\xi_1 + \dots + \xi_n}{n} \xrightarrow[n \to \infty]{\text{п.н.}} a\]
Поэтому при достаточно больших \(n\) среднее выборочное \(\overline{X}\) будет неплохой оценкой \(a\).

Оценим погрешность вычислений. Пусть \(\D \xi\) конечна, тогда по ЦПТ
\[\frac{S_n - na}{\sqrt{n \D \xi}} = \frac{n (\overline{X} - a)}{\sqrt{n \D \xi}} = \frac{\sqrt{n} (\overline{X} - a)}{\sqrt{\D \xi}} \xrightrightarrows{n \to \infty} Z \in N(0, 1)\]
По правилу трёх сигм \(P(|Z| < 3 \cdot 1) = 0,9973\). Тогда:
\[P\left(\frac{\sqrt{n} \left|\overline{X} - a\right|}{\sqrt{\D \xi}} < 3\right) \xrightarrow{n \to \infty} 0,9973\]
и поэтому можно считать, что \(\frac{\sqrt{n} \left|\overline{X} - a\right|}{\sqrt{\D \xi}} < 3\) и, следовательно:
\[\left|\overline{X} - a\right| < \frac{3\sqrt{\D \xi}}{\sqrt{n}}\]

По возможности надо брать \(\xi\) с минимальной дисперсией.
%</55>

\subsection{Вычисление определенных интегралов}

\[\int_a^b f(x)dx \defeq \lim_{\Delta x_i \to 0} \sum f(c_i) \Delta x_i\]
, где \(\Delta x_i\) --- длина отрезка разбиения, \(c_i \in i\)-тый отрезок разбиения.

\subsubsection{Квадратурные формулы}

\begin{enumerate}
    \item Формула прямоугольников

          Разбиваем \([a, b]\) на \(n\) равных частей, \(l = \frac{b - a}{n}\) --- длина интервала, \(x_i\) --- середине \(i\)-го интервала.

          \[I = \int_a^b \varphi(x)dx \approx \frac{b - a}{n} \sum_{i=1}^{n} \varphi(x_i) = I_n\]

          Погрешность: \(|I - I_n| \leq \frac{M\footnotemark}{n^2}\)
          \footnotetext{Число, зависящее от \(\varphi, a\) и \(b\).}

    \item Формула трапеций

          Разбиваем на \(n\) частей \([x_i, x_{i+1}], i = 0 \dots n - 1, y_i = \varphi(x_i)\).

          \[I = \int_a^b \varphi(x)dx = \frac{b - a}{2n}(y_0 + y_n + 2(y_1 + \dots + y_{n-1})) = I_n\]
          Погрешность: \(|I - I_n| \leq \frac{\frac{M}{2}}{n^2}\)

    \item Формула Симпсона

          Разбиваем \([a, b]\) на чётное число отрезков \(n = 2m, l = \frac{b - a}{n}, y_i = \varphi(x_i)\)

          \[I = \int_a^b \varphi(x)dx = \frac{b - a}{3n}\left(y_1 + y_n + 4(y_3 + y_5 + \dots + y_{n-1}) + 2(y_2 + y_4 + \dots + y_{n - 2})\right) = I_n\]
          Погрешность: \(|I - I_n| \leq \frac{C}{n^4}\)

    \item Метод Монте-Карло

          %<*56.1>
          Отличается от метода прямоугольников тем, что в качестве узлов берутся случайные числа.

          Пусть нужно посчитать \(\int_0^1 \varphi(x)dx\), \(\eta_i \in U(0, 1)\). Тогда \(\xi_i \coloneqq \varphi(\eta_i)\), плотность \(f_\eta(y) = 1, y \in [0, 1]\), \(0\) иначе.
          \[\E \xi_1 = \int_{ - \infty}^{ + \infty} \varphi(y)f_\eta(y)dy = \int_0^1 \varphi(y)dy = I\]
          \[I \approx \tilde{I}_n = \frac{1}{n} \sum_{i=1}^{n} \varphi(\eta_i)\]
          Погрешность: \(|I - I_n| \leq \frac{3 \sqrt{\D \xi_1}}{\sqrt{n}}\), где \(\D \xi_1 = \int_0^1 \varphi^2(y)dy = I^2\)

          Недостатки:
          \begin{enumerate}
              \item Медленная сходимость (корень вместо квадрата или четвёртой степени).
              \item Для оценки погрешности надо вычислить дисперсию.
              \item Оценка справедлива лишь с вероятностью, пусть и близкой к единице.
          \end{enumerate}

          В силу этого метод Монте-Карло не применяется для вычисления интегралов.
          %</56.1>
\end{enumerate}

%<*56.2>
\subsection{Вычисление кратных интегралов}

По квадратурным формулам (при достаточно высокой кратности интеграла) погрешность \( \leq C n^{ - 1 + \varepsilon}\)

Пусть требуется вычислить интеграл \(\idotsint_0^1 \varphi(x_1 \dots x_k) dx_1 \dots dx_k\). При методе Монте-Карло здесь достаточно набросать \(n\) случайных равномерно распределенных точек и взять среднее арифметическое значения \(\varphi\) в этих точках.
%</56.2>

\begin{example}
    Для вычисления объёма фигуры можем брать индикатор.
\end{example}

%<*56.3>
\subsection{Метод Монте-Карло расслоенной выборкой}

Пусть требуется вычислить интеграл \(\idotsint_0^1 \varphi(x_1 \dots x_k) dx_1 \dots dx_k\). Каждую из сторон \(k\)-мерного куба разобьем нам \(N\) равных частей. Тогда куб разобьется на \(N^k\) кубиков \(\Delta_i\) и в каждом из \(\Delta_i\) возьмём \(k\)-мерную равномерно распределенную точку \(\eta_i = (\eta_i^{(1)} \dots \eta_i^{(k)})\). Интеграл оценивается при помощи суммы
\[\tilde{I}_n = \frac{1}{n} \sum_{i=1}^{n} \varphi(\eta_i)\]
Погрешность: \(|\tilde{I}_n - I| \leq C n^{ - \frac{1}{2} - \frac{1}{k}}\)

\begin{example}
    При \(k = 1\) \(|\tilde{I}_n - I| \leq C n^{ - \frac{3}{2}}\)
\end{example}
%</56.3>

\section{Равномерность по Вейлю}

\begin{definition}
    Числовая последовательность \(X_0 \dots X_n \dots\) \textbf{равномерна по Вейлю}, если частота попадания точек на любой отрезок \([a, b] \subseteq [0, 1]\) стремится к его длине.
\end{definition}

\begin{theorem}
    Последовательность \(X_n = \{n \cdot \alpha\}\), где \(\alpha\) --- иррациональное число, а \(\{\}\) --- дробная часть. Тогда эта последовательность равномерна по Вейлю.
\end{theorem}

\begin{example}
    \(\{n \sqrt{2}\}, \{ne\}\)
\end{example}

\begin{theorem}\label{вейль}
    Если \(X_m\) равномерна по Вейлю, то \(\int_0^1 \varphi(x)dx\) можно считать как \(\frac{1}{n} \sum_{i=1}^{n} \varphi(X_i)\).
\end{theorem}

\begin{remark}
    Возможны ситуации, что последовательность \(X_n\) равномерна по Вейлю, но:
    \[\frac{1}{n} \sum_{i=1}^{n} \varphi(X_{2i - 1}, X_{2i}) \xrightarrow[n \to \infty]{} C \neq \iint_0^1 \varphi(x, y) dx dy\]
\end{remark}

\section{Парадокс первой цифры}

Рассмотрим множество степеней двойки. Оценим вероятность того, что \(7\) будет первой цифрой. По интуиции эта вероятность \(\frac{1}{9}\). Пусть \(m\) --- первая цифра числа \(2^n\). Тогда \(m \cdot 10^b \leq 2^n < (m + 1) \cdot 10^b\)

\[m \cdot 10^b \leq 2^n < (m + 1) \cdot 10^b\]
\[\log_{10} m + b \leq n \log_{10} 2 < \log_{10}(m + 1) + b\]
\[\log_{10} m \leq \{n \log_{10} 2\} < \log_{10}(m + 1)\]
\(\log_{10} 2\) --- иррациональное, поэтому по теореме~\ref{вейль} эта последовательность равномерна по Вейлю.
Тогда \(p_m = \log_{10}(m + 1) - \log_{10} m\). В частности, при \(m = 7\) \(p_7 = 0,056\).
