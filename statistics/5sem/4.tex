\chapter{18 сентября}

\section{Распределения в матстатистике}

\setcounter{subsection}{-1}

%<*6>
\subsection{Нормальное распределение}

\(X \in N(a, \sigma^2)\):
\[\E X = a, \D X = \sigma^2\]
\(N(0, 1)\) --- стандартное нормальное распределение

\subsection{Гамма-распределение}

\(X \in \Gamma_{\alpha, \lambda}\), если её плотность равна:
\[f(x) = \begin{cases}
        0,                                                                     & x \leq 0 \\
        \frac{\alpha^\lambda}{\Gamma(\lambda)} x^{\lambda - 1} e^{ -\alpha x}, & x > 0
    \end{cases}\]

\begin{prop}\itemfix
    \begin{enumerate}
        \item \(\E \xi = \frac{\lambda}{\alpha}, \D \xi = \frac{\lambda}{\alpha^2}\)
        \item Если \(\xi_1 \in \Gamma_{\alpha, \lambda_1}, \xi_2 \in \Gamma_{\alpha, \lambda_2}\), то \(\xi_1 + \xi_2 \in \Gamma_{\alpha, \lambda_1 + \lambda_2}\)
        \item \(\Gamma_{\alpha, 1} = E_\alpha\) --- показательное распределение.
        \item Если \(X_i \in E_\alpha\), то \(\sum_{i=1}^{n} X_i \in \Gamma_{\alpha, n}\)
        \item Если \(X \in N(0, 1)\), то \(X^2 \in \Gamma_{\frac{1}{2},\frac{1}{2}}\)
    \end{enumerate}
\end{prop}

\begin{remark}
    Гамма-распределение возникает в матстатистике как распределение квадрата стандартно нормально распределенной величины. Обобщим эту идею:
\end{remark}

\subsection{Распределение ``хи-квадрат''}

\begin{definition}
    Распределением \textbf{хи-квадрат} с \(k\) степенями свободы называется распределение суммы \(k\) квадратов независимых стандартных нормальных величин.
    \[\chi^2_k = X_1^2 + X_2^2 + \dots + X_k^2, \quad X_i \in N(0, 1)\]
\end{definition}
\begin{notation}
    \(\chi^2 \in H_k\)
\end{notation}

\begin{prop}\itemfix
    \begin{enumerate}
        \item \(\chi_k^2 \in \Gamma_{\frac{1}{2}, \frac{k}{2}}\)
        \item \(\chi_n^2 + \chi_m^2 = \chi_{n + m}^2\) --- по определению
        \item \(\E \chi_k^2 = \frac{\lambda}{\alpha} = \frac{\frac{k}{2}}{\frac{1}{2}} = k, \D \chi_k^2 = \frac{\lambda}{\alpha^2} = \frac{\frac{k}{2}}{\left(\frac{1}{2}\right)^2} = 2k\)
    \end{enumerate}
\end{prop}

\subsection{Распределение Стьюдента}

\begin{definition}
    Пусть случайные величины \(X_0, X_1 \dots X_k\) --- независимы и имеют стандартное нормальное распределение. Распределением \textbf{Стьюдента} с \(k\) степеней свободы называется распределение случайной величины
    \[t_k = \frac{X_0}{\sqrt{\frac{1}{k}(X_1^2 + \dots + X_k^2)}} = \frac{X_0}{\sqrt{\frac{1}{k} \chi_k^2}}\]
\end{definition}

\begin{prop}\itemfix
    \begin{enumerate}
        \item \(\E t_k = 0\)
        \item \(\D t_k = \frac{k}{k - 2}\)
    \end{enumerate}
\end{prop}

\subsection{Распределение Фишера-Снедекора}

\begin{definition}
    Распределение \(F_{m, n}\) называется распределением \textbf{Фишера-Снедекора} \textit{(или \textbf{\(F\)-распределением})} со степенями свободы \(m\) и \(n\) называется распределение случайной величины
    \[f_{m, n} = \frac{\frac{\chi_m^2}{m}}{\frac{\chi_n^2}{n}}\]
    , где \(\chi_n^2\) и \(\chi_m^2\) --- независимые случайные величины с распределением \(\chi^2\).
\end{definition}

\begin{prop}\itemfix
    \begin{enumerate}
        \item \(\E f_{m, n} = \frac{n}{n - 2}\)
        \item \(\D f_{m, n} = \frac{2n^2(m + n - 2)}{m(n - 2)^2(n - 4)}\)
        \item \(F_{m, n}(x) = P(f_{m, n} < x) = P\left(\frac{1}{f_{m, n}} > \frac{1}{x}\right) = P\left(f_{m, n} > \frac{1}{x}\right) = 1 - F_{m, n}\left(\frac{1}{x}\right)\)
    \end{enumerate}
\end{prop}

При \(n, k, m \to \infty\) эти распределения слабо сходятся к нормальному. При \(n > 30\) они достаточно близки.
%</6>

\section{Линейные преобразования нормальных выборок}

%<*7>
Пусть \(\vec{X} = (X_1 \dots X_n)\), где \(X_i \in N(0, 1)\) и независимы. Будем рассматривать линейные комбинации этого вектора. Пусть \(A\) --- невырожденная матрица размера \(n \times n\). Рассмотрим случайный вектор \(\vec{Y} = A \vec{X}\), где координаты случайного вектора \(Y_i = a_{i 1} X_1 + \dots + a_{i n} X_n\). Будем исследовать, что из себя представляют \(Y_i\) и их совместное распределение.

\begin{remark}
    Если \(\eta = a \xi + b\), то \(f_\eta(\xi) = \frac{1}{|a|} f_\xi\left(\frac{\xi - b}{a}\right)\)
\end{remark}

\begin{theorem}
    Пусть случайный вектор \(\vec{X}\) имеет плотность распределения \(f_{\vec{X}} (\vec{x})\) и \(A\) невырожденная матрица.

    Тогда случайный вектор \(\vec{Y} = A \vec{X} + \vec{b}\) имеет плотность
    \[f_{ \vec{Y}}(\vec{y}) = \frac{1}{|\det A|} \cdot f_{ \vec{X}}(A^{ - 1}(\vec{y} - \vec{b}))\]
\end{theorem}

\begin{remark}
    \(f_{ \vec{X}}(\vec{x})\) --- плотность \(\vec{X}\), если \(P(\vec{x} \in B) = \idotsint_{B} f_{ \vec{X}}(\vec{x}) d \vec{x}\)
\end{remark}

\begin{proof}
    \begin{align*}
        P(\vec{y} \in B) & = P(A \vec{x} + \vec{b} \in B)                              \\
                         & = P(\vec{x} \in A^{-1}(\vec{y} - \vec{b}))                  \\
                         & = \idotsint_{A^{-1}(B - \vec{b})} f_{ \vec{x}}(x) d \vec{x}
    \end{align*}

    Сделаем замену \(\vec{y} = A \vec{x} + \vec{b}\). Тогда \(A^{-1}(B - \vec{b})\) перейдёт в \(B\), \(\vec{x}\) перейдёт в \(A^{-1}(\vec{y} - \vec{b}), \vec{y} \in B\), \(d \vec{x}\) перейдёт  \(|J| d \vec{y}\), где \(J = |A^{-1}| = |A|^{ - 1}\)

    Итого:
    \[ = \idotsint_B f(A^{-1}(\vec{y} - \vec{b})) \cdot \frac{1}{|\det A|} d \vec{y} \Rightarrow f_{ \vec{Y}}(\vec{y}) = \frac{1}{|\det A|} f_{ \vec{X}}(A^{-1}(\vec{y} - \vec{b}))\]
\end{proof}

\begin{definition}
    \(A = C\) --- \textbf{ортогональна}, т.е. \(C\tran = C^{-1}, |\det C| = 1\)
\end{definition}

\begin{theorem}
    Пусть дан случайный вектор \(\vec{X} = (X_1 \dots X_n)\), где \(\forall i \ \ X_i \in N(0, 1)\) и \(X_i\) независимы, а \(C\) --- ортогональная матрица.

    Тогда координаты случайного вектора \(\vec{Y} = C \vec{X}\) независимы и также имеют стандартное нормальное распределение.
\end{theorem}
\begin{proof}
    Т.к. координаты \(X_i \in N(0, 1)\) и независимы, то плотность \(\vec{X}\):
    \[f_{ \vec{X}}(\vec{x}) = \prod_{i=1}^{n} f_i (x_i) = \prod_{i=1}^{n} \frac{1}{\sqrt{2 \pi}} e^{ - \frac{x_i^2}{2}} = \frac{1}{(2 \pi)^{\frac{n}{2}}} e^{ - \frac{1}{2} (x_1^2 + x_2^2 + \dots + x_n^2)} = \frac{1}{(2 \pi)^{\frac{n}{2}}} e^{ - \frac{1}{2} ||\vec{x}||^2}\]
    По предыдущей теореме:
    \[f_{ \vec{Y}}(\vec{y}) = f_{ \vec{X}}(C\tran \vec{y}) = \frac{1}{(2 \pi)^{\frac{n}{2}}} e^{ - \frac{1}{2} ||C\tran \vec{y}||^2} = \frac{1}{(2 \pi)^{\frac{n}{2}}} e^{ - \frac{1}{2} ||\vec{y}||^2} = \prod_{i=1}^{n} \frac{1}{\sqrt{2 \pi}} e^{ - \frac{1}{2} y_i^2} = \prod_{i=1}^{n} f_i(y_i)\]
    Следовательно, \(Y_i \in N(0, 1)\) и независимы.
\end{proof}
%</7>

%<*8>
\begin{lemma}[Фишера]
    Пусть случайный вектор \(\vec{X}\) состоит из независимых стандартных нормальных случайных величин, \(\vec{Y} = C \vec{X}\), где \(C\) --- ортогональная матрица. Тогда \(\forall k : 1 \leq k \leq n - 1\) случайная величина
    \[T(\vec{X}) = \left(\sum_{i=1}^{n} X_i^2\right) - Y_1^2 - \dots - Y_k^2\]
    не зависит от случайного вектора \(Y_1 \dots Y_k\) и имеет распределение \(H_{n - k}\)
\end{lemma}
\begin{proof}
    Т.к. \(C\) ортогональна:
    \[||\vec{Y}||^2 = ||C \vec{X}||^2 = ||\vec{X}||^2 = X_1^2 + \dots + X_n^2 = Y_1^2 + \dots + Y_n^2\]

    Отсюда
    \[T(\vec{X}) = \sum_{i=1}^{n} Y_i^2 - Y_1^2 - \dots - Y_k^2 = Y_{k+1}^2 + \dots + Y_n^2\]

    \(Y_{k+1} \dots Y_n\) --- независимы, имеют стандартное нормальное распределение и \(T(\vec{X}) \in H_{n - k}\) по определению распределения \(\chi^2\).

    \(T(\vec{X})\) не зависит от \(Y_1 \dots Y_k\), т.к. \(Y_{k+1} \dots Y_n\) по предыдущей лемме от них не зависят.
\end{proof}
%</8>

%<*9>
\begin{theorem}[основная]\itemfix
    \label{основная}
    \begin{itemize}
        \item \(X_1 \dots X_k\) независимы и имеют нормальное распределение с параметрами \(a\) и \(\sigma^2\)
        \item \(\overline{X}\) --- выборочное среднее
        \item \(S^2\) --- исправленная выборочная дисперсия
    \end{itemize}

    Тогда имеют место следующие распределения:
    \begin{enumerate}
        \item \[\sqrt{n} \cdot \frac{\overline{X} - a}{\sigma} \in N(0, 1)\]
        \item \[\sum_{i=1}^{n} \left(\frac{X_i - a}{\sigma}\right)^2 \in H_n\]
        \item \[\sum_{i=1}^{n} \left(\frac{X_i - \overline{X}}{\sigma}\right)^2 =  \frac{(n - 1) S^2}{\sigma^2} \in H_{n-1}\]
        \item \[\sqrt{n} \cdot \frac{\overline{X} - a}{S} \in T_{n-1}\]
        \item \(\overline{X}\) и \(S^2\) --- независимые случайные величины
    \end{enumerate}
\end{theorem}
\begin{proof}\itemfix
    \begin{enumerate}
        \item \[X_i \in N(a, \sigma^2) \Rightarrow \sum_{i=1}^{n} X_i \in N(na, n\sigma^2) \Rightarrow \overline{X} \in N\left(a, \frac{\sigma^2}{n}\right) \Rightarrow \frac{\sqrt{n}}{\sigma}(\overline{X} - a) \in N(0, 1)\]
        \item Верно, т.к. \(\cfrac{X_i - a}{\sigma} \in N(0, 1)\)
        \item \begin{align*}
                  \sum_{i=1}^{n} \left(\frac{X_i - \overline{X}}{\sigma}\right)^2 & = \sum_{i=1}^{n} \left(\frac{X_i - a}{\sigma} - \frac{ \overline{X} - a}{\sigma}\right)^2 \\
                                                                                  & = \sum_{i=1}^{n} (z_i - \overline{z})^2                                                   \\
              \end{align*}
              , где \[z_i = \frac{X_i - a}{\sigma} \in N(0, 1), \overline{z} = \frac{\sum_{i=1}^{n} z_i}{n} = \frac{\sum X_i - na}{\sigma n}= \frac{ \overline{X} - a}{\sigma}\]
              Поэтому можем считать, что \(X_i \in N(0, 1)\). Применим лемму Фишера.
              \[T(\vec{X}) = \sum_{i=1}^{n} \left(X_i - \overline{X}\right)^2 = n\D_B = n (\overline{X^2} - (\overline{X})^2) = \sum_{i=1}^{n} X_i^2 - n (\overline{X})^2 = \sum_{i=1}^{n} X_i^2 - Y_1^2\]
              , где
              \[Y_1^2 = n(\overline{X})^2 \quad Y_1 = \sqrt{n} \overline{X} = \frac{1}{\sqrt{n}} X_1 + \dots + \frac{1}{\sqrt{n}} X_n\]
              Так как длина\footnote{То есть норма строки как вектора.} строки \(\frac{1}{\sqrt{n}}, \dots , \frac{1}{\sqrt{n}}\) равна \(1\), можем\footnote{По некой теореме.} дополнить эту строку до ортогональной матрицы \(C\). Тогда \(Y_1\) --- первая координата случайного вектора \(\vec{Y} = C \vec{X}\) и по лемме Фишера \(T(\vec{X}) \in H_{n - 1}\) \setcounter{enumi}{4}
        \item \(T(\vec{X}) = \frac{(n - 1)S^2}{\sigma^2}\) не зависит от \(Y_1 = \sqrt{n} \overline{X} \Rightarrow S^2\) и \(\overline{X}\) независимы. \setcounter{enumi}{3}
        \item \[\sqrt{n} \frac{ \overline{X} - a}{S} = \sqrt{n} \frac{\overline{X} - a}{\sigma} \cdot \frac{1}{\sqrt{\frac{(n - 1)S^2}{\sigma^2} \cdot \frac{1}{n - 1}}} = \frac{X_0}{\sqrt{\frac{\chi_{n - 1}^2}{n - 1}}} \in T_{n - 1} \text{ , т.к.:}\]
              \(X_0 \in N(0, 1)\) по пункту 1, \(\frac{(n - 1)S^2}{\sigma^2} \in H_{n-1}\) по пункту 3 и \(X_0\) не зависит от \(\frac{(n - 1)S^2}{\sigma^2}\) по пункту 5.
    \end{enumerate}
\end{proof}
%</9>

\begin{remark}
    Эта часть была рассказана на практике 29 сентября.
\end{remark}

\subsection{Многомерные нормальные распределения}

\begin{definition}
    Пусть случайный вектор \(\vec{\xi} = (\xi_1 \dots \xi_n)\) имеет в средних \(\vec{a} = (\E \xi_1 \dots \E \xi_n)\), \(K\) --- симметричная положительно определенная метрица.

    Вектор \(\vec{\xi}\) имеет \textbf{многомерное нормальное} распределение с параметрами \(\vec{a}\) и \(K\), если его плотность:
    \[f_{ \vec{\xi}}(\vec{x}) = \frac{1}{\sqrt{2 \pi}^n \sqrt{\det K}} e^{ - \frac{1}{2} ((\vec{x} - \vec{a})\tran K^{-1} (\overline{x} - \overline{a}))}\]
\end{definition}
\begin{remark}
    \((\vec{x} - \vec{a})\tran K^{-1} (\vec{x} - \vec{a})\) --- положительно определенная квадратичная форма от \((x_1 \dots x_n)\)
\end{remark}

\begin{prop}\itemfix
    \begin{enumerate}
        \item Пусть \(\vec{\eta}\) состоит из независимых стандартных нормальных величин, \(B\) --- невырожденная матрица. Тогда \(\vec{\xi} = B \vec{\eta} + \vec{a}\) имеет многомерное нормальное распределение с параметрами \(\vec{a}, K = B\tran B\)
        \item Пусть \(\vec{\xi}\) имеет многомерное нормальное распределение с параметрами \(\vec{a}\) и \(K\). Тогда \(\eta = B^{-1}(\vec{\xi} - \vec{a})\), где \(B = \sqrt{K}\)\footnote{\(B\) существует по задаче 3 из 4-ой практики.}, состоит из независимы стандартных нормальных величин.
        \item \(K = \cov(\xi_i, \xi_j)\)
        \item Пусть \(\vec{\xi}\) имеет многомерное нормальное распределение спараметрами \(\vec{a}\) и \(K\). Координаты \(\vec{\xi}\) независимы тогда и только тогда, когда они не кореллированы, т.е. \(K\) --- диагональная.

              \begin{corollary}
                  Если \(\xi, \eta\) --- нормальные случайные величины и вектор \((\xi, \eta)\) имеет ненулевую плотность, то \(\xi\) и \(\eta\) независимы тогда и только тогда, когда они не кореллированы, т.е. \(r_{\xi, \eta} = 0\).
              \end{corollary}
    \end{enumerate}
\end{prop}

\begin{theorem}[многомерная центральная предельная теорема]
    Среднее арифметическое независимых одинаково распределенных случайных векторов слабо сходится к многомерному нормальному распределению.
\end{theorem}
