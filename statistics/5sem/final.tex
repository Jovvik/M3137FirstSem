\documentclass[12pt, a4paper]{article}

%<*preamble>
% Math symbols
\usepackage{amsmath, amsthm, amsfonts, amssymb}
\usepackage{accents}
\usepackage{esvect}
\usepackage{mathrsfs}
\usepackage{mathtools}
\mathtoolsset{showonlyrefs}
\usepackage{cmll}
\usepackage{stmaryrd}
\usepackage{physics}
\usepackage[normalem]{ulem}
\usepackage{ebproof}
\usepackage{extarrows}

% Page layout
\usepackage{geometry, a4wide, parskip, fancyhdr}

% Font, encoding, russian support
\usepackage[russian]{babel}
\usepackage[sb]{libertine}
\usepackage{xltxtra}

% Listings
\usepackage{listings}
\lstset{basicstyle=\ttfamily,breaklines=true}
\setmonofont{Inconsolata}

% Miscellaneous
\usepackage{array}
\usepackage{calc}
\usepackage{caption}
\usepackage{subcaption}
\captionsetup{justification=centering,margin=2cm}
\usepackage{catchfilebetweentags}
\usepackage{enumitem}
\usepackage{etoolbox}
\usepackage{float}
\usepackage{lastpage}
\usepackage{minted}
\usepackage{svg}
\usepackage{wrapfig}
\usepackage{xcolor}
\usepackage[makeroom]{cancel}

\newcolumntype{L}{>{$}l<{$}}
    \newcolumntype{C}{>{$}c<{$}}
\newcolumntype{R}{>{$}r<{$}}

% Footnotes
\usepackage[hang]{footmisc}
\setlength{\footnotemargin}{2mm}
\makeatletter
\def\blfootnote{\gdef\@thefnmark{}\@footnotetext}
\makeatother

% References
\usepackage{hyperref}
\hypersetup{
    colorlinks,
    linkcolor={blue!80!black},
    citecolor={blue!80!black},
    urlcolor={blue!80!black},
}

% tikz
\usepackage{tikz}
\usepackage{tikz-cd}
\usetikzlibrary{arrows.meta}
\usetikzlibrary{decorations.pathmorphing}
\usetikzlibrary{calc}
\usetikzlibrary{patterns}
\usepackage{pgfplots}
\pgfplotsset{width=10cm,compat=1.9}
\newcommand\irregularcircle[2]{% radius, irregularity
    \pgfextra {\pgfmathsetmacro\len{(#1)+rand*(#2)}}
    +(0:\len pt)
    \foreach \a in {10,20,...,350}{
            \pgfextra {\pgfmathsetmacro\len{(#1)+rand*(#2)}}
            -- +(\a:\len pt)
        } -- cycle
}

\providetoggle{useproofs}
\settoggle{useproofs}{false}

\pagestyle{fancy}
\lfoot{M3137y2019}
\cfoot{}
\rhead{стр. \thepage\ из \pageref*{LastPage}}

\newcommand{\R}{\mathbb{R}}
\newcommand{\Q}{\mathbb{Q}}
\newcommand{\Z}{\mathbb{Z}}
\newcommand{\B}{\mathbb{B}}
\newcommand{\N}{\mathbb{N}}
\renewcommand{\Re}{\mathfrak{R}}
\renewcommand{\Im}{\mathfrak{I}}

\newcommand{\const}{\text{const}}
\newcommand{\cond}{\text{cond}}

\newcommand{\teormin}{\textcolor{red}{!}\ }

\DeclareMathOperator*{\xor}{\oplus}
\DeclareMathOperator*{\equ}{\sim}
\DeclareMathOperator{\sign}{\text{sign}}
\DeclareMathOperator{\Sym}{\text{Sym}}
\DeclareMathOperator{\Asym}{\text{Asym}}

\DeclarePairedDelimiter{\ceil}{\lceil}{\rceil}

% godel
\newbox\gnBoxA
\newdimen\gnCornerHgt
\setbox\gnBoxA=\hbox{$\ulcorner$}
\global\gnCornerHgt=\ht\gnBoxA
\newdimen\gnArgHgt
\def\godel #1{%
    \setbox\gnBoxA=\hbox{$#1$}%
    \gnArgHgt=\ht\gnBoxA%
    \ifnum     \gnArgHgt<\gnCornerHgt \gnArgHgt=0pt%
    \else \advance \gnArgHgt by -\gnCornerHgt%
    \fi \raise\gnArgHgt\hbox{$\ulcorner$} \box\gnBoxA %
    \raise\gnArgHgt\hbox{$\urcorner$}}

% \theoremstyle{plain}

\theoremstyle{definition}
\newtheorem{theorem}{Теорема}
\newtheorem*{definition}{Определение}
\newtheorem{axiom}{Аксиома}
\newtheorem*{axiom*}{Аксиома}
\newtheorem{lemma}{Лемма}

\theoremstyle{remark}
\newtheorem*{remark}{Примечание}
\newtheorem*{exercise}{Упражнение}
\newtheorem{corollary}{Следствие}[theorem]
\newtheorem*{statement}{Утверждение}
\newtheorem*{corollary*}{Следствие}
\newtheorem*{example}{Пример}
\newtheorem{observation}{Наблюдение}
\newtheorem*{prop}{Свойства}
\newtheorem*{obozn}{Обозначение}

% subtheorem
\makeatletter
\newenvironment{subtheorem}[1]{%
    \def\subtheoremcounter{#1}%
    \refstepcounter{#1}%
    \protected@edef\theparentnumber{\csname the#1\endcsname}%
    \setcounter{parentnumber}{\value{#1}}%
    \setcounter{#1}{0}%
    \expandafter\def\csname the#1\endcsname{\theparentnumber.\Alph{#1}}%
    \ignorespaces
}{%
    \setcounter{\subtheoremcounter}{\value{parentnumber}}%
    \ignorespacesafterend
}
\makeatother
\newcounter{parentnumber}

\newtheorem{manualtheoreminner}{Теорема}
\newenvironment{manualtheorem}[1]{%
    \renewcommand\themanualtheoreminner{#1}%
    \manualtheoreminner
}{\endmanualtheoreminner}

\newcommand{\dbltilde}[1]{\accentset{\approx}{#1}}
\newcommand{\intt}{\int\!}

% magical thing that fixes paragraphs
\makeatletter
\patchcmd{\CatchFBT@Fin@l}{\endlinechar\m@ne}{}
{}{\typeout{Unsuccessful patch!}}
\makeatother

\newcommand{\get}[2]{
    \ExecuteMetaData[#1]{#2}
}

\newcommand{\getproof}[2]{
    \iftoggle{useproofs}{\ExecuteMetaData[#1]{#2proof}}{}
}

\newcommand{\getwithproof}[2]{
    \get{#1}{#2}
    \getproof{#1}{#2}
}

\newcommand{\import}[3]{
    \subsection{#1}
    \getwithproof{#2}{#3}
}

\newcommand{\given}[1]{
    Дано выше. (\ref{#1}, стр. \pageref{#1})
}

\renewcommand{\ker}{\text{Ker }}
\newcommand{\im}{\text{Im }}
\renewcommand{\grad}{\text{grad}}
\newcommand{\rg}{\text{rg}}
\newcommand{\defeq}{\stackrel{\text{def}}{=}}
\newcommand{\defeqfor}[1]{\stackrel{\text{def } #1}{=}}
\newcommand{\itemfix}{\leavevmode\makeatletter\makeatother}
\newcommand{\?}{\textcolor{red}{???}}
\renewcommand{\emptyset}{\varnothing}
\newcommand{\longarrow}[1]{\xRightarrow[#1]{\qquad}}
\DeclareMathOperator*{\esup}{\text{ess sup}}
\newcommand\smallO{
    \mathchoice
    {{\scriptstyle\mathcal{O}}}% \displaystyle
    {{\scriptstyle\mathcal{O}}}% \textstyle
    {{\scriptscriptstyle\mathcal{O}}}% \scriptstyle
    {\scalebox{.6}{$\scriptscriptstyle\mathcal{O}$}}%\scriptscriptstyle
}
\renewcommand{\div}{\text{div}\ }
\newcommand{\rot}{\text{rot}\ }
\newcommand{\cov}{\text{cov}}

\makeatletter
\newcommand{\oplabel}[1]{\refstepcounter{equation}(\theequation\ltx@label{#1})}
\makeatother

\newcommand{\symref}[2]{\stackrel{\oplabel{#1}}{#2}}
\newcommand{\symrefeq}[1]{\symref{#1}{=}}

% xrightrightarrows
\makeatletter
\newcommand*{\relrelbarsep}{.386ex}
\newcommand*{\relrelbar}{%
    \mathrel{%
        \mathpalette\@relrelbar\relrelbarsep
    }%
}
\newcommand*{\@relrelbar}[2]{%
    \raise#2\hbox to 0pt{$\m@th#1\relbar$\hss}%
    \lower#2\hbox{$\m@th#1\relbar$}%
}
\providecommand*{\rightrightarrowsfill@}{%
    \arrowfill@\relrelbar\relrelbar\rightrightarrows
}
\providecommand*{\leftleftarrowsfill@}{%
    \arrowfill@\leftleftarrows\relrelbar\relrelbar
}
\providecommand*{\xrightrightarrows}[2][]{%
    \ext@arrow 0359\rightrightarrowsfill@{#1}{#2}%
}
\providecommand*{\xleftleftarrows}[2][]{%
    \ext@arrow 3095\leftleftarrowsfill@{#1}{#2}%
}

\allowdisplaybreaks

\newcommand{\unfinished}{\textcolor{red}{Не дописано}}

% Reproducible pdf builds 
\special{pdf:trailerid [
<00112233445566778899aabbccddeeff>
<00112233445566778899aabbccddeeff>
]}
%</preamble>


\usepackage{sectsty}

\allsectionsfont{\raggedright}
\sectionfont{\fontsize{14}{15}\selectfont}
\subsectionfont{\fontsize{14}{15}\selectfont}

\lhead{Матстат, билеты}
\rfoot{}
\lfoot{}

\settoggle{useproofs}{true}

\renewcommand{\import}[3]{
    \section{#1}
    \getwithproof{#2}{#3}
}

\newcommand{\notenough}{
    \textcolor{red}{Кажется, тут недостаточно написано.}
}

\begin{document}

\import{Точечные оценки. Их свойства: состоятельность, несмещенность, эффективность.}{2}{1}

\import{Точечные оценки моментов. Свойства оценок математического ожидания и дисперсии.}{2}{2.1}
\get{2}{2.2}

\import{Метод моментов. Пример.}{2}{3}

\import{Метод максимального правдоподобия. Пример.}{3}{4}
\get{3}{4.1}
\get{3}{4.2}

\import{Информация Фишера. Неравенство Рао-Крамера (без док--ва).}{3}{5}

\import{Основные распределения математической статистики: хи--квадрат, Стьюдента, Фишера-Снедекора. Их свойства.}{4}{6}

\import{Линейные преобразования нормальных выборок. Теорема об ортогональном преобразовании. }{4}{7}

\import{Лемма Фишера.}{4}{8}

\import{Основная теорема о связи точечных оценок нормального распределения и основных распределений статистики.}{4}{9}

\import{Квантили распределений (оба определения). Функции для их вычисления в EXCEL.}{5}{10}

\import{Интервальные оценки. Определения, смысл, терминология.}{5}{11}
\notenough

\import{Доверительный интервал для математического ожидания нормального распределения при известном \(\sigma\).}{5}{12}

\import{Доверительный интервал для математического ожидания нормального распределения при неизвестном \(\sigma\).}{5}{13}

\import{Доверительный интервал для дисперсии нормального распределения при неизвестном \(a\).}{5}{14}

\import{Доверительный интервал для дисперсии нормального распределения при известном \(a\).}{5}{15}

\import{Проверка статистических гипотез. Определения, терминология. Уровень значимости и мощность критерия.}{6}{16}
\get{6}{16.2}

\import{Способы сравнения критериев проверки гипотез.}{6}{17}

\import{Построение критериев согласия (основные принципы).}{6}{18}

\import{Гипотеза о среднем нормальной совокупности с известной дисперсией.}{6}{19}

\import{Гипотеза о среднем нормальной совокупности с неизвестной дисперсией.}{6}{20}

\import{Доверительные интервалы как критерии гипотез о параметрах распределения.}{6}{21}

\import{Критерий хи--квадрат для параметрической гипотезы.}{7}{22}

\import{Критерий хи--квадрат для гипотезы о распределении.}{7}{23}

Дальше все аналогично с прошлым билетом.

\import{Критерий Колмогорова для гипотезы о распределении.}{7}{24}

\import{Критерий Колмогорова-Смирнова.}{7}{25}

\import{Критерий Фишера.}{7}{26}

\import{Критерий Стьюдента.}{7}{27}

\import{Понятие статистической зависимости. Корреляционное облако и корреляционная таблица. Первоначальные выводы по ним.}{8}{28}

\import{Критерий хи--квадрат для проверки независимости.}{8}{29}

\import{Однофакторный дисперсионный анализ. Общая, межгрупповая и внутригрупповая дисперсии. Теорема о разложении дисперсии.}{8}{30}

\import{Однофакторный дисперсионный анализ. Проверка гипотезы о влиянии фактора.}{8}{31}

\import{Математическая модель регрессии. Основные понятия и определения. Метод наименьших квадратов.}{9}{32}

\import{Вывод уравнения линейной парной регрессии. Геометрический смысл прямой регрессии.}{9}{33}

\import{Выборочный коэффициент линейной корреляции. Проверка гипотезы о его значимости.}{9}{34}

\import{Выборочное корреляционное отношение, его свойства.}{9}{35}

\import{Свойства ошибок в модели линейной парной регрессии. Анализ дисперсии фактора--результата. Коэффициент детерминации, его свойства.}{10}{36}

\import{Проверка гипотезы о значимости уравнения линейной регрессии. Связь между коэффициентом детерминации и коэффициентом линейной корреляции.}{10}{37}

\import{Теорема Гаусса-Маркова.}{10}{38}

\import{Стандартные ошибки коэффициентов регрессии. Их доверительные интервалы.}{10}{39}

\import{Прогнозирование в модели линейной парной регрессии. Стандартная ошибка прогноза, доверительный интервал прогноза.}{10}{40}

\import{Общая модель линейной регрессии. Вывод нормального уравнения (свойство существования квадратного корня симметрической матрицы без доказательства).}{11}{41}

\import{Свойства ОНМК в уравнении общей линейной регрессии.}{11}{42}

\import{Основная теорема об ОМНК (п.2 без доказательства).}{11}{43}

\import{Мультиколлинеарность, ее неприятные последствия. Основные принципы отбора факторов в модель общей линейной регрессии.}{12}{44}

\import{Стандартная ошибка общей линейной регрессии и стандартные ошибки коэффициентов регрессии. Проверка гипотезы о значимости отдельного коэффициента регрессии.}{12}{45.1}
\get{12}{45.2}

\import{Уравнение регрессии в стандартных масштабах. Смысл стандартизованных коэффициентов и частных коэффициентов эластичности. Разложение влияния фактора на прямое и косвенное.}{12}{46}

\import{Коэффициенты детерминации и множественной корреляции, их свойства. Проверка гипотезы о значимости уравнения регрессии в целом.}{12}{47}

\import{Взвешенный МНК.}{13}{48}

\import{Приемы сведения нелинейных регрессий к линейным.}{13}{49}

\import{Математические датчики случайных чисел.}{14}{50}

\import{Моделирование случайных величин методом обратной функции (включая дискретный случай).}{14}{51.1}
\get{14}{51.2}
\label{51конец}

\import{Моделирование нормальной случайной величины.}{14}{52}

\import{Быстрый показательный датчик.}{14}{53}

\import{Моделирование дискретных случайных величин.}{14}{54}
\textcolor{red}{Возможно, ещё нужно рассказать про моделирование через обратную функцию, см. конец билета~\ref{51конец}, стр. \pageref{51конец}.}

\import{Метод Монте-Карло. Общая постановка, оценка погрешности.}{15}{55}

\import{Вычисление определенного и кратного интегралов методом Монте-Карло. Метод расслоенной выборки.}{15}{56.1}
\get{15}{56.2}
\get{15}{56.3}

\end{document}
