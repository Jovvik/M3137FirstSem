\documentclass[12pt, a4paper]{article}

%<*preamble>
% Math symbols
\usepackage{amsmath, amsthm, amsfonts, amssymb}
\usepackage{accents}
\usepackage{esvect}
\usepackage{mathrsfs}
\usepackage{mathtools}
\mathtoolsset{showonlyrefs}
\usepackage{cmll}
\usepackage{stmaryrd}
\usepackage{physics}
\usepackage[normalem]{ulem}
\usepackage{ebproof}
\usepackage{extarrows}

% Page layout
\usepackage{geometry, a4wide, parskip, fancyhdr}

% Font, encoding, russian support
\usepackage[russian]{babel}
\usepackage[sb]{libertine}
\usepackage{xltxtra}

% Listings
\usepackage{listings}
\lstset{basicstyle=\ttfamily,breaklines=true}
\setmonofont{Inconsolata}

% Miscellaneous
\usepackage{array}
\usepackage{calc}
\usepackage{caption}
\usepackage{subcaption}
\captionsetup{justification=centering,margin=2cm}
\usepackage{catchfilebetweentags}
\usepackage{enumitem}
\usepackage{etoolbox}
\usepackage{float}
\usepackage{lastpage}
\usepackage{minted}
\usepackage{svg}
\usepackage{wrapfig}
\usepackage{xcolor}
\usepackage[makeroom]{cancel}

\newcolumntype{L}{>{$}l<{$}}
    \newcolumntype{C}{>{$}c<{$}}
\newcolumntype{R}{>{$}r<{$}}

% Footnotes
\usepackage[hang]{footmisc}
\setlength{\footnotemargin}{2mm}
\makeatletter
\def\blfootnote{\gdef\@thefnmark{}\@footnotetext}
\makeatother

% References
\usepackage{hyperref}
\hypersetup{
    colorlinks,
    linkcolor={blue!80!black},
    citecolor={blue!80!black},
    urlcolor={blue!80!black},
}

% tikz
\usepackage{tikz}
\usepackage{tikz-cd}
\usetikzlibrary{arrows.meta}
\usetikzlibrary{decorations.pathmorphing}
\usetikzlibrary{calc}
\usetikzlibrary{patterns}
\usepackage{pgfplots}
\pgfplotsset{width=10cm,compat=1.9}
\newcommand\irregularcircle[2]{% radius, irregularity
    \pgfextra {\pgfmathsetmacro\len{(#1)+rand*(#2)}}
    +(0:\len pt)
    \foreach \a in {10,20,...,350}{
            \pgfextra {\pgfmathsetmacro\len{(#1)+rand*(#2)}}
            -- +(\a:\len pt)
        } -- cycle
}

\providetoggle{useproofs}
\settoggle{useproofs}{false}

\pagestyle{fancy}
\lfoot{M3137y2019}
\cfoot{}
\rhead{стр. \thepage\ из \pageref*{LastPage}}

\newcommand{\R}{\mathbb{R}}
\newcommand{\Q}{\mathbb{Q}}
\newcommand{\Z}{\mathbb{Z}}
\newcommand{\B}{\mathbb{B}}
\newcommand{\N}{\mathbb{N}}
\renewcommand{\Re}{\mathfrak{R}}
\renewcommand{\Im}{\mathfrak{I}}

\newcommand{\const}{\text{const}}
\newcommand{\cond}{\text{cond}}

\newcommand{\teormin}{\textcolor{red}{!}\ }

\DeclareMathOperator*{\xor}{\oplus}
\DeclareMathOperator*{\equ}{\sim}
\DeclareMathOperator{\sign}{\text{sign}}
\DeclareMathOperator{\Sym}{\text{Sym}}
\DeclareMathOperator{\Asym}{\text{Asym}}

\DeclarePairedDelimiter{\ceil}{\lceil}{\rceil}

% godel
\newbox\gnBoxA
\newdimen\gnCornerHgt
\setbox\gnBoxA=\hbox{$\ulcorner$}
\global\gnCornerHgt=\ht\gnBoxA
\newdimen\gnArgHgt
\def\godel #1{%
    \setbox\gnBoxA=\hbox{$#1$}%
    \gnArgHgt=\ht\gnBoxA%
    \ifnum     \gnArgHgt<\gnCornerHgt \gnArgHgt=0pt%
    \else \advance \gnArgHgt by -\gnCornerHgt%
    \fi \raise\gnArgHgt\hbox{$\ulcorner$} \box\gnBoxA %
    \raise\gnArgHgt\hbox{$\urcorner$}}

% \theoremstyle{plain}

\theoremstyle{definition}
\newtheorem{theorem}{Теорема}
\newtheorem*{definition}{Определение}
\newtheorem{axiom}{Аксиома}
\newtheorem*{axiom*}{Аксиома}
\newtheorem{lemma}{Лемма}

\theoremstyle{remark}
\newtheorem*{remark}{Примечание}
\newtheorem*{exercise}{Упражнение}
\newtheorem{corollary}{Следствие}[theorem]
\newtheorem*{statement}{Утверждение}
\newtheorem*{corollary*}{Следствие}
\newtheorem*{example}{Пример}
\newtheorem{observation}{Наблюдение}
\newtheorem*{prop}{Свойства}
\newtheorem*{obozn}{Обозначение}

% subtheorem
\makeatletter
\newenvironment{subtheorem}[1]{%
    \def\subtheoremcounter{#1}%
    \refstepcounter{#1}%
    \protected@edef\theparentnumber{\csname the#1\endcsname}%
    \setcounter{parentnumber}{\value{#1}}%
    \setcounter{#1}{0}%
    \expandafter\def\csname the#1\endcsname{\theparentnumber.\Alph{#1}}%
    \ignorespaces
}{%
    \setcounter{\subtheoremcounter}{\value{parentnumber}}%
    \ignorespacesafterend
}
\makeatother
\newcounter{parentnumber}

\newtheorem{manualtheoreminner}{Теорема}
\newenvironment{manualtheorem}[1]{%
    \renewcommand\themanualtheoreminner{#1}%
    \manualtheoreminner
}{\endmanualtheoreminner}

\newcommand{\dbltilde}[1]{\accentset{\approx}{#1}}
\newcommand{\intt}{\int\!}

% magical thing that fixes paragraphs
\makeatletter
\patchcmd{\CatchFBT@Fin@l}{\endlinechar\m@ne}{}
{}{\typeout{Unsuccessful patch!}}
\makeatother

\newcommand{\get}[2]{
    \ExecuteMetaData[#1]{#2}
}

\newcommand{\getproof}[2]{
    \iftoggle{useproofs}{\ExecuteMetaData[#1]{#2proof}}{}
}

\newcommand{\getwithproof}[2]{
    \get{#1}{#2}
    \getproof{#1}{#2}
}

\newcommand{\import}[3]{
    \subsection{#1}
    \getwithproof{#2}{#3}
}

\newcommand{\given}[1]{
    Дано выше. (\ref{#1}, стр. \pageref{#1})
}

\renewcommand{\ker}{\text{Ker }}
\newcommand{\im}{\text{Im }}
\renewcommand{\grad}{\text{grad}}
\newcommand{\rg}{\text{rg}}
\newcommand{\defeq}{\stackrel{\text{def}}{=}}
\newcommand{\defeqfor}[1]{\stackrel{\text{def } #1}{=}}
\newcommand{\itemfix}{\leavevmode\makeatletter\makeatother}
\newcommand{\?}{\textcolor{red}{???}}
\renewcommand{\emptyset}{\varnothing}
\newcommand{\longarrow}[1]{\xRightarrow[#1]{\qquad}}
\DeclareMathOperator*{\esup}{\text{ess sup}}
\newcommand\smallO{
    \mathchoice
    {{\scriptstyle\mathcal{O}}}% \displaystyle
    {{\scriptstyle\mathcal{O}}}% \textstyle
    {{\scriptscriptstyle\mathcal{O}}}% \scriptstyle
    {\scalebox{.6}{$\scriptscriptstyle\mathcal{O}$}}%\scriptscriptstyle
}
\renewcommand{\div}{\text{div}\ }
\newcommand{\rot}{\text{rot}\ }
\newcommand{\cov}{\text{cov}}

\makeatletter
\newcommand{\oplabel}[1]{\refstepcounter{equation}(\theequation\ltx@label{#1})}
\makeatother

\newcommand{\symref}[2]{\stackrel{\oplabel{#1}}{#2}}
\newcommand{\symrefeq}[1]{\symref{#1}{=}}

% xrightrightarrows
\makeatletter
\newcommand*{\relrelbarsep}{.386ex}
\newcommand*{\relrelbar}{%
    \mathrel{%
        \mathpalette\@relrelbar\relrelbarsep
    }%
}
\newcommand*{\@relrelbar}[2]{%
    \raise#2\hbox to 0pt{$\m@th#1\relbar$\hss}%
    \lower#2\hbox{$\m@th#1\relbar$}%
}
\providecommand*{\rightrightarrowsfill@}{%
    \arrowfill@\relrelbar\relrelbar\rightrightarrows
}
\providecommand*{\leftleftarrowsfill@}{%
    \arrowfill@\leftleftarrows\relrelbar\relrelbar
}
\providecommand*{\xrightrightarrows}[2][]{%
    \ext@arrow 0359\rightrightarrowsfill@{#1}{#2}%
}
\providecommand*{\xleftleftarrows}[2][]{%
    \ext@arrow 3095\leftleftarrowsfill@{#1}{#2}%
}

\allowdisplaybreaks

\newcommand{\unfinished}{\textcolor{red}{Не дописано}}

% Reproducible pdf builds 
\special{pdf:trailerid [
<00112233445566778899aabbccddeeff>
<00112233445566778899aabbccddeeff>
]}
%</preamble>


\lhead{Матстат \textit{(практика)}}
\cfoot{}
\rfoot{22.9.2021}

\begin{document}

\[\ln L (\vec{x}, a, \sigma^2) = - n \ln \sigma - \frac{n}{2} \ln (2\pi) - \frac{1}{2\sigma^2} \sum (x_i - a)^2\]
\[\frac{\partial}{\partial a} = \frac{1}{\sigma^2} (n \overline{x} - na) \quad \frac{\partial}{\partial \sigma} = \frac{1}{\sigma^3} \sum (x_i - a)^2 - \frac{n}{\sigma}\]
\[\begin{cases}
        \hat{a} = \overline{x} \\
        \hat{\sigma^2} = D_B
    \end{cases} \quad M \coloneqq (\overline{x}, D_B)\]
\[\frac{\partial^2}{\partial a^2} = -\frac{n}{\sigma^2} = - \frac{n}{D_B}\]
\[\frac{\partial^2}{\partial \sigma^2} = - \frac{3}{\sigma^4} \sum (x_i - a)^2 + \frac{n}{\sigma^2} = - \frac{3}{D_B^2} \sum (x_i - \overline{x})^2 + \frac{n}{D_B} = - \frac{3n}{D_B} + \frac{n}{D_B} = - \frac{2n}{D_B}\]
\[\frac{\partial^2}{\partial a \partial \sigma} = - \frac{2}{\sigma^3} (n \overline{x} - na) = 0\]
\[d^2 L(M) = - \frac{n}{D_B} (da)^2 + 2 \cdot 0 \cdot dad\sigma - \frac{2n}{D_B} (d\sigma)^2 = -\frac{n (da)^2 + 2n (d\sigma)^2}{D_B} < 0\]
Таким образом, \(M\) --- точка максимума.

\begin{remark}
    Можно было посчитать по теореме Сильвестра.
\end{remark}

\begin{exercise}
    \(X \in B_p\). Найти оценку параметра \(p\) методом максимального правдоподобия.
\end{exercise}

\begin{solution}
    \[L(\vec{X}, p) = \prod_{i=1}^{n} (1 - p)^{n - n \overline{x}} \cdot p^{n \overline{x}}\]
    \[\ln L(\vec{X}, p) = \sum_{i=1}^{n} \ln(1 - p) \cdot \ln p \cdot (n - n \overline{x}) \cdot n \overline{x} = \ln(1 - p) \cdot (n - n \overline{x}) + \ln p \cdot n \overline{x}\]
    \[\frac{\partial \ln L}{\partial p} = - \frac{n - n \overline{x}}{1 - p} + \frac{n \overline{x}}{p} = 0\]
    \[n \overline{x} (1 - p) + (n \overline{x} - n)p = 0\]
    \[p = \overline{x}\]
\end{solution}

\begin{exercise}
    \(X \in E_\alpha\). \(E_\alpha\) --- регулярное ли семейство? Найти \(I(\alpha)\).
\end{exercise}
\begin{solution}
    \(f_\alpha(x) = \begin{cases}
        0,                      & x < 0    \\
        \alpha e^{ -\alpha x} , & x \geq 0
    \end{cases}\)

    Носитель \(C = (0, \infty)\). Можно выкинуть точку \(0\), т.к. она имеет меру \(0\).

    \[\frac{\partial}{\partial \alpha} \ln f_\alpha(x) = \frac{\partial}{\partial \alpha} (\ln \alpha - \alpha x) = \frac{1}{\alpha} - x\]
    Эта функция непрерывна \(\forall \alpha \in C\).

    \[I(\alpha) = \E\left( \frac{\partial}{\partial \alpha} \ln(f_\alpha(X)) \right) = \E \left(X - \frac{1}{\alpha}\right) = \E\left(X - \E X\right) = \D X = \frac{1}{\alpha^2}\]
\end{solution}

\begin{exercise}
    То же самое, но для \(X \in E_{\frac{1}{\alpha}}\)
\end{exercise}
\begin{solution}
    \[f_\alpha(x) = \begin{cases}
            0,                                       & x < 0 \\
            \frac{1}{\alpha} e^{ -\frac{x}{\alpha}}, & x > 0
        \end{cases}\]
    \[\frac{\partial}{\partial \alpha} \ln f_\alpha(x) = \frac{\partial}{\partial \alpha} \left(\ln\frac{1}{\alpha} - \frac{x}{\alpha}\right) = - \frac{1}{\alpha} + \frac{x}{\alpha^2}\]
    \[I(\alpha) = \E\left(\frac{\partial}{\partial a} \ln f_\alpha(x)\right)^2 = \E\left(\frac{x}{\alpha^2} - \frac{1}{\alpha}\right)^2 = \frac{1}{\alpha^4} \E(X - \alpha)^2 = \frac{1}{\alpha^4} \D X = \frac{1}{\alpha^2}\]

    \[\alpha^* = \overline{x}\]
    \[\D \alpha^* = \D \overline{x} = \frac{\D x}{n} = \frac{\alpha^2}{n}\]
    \[\D \alpha^* \geq \frac{1}{n I(\alpha)}\]
    \[\frac{\alpha^2}{n} = \frac{\alpha^2}{n}\]
    Таким образом, оценка эффективная.
\end{solution}

\begin{exercise}
    Для \(X \in U(0, \theta)\) найти информацию фишера, проверить регулярность.

    \(\theta^* = 2 \overline{X}^*\) --- по м. моменты, \(\tilde \theta = \frac{n + 1}{n} X_{(n)}\) --- ОМП.
\end{exercise}
% \begin{solution}
%     \[f_\theta(x) = \begin{cases}
%             \frac{1}{\theta}, & x \in [0, \theta] \\
%             0,                & \text{иначе}
%         \end{cases}\]
%     \[C = (0, + \infty)\]
%     \[\frac{\partial}{\partial \theta} \ln f_\theta(x) = \]
% \end{solution}

\end{document}
