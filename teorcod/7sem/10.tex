\chapter{11 ноября}

\subsection{Идеалы}

\begin{definition}
    Подмножество \(I\) кольца \(R\) называется правым \textit{(или левым)}
    \textbf{идеалом}, если:
    \begin{enumerate}
        \item \((I, \{+\})\) является подгруппой \((R, \{+\})\)
        \item \(\forall r \in R, x \in I : xr \in I\)
        \textit{(или \(rx \in I\)) для правого идеала}
    \end{enumerate}
\end{definition}

\unfinished

\begin{definition}
	Область целостности, в которой все идеалы являются главными,
    называется \textbf{кольцом главных идеалов}.
\end{definition}

\begin{definition}
    Идеал \(I\) кольца \(R\) называется \textbf{максимальным}, если
    \(I \neq R\) и всякий прочий идеал \(J\), содержащий \(I\), является \(I\) или \(R\).
\end{definition}

\unfinished

\begin{definition}
    Бинарное отношение \(\sim\) на группе \(G\) называется отношением
    \textbf{конгруэнтности}, если оно является отношением эквивалентности
    и \(\forall a,b,c \in G : (a \sim b) \Rightarrow a \cdot c \sim b \cdot c\).
\end{definition}

\unfinished

Если \(R\) --- кольцо главных идеалов и \(\ev{a} \in R\),
то соответствующее факторкольцо обозначают \(\faktor{R}{aR}\)

\begin{theorem}
    Если \(I \subset R\) является максимальным идеалом,
    то \(\faktor{R}{I}\) является полем.
\end{theorem}
\begin{proof}
    Необходимо показать, что для всякого ненулевого \(a + I \in \faktor{R}{I}\)
    существует обратный \(b + I : (a + I)(b + I) = 1 + I\).
    
    Если \(a + I \neq 0\), то \(a \notin I\).
    
    Множество \(J = \{ax + m \mid x \in R, m \in I\}\) является идеалом,
    т.к.:
    \begin{enumerate}
        \item \((ax_1 + m_1) \pm (ax_2 + m_2) = a(x_1 \pm x_2) + (m_1 \pm m_2)
        \) и \((x_1 \pm x_2) \in R, (m_1 \pm m_2) \in I\)
        \item \unfinished
    \end{enumerate}
    
    \unfinished
\end{proof}

\begin{theorem}
    Пусть \(R\) --- кольцо главных идеалов, \(p\) --- его неприводимый элемент.
    Тогда факторкольцо \(\faktor{R}{pR}\) является полем.
\end{theorem}
\begin{proof}
    Докажем, что \(I = \ev{p}\) является максимальным.
    Предположим обратное, т.е. что существует идеал \(J \neq R : I \subset J\).
    
    Т.к. \(R\) --- кольцо главных идеалов, то \(J\) является главным идеалом
    и \(\exists q \in R : J = \ev{q}\). \unfinished 
\end{proof}

В дальнейшем такие поля будут обозначаться просто \(\faktor{R}{p}\)

\begin{example}\itemfix
    \begin{itemize}
        \item \(GF(2) = \faktor{\Z}{2}\)
        \item Пусть \(p\) --- простое число. \(\Z\) --- кольцо главных идеалов,
        следовательно, \(\faktor{\Z}{p}\) является полем
        и кроме того, это \(GF(p)\).
        \item Многочлен \(x^2 + 1\) неприводим над \(\R\), \(\R[x]\) --- кольцо
        главных идеалов, следовательно, \(\faktor{\R[x]}{\ev{x^2 + 1}}\)
        \unfinished
    \end{itemize}
\end{example}

\unfinished

Если мы хотим построить циклический код длины \(n\), то
нам нужен \(g(x) : g(x) \mid (x^n - 1) \Rightarrow \prod_{i \in J} f_i(x),
J \subset \{0 \dots l - 1\}\).
Если все \(f_i(x)\) различны,
то есть \(2^l - 2\) нетривиальных циклических кода.

Циклические коды над \(GF(q)\) длины \(n = q^m - 1\) называются примитивными.

\begin{theorem}
    Пусть \(\beta_1 \dots \beta_r \in GF(q^m)\) --- корни порождающего
    многочлена \(g(x)\) примитивного циклического кода \(\mathcal{C}\)
    длины \(n\) над полем \(GF(q)\).
    
    Многочлен \(c(x) \in GF(q)[x]\) является кодовым тогда и только тогда,
    когда \(c(\beta_1) = \dots = c(\beta_r) = 0\).
\end{theorem}
\begin{proof}
    \unfinished
\end{proof}

\unfinished % проверочная матрица

\begin{example}
    Рассмотрим поле \(GF(2^3)\). В нем есть элементы \(0, 1, \alpha\)
    и следующие элементы:
    
    \begin{table}[H]
        \centering
        \begin{tabular}{RC}
            1 & \alpha^0 \\
            \alpha & \alpha^1 \\
            \alpha^2 & \alpha^2 \\
            \alpha + 1 & \alpha^3 \\
            \alpha^2 + \alpha & \alpha^4 \\
            \alpha^2 + \alpha + 1 & \alpha^5 \\
            \alpha^2 + 1 & \alpha^6 \\
            1 & \alpha^7
        \end{tabular}
    \end{table}
    
    Рассмотрим многочлен \(g(x) = x^3 + x + 1\) и его корни:
    \(0, \alpha^2, \alpha^4\).
    
    Тогда проверочная матрица над \(GF(2^3)\) это:
    \[H = \begin{pmatrix}
        1 & \alpha & \alpha^2 & \alpha + 1 & \alpha^2 + \alpha & \alpha^2 + \alpha + 1 & \alpha^2 + 1 \\
        1 & \alpha^2 & \alpha^2 + \alpha & \alpha^2 + 1 & \alpha & \alpha + 1 & \alpha^2 + \alpha + 1 \\
        1 & \alpha^2 + \alpha & \alpha & \alpha^2 + \alpha + 1 & \alpha^2 + \alpha^2 + 1 & \alpha^2 + 1 & \alpha + 1
    \end{pmatrix}\]
    
    \unfinished
\end{example}

\section{Коды Боуза-Чоудхури-Хоквингема}

\begin{definition}
    Кодом \textbf{БЧХ} над \(GF(q)\) длины \(n\) с конструктивным расстоянием 
    \(\delta\) называется циклический код наибольшей возможной размерности,
    порождающий многочлен которого имеет корни \(a^b \dots a^{b + \delta - 2}\),
    где \(\alpha \in GF(q^m)\) --- примитивный корень степени \(n\) из 1.
\end{definition}

\begin{remark}\itemfix
    \begin{itemize}
        \item По теореме Лагранжа \(n \mid (q^m - 1)\).
        Если невозможно подобрать такое \(m\), то кода БЧХ не существует.
        \item Если \(n = q^m - 1\), то код БЧХ называется примитивным.
        \item Если \(b = 1\), то это код БЧХ в узком смысле.
        \item Если \(m = 1\), то это код Рида-Соломона.
    \end{itemize}
\end{remark}

\begin{definition}
    Если порождающий многочлен циклического кода длины \(n\)
    над \(GF(q)\) имеет корни \(\alpha^b \dots \alpha^{b + \delta - 2}\),
    где \(\alpha \in GF(q^m)\) --- примитивный корень степени \(n\) из 1,
    то минимальное расстояние этого кода \(d \geq \delta\).
\end{definition}
\begin{proof}
    \unfinished
\end{proof}

\unfinished

Как правило, используют коды БЧХ в узком смысле,
но иногда коды БЧХ в широком смысле позволяют выиграть в размерности.

До недавнего времени использовались в основном коды БЧХ,
только недавно его стал вытеснять код LDPC, но внезапно оказалось,
что на достаточно высоких скоростях его не успевают декодировать.

\subsection{Коды Рида-Соломона}

\begin{definition}
    Код Рида-Соломона --- код БЧХ длины \(q - 1\) над \(GF(q)\).
\end{definition}

Минимальный многочлен \(\beta \in GF(q)\) над \(GF(q)\) это
\(M_\beta(x) = x - \beta\).

Порождающий многочлен кода Рида-Соломона имеет вид
\(g(x) = \prod_{i = 0}^{\delta - 2} (x - \alpha^{b + i})\).

Размерность: \(k = n - \delta + 1\)

Минимальное расстояние по теореме \(d \geq \delta\).
С другой стороны, граница Синглтона:
\(d \leq n - k + 1 = \delta \Rightarrow d = n - k + 1\).
Таким образом, код Рида-Соломона имеет максимальное достижимое расстояние.

\subsection{Декодирование кодов БЧХ}

Рассмотрим исправление ошибок в векторе \(y = c + e\),
\(y(x) = a(x)g(x) + e(x)\).

Синдром: \(S_i = y(\alpha^{b + i})
= a(\alpha^{b + i})g(\alpha^{b + i}) + e(\alpha^{b + i}) = e(\alpha^{b + i}),
0 \leq i < \delta - 1\).

Пусть ошибки произошли в неизвестных нам позициях \(j_1 \dots j_t,
t \leq \floor{(\delta - 1) / 2}\) 

\unfinished