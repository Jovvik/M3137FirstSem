\chapter{25 ноября}

\unfinished

\subsection{QR--коды}

\begin{definition}
    Число \(y\) называется \textbf{квадратичным вычетом
    \textit{(quadratic residue)}} по модулю \(n\), если существует число \(x\)
    такое, что \(y = x^2 \mod n\).
\end{definition}

\begin{prop}\itemfix
    \begin{itemize}
        \item \((n - x)^2 \equiv x^2 \mod n \Rightarrow\)
        \(1^2, 2^2 \dots ((n - 1) / 2)^2\) являются  \unfinished
        \item \unfinished
    \end{itemize}
\end{prop}

Это не те QR--коды, про которые вы подумали, черно-белые квадраты
это Quick Response code, где используется код Рида-Соломона.

В Blu-Ray дисках данные кодируются так называемым \textbf{пикетным}
кодом. \unfinished.
Оптические диски имеют свойство повреждаться, при этом повреждения
состоят из большого числа последовательных ошибок.
Пикетный код состоит из двух кодов --- один кодирует данные, другой
служебные данные. \unfinished

CRC \textit{(Cyclic Redundancy Check)} --- циклический код. 
Контрольная сумма, т.е. многочлен проверочных символов \(r(x)\)
для многочлена данных \(a(x)\) вычисляется с помощью формулы
систематического кодирования \(r(x) \equiv x^{N - K} a(x) \mod g(x),
\deg a(x) \leq K - 1\).
\unfinished.

Выводы:
\begin{itemize}
    \item Циклические коды допускают более компактное задание по сравнению
    с линейными блоковыми кодами.
    \item Коды БЧХ можно строить с заданным минимальным расстоянием.
    \item Коды Рида-Соломона --- коды БЧХ на границе Синглтона.
    \item Существуют алгоритмы декодирования кодов БЧХ с исправлением
    \(\floor{(\delta - 1) / 2}\) ошибок со сложностью
    \(\mathcal{O}(n\delta + \delta^2)\).
    \item \unfinished
\end{itemize}

\section{Альтернантные коды и криптосистема Мак-Элиса}

\begin{theorem}
    Если \(c = (c_0 \dots c_{n-1})\) --- кодовое слово кода Рида-Соломона
    над \(GF(q)\) в узком смысле тогда и только тогда, когда
    \(c_i = f(\alpha^i), 0 \leq i < n\),
    т.е. \(c = \operatorname{ev}\footnote{Evaluate.}(f)\),
    где \(\deg f(x) < k, f(x) \in GF(q)[x]\). 
\end{theorem}
\begin{proof}
    \[\alpha^n \defeq 1 \Rightarrow 0 = \alpha^{in} - 1 =
    (\alpha^i - 1)\sum_{j = 0}^{n-1} \alpha^{ij}, \quad 0 < i < n\]
    Т.к. по определению БЧХ \(\alpha^i \neq 1\),
    то \(\sum_{j = 0}^{n-1} \alpha^{ij} = 0\).
    
    \(c\) --- кодовое слово тогда и только тогда, когда \(cH\tran = 0\), т.е.
    \(\sum_{i = 0}^{n-1} c_i\alpha^{ij} = 0, 1 \leq j \leq n - k\).
    \begin{align*}
        \sum_{i = 0}^{n-1} f(\alpha^i) \alpha^{ij}
        & = \sum_{i = 0}^{n-1} \left(\sum_{t = 0}^{k-1} f_t\alpha^{it}\right) \alpha^{ij} \\
        & = \sum_{t = 0}^{k-1} f_t \underbrace{\sum_{i = 0}^{n-1} \alpha^{i(j + t)}}_{0} \\
        & = 0
    \end{align*}
    \unfinished
\end{proof}

Из этого результата можно получить альтернативное определение кода
Рида-Соломона:
\begin{definition}
    \((n,k,n -k + 1)\) кодом \textbf{Рида-Соломона} называется
    множество векторов \(c = (c_0 \dots c_{n-1})\), где
    \(c_i = f(\alpha^i), \deg f(x) < k, f(x) \in GF(q)[x], \alpha^i \in GF(q)\)
    --- различные значения, называемые \textbf{локаторами}.
\end{definition}

\unfinished

\begin{definition}[обобщенные коды Рида-Соломона]
    \unfinished
\end{definition}

\begin{definition}
    \textbf{Алтернантным кодом} длины \(n\) над полем \(GF(q)\) называется
    код \(\mathcal{A}(n,r,a,u)\) с проверочной матрицей
    \[H = \begin{pmatrix}
        \alpha_0^0 & \alpha_1^0 & \dots & \alpha_{n-1}^0 \\
        \alpha_0^1 & \alpha_1^1 & \dots & \alpha_{n-1}^1 \\
        \vdots & \vdots & \ddots & \vdots \\
        \alpha_0^{n-1} & \alpha_1^{n-1} & \dots & \alpha_{n-1}^{n-1}
    \end{pmatrix}\operatorname{diag}(u_0\dots u_{n-1})\]
    , где \(\alpha_i \in GF(q^m)\) --- различные элементы и
    \(u_i \in GF(q^m) \setminus \{0\}\).
\end{definition}

\begin{itemize}
    \item \unfinished
\end{itemize}

\begin{theorem}
    Пусть \(m \mid (n - h)\).
    Существует альтернантный \((n,k \geq h,d \geq \delta)\) код над \(GF(q)\)
    такой, что
    \[\sum_{i = 1}^{\delta - 1}(q -1)^i \binom{i}{n} < (q^m - 1)^{(n - h) / m}\]
\end{theorem}
\begin{proof}
    Пусть вектор локаторов фиксирован.
    
    Рассмотрим вектор без элементов--нулей
    \(y \in GF(q)^n \setminus \{0\}\).
    Оценим число обобщенных кодов Рида--Соломона, содержащих этот вектор, т.е.
    \(|\{GRS(n,k',a,v) \mid y \in GRS(n,k',a,v)\}|\).
    
    \unfinished
\end{proof}

\unfinished

\subsection{Криптосистема Мак--Элиса}

Пусть дана порождающая матрица \(G\) кода, для которого известен
эффективный алгоритм исправления \(t\) ошибок.

Будем использовать как секретный ключ \unfinished,
а как открытый ключ пару \(\Gamma = QGP, t\).
Сообщение \(x\) шифруется как \(y = x\Gamma + e\), где \(e\) --- случайный
вектор веса не более \(t\).

Дешифрование \(y\) происходит следующим образом:
\(y' = yP^{-1} = xQG + eP^{-1} = xQG + e^t\).
Вектор \(e'\) можно найти из \(y' - e' = (xQ \mid xQA)\).

Поиск секретного ключа по открытому --- задача эквивалентности кодов,
которая просто решается для кодов Рида-Соломона, но не для кодов Гоппы.

Восстановление сообщения по криптограмме --- задача исправления декодирования
линейного кода.

Недостаток: большой размер открытого ключа: \( \approx 512\) Кбит.

\subsection{Криптосистема Нидеррайтера}

\unfinished