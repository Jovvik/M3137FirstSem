\chapter{16 сентября}

\begin{definition}
    Для группы \(\mathcal{G} = (G, +)\) с подгруппой \(\mathcal{H} = (H, +)\) назовём \textbf{смежным классом} для \(x \in G\) множество:
    \[x + H = \{x + h \mid h \in H\}\]
\end{definition}

\begin{example}
    Для группы \(\mathbb{Z}\) группа чётных чисел \(2\mathbb{Z}\) является подгруппой. Для \(1\) смежный класс --- все нечётные числа, для \(2\) --- все чётные.
\end{example}

\unfinished

Иногда демодулятор может сообщить, что некоторый полученный символ ненадежен.
Это называется \textbf{стиранием}.
Также стиранием считаются потери пакетов в компьютерных сетях (UDP).

Стирания исправлять проще, чем ошибки, т.к. мы знаем, где они произошли.

\begin{definition}
    \textbf{Весовой спектр кода} это \(A_i = |\{c \in C \mid \mathrm{wt}(c) = i\}|\)
\end{definition}

Для двоичного симметричного канала с переходной вероятностью \(p\) вероятность необнаружения ошибки это:
\[P \{S = 0\} = \sum_{i=d}^{n} A_i p^i (1 - p)^{n - i} \leq \sum_{i=d}^{n} C_n^i p^i (1 - p)^{n - 1}\]
\unfinished

Для аддитивного гауссовского канала с двоичной модуляцией вероятность ошибки мягкого декодирования по максимуму правдоподобия линейного блокового кода:
\[P \leq \sum_{i=d}^{n} A_i Q \left(\sqrt{2i \frac{E_s}{N_0}}\right)
    = \sum_{i=d}^{n} A_i Q \left(\sqrt{2iR \frac{E_b}{N_0}}\right)
    = \frac{1}{2} \sum_{i=d}^{n} A_i \mathrm{erfc}\left(\sqrt{iR \frac{E_s}{N_0}}\right)\]

\begin{definition}
    \textbf{Полное декодирование по минимальному расстоянию} --- нахождение по \(y\) ближайшего кодового слова \(c = \mathrm{argmin}_{c \in C} d(c, y)\)
\end{definition}

\begin{definition}
    \textbf{Информационная совокупность \textit{(ИС)}} --- множество из \(k\) позиций в кодовом слове,
    значения которых однозначно определяют значения символов на остальных позициях кодового слова.
\end{definition}

\begin{definition}
    Если \(\gamma = \{j_1 \dots j_k\}\) --- информационная совокупность,
    то остальные позиции кодового слова называются \textbf{проверочной совокупностью}.
\end{definition}

Если \(\gamma\) --- информационная совокупность, то матрица из столбцов порождающей матрицы с номерами \(j_1 \dots j_k\), обратима.
Почему? \unfinished
Будем обозначать такую матрицу \(M(\gamma)\).

\begin{remark}
    Эта матрица не единственна.
\end{remark}

Пусть \(G(\gamma) = M(\gamma) G\) --- порождающая матрица, содержащая единичную подматрицу на столбцах \(j_1 \dots j_k\).

Информационные совокупности позволяют более эффективно проверять коды,
т.к. стандартный метод требует таблицу размера \(2^k\).

\begin{theorem}
    Алгоритм декодирования по информационной совокупности обеспечивает полное декодирование по минимальному расстоянию.
\end{theorem}
\begin{proof}
    Необходимо доказать, что для всякого исправимого вектора ошибки \(r\)
    существует информационная совокупность, свободная от ошибок.

    Пусть \(c\) --- единственное решение задачи декодирования по минимальному расстоянию.
    Тогда \(e = r - c\) --- вектор ошибки, \(E = \mathrm{supp}(e)\) --- множество позиций ненулевых элементов \(e\) и \(|E| \leq n - k\).

    Пусть \(N = \{1 \dots n\}\).
    Предположим, что \(N \setminus E\) не содержит информационных совокупностей,
    тогда существуют различные кодовые слова, отличающиеся от \(r\) в позициях \(E\).
    Но тогда \(c\) не единственно, противоречие.
\end{proof}

Сложность декодирования для \((n, k)_q\) кода экспоненциальная.

\subsection{Покрытия}

\begin{definition}
    \textbf{Покрытием} \(M(n,m,t)\) называется такой набор \(F \subset 2^{N_n}\), где \(N_n = \{1 \dots n\}, \forall f \in F \ \ |f| = m\) и любое \(t\)-элементное подмножество \(N_n\) содержится в одном из \(f \in F\).
\end{definition}

Для декодирования по информационной совокупности с исправлением не более \(t\) ошибок необходимо покрыть все исправимые конфигурации ошибок.
Элементы такого покрытия --- проверочные совокупности.

\begin{example}
    \unfinished
\end{example}

Построение минимального покрытия --- NP-полная задача.
Но есть итеративный приблизительный алгоритм.

\subsection{Дуальные коды}

\unfinished

\subsection{Граница Хэмминга}

\begin{theorem}
    Для любого \(q\)-ичного кода с минимальным расстоянием \(d = 2t + 1\) число кодовых слов удовлетворяет
    \[A_q(n, d) \leq \frac{q^n}{\sum_{i=0}^{t} C_n^i (q - 1)^i}\]
\end{theorem}
\begin{proof}
    Если код способен исправить \(t\) ошибок, то вокруг всех кодовых слов можно
    описать хэмминговы шары радиуса \(t\), не пересекающиеся друг с другом.
\end{proof}

\subsubsection{Асимптотическая оценка}

\begin{align*}
    A_q(n, d)
     & \leq \frac{q^n}{\sum_{i=0}^{t} C_n^i (q - 1)^i}                                                           \\
     & \leq \frac{q^n}{C_n^{\frac{d - 1}{2}} (q - 1)^{\frac{d - 1}{2}}}                                          \\
     & = \frac{q^n \left(\frac{d - 1}{2}\right)!\left(n - \frac{d - 1}{2}\right)!}{n! (q - 1)^{\frac{d - 1}{2}}} \\
     & = \?                                                                                                      \\
\end{align*}

\subsection{Граница Варшамова--Гилберта}

\begin{theorem}
    Существует \(q\)-ичный код длины \(n\) с минимальным расстоянием \(d\), число слов которого удовлетворяет:
    \[A_q(n, d) \geq \frac{q^n}{\sum_{i=0}^{d - 1} C_n^i (q - 1)^i}\]
\end{theorem}
\begin{proof}
    Если код \(C\) имеет максимальную мощность, для любого вектора \(x \notin C\)
    существует кодовое слово \(c\) такое, что \(d(x, c) \leq d - 1\).
    \unfinished
\end{proof}

\subsection{Граница Варшамова--Гилберта для линейных кодов}

\begin{theorem}
    Если
    \[q^r > \sum_{i=0}^{d - 2} C_{n - 1}^i (q - 1)^i\]
    , то существует линейный код над \(GF(q)\) длины \(n\)
    с минимальным расстоянием не менее \(d\) и не более чем \(r = n - k\)
    проверочными символами.
\end{theorem}
\begin{proof}
    Построим матрицу \(H\) размера \((n - k) \times n\) такую, что
    любые \(d - 1\) её столбцов линейно независимы.

    Первый столбец пусть будет произвольным ненулевым вектором.
    Если уже выбраны \(j\) столбцов, то \(j + 1\)-й столбец не может быть
    никакой линейной комбинацией любых \(d - 2\) выбранных столбцов.
    Таких столбцов \(\sum_{i=0}^{d-2} C_j^i (q - 1)^i\).
    Пока не запрещены все \(q^{n - k}\) столбцов, то можно выбрать ещё хотя бы один столбец.
\end{proof}

\unfinished

\subsection{Граница Грайсмера}

\begin{notation}
    \(N(k, d)\) --- минимальная длина двоичного линейного кода размерности
    \(k\) с минимальным расстоянием \(d\).
\end{notation}

\begin{theorem}
    \(N(k, d) \geq d + N(k - 1, \ceil{d / 2})\)
\end{theorem}
\begin{proof}
    Пусть порождающая матрица \((n, k, d)\) кода \(C\) наименьшей длины \(n = N(k, d)\) имеет вид:
    \[G = \begin{pmatrix}
            0 & \dots & 0 & 1 & \dots & 1 \\
              & G'    &   &   & *
        \end{pmatrix}\]
        \unfinished
\end{proof}