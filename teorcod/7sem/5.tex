\chapter{30 сентября}

По формуле Байеса:
\[\alpha_i(s) = \sum_{\tilde{s} \in V_{i-1}} \alpha_{i-1}(\tilde{s}) \gamma_i(\tilde{s}, s), s \in V_i\]
\[\beta_i(\tilde{s}) = \sum_{s \in V_{i+1}} \gamma_{i+1}(\tilde{s}, s) \beta_{i+1}(\tilde{s})\]
\unfinished

Начальные значения для рекуррентной формулы:
\[\beta_n'(s) = \beta_n(s) = \begin{cases}
    1, & s = 0 \\
    0, & s \neq 0
\end{cases}\]
\begin{align}
    \gamma_{i+1}(s', s) & = \? \\
\end{align}

\unfinished

\subsection{Выводы}

Метод порядковых статистик позволяет выполнить мягкое декодирование произвольного линейного блокового кода. Увеличение сложности позволяет приблизиться к декодированию по максимум правдоподобия.

Всякий линейный блоковый код может быть представлен в виде решетки.

Алгоритм Витерби реализует декодирование линейного кода по максимуму правдоподобия.

Алгоритм Бала-Коке-Елинека-Равива реализует посимвольное декодирование линейного кода по максимуму апостериорной вероятности.

\section{Сверточные коды}

Задача кодера --- сделать передаваемые символы статистически зависимыми.
Сверточный код отображает автоматом блоки данных в кадры кодового слова.

\begin{example}
    Простейший автомат --- регистр сдвига.
\end{example}

\unfinished