\chapter{9 сентября}

\subsection{Вычисление \(\mathrm{FIRST}\)}

Определим массив \textit{(или map)} \(\mathrm{FIRST[]} : N \to 2^{\Sigma \cup \{\varepsilon\}}\), который будет возвращать \(\mathrm{FIRST}\) от нетерминалов.

\begin{lemma}\itemfix
    \begin{itemize}
        \item \(\alpha = c\beta \Rightarrow \mathrm{FIRST}(\alpha) = \{c\}\)
        \item \(\alpha = \varepsilon \Rightarrow \mathrm{FIRST}(\alpha) = \{\varepsilon\}\)
        \item \(\alpha = A\beta \Rightarrow \mathrm{FIRST}(\alpha) = \mathrm{FIRST}[A] \setminus \varepsilon \cup (\mathrm{FIRST}(\beta)\ \mathrm{if}\ \varepsilon \in \mathrm{FIRST}[A])\)
    \end{itemize}
\end{lemma}
\begin{proof}
    Очевидно.
\end{proof}

По лемме мы можем найти \(\mathrm{FIRST}[]\) следующим алгоритмом:

\begin{minted}[escapeinside=||,mathescape=true]{python}
while (change):
    for |$A \to \alpha \in \Gamma$|:
        FIRST[A] |$\cup$|= FIRST(|$\alpha$|)
\end{minted}

Докажем, что итоговый массив \(\mathrm{FIRST}[]\) как функция от \(N\) равен функции \(\mathrm{FIRST}\).
\begin{proof}
    Очевидно, что \(\mathrm{FIRST}[A] \subset \mathrm{FIRST}(A)\), т.к. мы не добавляем лишнего \textit{(по лемме)}.

    Докажем, что \(\mathrm{FIRST}[A] \supset \mathrm{FIRST}(A)\) от противного --- пусть \(\exists c : c \in \mathrm{FIRST}(A), c \notin \mathrm{FIRST}[A]\). Среди всех таких \(c\) найдем такое, что вывод \(A \Rightarrow^k c\xi\) имеет минимальную длину, т.е. \(k \to \min\). Можем расписать \(A \Rightarrow^k c\xi\) как \(A \Rightarrow \alpha \Rightarrow^{k - 1} c\xi\) для некоторого \(\alpha\).

    Рассмотрим структуру \(\alpha\). Это некоторая строка \(x_1 x_2 \dots x_k\), при этом все символы с \(x_1\dots x_{i - 1}\)\footnote{Эта последовательность может быть пустой.} породили пустые строки, и \(x_i\) породил строку, начинающуюся с \(c\). Т.к. \(k \to \min\), то \(c \in \mathrm{FIRST}[x_i]\), т.к. \(c \in \mathrm{FIRST}(x_i)\). Но тогда на последней итерации алгоритма, когда рассматривается правило \(A \to \alpha\), в \(\mathrm{FIRST}[A]\) должно было добавиться \(\mathrm{FIRST}(A)\), в котором лежит \(c\). Противоречие.
\end{proof}

По массиву \(\mathrm{FIRST}[]\) можно восстановить \(\mathrm{FIRST}(\alpha)\) для любого \(\alpha\) по лемме.

\subsection{Вычисление \(\mathrm{FOLLOW}\)}

\begin{minted}[escapeinside=||,mathescape=true]{python}
FOLLOW[S] = |$\{\$\}$|
while (change):
    for |$A \to \alpha \in \Gamma$|:
        for |$B : \alpha = \beta B \gamma$|:
            FOLLOW(B) |$\cup$|= FIRST(|$\gamma$|)|$\setminus \varepsilon \cup{}$|(FOLLOW(|$A$|) if |$\varepsilon \in{}$| FIRST(|$\gamma$|))
\end{minted}

\begin{proof}
    Аналогично \(\mathrm{FIRST}\).
\end{proof}


\begin{example}
    Вспомним грамматику арифметических выражений с прошлой лекции:
    \begin{itemize}
        \item \(E \to E + T\)
        \item \(E \to T\)
        \item \(T \to T * F\)
        \item \(T \to F\)
        \item \(F \to (E)\)
        \item \(F \to n\)
    \end{itemize}

    \begin{center}
        \begin{tabular}{C|C|C}
              & \mathrm{FIRST} & \mathrm{FOLLOW} \\ \hline
            E & (n             & \$\ +\ )        \\
            T & (n             & \$\ +\ *\ )     \\
            F & (n             & \$\ +\ *\ )
        \end{tabular}
    \end{center}
\end{example}

Для правил \(E \to E + T, E \to T\) множества \(\mathrm{FIRST}\) от правых частей пересекаются, следовательно эта грамматика не \(LL(1)\). Это вызвано простой проблемой --- эти два правила образуют левую рекурсию и очевидно условие 1 теоремы не выполнено.

\subsection{Доказательство теоремы о характеризации \(LL(1)\)}

\begin{proof}\itemfix
    Предположим, что \(\Gamma \notin LL(1)\).
    \begin{enumerate}
        \item \(\xi\) не породил \(\varepsilon\), \(\eta\) не породил \(\varepsilon\)

              Тогда \(c \in \mathrm{FIRST}(\xi)\) и \(c \in FIRST(\eta)\)

        \item \(\xi\) породил \(\varepsilon\), \(\eta\) породил \(\varepsilon\)

              Тогда \(\varepsilon \in \mathrm{FIRST}(\xi)\) и \(\varepsilon \in \mathrm{FIRST}(\eta)\)

        \item \(\xi\) породил \(\varepsilon\), \(\eta\) не породил \(\varepsilon\)

              Тогда \(\varepsilon \in \mathrm{FIRST}(\xi), c \in \mathrm{FOLLOW}(A)\) и \(c \in \mathrm{FIRST}(\eta)\)
    \end{enumerate}

    Таким образом, если \(\Gamma \notin LL(1)\), то условие теоремы не выполнено.

    \textcolor{red}{В обратную сторону доказательство не написано.} % TODO
\end{proof}

\subsection{Проблемы грамматик}

В этой части мы обсудим типичные причины, по которым грамматика может быть \(\notin LL(1)\).

\subsubsection{Левая рекурсия}

\begin{definition}
    \(A \Rightarrow^{ +} A\alpha\) --- \textbf{левая рекурсия}
\end{definition}

\begin{statement}
    Левая рекурсия \(\notin LL(1)\). \textit{(почти всегда)}
\end{statement}
\begin{proof}
    Рассмотрим непосредственную левую рекурсию: \(A \to A\alpha\). Пусть ещё есть правило \(A \to \beta\).\footnote{Мы не теряем общности, т.к. иначе \(A\) --- бесполезный нетерминал. Мы не рассматриваем грамматики с такими нетерминалами, т.к. их можно убрать.} Рассмотрим \(c \in \mathrm{FIRST}(\beta)\), тогда ещё \(c \in \mathrm{FIRST}(A\alpha)\), следовательно \(\Gamma \notin LL(1)\). Если же \(\nexists c \in \mathrm{FIRST}(\beta)\) для любого \(\beta\), то \(\Gamma \in LL(1)\).
\end{proof}

От левой рекурсии можно избавиться следующим преобразованием для всех \(A \to A\alpha, A \to \beta\):
\begin{itemize}
    \item \(A \to \beta A'\)
    \item \(A' \to \alpha A'\)
    \item \(A' \to \varepsilon\)
\end{itemize}

Никто не гарантирует, что после такого преобразования \(\Gamma' \in LL(1)\), т.к. у грамматики могут быть другие проблемы.

\subsubsection{Правое ветвление}

\begin{definition}[правое ветвление]
    \(A \to \alpha \beta, A \to \alpha \gamma\)
\end{definition}

\begin{statement}
    Правое ветвление \(\notin LL(1)\). \textit{(почти всегда)}
\end{statement}

Преобразование, удаляющее правое ветвление:
\begin{itemize}
    \item \(A \to \alpha A'\)
    \item \(A' \to \beta\)
    \item \(A' \to \gamma\)
\end{itemize}

\subsubsection{Пример}

Преобразуем грамматику арифметических выражений:

\begin{itemize}
    \item \(E \to TE'\)
    \item \(E' \to + TE'\)
    \item \(E' \to \varepsilon\)
    \item \(T \to FT'\)
    \item \(T' \to *FT'\)
    \item \(T' \to \varepsilon\)
    \item \(F \to (E)\)
    \item \(F \to n\)
\end{itemize}

Посчитаем \(\mathrm{FIRST}\) и \(\mathrm{FOLLOW}\):
\begin{center}
    \begin{tabular}{C|L|L}
           & \mathrm{FIRST} & \mathrm{FOLLOW} \\ \hline
        E  & n\ (           & \$\  )          \\
        E' & \varepsilon\ + & \$\  )          \\
        T  & n\ (           & +\ \$\ )        \\
        T' & \varepsilon\ * & +\ \$\  )       \\
        F  & n\ (           & *\  +\  \$\ )   \\
    \end{tabular}
\end{center}

Ура, эта грамматика \(\in LL(1)\)!

\section{Построение парсеров}

Есть два метода построения деревьев разбора --- сверху и снизу. Для \(LL\) грамматик используется сверху, для \(LR\) \textit{(определим позже)} --- снизу.

Для каждого нетерминала определим функцию, которая возвращает дерево разбора с корнем в этом нетерминале.

Также у нас есть контекст, в котором есть \texttt{token} --- рассматриваемый токен и функция \texttt{nextToken()} --- достает новый токен. Для каждой функции инвариант --- при вызове функции \(f\) \texttt{token} есть нетерминал \(f\).

\begin{example}
    Пусть есть правила \(A \to \alpha_1, A \to \alpha_2 \dots A \to \alpha_k\).

    \[\mathrm{FIRST1}(\alpha) \coloneqq \mathrm{FIRST}(\alpha) \setminus \varepsilon \cup (\mathrm{FOLLOW}(A)\ \mathrm{if}\ \varepsilon \in \mathrm{FIRST}(\alpha))\]

    \begin{minted}[escapeinside=||,mathescape=true]{cpp}
A(): Tree
    r = Tree(A)
    switch (token)
        case FIRST1(|$\alpha_1$|)
            f|$\alpha_1$|
        case FIRST1(|$\alpha_2$|)
            f|$\alpha_2$|
        |$\vdots$|
    return r
    \end{minted}
    , где \(f\alpha_i\) --- блок обработки \(\alpha_i\).

    Пусть \(\alpha_i = X_1X_2\dots X_t\). Тогда \(f\alpha_i\):
    \begin{minted}[escapeinside=||,mathescape=true]{cpp}
f|$\alpha_i$|:
    |$\vdots$|
    // $X_i$ = c --- terminal
    assert |$X_i$| == token
    nextToken()
    r.addChild(|$X_i$|)

    // $X_j$ = A --- nonterminal
    r.addChild(|$X_j$|)
    \end{minted}
\end{example}
