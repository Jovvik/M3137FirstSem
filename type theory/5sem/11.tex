\chapter{15 ноября}

Все, что мы доказывали, как-то не очень интересно --- это тождества. Что если мы хотим доказать например \(a \leq b\)? Для этого для начала надо уметь это записать.

\begin{definition}[меньше или равно]
    \(a \leq b\) это \(\exists x.a + b = b\)
\end{definition}

Несложно догадаться, что у нас экзистенциальный тип. В Arend все экзистенциальные типы это пары вида \((x : A, a + x = b : A \to \mathrm{\backslash Type})\). Таким образом, наш тип это \(\mathrm{\backslash Sigma}\ (x : \mathrm{Nat})\ (a + x = b)\).

\begin{example}
    Хотим доказать \(5 \leq 12\), тогда доказательство это \((7, \mathrm{idp}) : \mathrm{\backslash Sigma}\ (x : \mathrm{Nat})\ (5 + x = 12)\)
\end{example}

Определим наш тип ``\( \leq \)'' индуктивно:
\begin{minted}{haskell}
\data loe (a b : Nat) \with
    | zero, _ => base
    | suc a', suc b' => next (p : loe a' b')
\end{minted}

\begin{remark}
    \texttt{base} и \texttt{next} --- конструкторы типа, а не магическая вещь для индукции.
\end{remark}

\begin{example}\itemfix
    \begin{itemize}
        \item \verb!base: loe 0 15!
        \item \verb!next (next (base)): loe 1 16!
    \end{itemize}
\end{example}

Эти два определения эквивалентны. Докажем это.

Из индуктивного в экзистенциальный тип:
\begin{minted}{haskell}
\func f1 {a b : Nat} (p1 : loe a b) : loe' a b
    | {0}, {b}, base => (b, idp)
    | {suc a}, {suc b}, next (pr1) =>
        \let (pb, ppr) => f1 pr1 \in (pb, pmap suc ppr)
\end{minted}

В обратную сторону:
\begin{minted}{haskell}
\func f2 {a b : Nat} (p1 : loe' a b) : loe a b
    | {0}, _ => base
    | {suc a}, {0}, (x, p) => absurd (transport (\lam t => \case t \with {
        | 0 => Empty
        | suc n => \Sigma}) p ())
\end{minted}

Это не удобно писать, поэтому напишем \texttt{contradiction}\footnote{Для этого надо добавить \texttt{dependencies: [arend-lib]
} в \texttt{arend.yaml}}. Эта конструкция умеет доказывать противоречия за 1--2 шага. Таким образом:

\begin{minted}{haskell}
\func f2 {a b : Nat} (p1 : loe' a b) : loe a b
    | {0}, _ => base
    | {suc a}, {0}, (x, p) => contradiction
    | {suc a}, {suc b}, (x, p) => next (f2 (x, pmap minus1 p))
\end{minted}

Докажем часть домашнего задания:
\begin{minted}{haskell}
\func plus-assoc {a b c : Nat} : (a + b) + c = a + (b + c) \elim c
    | 0 => idp
    | suc c => {?}
\end{minted}

Можно вместо \texttt{\{?\}} написать \texttt{pmap suc (plus-assoc)}, но это скучно. Можем написать \texttt{rewrite plus-assoc idp}, который докажет нам \texttt{suc ((a + b) + c) = suc (a + (b + c))}, переписав \texttt{(a + b) + c} на \texttt{a + (b + c)}, т.к. есть доказательство \texttt{(a + b) + c = a + (b + c)}. Это звучит как магия, но на самом деле делается так:
\begin{minted}{haskell}
transport (\lam x => (a + b) + suc x = suc x) plus-assoc idp
\end{minted}

\section{Классы}

\begin{example}
    Группа: \(\ev{R, +, e, {}^{-1}}\), такие что:
    \begin{itemize}
        \item \(e + x = x\)
        \item \(x + e = x\)
        \item \(x + x^{-1} = e\)
        \item \(x^{-1} + x = e\)
    \end{itemize}

    Попробуем описать что-то подобное в Arend.
\end{example}

\begin{minted}{haskell}
    \instance OrdNat : Preorder Nat
    | <= (a b : Nat) => TruncP (\Sigma (r : Nat) (r + a = b))
    | <=-reflexive => inP (0, idp)
    | <=-transitive {x} {y} {z} =>
      \lam (t1 : TruncP (\Sigma (r : Nat) (r + x = y)))
           (t2 : TruncP (\Sigma (r : Nat) (r + y = z))) =>
        \case t1, t2 \with {
          | inP (d1, p1), inP (d2, p2) => inP (d2 + d1, plus-assoc *> (rewrite p1 p2))
        }
\end{minted}

% Вспомним теорему Гливенко, которая гласит, что если \(p\) --- доказательство \(x\) в ИИВ, то \(q\) --- доказательство \(\neg \neg x\) в ИИВ, \(p \to q\).

У \texttt{TruncP} есть математическое объяснение, и есть программистское. У нас есть различные доказательства и мы их все объявляем равными.

План дальнейших лекций:
\begin{enumerate}
    \item \texttt{Set}/\texttt{Prop}
    \item Теорема Диаконеску: теория множеств + ИИВ + аксиома выбора это классическая логика.
\end{enumerate}

Дальше есть несколько вариантов:
\begin{enumerate}
    \item Гомотопическая теория типов (математически)
    \item Другие языки: возможно Idris, Coq
    \item Другие исчисления, возможно \(F_\Omega\), линейные типы.
\end{enumerate}
