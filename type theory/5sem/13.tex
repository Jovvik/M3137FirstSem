\chapter{30 ноября}

\section{Парадокс Жирара}

\(\lambda\)-куб был придуман только в 1991 году. Начиналось всё с интуиционистской типовой системы (Мартин --- Лёф), которая была разработана в 1977,78гг., а сам парадокс Жирара был найден ещё в 1972 году. Поэтому тот парадокс, который мы будем рассматривать --- не совсем тот, что был представлен в то время.

Очевидно, выразительность \(\lambda\)-куба ограничена.

\begin{example}
    Что такое топология? \(X, \Omega \subseteq \mathcal{P}(X)\). \(\mathcal{P}(X)\) это \(\mathcal{P}(X): X \to *\). Возьмём \(x \in \mathcal{P}(X)\), это \(x : X \to *\). Где оно может жить в \(\lambda\)-кубе?

    Топология это \((X \to *) \to *\), т.е. мы про подмножество говорим, подходит ли оно.

    \begin{center}
        \begin{tabular}{lll}\toprule
            Сорт & Объекты                          & Тип           \\ \midrule
            0    & Все значения (\(\lambda\)-термы) &               \\
            1    & Типы (утверждения)               & \(*\)         \\
            2    & Рода                             & \(* \to *\)   \\
            3    & Сорта                            & \(\square\)   \\
            4    &                                  & \(\triangle\) \\
            \bottomrule
        \end{tabular}
    \end{center}

    Таким образом, у нас типовая система с 5 уровнями:
    \[1, 2, x : * \quad * : \square \quad \square : \triangle\]
\end{example}

\subsection{Обобщение \(\lambda\)-куба}

\(\sphericalangle \lambda \mathrm{HOL}\), которая оперирует над \(*, \square, \triangle\) и \(S = \{(*,*), (\square,\square), (\square,*)\}\). Утверждается, что если добавить \((\triangle,*)\), то система не является противоречивой.

Что такое \((\triangle,*)\)? Это конструкции в стиле \(\Pi x^\square.\text{значения}\).

Добавление \((\triangle, \square)\) делает систему противоречивой.

\[\underbrace{\overbrace{(*,*), (\square,\square), (\square,*) \quad (\triangle, \square)}^{U^-} \quad (\triangle,*)}_U\]

\(U^-\) тоже противоречива.

Можно рассмотреть систему \(*:*\), в ней будут \(*,\mathrm{значения},* \to *\). Тем самым мы добавляем себе ещё возможность появления парадоксов, например парадокса Рассела.

Вспомним топологию: \(\underbrace{\underbrace{(X \to *)}_\square \to *}_\square\). Мы хотим делать что-то с топологиями, то есть уметь оперировать над \(\square\). На самом деле топология --- квантор всеобщности:
\[\forall \alpha^T.\varphi(\alpha) = A(T, \varphi) \quad S : A(T) \approx (T \to *) \to *\]

Гениальная идея: не будем заводить \(\square\), а скажем, что \(X \to * : \square\). К сожалению, в таком случае опять возникает парадокс.

Что делает наличие пра вила \((\triangle, \square)\)? Оно позволяет написать выражение \(F\) с типом \(\varphi\), причём \(\varphi\) --- любой, которое не заканчивает исполнение. Таким образом, любой тип обитаем. Противоречивость из этого очевидна.

\subsection{Парадокс Бурали-Форте}

\begin{enumerate}
    \item ``Не существует максимального ординала'' или ``не существует множества всех ординалов''.

          \begin{definition}
              Ординал --- транзитивное, вполне упорядоченное множество.
          \end{definition}

          Пусть \(S\) --- множество всех ординалов. Оно является ординалом. Получается, что \(S \in S\). Это парадокс Кантора. Не хотелось бы его использовать, ибо в типовых системах с ним сложно.

    \item  Фундированное множество \(X\) --- множество, где нет бесконечной цепочки ``\(\in\)'':
          \[X \ni x_1 \ni x_2 \ni x_3 \ni \dots \ni x_n\]

          Множество всех фундированных множеств \(W\) фундировано, т.е. \(W \ni W\). Но тогда оно не фундировано, т.к. \(W \ni W \ni \dots\). Теперь мы не используем парадокс Кантора.
\end{enumerate}

\subsection{Множества всех множеств}

Рассмотрим две операции, \(\sigma\) и \(\tau\).
\[\sigma : X \to \mathcal{P} X \quad \tau : \mathcal{P} X \to X\]
\(\sigma X = \{a \mid a < X\}\) --- начальные отрезки до \(X\).

\(\tau X\) --- ординал, соответствующий \(X\).
\begin{example}
    \(X = \{0, 1, 2, 3\}, \tau X = 4\)
\end{example}
\[\sigma \tau X = \{\beta \mid \beta < \tau X\} = \{\tau \sigma \alpha \mid \alpha \in X\}\]

\subsection{Парадоксальный универсум}

\begin{definition}
    Пусть \(U\) --- универсум, \(\sigma : U \to \mathcal{P}U, \tau : \mathcal{P}U \to U\). Если для всех \(x \in \mathcal{P} U\) \(\sigma \tau X = \{\tau \sigma x \mid x \in X\}\), то \(U\) называется \textbf{парадоксальным универсумом}.
\end{definition}

\begin{definition}
    Если \(y = \sigma x\), то \(y < x\).
\end{definition}

Можем заметить, что \(\tau \sigma y < \tau \sigma x\).

\begin{definition}
    Будем говорить, что множество \(X\) \textbf{индуктивно}, если для каждого \(x : y < x \Rightarrow y \in X\) верно что \(x \in X\). Таким образом, это множество, на котором выполнена трансфинитная индукция, т.е. индукция по большим ординалам, чем натуральные числа.
\end{definition}

\begin{definition}
    \(x\) \textbf{фундировано}, если \(x\) принадлежит всем индуктивным множествам.
\end{definition}

\(\sphericalangle \Omega = \tau \{x \mid x\ \mathrm{фундировано}\}\)

\begin{enumerate}
    \item \(\Omega\) фундировано.\label{противоречие универсума 1}
          \begin{proof}
              Опустим.
          \end{proof}
    \item \(\Omega\) не фундировано.\label{противоречие универсума 2}
          \begin{proof}
              Опустим.
          \end{proof}
\end{enumerate}

Как это выразить в системе типов \(U\)\footnote{Да, у нас коллизия обозначений с универсумом, очень жаль.}?
\begin{itemize}
    \item \(\mathcal{P} X : X \to *\)
    \item \(U : \square\)
    \item \(\sigma : U \to \mathcal{P} U\)
    \item \(\tau : \mathcal{P} U \to U\)
    \item \(o : \forall S^{\mathcal{P} U}.\sigma(\tau(x)) = \lambda u^U.\exists x^U.(S\ x) \with u = \tau(\sigma(x))\)
\end{itemize}
В этом контексте можно доказать~\ref{противоречие универсума 1} и~\ref{противоречие универсума 2}. Получить такое противоречие можно только в \(U\), т.к. \(\mathcal{P} U : \square \to *\) мы не умеем строить без правила \((\square, \triangle)\)

Спасаться от этого парадокса можно тем, что у нас типы живут в разных вселенных.
