\chapter{14 сентября}

\section{Формализация \(\lambda\)-исчисления}

\begin{definition}
    \textbf{Пред-\(\lambda\)-терм} определяется индуктивно как одно из:
    \begin{enumerate}
        \item \(x\) --- переменная
        \item \((L\ L)\) --- применение
        \item \((\lambda x.L)\) --- абстракция
    \end{enumerate}
\end{definition}

Почему \underline{пред}-\(\lambda\)-терм? Мы не хотим различать \(\lambda x.x\) и \(\lambda y.y\).

\begin{definition}
    \textbf{\(\alpha\)-эквивалентность} --- обозначается \(A =_\alpha B\) и выполняется, если\footnote{И только если.}:
    \begin{enumerate}
        \item \(A \equiv x, B \equiv x\) --- одна и та же переменная
        \item \(A \equiv P\ Q, B \equiv R\ S, P =_\alpha R, Q =_\alpha S\)
        \item \(A \equiv \lambda x.P, B \equiv \lambda y.Q\) и существует \(t\) --- новая переменная, такая что \(P[x \coloneqq t] =_\alpha Q[y \coloneqq t]\)
    \end{enumerate}
\end{definition}

\begin{definition}
    % Свободное вхождение, свобода для подстановки, замена свободных вхождений
    Свобода для подстановки: \(A[x \coloneqq B]\), никакое свободное вхождение переменной в \(B\) не станет связанным.
\end{definition}

\begin{definition}[\(\lambda\)-терм]
    Множество всех \(\lambda\)-термов это \(\Lambda /_{ =_\alpha}\)
\end{definition}

\begin{definition}[\(\beta\)-редекс]
    Выражение вида \((\lambda x.A)\ B\)
\end{definition}

\begin{definition}[\(\beta\)-редукция]
    Обозначается \(A \to_\beta B\) и выполняется, если выполняется одно из:
    \begin{enumerate}
        \item \(A \equiv P\ Q, B \equiv R\ S\) и либо \(P \to_\beta R\) и \(Q =_\alpha S\), либо \(P =_\alpha R\) и \(Q \to_\beta S\).
        \item \(A \equiv \lambda x.P, B \equiv \lambda x.Q\) и \(P \to_\beta Q\)
        \item \(A \equiv (\lambda x.P)\ Q, B \equiv P [x \coloneqq Q]\) и \(Q\) свободно для подстановки.
    \end{enumerate}
\end{definition}

\begin{definition}
    Придуман Моисеем Шейнфинкелем.

    \(I \coloneqq \lambda x.x\) --- Identit\"
    at\footnote{Тождество (с немецкого)}
\end{definition}

\begin{definition}\itemfix
    \begin{itemize}
        \item \(K = \lambda x.\lambda y.x\)
        \item \(\Omega = \omega\ \omega\)
        \item \(\omega = \lambda x.x\ x\)
    \end{itemize}
\end{definition}

\begin{example}
    % https://tikzcd.yichuanshen.de/#N4Igdg9gJgpgziAXAbVABwnAlgFyxMJZABgBpiBdUkANwEMAbAVxiRAGkBJAHW4HkAtjADmdEAF9S6TLnyEUAJnJVajFmwAUG3gzoCARlDoACAB4A6HXsMmAnudMBKY52e9BIsZOnY8BImQAjCr0zKyIINrcugZGxvauxu5CohJSIBi+cgGkCiFq4SCcEiowUMLwRKAAZgBOEAJIgdQ4EE3UDBAQaDnVjHAwKrr6MAwACjJ+8iC1WMIAFjgg1KHqEbwjOF7pdQ1IZCCt7aphbBswW2k19Y2IB0eISidrIOeX3iC7t-dtdx1dPRQgQAHGQ+gwBkM6CNxpNshFZgslisCmduJtttc9o8Wr8AMwo07rdEXLwUcRAA
    \[\begin{tikzcd}[ampersand replacement=\&]
            KI\Omega \arrow{rr}{\beta} \arrow{d}{\beta} \arrow[loop] \&  \& ((\lambda x.\lambda y.x) I) \Omega \arrow[loop] \\
            (\lambda y.I) \Omega \arrow[loop left] \arrow{d}{\beta}                                                              \&  \&                                                                                          \\
            I                                                                                                    \&  \&
        \end{tikzcd}\]
\end{example}

\begin{definition}
    \(R\) обладает \textbf{ромбовидным свойством} \textit{(diamond)}, если для любых \(a,b,c\), таких что:
    \begin{enumerate}
        \item \(aRb, aRc\)
        \item \(b \neq c\)
    \end{enumerate}
    существует \(d\): \(bRd\) и \(cRd\).

    \begin{center}
        \begin{tikzcd}
            & \arrow{ld} A \arrow{rd} & \\
            B \arrow{rd} & & \arrow{ld} C \\
            & D &
        \end{tikzcd}
    \end{center}
\end{definition}

\begin{example}
    \( > \) на \(\mathbb{Z}^+\) не ромбовидно: для \(a = 3, b = 2, c = 1\) выполнено условие, но \(\nexists d\).

    \( > \) на \(\R\) ромбовидно.
\end{example}

\begin{definition}[\(\beta\)-редуцируемость]
    Рефлексивное, транзитивное замыкание отношения \( \to_\beta\), обозначается \( \twoheadrightarrow_\beta\).
\end{definition}

\begin{theorem}[Чёрча-Россера]
    \label{чр}
    \(\beta\)-редуцируемость обладает ромбовидным свойством.
\end{theorem}

\begin{definition}
    \(\rightrightarrows_\beta\) --- параллельная \(\beta\)-редукция, выполняется если:
    \begin{enumerate}
        \setcounter{enumi}{-1}
        \item \(A =_\alpha B\)
        \item \(A \equiv P\ Q, B \equiv R\ S\) и \(P \rightrightarrows_\beta R\) и \(Q \rightrightarrows_\beta S\).
        \item Аналогично \(\beta\)-редукции.
        \item Аналогично \(\beta\)-редукции.
    \end{enumerate}
\end{definition}

\begin{lemma}
    \((\rightrightarrows_\beta)\) обладает ромбовидным свойством.
\end{lemma}

\begin{lemma}
    Если \(R\) обладает ромбовидным свойством, то \(R^*\) обладает ромбовидным свойством.
\end{lemma}
\begin{proof}
    Две индукции.
\end{proof}

\begin{lemma}
    \((\rightrightarrows_\beta)^* \subseteq ( \twoheadrightarrow_\beta)\)
\end{lemma}

\begin{proof}[Доказательство теоремы \nameref{чр}]
    Заметим, что:
    \begin{enumerate}
        \item \((\rightrightarrows_\beta)^* \subseteq ( \twoheadrightarrow_\beta)\) --- из леммы
        \item \(( \twoheadrightarrow_\beta) \subseteq (\rightrightarrows_\beta)^*\) --- из определения
        \item Т.к. \((\rightrightarrows_\beta)^*\) обладает р.с., то и \(( \twoheadrightarrow_\beta)\) обладает р.с.
    \end{enumerate}
\end{proof}

\begin{corollary}
    У \(\lambda\)-выражения существует не более одной нормальной формы.
\end{corollary}
\begin{proof}
    Пусть \(A\) имеет две нормальные формы: \(A \twoheadrightarrow_\beta B, A \twoheadrightarrow_\beta C\) и \(B \neq_\alpha C\). Тогда есть \(D\): \(B \twoheadrightarrow_\beta D\) и \(C \twoheadrightarrow_\beta D\). Противоречие.
\end{proof}

\begin{definition}
    \textbf{Нормальный порядок редукции} --- редуцируем самый левый редекс.
\end{definition}

\begin{theorem}
    Если нормальная форма существует, она может быть получена нормальным порядком редукции.
\end{theorem}

\begin{remark}
    Нижеследующее объяснение --- с практики.
\end{remark}

Рассмотрим \(Y\ f =_\beta f\ (Y\ f) =_\beta f\ (f\ (Y\ f)) =_\beta \dots \). Можно считать, что у \(f\) сколько угодно аргументов, первый аргумент можно считать указателем на свой рекурсивный вызов.

\begin{example}
    Числа Фибоначчи:
    \begin{minted}{haskell}
fib a b n =
    if n = 0 then a
    else fib b (a + b) (n - 1)
    \end{minted}

    Здесь решение уравнения заметано под ковер, в \(\lambda\)-исчислении оно видно:
    \[\mathrm{Fib} = \lambda f.\lambda a.\lambda b.\lambda n.(\mathrm{IsZero}\ n)\ a\ (f\ f\ a\ (a + b)\ (n - 1))\]
    Здесь \(f\) передается само себе, чтобы иметь ссылку на себя для рекурсивного вызова.

    Для работы \(\mathrm{Fib}\) нужно дать его самому себе: \(\mathrm{Fib}\ \mathrm{Fib}\ 1\ 1\ 10\).
\end{example}
