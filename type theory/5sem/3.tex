\chapter{21 сентября}

В \(\lambda\)-исчислении можно сделать:
\begin{enumerate}
    \item Целые числа, где \(\ev{a, b} \leftrightarrow a - b\)
    \item Рациональные числа в виде дробей
    \item Матлогику?
\end{enumerate}

Попытки сделать матлогику всегда приводили к парадоксам.

Оказывается, нельзя относиться к любому выражению как к логическому.

\begin{notation}
    \(\supset\) --- импликация
\end{notation}

\begin{example}
    Рассмотрим комбинатор \(\Phi_A =_\beta A \supset \Phi_A\). Это \(Y\ (\lambda f.\lambda a.a \supset f\ a)\).

    Добавим аксиому \((A \supset (A \supset B)) \supset (A \supset B)\). Если такой аксиомы нет, то теория грустная.

    Мы также хотим, чтобы если \(X =_\beta Y\), то \(X \supset Y\).

    Каким-то образом мы получим парадокс.
    % Тогда для фиксированного \(A\):
    % \[(A \supset (A \supset \Phi_A)) \supset (A \supset \Phi_A)\]
    % Так как \(\Phi_A =_\beta A \supset \Phi_A\):
    % \[(A \supset \Phi_A) \supset (A \supset (A \supset \Phi_A))\]
    % \[(A \supset \Phi_A) \supset \Phi_A\]
    % \[(A \supset \Phi_A) \supset (A \supset \Phi_A)\]
\end{example}

\section{Просто-типизированное \(\lambda\)-исчисление}

\begin{definition}[типовые переменные]\itemfix
    \begin{itemize}
        \item \(\alpha, \beta, \gamma\) --- атомарные
        \item \(\tau, \sigma\) --- составные
    \end{itemize}
\end{definition}

2 традиции:
\begin{enumerate}
    \item Исчисление по Чёрчу
    \item Исчисление по Карри
\end{enumerate}

Мы сначала рассмотрим исчисление по Карри.

\subsection{Исчисление по Карри}

Типизация: \(\Gamma \vdash A : \tau\), \(\Gamma = \{x_1 : \tau_1, x_2 : \tau_2 \dots\}\)

Правила:
\begin{enumerate}
    \item \begin{prooftree}
              \infer0[Ax.]{\Gamma, x_1 : \tau_1 \vdash x_1 : \tau_1}
          \end{prooftree}
    \item \begin{prooftree}
              \hypo{\Gamma \vdash A : \sigma \to \tau}
              \hypo{\Gamma \vdash B : \sigma}
              \infer2{\Gamma \vdash A\ B : \tau}
          \end{prooftree}
    \item \begin{prooftree}
              \hypo{\Gamma, x : \tau \vdash A : \sigma}
              \infer1{\Gamma \vdash \lambda x.A : \tau \to \sigma}
          \end{prooftree}
\end{enumerate}

\begin{example}
    \[\lambda f^{\alpha \to \alpha}.\lambda x^\alpha.f\ (f\ x) : (\alpha \to \alpha) \to \alpha \to \alpha\]
    Подгоним доказательство под результат:
    \[\begin{prooftree}
            \infer0{f : \alpha \to \alpha \vdash f : \alpha \to \alpha}
            \infer0{\Gamma \vdash f : \alpha \to \alpha}
            \infer0{\Gamma \vdash x : \alpha}
            \infer2{\Gamma \vdash f\ x : \alpha}
            \infer2{f : \alpha \to \alpha, x : \alpha \vdash f\ (f\ x) : \alpha}
            \infer1{f : \alpha \to \alpha \vdash \lambda x.f\ (f\ x) : \alpha \to \alpha}
            \infer1{\lambda f.\lambda x.f\ (f\ x) : (\alpha \to \alpha) \to (\alpha \to \alpha)}
        \end{prooftree}\]
\end{example}

\begin{theorem}
    Если \(\Gamma \vdash A : \tau\), то любое подвыражение имеет тип.
\end{theorem}
\begin{proof}
    По индукции по длине.

    \begin{itemize}
        \item [База.] Это правило 1.
        \item [Переход.] Пусть любое выражение длиной \( < n\) символов обладает искомым свойством. Покажем искомое для \(A : |A| = n\). Рассмотрим варианты того, по какому правилу доказана типизируемость \(A\):
              \begin{enumerate}
                  \item Второе правило: \(B\) и \(C\) короче \(A\), следовательно для них искомое верно.
                  \item Третье правило: аналогично для \(x, B\)
              \end{enumerate}
    \end{itemize}
\end{proof}

\begin{theorem}[Subject reduction, о редукции]
    Если \(\Gamma \vdash A : \sigma\) и \(A \twoheadrightarrow_\beta B\), то \(\Gamma \vdash B : \sigma\)
\end{theorem}
\begin{proof}
    Скучно.

    Самая интересная часть: рассмотрим \(A \to_\beta B\). Случаи:
    \begin{enumerate}
        \item \(\lambda x.A \to \lambda x.B\) --- индукция
        \item \(A\ B\) --- индукция
        \item \((\lambda x.A)\ B \to A[x \coloneqq B]\)

              По теореме о типизации подвыражений, \((\lambda x^{\tau \to \sigma}.A^\sigma)\ B^\tau : \sigma\). Кроме того, доказывается \((A [x \coloneqq B]) : \sigma\).
    \end{enumerate}
\end{proof}

\begin{lemma}
    Если \(\Gamma, x : \tau \vdash A : \sigma\),\(\Gamma \vdash B : \tau\), то \(\Gamma \vdash A[x \coloneqq B] : \sigma\)
\end{lemma}

\begin{theorem}[Чёрча-Россера]
    Если \(\Gamma \vdash M : \sigma\) и существуют \(N, P : M \twoheadrightarrow_\beta N, M \twoheadrightarrow_\beta P\), то найдется такой \(S\), что \(\Gamma \vdash S : \sigma\) и \(N \twoheadrightarrow_\beta S\) и \(P \twoheadrightarrow_\beta S\)
\end{theorem}

\subsection{Исчисление по Чёрчу}

Язык:
\begin{itemize}
    \item \(x\) --- переменная
    \item \(A\ B\) --- аппликация
    \item \(\lambda x^\tau.P\) --- абстракция
\end{itemize}

\begin{notation}
    Когда нужно различить исчисления, будем писать \( \vdash_{\mathrm{Ч}}\) или \( \vdash_{\mathrm{К}}\)
\end{notation}

\begin{theorem}
    Если контекст \(\Gamma\) и выражение \(P\) типизируется, то \(\Gamma \vdash_{\mathrm{Ч}} P : \sigma\)
\end{theorem}

\begin{example}
    \[ \vdash_{\mathrm{К}} \lambda x.x : \alpha \to \alpha\]
    \[ \vdash_{\mathrm{К}} \lambda x.x : \beta \to \beta\]
    \[ \vdash_{\mathrm{Ч}} \lambda x^\sigma.x : \sigma \to \sigma\]
\end{example}
