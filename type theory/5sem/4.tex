\chapter{28 сентября}

\subsection{Противоречивость нетипизированного \(\lambda\)-исчисления}

\?

\begin{enumerate}
    \item Логические выражения
    \item Запрещенные выражения
\end{enumerate}

\(Y\) явно нехорошее выражение. \(\Phi_A =_\beta \Phi_A \supset A\)

Добавим очевидные аксиомы:
\begin{enumerate}
    \item \(A =_\beta B\), то \( \vdash A \supset B, \vdash B \supset A\). Почему? Потому что мы хотим, чтобы \(\sin 0 = 0\), а не только \(\sin 0 \to 0\)
    \item \((A \supset A \supset B) \supset (A \supset B)\)
    \item \(A, A \supset B\), тогда \(B\)
\end{enumerate}

Тогда заметим, что при любом \(A, \vdash A\):

\begin{align*}
    \Phi_A                              & \supset \Phi_A                  \\
    \Phi_A                              & \supset (\Phi_A \supset A)      \\
    (\Phi_A \supset (\Phi_A \supset A)) & \supset (\Phi_A \supset \Phi_A) \\
    \Phi_A                              & \supset A                       \\
    \Phi_A                                                                \\
    A
\end{align*}

\section{Изоморфизм Карри-Ховарда}

\[\begin{prooftree}
        \infer0[\(x \notin \Gamma\)]{\Gamma, x : \tau \vdash x : \tau}
    \end{prooftree}\]
\[\begin{prooftree}
        \infer0[\(x \notin \Gamma\)]{\Gamma, x : \tau \vdash x : \tau}
    \end{prooftree}\]
\[\begin{prooftree}
        \infer0{\Gamma, \tau \vdash \tau}
    \end{prooftree}\]
\[\begin{prooftree}
        \hypo{\Gamma \vdash A : \sigma \to \tau}
        \hypo{\Gamma \vdash B : \sigma}
        \infer2{\Gamma \vdash A\ B : \tau}
    \end{prooftree}\]
\[\begin{prooftree}
        \hypo{\Gamma \vdash \sigma \to \tau}
        \hypo{\Gamma \vdash \sigma}
        \infer2{\Gamma \vdash \tau}
    \end{prooftree}\]
\[\begin{prooftree}
        \hypo{\Gamma, x : \sigma \vdash A : \tau}
        \infer1[\(x \notin \Gamma\)]{\Gamma \vdash \lambda x.A : \sigma \to \tau}
    \end{prooftree}\]
\[\begin{prooftree}
        \hypo{\Gamma, \sigma \vdash \tau}
        \infer1{\Gamma \vdash \sigma \to \tau}
    \end{prooftree}\]

\begin{theorem}[об изоморфизме Карри-Ховарда]\itemfix
    \begin{enumerate}
        \item \(\Gamma \vdash_{\lambda \to } A : \tau\), то \(|\Gamma| \vdash_{\to} \tau\)
        \item Если \(\Delta \vdash_{\to} \tau\), то найдутся \(\Gamma, A : |\Gamma| = \Delta, \Gamma \vdash A : \tau\)
    \end{enumerate}
\end{theorem}

\begin{definition}
    \[|\Gamma| \coloneqq \{\tau \mid x : \tau \in \Gamma\}\]
\end{definition}

\subsection{Импликационный фрагмент ИИВ}

У нас есть только импликация. % и тут распил блетб

Есть три правила: \(I_{\to}, E_{\to}, \mathrm{Ax}\). Будет ли у нас всё работать? Утверждается, что да.

\begin{obozn}
    Доказуемость в И.Ф. ИИВ будем обозначать \(\Gamma \vdash_{\to} \tau\)
\end{obozn}

\begin{definition}[язык И.Ф. ИИВ]
    \[\tau \Coloneqq \alpha \mid (\tau \to \tau)\]
\end{definition}

\begin{theorem}
    Импликационный фрагмент ИИВ замкнут относительно доказуемости:
    Если \(\Gamma \vdash \tau\) и \(\tau\) содержит только пропозиционные переменные и импликацию, то \(\Gamma \vdash_{\to} \tau\). Обратное очевидно верно.
\end{theorem}
\begin{proof}
    Рассмотрим \(\Gamma^*\) --- множество формул, замкнутых по доказуемости.
    \[\tau \in \Gamma^* \Leftrightarrow \Gamma^* \vdash \tau\]

    \begin{obozn}
        \(\Gamma\) --- множество формул, тогда \(\Gamma^*\) --- замыкание этого множества по доказуемости, а \(\Gamma^{\stackrel{*}{\to}}\) --- замыкание по доказуемости в ИФИИВ.
    \end{obozn}

    \underline{Рассмотрим множество миров}: \(\Gamma^{\stackrel{*}{\to}} \preceq \Delta^{\stackrel{*}{\to}}\), если \(\Gamma^{\stackrel{*}{\to}} \subseteq \Delta^{\stackrel{*}{\to}}, \Delta^{\stackrel{*}{\to}}\) --- замкн., \(\Gamma^* \Vdash \tau\), если \(\tau \in \Gamma^*\)

    \begin{statement}
        \(\Gamma^*\) образует модель Крипке.
    \end{statement}

    \begin{definition}[модель Крипке]\itemfix
        \begin{enumerate}
            \item Множество миров, упорядоченных отношением \(\preceq\)
            \item \(\Vdash\) такое, что если \(\Gamma \Vdash \alpha\), то \(\Gamma \preceq \Delta\), то \(\Delta \Vdash \alpha\).
        \end{enumerate}

        Тогда \(\Gamma \Vdash \tau \to \sigma\) тогда и только тогда, когда в любом \(\Gamma \preceq \Delta\) из \(\Delta \Vdash \tau\), следует \(\Delta \Vdash \sigma\).
    \end{definition}

    \begin{statement}
        \(\tau \in \Gamma^{\stackrel{*}{\to}}\) тогда и только тогда, когда \(\Gamma^{\stackrel{*}{\to}} \vdash_{\mathrm{и}} \tau\)
    \end{statement}
    \begin{proof}
        Индукция по структуре \(\tau\).
        \begin{itemize}
            \item [База.] \(\tau \equiv \alpha\)
                  \begin{itemize}
                      \item [\( \Rightarrow \)] \(\alpha \in \Gamma^{\stackrel{*}{\to}}\), то \(\alpha \vdash_{\mathrm{и}} \alpha\)
                      \item [\( \Leftarrow \)] \(\alpha \vdash_{\mathrm{и}} \alpha\), тогда очевидно \(\alpha \in \Gamma^{\stackrel{*}{\to}}\)
                  \end{itemize}
            \item [Переход.] \(\tau \equiv \delta \to \pi\)
                  \begin{itemize}
                      \item [\( \Rightarrow \)] \(\sigma \to \pi \in \Gamma^{\stackrel{*}{\to}}\), то \(\Gamma^{\stackrel{*}{\to}} \vdash_{\to} \sigma \to \pi\)
                      \item [\( \Leftarrow \)] \(\Gamma^{\stackrel{*}{\to}} \vdash_{\mathrm{и}} (\sigma \to \pi)\). Значит, \(\Gamma^{\stackrel{*}{\to}} \Vdash \sigma \to \pi\).
                  \end{itemize}
        \end{itemize}
    \end{proof}

    Рассмотрим \(\Gamma^{\stackrel{*}{\to}} \preceq \Delta : \Delta \Vdash \sigma\), то \(\Delta \Vdash \pi\). Значит, \(\Delta \vdash \sigma\). Значит, \(\sigma \in \Delta\), т.е. \(\Delta \vdash_{\to} \sigma\). Значит, \(\Delta \vdash_{\to} \pi\) по M.P., т.к. \(\Gamma^{\stackrel{*}{\to}} \Vdash \sigma \to \pi \Rightarrow \Delta \Vdash \sigma \to \pi \Rightarrow \Delta \vdash_{\to} \sigma \to \pi\)

    \begin{statement}
        \(\Gamma \Vdash \tau \Leftrightarrow \Gamma |_{\to} \tau\)
    \end{statement}
    \begin{proof}\itemfix
        \begin{itemize}
            \item [\( \Rightarrow \)] \(\Gamma \Vdash \tau\).
                  \begin{enumerate}
                      \item \(\Gamma \Vdash \alpha\), т.е. \(\alpha \in \Gamma\), т.е. \(\alpha \vdash_{ \to } \alpha\)
                      \item \(\Gamma \Vdash \sigma \to \pi\).

                            Рассмотрим \(\Gamma \preceq \Delta\), причём \(\Delta \Vdash \sigma\), тогда \(\Delta \Vdash \pi\). Т.е. по индукционному предположению \(\Delta \vdash_{ \to }\sigma\). Пусть \(\Delta = \{\Gamma, \sigma\}^*\). Тогда \(\Gamma, \sigma \vdash_{ \to } \sigma\).

                            Тогда \(\Gamma, \sigma \Vdash \pi\) по индукционному предложил и определению \( \Vdash \). Тогда \(\Gamma, \sigma \vdash_{ \to } \pi\), т.е. \(\Gamma \vdash_{ \to } \sigma \to \pi\)
                  \end{enumerate}
            \item [\( \Leftarrow \)] \(\Gamma \vdash_{ \to } \tau\), тогда \(\Gamma \Vdash \tau\)
                  \begin{enumerate}
                      \item \(\tau \equiv \alpha\) --- очевидно.
                      \item \(\tau \equiv \sigma \to \pi\). Дано, что \(\Gamma_{ \to } \vdash \sigma \to \pi\).

                            Пусть \(\Delta \Vdash \sigma\). \(\Gamma \preceq \Delta\). Тогда \(\Delta \vdash_{ \to } \sigma\) по индукционному предположению. \(\Gamma \vdash_{ \to } \sigma \to \pi\), т.е. \(\Delta \vdash_{ \to } \sigma \to \pi\). По M.P. \(\Delta \vdash_{ \to } \pi\). По индукционному предположению \(\Delta \Vdash \pi\). Т.е. \(\Gamma \Vdash \sigma \to \pi\). \textcolor{red}{В лекции было \(\models\).}
                  \end{enumerate}
        \end{itemize}
    \end{proof}

    Схема доказательства:
    \begin{enumerate}
        \item \(\tau \in \Gamma^*\), если \(\Gamma^* \vdash_{\text{и}} \tau\)
        \item \(\Gamma^* \Vdash \tau\)
        \item \(\Gamma^* \Vdash \tau\) тогда и только тогда, когда \(\Gamma^* \vdash_{ \to } \tau\)
    \end{enumerate}
\end{proof}

\begin{obozn}
    \(\lambda_{ \to }\) --- типизированное \(\lambda\)-исчисление.
\end{obozn}

\begin{enumerate}
    \item Обитаемость: \(\stackrel{?}{\Gamma} {}\vdash{} ? : \tau\) --- по изоморфизму Карри-Ховарда и теореме об эквивалентности \(\Gamma \vdash \tau\)
    \item Вывод (реконструкция): \(\Gamma \vdash A : ?\)
    \item Проверка: \(\Gamma \vdash A : \tau\)
\end{enumerate}

Пункты 2 и 3 это одно и то же.
