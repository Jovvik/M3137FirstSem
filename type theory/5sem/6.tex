\chapter{12 октября}

\section{Абстрактные типы данных}

ООП = АТД + наследование.

\begin{example}[стек]
    \[\texttt{push}: \alpha \to \alpha \texttt{ stack} \to \alpha \texttt{ stack}\]
    \[\texttt{pop}: \alpha \texttt{ stack} \to (\alpha \cdot \alpha \texttt{ stack})\]
    \[\texttt{empty} : \alpha \texttt{ stack}\]
\end{example}

Что мы понимаем под \(\exists \alpha.\varphi\)? \(\varphi\) --- интерфейс и существует где-то в природе тип, который этому интерфейсу соответствует.

Для стека:
\[\exists \alpha.\underbrace{(\eta \to \alpha \to \alpha)}_{\texttt{push}} \with \underbrace{(\alpha \to \alpha \with \eta)}_{\texttt{pop}} \with \underbrace{\eta}_{\texttt{empty}}\]

Правила вывода:
\[\begin{prooftree}
        \hypo{\Gamma \vdash \varphi [x \coloneqq \theta]}
        \infer1{\exists x.\varphi}
    \end{prooftree}\]

\[\begin{prooftree}
        \hypo{\Gamma, \psi \vdash \varphi}
        \hypo{\Gamma \vdash \exists x.\psi}
        \infer2[\(x \notin \mathrm{FV}(\Gamma)\)]{\Gamma \vdash \varphi}
    \end{prooftree}\]

\[\begin{prooftree}
        \hypo{\Gamma \vdash M : \sigma[\alpha \coloneqq \tau]}
        \infer1{\Gamma \vdash \texttt{pack } \tau, M \texttt{ to } \exists \alpha.\sigma : \exists \alpha.\sigma}
    \end{prooftree}\]

% Здесь \texttt{pack} \(\tau, M\) соответствует объекту, который мы подставляем вместо абстрактного типа данных, а

\textcolor{red}{TBD.}
