\chapter{19 октября}

\section{Типовая система Хиндли-Милнера}

Мы рассмотрели две системы типов:
\begin{enumerate}
    \item Просто типизированное лямбда исчисление: недостаточно выразительно
    \item Система F: местами выразительна, местами недостаточно. Кроме того, потеряна разрешимость.
\end{enumerate}

Ограничим излишнюю свободу системы F.

\begin{definition}[ранг типа]
    Пусть \(\sigma\) --- тип без кванторов. \(R \subset \text{тип} \times \N_0\), такое что:
    \begin{enumerate}
        \item \(R(\sigma, 0)\)
        \item Если \(R(\tau, k)\), то \(R(\forall \alpha.\tau,\max(k, 1))\)
        \item Если \(R(\tau_0, k)\) и \(R(\tau_1, k + 1)\), то \(R(\tau_0 \to \tau_1, k + 1)\).
    \end{enumerate}
\end{definition}

\begin{example}
    \[R(\alpha, 0) \Rightarrow R(\alpha, 5) \Rightarrow R(\forall \alpha.\alpha, 5)\]
    \[R(\alpha \to \alpha, 0) \Rightarrow R(\forall \alpha.\alpha \to \alpha, 1)\]
\end{example}

\begin{definition}[Типовая система Хиндли-Милнера]
    Рассмотрим \(\lambda\)-исчисление 2 порядка по Карри.

    Типы:
    \begin{enumerate}
        \item Типы без кванторов: \(\tau = \alpha \mid (\tau \to \tau)\)
        \item Типовые схемы: \(\sigma = \forall \alpha.\sigma \mid \tau\)
    \end{enumerate}
\end{definition}

\begin{definition}[Отношение ``\textbf{быть частным случаем}'' (специализация)]
    \(\sigma_1 \sqsubseteq \sigma_2\) (\(\sigma_2\) --- частный случай \(\sigma_1\)), если \(\sigma_1 \equiv \forall \alpha_1 \dots \forall \alpha_n.\tau_1\), \(\sigma \equiv \forall \beta_1 \dots \beta_k.\tau_1[\alpha_1 \coloneqq \theta_1 \dots \alpha_n \coloneqq \theta_n]\) и новые \(\beta_1 \dots \beta_k\) не входят свободно в \(\sigma_1\).
\end{definition}

\begin{example}
    \(\tau \sqsubseteq \texttt{string}\)
    \[\begin{tikzcd}
            & \texttt{string} \\
            \gamma \arrow[bend left = 20]{ru}{\gamma[\gamma \coloneqq \texttt{string}]}\arrow[bend right = 10]{rd}\arrow{r}{\gamma[\gamma \coloneqq \alpha \to \beta]} & \alpha \to \beta \arrow{r} & \texttt{string} \to \texttt{int} \\
            & (\alpha \to \beta) \to (\texttt{string} \to \texttt{int})
        \end{tikzcd}\]
\end{example}

\begin{example}
    \[\forall \alpha.\alpha \to \alpha \sqsubseteq \forall \beta.(\beta \to \beta) \to (\beta \to \beta)\]
\end{example}

Правила вывода:
\[\begin{prooftree}
        \infer0[\(\alpha \notin \mathrm{FV}(\Gamma)\)]{\Gamma, x : \sigma \vdash x : \sigma}
    \end{prooftree}\]
\[\begin{prooftree}
        \hypo{\Gamma \vdash A : \tau \to \tau'}
        \hypo{\Gamma \vdash B : \tau}
        \infer2{\Gamma \vdash A\ B : \tau'}
    \end{prooftree}\]
\[\begin{prooftree}
        \hypo{\Gamma, x : \tau \vdash A : \tau'}
        \infer1{\Gamma : \lambda x.A : \tau \to \tau'}
    \end{prooftree}\]
\[\begin{prooftree}
        \hypo{\Gamma \vdash A : \sigma}
        \hypo{\Gamma, x : \sigma \vdash B : \tau}
        \infer2{\Gamma \vdash \mathrm{let}\ x = A\ \mathrm{in}\ B : \tau}
    \end{prooftree}\]
\[\begin{prooftree}
        \hypo{\Gamma \vdash A : \sigma'}
        \infer1[\(\sigma' \sqsubseteq \sigma\)]{\Gamma \vdash A : \sigma}
    \end{prooftree}\]
\[\begin{prooftree}
        \hypo{\Gamma \vdash A : \sigma}
        \infer1[\(\alpha \notin \mathrm{FV}(\Gamma)\)]{\Gamma \vdash A : \forall \alpha.\sigma}
    \end{prooftree}\]
\[\mathrm{let}\ x = A\ \mathrm{in}\ B \twoheadrightarrow_\beta B[x \coloneqq A]\]

Казалось бы, \(\mathrm{let}\) похож на \((\lambda x.B)\ A\). Однако, мы разрешаем кванторы в \(A\).

\begin{example}
    \[I \equiv \lambda x.x : \forall \alpha.\alpha \to \alpha\]
    \[\sphericalangle (I\ 1, I\ \texttt{\dq a\dq}) \quad I\ 1 : \texttt{int}, I\ \texttt{\dq a\dq} : \texttt{string}\]
\end{example}

\begin{enumerate}
    \item \[\mathrm{let}\ \underbrace{I = \lambda x.x}_{I : \forall \alpha.\alpha \to \alpha}\ \mathrm{in}\ (I\ 1, I\ \texttt{\dq a\dq})\]
    \item То же самое, но без let:
          \[(\lambda i.(i\ 1,i\ \texttt{\dq a\dq}))\ (\lambda x.x)\]
          В этом варианте тип внутри лямбды без кванторов. В силу этого операцию \((i\ 1,i\ \texttt{\dq a\dq})\) сложно выполнить --- нужно угадать, какой тип подставлять.
\end{enumerate}

Эта система очевидно \'{у}же, чем System F. Мы её сузили, чтобы получить разрешимость.

\subsection{Алгоритм \(W\)}

Мы хотим решить \(? \vdash A : ?\), т.е найти контекст и тип выражения, притом наиболее общие.

\(W(\Gamma, E) \Rightarrow (\tau, S)\) --- по контексту и выражению получаем тип и подстановку.

\begin{enumerate}
    \item \(E \equiv x, x \in \Gamma, x : \sigma_x = \forall \alpha_1 \dots \alpha_n.\tau\). Тогда введём новые переменные \(\beta_1 \dots \beta_n\) и результатом будет:
          \[W(\Gamma, E) = (\forall \beta_1 \dots \beta_n.\tau, \emptyset)\]
    \item \(E \equiv \lambda x.P\). Пусть \(W(\Gamma \cup \{x : \gamma\}, P) = (\tau_P, S_1)\).
          \[W(\Gamma, E) = (S_1(\gamma \to \tau_P), S_1)\]
    \item \(E \equiv P\ Q\). Введём новый тип \(\gamma\). Пусть \(\mho\) --- вызов алгоритма унификации и:
          \[W(\Gamma, P) = (\tau_P, S_1) \quad W(S_1 \Gamma, Q) = (\tau_Q, S_2) \quad \mho(S_2 \tau_P, \tau_Q \to \gamma) = S_3\]
          Тогда:
          \[W(\Gamma, E) = (S_3 \gamma, S_3 \circ S_2 \circ S_1)\]
    \item \(E \equiv \mathrm{let}\ x = P\ \mathrm{in}\ Q\). Пусть:
          \[W(\Gamma, P) = (\tau_P, S_1) \quad W(S_1 \Gamma \cup \{x : \forall\footnote{Кванторы по всем свободным в \(\tau_f\) переменным.}.\tau_f\}, Q) = (S_2, \tau_Q)\]
          \[W(\Gamma, E) = (\tau_Q, S_2 \circ S_1)\]
\end{enumerate}

\begin{example}
    \[\mathrm{let}\ I = \lambda x.x\ \mathrm{in}\ (I\ 1, I\ \mlq\mrq)\]
    Согласно 4 пункту алгоритма:
    \[I : \forall \alpha.\alpha \to \alpha \vdash (I\ 1, I\ \mlq\mrq)\]
\end{example}

Мы теряем полноту по Тьюрингу, т.к. это частный случай системы F. Мы такого не хотим, поэтому добавим чего-нибудь, что её нам даст.

\begin{enumerate}
    \item Правило для \(Y\):
          \[Y : \forall \alpha.(\alpha \to \alpha) \to \alpha\]

          Теория становится противоречивой --- вместо \(\alpha \to \alpha\) всегда можно подставить \(\mathrm{id}\) и получить любое \(\alpha\).

    \item \(\sphericalangle \texttt{IntList} = (\texttt{int} \with \texttt{IntList}) \lor \texttt{Nil}\). Это какое-то уравнение. Как его решить?

          Введём \(\mu\)-оператор: \(\mu\ \eta.(\texttt{int} \with \eta) \lor \texttt{Nil}\) или в общем случае \(\mu\ \eta.\tau\) --- тип, решающий уравнение \(\eta = \tau\)
\end{enumerate}

Есть две традиции решения таких уравнений:
\begin{enumerate}
    \item Экви-рекурсивные: \(\mu\) существует как тип (Java).
    \item Изо-рекурсивные: ищем \(\mu \eta.\tau(\eta)\) вводятся две операции
          \begin{enumerate}
              \item Roll: \(\tau(\eta) \to \eta\)
              \item Unroll: \(\eta \to \tau(\eta)\)
          \end{enumerate}
\end{enumerate}

Итого мы взяли систему F и:
\begin{enumerate}
    \item Запретили типы с неповерхностными кванторами
    \item Добавили \texttt{let}-полиморфизм
    \item Добавили противоречивость через \(Y\)-комбинатор и решение уравнений на типах.
\end{enumerate}
