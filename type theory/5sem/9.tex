\chapter{2 ноября}

\textcolor{red}{Первые полчаса лекции пропущены.}

\begin{definition}
    \textbf{Путь} между \(a\) и \(b\) в пространстве \(X\) --- функция \(f : [0, 1] \to X, f(0) = a, f(1) = b\) и \(f\) непрерывно. Если есть путь из \(a\) в \(b\), то мы считаем \(a\) и \(b\) равными.
\end{definition}

\begin{definition}
    \textbf{Интервальный тип}: \(I = [left, right]\)
\end{definition}

Почему не \(I = \{left, right\}\)?

\begin{example}
    Докажем, что \(2 + 1 = 1 + 2\). Рассмотрим \(f\) --- потенциальный путь \(f(left) = 1 + 2, f(right) = 2 + 1\)
\end{example}

\begin{definition}
    Отображение \textbf{непрерывно}, если\footnote{И только если.} прообразо открытого множества открыт.
\end{definition}

\begin{example}
    \(f : \{0, 1\} \to \{2, 3\}\), дискретная топология. Непрерывна ли \(f x = x + 2\)?

    Открытые в \(\{2, 3\} : \{\}, \{2\}, \{3\}, \{2, 3\}\)

    \[f^{-1}(\{\}) = \{\}, f^{-1}(\{2\}) = \{0\}, f^{-1}(\{3\}) = \{1\}, f^{-1}(\{2, 3\}) = \{0, 1\}\]
    Таким образом \(f\) непрерывна.
\end{example}

\begin{example}
    \(\sphericalangle f : \R \to \R, f(x) = \floor{x}\)

    \(f^{-1}((0.1, 0.2)) = 0\), что не является открытым множеством.

    Кроме того, \(f^{-1}([0, 5]) = \{0,1,2,3,4,5\}\)
\end{example}

\begin{definition}
    Пространство \(X\) \textbf{не связно}, если существует \(P, Q : X = P \cup Q, P \cap Q = \emptyset, P \neq X, Q \neq X, P, Q\) открыты.

    Множество \textbf{связно}, если является связным пространством под индуцированной топологией (или при сужении топологии на него, то же самое).
\end{definition}

\begin{example}
    \((0, 1) \cup (2, 3)\) --- не связно.
\end{example}

\begin{example}
    \((0, 1) \cup (2, 3]\) --- тоже не связно, т.к. \((2, 3]\) открыто в \((0, 1) \cup (2, 3]\)
\end{example}

\begin{example}
    \((0, 1) \cup (2, 3)\) в дискретной топологии --- не связно.
\end{example}

\begin{exercise}
    Предложите топологию, в которой пространство \((0, 1) + (2, 3)\) связно. Тривиальная топология.
\end{exercise}

\begin{definition}
    Пространство \textbf{линейно связно}, если существует путь из любой точки в любую.
\end{definition}

Первое определение гласит, что мы не можем провести границу, а второе --- что не можем \?. % TODO

\begin{definition}
    \textbf{Равенство} --- тип пути из \(a : X\) в \(b : X\). Равенство обитаемо, если такой путь существует.
\end{definition}

Такое определение сложно уместить в язык программирования, т.к. \([0, 1]\) не влезает в компьютер, поэтому придуман интервальный тип \(I = [left, right]\).

\begin{example}
    \(\sphericalangle 1 : \mathrm{Nat}, 2 : \mathrm{Nat}\), в \(\mathrm{Nat}\) дискретная топология. Нужно построить \(f : f( 0) = 1, f(1) = 2\).
\end{example}

Рассмотрим \texttt{Path} в Arend:
\begin{minted}{haskell}
\data Path (A : I -> \Type) (a : A left) (a' A right)
    | path (\Pi (i : I) -> A i)
\end{minted}

\(I\) --- интервальный тип, \(A\) --- отображение интервал \( \to \) тип. \(\mathrm{path}\) --- единственный конструктор, принимает функцию, сопоставляющую точки интервала значение \(A\) в этой точке.

Докажем в Arend, что \(1 = 1\):
\begin{minted}{haskell}
\func oneone : (1 = 1) => idp
\end{minted}

Если мы не знаем, что сделать, мы можешь написать \texttt{?}, то выражение условно принимается. Иногда работают рефакторинги, которые заменят \texttt{?} на доказательство.

Фигурные скобки вокруг аргумента обозначают, что аргумент неявный. Запишем в явном виде:

\begin{minted}{haskell}
\func oneone : (1 = 1) => idp {Nat} {1}
\end{minted}

\texttt{oneone} --- значение зависимого типа равенства.

\begin{minted}{haskell}
\func arar : ((1 Nat.+ 2) = (2 Nat.+ 1)) => idp
\end{minted}

\texttt{idp} это \texttt{refl} из Lean.

У нас интенсиональное равенство.

Построим фукнцию с типовым параметром:
\begin{minted}{haskell}
\func second (t : \Type) (ttt : \Pi (x : t) -> t) (y : t) : t => ttt y
\end{minted}
Здесь \texttt{ttt} --- функция \(t \to t\), мы её применяем к \(y : t\) и получаем \(ttt\ y : t\)

\begin{minted}{haskell}
\func third (t : \Type) (ttt : \Pi (x : t) -> t) (y : t) => ttt y = ttt y
\end{minted}
