\documentclass[12pt, a4paper]{article}

%<*preamble>
% Math symbols
\usepackage{amsmath, amsthm, amsfonts, amssymb}
\usepackage{accents}
\usepackage{esvect}
\usepackage{mathrsfs}
\usepackage{mathtools}
\mathtoolsset{showonlyrefs}
\usepackage{cmll}
\usepackage{stmaryrd}
\usepackage{physics}
\usepackage[normalem]{ulem}
\usepackage{ebproof}
\usepackage{extarrows}

% Page layout
\usepackage{geometry, a4wide, parskip, fancyhdr}

% Font, encoding, russian support
\usepackage[russian]{babel}
\usepackage[sb]{libertine}
\usepackage{xltxtra}

% Listings
\usepackage{listings}
\lstset{basicstyle=\ttfamily,breaklines=true}
\setmonofont{Inconsolata}

% Miscellaneous
\usepackage{array}
\usepackage{calc}
\usepackage{caption}
\usepackage{subcaption}
\captionsetup{justification=centering,margin=2cm}
\usepackage{catchfilebetweentags}
\usepackage{enumitem}
\usepackage{etoolbox}
\usepackage{float}
\usepackage{lastpage}
\usepackage{minted}
\usepackage{svg}
\usepackage{wrapfig}
\usepackage{xcolor}
\usepackage[makeroom]{cancel}

\newcolumntype{L}{>{$}l<{$}}
    \newcolumntype{C}{>{$}c<{$}}
\newcolumntype{R}{>{$}r<{$}}

% Footnotes
\usepackage[hang]{footmisc}
\setlength{\footnotemargin}{2mm}
\makeatletter
\def\blfootnote{\gdef\@thefnmark{}\@footnotetext}
\makeatother

% References
\usepackage{hyperref}
\hypersetup{
    colorlinks,
    linkcolor={blue!80!black},
    citecolor={blue!80!black},
    urlcolor={blue!80!black},
}

% tikz
\usepackage{tikz}
\usepackage{tikz-cd}
\usetikzlibrary{arrows.meta}
\usetikzlibrary{decorations.pathmorphing}
\usetikzlibrary{calc}
\usetikzlibrary{patterns}
\usepackage{pgfplots}
\pgfplotsset{width=10cm,compat=1.9}
\newcommand\irregularcircle[2]{% radius, irregularity
    \pgfextra {\pgfmathsetmacro\len{(#1)+rand*(#2)}}
    +(0:\len pt)
    \foreach \a in {10,20,...,350}{
            \pgfextra {\pgfmathsetmacro\len{(#1)+rand*(#2)}}
            -- +(\a:\len pt)
        } -- cycle
}

\providetoggle{useproofs}
\settoggle{useproofs}{false}

\pagestyle{fancy}
\lfoot{M3137y2019}
\cfoot{}
\rhead{стр. \thepage\ из \pageref*{LastPage}}

\newcommand{\R}{\mathbb{R}}
\newcommand{\Q}{\mathbb{Q}}
\newcommand{\Z}{\mathbb{Z}}
\newcommand{\B}{\mathbb{B}}
\newcommand{\N}{\mathbb{N}}
\renewcommand{\Re}{\mathfrak{R}}
\renewcommand{\Im}{\mathfrak{I}}

\newcommand{\const}{\text{const}}
\newcommand{\cond}{\text{cond}}

\newcommand{\teormin}{\textcolor{red}{!}\ }

\DeclareMathOperator*{\xor}{\oplus}
\DeclareMathOperator*{\equ}{\sim}
\DeclareMathOperator{\sign}{\text{sign}}
\DeclareMathOperator{\Sym}{\text{Sym}}
\DeclareMathOperator{\Asym}{\text{Asym}}

\DeclarePairedDelimiter{\ceil}{\lceil}{\rceil}

% godel
\newbox\gnBoxA
\newdimen\gnCornerHgt
\setbox\gnBoxA=\hbox{$\ulcorner$}
\global\gnCornerHgt=\ht\gnBoxA
\newdimen\gnArgHgt
\def\godel #1{%
    \setbox\gnBoxA=\hbox{$#1$}%
    \gnArgHgt=\ht\gnBoxA%
    \ifnum     \gnArgHgt<\gnCornerHgt \gnArgHgt=0pt%
    \else \advance \gnArgHgt by -\gnCornerHgt%
    \fi \raise\gnArgHgt\hbox{$\ulcorner$} \box\gnBoxA %
    \raise\gnArgHgt\hbox{$\urcorner$}}

% \theoremstyle{plain}

\theoremstyle{definition}
\newtheorem{theorem}{Теорема}
\newtheorem*{definition}{Определение}
\newtheorem{axiom}{Аксиома}
\newtheorem*{axiom*}{Аксиома}
\newtheorem{lemma}{Лемма}

\theoremstyle{remark}
\newtheorem*{remark}{Примечание}
\newtheorem*{exercise}{Упражнение}
\newtheorem{corollary}{Следствие}[theorem]
\newtheorem*{statement}{Утверждение}
\newtheorem*{corollary*}{Следствие}
\newtheorem*{example}{Пример}
\newtheorem{observation}{Наблюдение}
\newtheorem*{prop}{Свойства}
\newtheorem*{obozn}{Обозначение}

% subtheorem
\makeatletter
\newenvironment{subtheorem}[1]{%
    \def\subtheoremcounter{#1}%
    \refstepcounter{#1}%
    \protected@edef\theparentnumber{\csname the#1\endcsname}%
    \setcounter{parentnumber}{\value{#1}}%
    \setcounter{#1}{0}%
    \expandafter\def\csname the#1\endcsname{\theparentnumber.\Alph{#1}}%
    \ignorespaces
}{%
    \setcounter{\subtheoremcounter}{\value{parentnumber}}%
    \ignorespacesafterend
}
\makeatother
\newcounter{parentnumber}

\newtheorem{manualtheoreminner}{Теорема}
\newenvironment{manualtheorem}[1]{%
    \renewcommand\themanualtheoreminner{#1}%
    \manualtheoreminner
}{\endmanualtheoreminner}

\newcommand{\dbltilde}[1]{\accentset{\approx}{#1}}
\newcommand{\intt}{\int\!}

% magical thing that fixes paragraphs
\makeatletter
\patchcmd{\CatchFBT@Fin@l}{\endlinechar\m@ne}{}
{}{\typeout{Unsuccessful patch!}}
\makeatother

\newcommand{\get}[2]{
    \ExecuteMetaData[#1]{#2}
}

\newcommand{\getproof}[2]{
    \iftoggle{useproofs}{\ExecuteMetaData[#1]{#2proof}}{}
}

\newcommand{\getwithproof}[2]{
    \get{#1}{#2}
    \getproof{#1}{#2}
}

\newcommand{\import}[3]{
    \subsection{#1}
    \getwithproof{#2}{#3}
}

\newcommand{\given}[1]{
    Дано выше. (\ref{#1}, стр. \pageref{#1})
}

\renewcommand{\ker}{\text{Ker }}
\newcommand{\im}{\text{Im }}
\renewcommand{\grad}{\text{grad}}
\newcommand{\rg}{\text{rg}}
\newcommand{\defeq}{\stackrel{\text{def}}{=}}
\newcommand{\defeqfor}[1]{\stackrel{\text{def } #1}{=}}
\newcommand{\itemfix}{\leavevmode\makeatletter\makeatother}
\newcommand{\?}{\textcolor{red}{???}}
\renewcommand{\emptyset}{\varnothing}
\newcommand{\longarrow}[1]{\xRightarrow[#1]{\qquad}}
\DeclareMathOperator*{\esup}{\text{ess sup}}
\newcommand\smallO{
    \mathchoice
    {{\scriptstyle\mathcal{O}}}% \displaystyle
    {{\scriptstyle\mathcal{O}}}% \textstyle
    {{\scriptscriptstyle\mathcal{O}}}% \scriptstyle
    {\scalebox{.6}{$\scriptscriptstyle\mathcal{O}$}}%\scriptscriptstyle
}
\renewcommand{\div}{\text{div}\ }
\newcommand{\rot}{\text{rot}\ }
\newcommand{\cov}{\text{cov}}

\makeatletter
\newcommand{\oplabel}[1]{\refstepcounter{equation}(\theequation\ltx@label{#1})}
\makeatother

\newcommand{\symref}[2]{\stackrel{\oplabel{#1}}{#2}}
\newcommand{\symrefeq}[1]{\symref{#1}{=}}

% xrightrightarrows
\makeatletter
\newcommand*{\relrelbarsep}{.386ex}
\newcommand*{\relrelbar}{%
    \mathrel{%
        \mathpalette\@relrelbar\relrelbarsep
    }%
}
\newcommand*{\@relrelbar}[2]{%
    \raise#2\hbox to 0pt{$\m@th#1\relbar$\hss}%
    \lower#2\hbox{$\m@th#1\relbar$}%
}
\providecommand*{\rightrightarrowsfill@}{%
    \arrowfill@\relrelbar\relrelbar\rightrightarrows
}
\providecommand*{\leftleftarrowsfill@}{%
    \arrowfill@\leftleftarrows\relrelbar\relrelbar
}
\providecommand*{\xrightrightarrows}[2][]{%
    \ext@arrow 0359\rightrightarrowsfill@{#1}{#2}%
}
\providecommand*{\xleftleftarrows}[2][]{%
    \ext@arrow 3095\leftleftarrowsfill@{#1}{#2}%
}

\allowdisplaybreaks

\newcommand{\unfinished}{\textcolor{red}{Не дописано}}

% Reproducible pdf builds 
\special{pdf:trailerid [
<00112233445566778899aabbccddeeff>
<00112233445566778899aabbccddeeff>
]}
%</preamble>


\lhead{Теория типов \textit{(практика)}}
\cfoot{}
\rfoot{21.9.2021}

\begin{document}

\section*{Домашнее задание №2: <<формализация лямбда-исчисления>>}

\begin{enumerate}
    \item Придумайте грамматику для лямбда-выражений, однозначно разбирающую любое выражение
          (в частности, учитывающую все сокращения скобок в записи).
          \begin{solution}\itemfix
              \begin{itemize}
                  \item \(S \to \lambda T.S\)
                  \item \(S \to S\ S\)
                  \item \(S \to T\)
                  \item \(S \to (S)\)
                  \item \(T \to \Sigma\)
              \end{itemize}
          \end{solution}
    \item Приведите пример лямбда-выражения, корректная бета-редукция которого невозможна без переименования
          связанных переменных. Возможно ли, чтобы в этом выражении все переменные в лямбда-абстракциях
          были различными?
          \begin{solution}
              \((\lambda y.\lambda x.y\ x)\ (\lambda x.x)\)
          \end{solution}
    \item Два выражения $A$ и $B$ назовём родственными, если существует $C$, что
          $A \twoheadrightarrow_\beta C$ и $B \twoheadrightarrow_\beta C$.
          Как соотносится родственность и бета-эквивалентность?
          \begin{solution}
              \begin{notation}
                  \(A\) родственн. \(B\) \(\Leftrightarrow\) \(A \sim B\)
              \end{notation}

              \[A \sim B \Rightarrow \begin{cases}
                      A \twoheadrightarrow_\beta C \\
                      B \twoheadrightarrow_\beta C
                  \end{cases} \Rightarrow \begin{cases}
                      A =_\beta C \\
                      B =_\beta C
                  \end{cases} \Rightarrow A =_\beta B\]

              В обратную сторону тоже верно по ромбовидному свойству.
          \end{solution}

    \item Рассмотрим представление лямбда-выражений де Брауна (de Bruijn): вместо имени связанной переменной будем
          указывать число промежуточных лямбда-абстракций между связывающей абстрацией и переменной.
          Например, $\lambda x.\lambda y.y\ x$ превратится в $\lambda.\lambda.0\ 1$.

          Докажите, что $A =_\alpha B$ тогда и только тогда, когда представления де Брауна для $A$ и $B$ совпадают.
          Сформулируйте правила (алгоритмы) для подстановки термов и бета-редукции для этого представления.

    \item Как мы знаем, $\Omega \rightarrow_\beta \Omega$. А существуют ли такие лямбда-выражения
          $A$ и $B$ ($A \ne_\alpha B$), что $A \rightarrow_\beta B$ и $B \rightarrow_\beta A$?

          \begin{solution}
              \(A = \omega\ (\lambda x.\omega\ x), B = (\lambda x.\omega\ x)\ (\lambda x.\omega\ x)\)
          \end{solution}

    \item Рассмотрим следующие лямбда-выражения для задания алгебраических типов:

          \begin{tabular}{lll}
              Обозначение & лямбда-терм                             & название                 \\\hline
              $Case$      & $\lambda l.\lambda r.\lambda c.c\ l\ r$ & сопоставление с образцом \\
              $InL$       & $\lambda l.(\lambda x.\lambda y.x\ l)$  & левая инъекция           \\
              $InR$       & $\lambda r.(\lambda x.\lambda y.y\ r)$  & правая инъекция          \\
          \end{tabular}

          Сопоставление с образцом --- это функция от значения алгебраического типа и двух действий $l$ и $r$,
          которая выполняет действие $l$, если значение создано <<левым>> конструктором, и $r$ в случае
          <<правого>> конструктора. Иными словами, $Case\ l\ r\ c$ --- это аналог
          \verb!case c { InL x -> l x; InR x -> r x }!.

          Заметим, что список (например, целых чисел) — это алгебраический тип:

          \verb!List = Nil | Cons Integer List!.

          Можно сконструировать значение данного типа: \verb!Cons 3 (Cons 5 Nil)!.
          Можно, например, вычислить его длину:
          \begin{verbatim}
length Nil = 0
length (Cons _ tail) = length tail + 1
\end{verbatim}

          Определим $Nil = InL\ 0$, а $Cons\ a\ b = InR\ (MkPair\ a\ b)$. Заметим, что теперь списки могут быть впрямую
          перенесены в лямбда выражения.

          Определите следующие функции в лямбда-исчислении для списков:
          \begin{enumerate}
              \item вычисление длины списка;
                    \[Y\ \lambda f.\lambda l.\mathrm{Case}\ (\lambda x.0)\ (\lambda p.( + 1)\ (f\ (\mathrm{PrR}\ p)))\ l\]
              \item построение списка длины $n$ из элементов $0, 1, 2, \dots, n-1$;
                    \[\lambda n.(Y\ \lambda f.(\lambda n'.\lambda m.(\mathrm{Eq}\ n'\ m)\ \mathrm{Nil}\ (\mathrm{Cons}\ m\ (f\ n'\ (\mathrm{inc}\ m)))))\ n\ 0\]
              \item разворот списка: из списка $a_1, a_2, \dots, a_n$ сделать список $a_n, a_{n-1}, \dots, a_1$;
                    \[\mathrm{Add} = \lambda e.Y\ \lambda f.\mathrm{Case}\ (\lambda x.\mathrm{Cons}\ e\ \mathrm{Nil})\ (\lambda p.\mathrm{Cons}\ (\mathrm{PrL}\ p)\ (f\ (\mathrm{PrR}\ p)))\]
                    \[\mathrm{Reverse} = Y\ \lambda f.\mathrm{Case}\ (\lambda x.\mathrm{Nil})\ (\lambda p.\mathrm{Add}\ (\mathrm{PrR}\ p)\ (f\
                        (\mathrm{PrL}\ p)))\]
              \item функцию высшего порядка $map$, которая по функции $f$ и списку $a_1, a_2, \dots, a_n$
                    строит список $f(a_1), f(a_2), \dots, f(a_n)$.
                    \[\lambda g.Y\ \lambda f.\mathrm{Case}\ (\lambda x.\mathrm{Nil})\ (\lambda p.\mathrm{Cons}\ (g\ (\mathrm{PrL}\ p))\ (f\ (\mathrm{PrR}\ p)))\]
          \end{enumerate}

          Решением задачи является полный текст соответствующего лямбда-выражения с объяснениями механизма его работы.
          Используйте интерпретатор лямбда-выражений $lci$ или аналогичный для демонстрации результата.

    \item Чёрчевские нумералы соответствуют натуральным числам в аксиоматике Пеано.
          \begin{enumerate}
              \item Предложите <<двоичные нумералы>> --- способ кодирования чисел, аналогичный двоичной системе
                    (такой, при котором длина записи числа соответствует логарифму числового значения).
                    \[\lambda n.\mathrm{Reverse}\ ((Y\ \lambda f.\lambda r.(\mathrm{IsZero}\ r)\ \mathrm{Nil}\ (\mathrm{Cons}\ ((\mathrm{IsEven}\ r)\ 0\ 1)\ (f\ (\mathrm{DivBy2}\ r))))\ n)\]
              \item Предложите реализацию функции (+1) в данном представлении.
                    \begin{verbatim}
BinaryPlus1 = \n.Reverse(
    (Y \f.\l.\r.Case 
        (\x.
            (IsZero r)
                Nil
                (Cons 1 Nil))
        (\p.
            (IsZero r)
                (Cons (PrL p) (f (PrR p) 0))
                ((IsZero (PrL p))
                    (Cons 1 (f (PrR p) 0))
                    (Cons 0 (f (PrR p) 1))))
        l)
    (Reverse n)
    1);
                    \end{verbatim}
              \item Предложите реализацию лямбда-выражения преобразования числа из двоичного нумерала в чёрчевский.
                    \begin{verbatim}
ToChurch = \b.
(Y \f.\l.\s.Case
    (\x.s)
    (\p.f
        (PrR p)
        (Plus (Plus s s) (PrL p)))
    l)
b 0;
                    \end{verbatim}
          \end{enumerate}

          Аналогично прошлому заданию, решение должно содержать полный код лямбда-выражения вместе с объяснением механизма его работы.

\end{enumerate}

\end{document}
