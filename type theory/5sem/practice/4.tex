\documentclass[12pt, a4paper]{article}

%<*preamble>
% Math symbols
\usepackage{amsmath, amsthm, amsfonts, amssymb}
\usepackage{accents}
\usepackage{esvect}
\usepackage{mathrsfs}
\usepackage{mathtools}
\mathtoolsset{showonlyrefs}
\usepackage{cmll}
\usepackage{stmaryrd}
\usepackage{physics}
\usepackage[normalem]{ulem}
\usepackage{ebproof}
\usepackage{extarrows}

% Page layout
\usepackage{geometry, a4wide, parskip, fancyhdr}

% Font, encoding, russian support
\usepackage[russian]{babel}
\usepackage[sb]{libertine}
\usepackage{xltxtra}

% Listings
\usepackage{listings}
\lstset{basicstyle=\ttfamily,breaklines=true}
\setmonofont{Inconsolata}

% Miscellaneous
\usepackage{array}
\usepackage{calc}
\usepackage{caption}
\usepackage{subcaption}
\captionsetup{justification=centering,margin=2cm}
\usepackage{catchfilebetweentags}
\usepackage{enumitem}
\usepackage{etoolbox}
\usepackage{float}
\usepackage{lastpage}
\usepackage{minted}
\usepackage{svg}
\usepackage{wrapfig}
\usepackage{xcolor}
\usepackage[makeroom]{cancel}

\newcolumntype{L}{>{$}l<{$}}
    \newcolumntype{C}{>{$}c<{$}}
\newcolumntype{R}{>{$}r<{$}}

% Footnotes
\usepackage[hang]{footmisc}
\setlength{\footnotemargin}{2mm}
\makeatletter
\def\blfootnote{\gdef\@thefnmark{}\@footnotetext}
\makeatother

% References
\usepackage{hyperref}
\hypersetup{
    colorlinks,
    linkcolor={blue!80!black},
    citecolor={blue!80!black},
    urlcolor={blue!80!black},
}

% tikz
\usepackage{tikz}
\usepackage{tikz-cd}
\usetikzlibrary{arrows.meta}
\usetikzlibrary{decorations.pathmorphing}
\usetikzlibrary{calc}
\usetikzlibrary{patterns}
\usepackage{pgfplots}
\pgfplotsset{width=10cm,compat=1.9}
\newcommand\irregularcircle[2]{% radius, irregularity
    \pgfextra {\pgfmathsetmacro\len{(#1)+rand*(#2)}}
    +(0:\len pt)
    \foreach \a in {10,20,...,350}{
            \pgfextra {\pgfmathsetmacro\len{(#1)+rand*(#2)}}
            -- +(\a:\len pt)
        } -- cycle
}

\providetoggle{useproofs}
\settoggle{useproofs}{false}

\pagestyle{fancy}
\lfoot{M3137y2019}
\cfoot{}
\rhead{стр. \thepage\ из \pageref*{LastPage}}

\newcommand{\R}{\mathbb{R}}
\newcommand{\Q}{\mathbb{Q}}
\newcommand{\Z}{\mathbb{Z}}
\newcommand{\B}{\mathbb{B}}
\newcommand{\N}{\mathbb{N}}
\renewcommand{\Re}{\mathfrak{R}}
\renewcommand{\Im}{\mathfrak{I}}

\newcommand{\const}{\text{const}}
\newcommand{\cond}{\text{cond}}

\newcommand{\teormin}{\textcolor{red}{!}\ }

\DeclareMathOperator*{\xor}{\oplus}
\DeclareMathOperator*{\equ}{\sim}
\DeclareMathOperator{\sign}{\text{sign}}
\DeclareMathOperator{\Sym}{\text{Sym}}
\DeclareMathOperator{\Asym}{\text{Asym}}

\DeclarePairedDelimiter{\ceil}{\lceil}{\rceil}

% godel
\newbox\gnBoxA
\newdimen\gnCornerHgt
\setbox\gnBoxA=\hbox{$\ulcorner$}
\global\gnCornerHgt=\ht\gnBoxA
\newdimen\gnArgHgt
\def\godel #1{%
    \setbox\gnBoxA=\hbox{$#1$}%
    \gnArgHgt=\ht\gnBoxA%
    \ifnum     \gnArgHgt<\gnCornerHgt \gnArgHgt=0pt%
    \else \advance \gnArgHgt by -\gnCornerHgt%
    \fi \raise\gnArgHgt\hbox{$\ulcorner$} \box\gnBoxA %
    \raise\gnArgHgt\hbox{$\urcorner$}}

% \theoremstyle{plain}

\theoremstyle{definition}
\newtheorem{theorem}{Теорема}
\newtheorem*{definition}{Определение}
\newtheorem{axiom}{Аксиома}
\newtheorem*{axiom*}{Аксиома}
\newtheorem{lemma}{Лемма}

\theoremstyle{remark}
\newtheorem*{remark}{Примечание}
\newtheorem*{exercise}{Упражнение}
\newtheorem{corollary}{Следствие}[theorem]
\newtheorem*{statement}{Утверждение}
\newtheorem*{corollary*}{Следствие}
\newtheorem*{example}{Пример}
\newtheorem{observation}{Наблюдение}
\newtheorem*{prop}{Свойства}
\newtheorem*{obozn}{Обозначение}

% subtheorem
\makeatletter
\newenvironment{subtheorem}[1]{%
    \def\subtheoremcounter{#1}%
    \refstepcounter{#1}%
    \protected@edef\theparentnumber{\csname the#1\endcsname}%
    \setcounter{parentnumber}{\value{#1}}%
    \setcounter{#1}{0}%
    \expandafter\def\csname the#1\endcsname{\theparentnumber.\Alph{#1}}%
    \ignorespaces
}{%
    \setcounter{\subtheoremcounter}{\value{parentnumber}}%
    \ignorespacesafterend
}
\makeatother
\newcounter{parentnumber}

\newtheorem{manualtheoreminner}{Теорема}
\newenvironment{manualtheorem}[1]{%
    \renewcommand\themanualtheoreminner{#1}%
    \manualtheoreminner
}{\endmanualtheoreminner}

\newcommand{\dbltilde}[1]{\accentset{\approx}{#1}}
\newcommand{\intt}{\int\!}

% magical thing that fixes paragraphs
\makeatletter
\patchcmd{\CatchFBT@Fin@l}{\endlinechar\m@ne}{}
{}{\typeout{Unsuccessful patch!}}
\makeatother

\newcommand{\get}[2]{
    \ExecuteMetaData[#1]{#2}
}

\newcommand{\getproof}[2]{
    \iftoggle{useproofs}{\ExecuteMetaData[#1]{#2proof}}{}
}

\newcommand{\getwithproof}[2]{
    \get{#1}{#2}
    \getproof{#1}{#2}
}

\newcommand{\import}[3]{
    \subsection{#1}
    \getwithproof{#2}{#3}
}

\newcommand{\given}[1]{
    Дано выше. (\ref{#1}, стр. \pageref{#1})
}

\renewcommand{\ker}{\text{Ker }}
\newcommand{\im}{\text{Im }}
\renewcommand{\grad}{\text{grad}}
\newcommand{\rg}{\text{rg}}
\newcommand{\defeq}{\stackrel{\text{def}}{=}}
\newcommand{\defeqfor}[1]{\stackrel{\text{def } #1}{=}}
\newcommand{\itemfix}{\leavevmode\makeatletter\makeatother}
\newcommand{\?}{\textcolor{red}{???}}
\renewcommand{\emptyset}{\varnothing}
\newcommand{\longarrow}[1]{\xRightarrow[#1]{\qquad}}
\DeclareMathOperator*{\esup}{\text{ess sup}}
\newcommand\smallO{
    \mathchoice
    {{\scriptstyle\mathcal{O}}}% \displaystyle
    {{\scriptstyle\mathcal{O}}}% \textstyle
    {{\scriptscriptstyle\mathcal{O}}}% \scriptstyle
    {\scalebox{.6}{$\scriptscriptstyle\mathcal{O}$}}%\scriptscriptstyle
}
\renewcommand{\div}{\text{div}\ }
\newcommand{\rot}{\text{rot}\ }
\newcommand{\cov}{\text{cov}}

\makeatletter
\newcommand{\oplabel}[1]{\refstepcounter{equation}(\theequation\ltx@label{#1})}
\makeatother

\newcommand{\symref}[2]{\stackrel{\oplabel{#1}}{#2}}
\newcommand{\symrefeq}[1]{\symref{#1}{=}}

% xrightrightarrows
\makeatletter
\newcommand*{\relrelbarsep}{.386ex}
\newcommand*{\relrelbar}{%
    \mathrel{%
        \mathpalette\@relrelbar\relrelbarsep
    }%
}
\newcommand*{\@relrelbar}[2]{%
    \raise#2\hbox to 0pt{$\m@th#1\relbar$\hss}%
    \lower#2\hbox{$\m@th#1\relbar$}%
}
\providecommand*{\rightrightarrowsfill@}{%
    \arrowfill@\relrelbar\relrelbar\rightrightarrows
}
\providecommand*{\leftleftarrowsfill@}{%
    \arrowfill@\leftleftarrows\relrelbar\relrelbar
}
\providecommand*{\xrightrightarrows}[2][]{%
    \ext@arrow 0359\rightrightarrowsfill@{#1}{#2}%
}
\providecommand*{\xleftleftarrows}[2][]{%
    \ext@arrow 3095\leftleftarrowsfill@{#1}{#2}%
}

\allowdisplaybreaks

\newcommand{\unfinished}{\textcolor{red}{Не дописано}}

% Reproducible pdf builds 
\special{pdf:trailerid [
<00112233445566778899aabbccddeeff>
<00112233445566778899aabbccddeeff>
]}
%</preamble>


\lhead{Теория типов \textit{(практика)}}
\cfoot{}
\rfoot{5.10.2021}

\begin{document}

\section*{Домашнее задание №4: <<просто-типизированное лямбда исчисление>>}
\begin{enumerate}
      \item Сформулируйте аксиомы для просто типизированного исчисления по Чёрчу.
            \emph{Указание:} аксиомы должны быть согласованы с типами аргументов
            лямбда-абстракций.
            \begin{solution}\itemfix
                  \begin{enumerate}
                        \item \(\begin{prooftree}
                                    \infer0{\Gamma, x : \tau \vdash x : \tau}
                              \end{prooftree}\)
                        \item \(\begin{prooftree}
                                    \hypo{\Gamma \vdash A : \sigma \to \tau}
                                    \hypo{\Gamma \vdash B : \sigma}
                                    \infer2{\Gamma \vdash A\ B : \tau}
                              \end{prooftree}\)
                        \item \(\begin{prooftree}
                                    \hypo{\Gamma, x : \tau \vdash A : \sigma}
                                    \infer1{\Gamma \vdash \lambda x^\tau.A : \tau \to \sigma}
                              \end{prooftree}\)
                  \end{enumerate}
            \end{solution}
      \item Рассмотрим типизацию по Чёрчу. Определим стирающее преобразование $|\cdot|: \Lambda \rightarrow \Lambda_\textrm{ч}$:

            $$|A| = \left\{ \begin{array}{ll} \alpha,   & A = \alpha           \\
             |P| |Q|,       & A = P Q              \\
             \lambda x.|P|, & A = \lambda x^\tau.P\end{array}  \right.$$

            Верно ли следующее: если $P \rightarrow_\beta Q$ и $|P'|=P$, $|Q'|=Q$, то $P'\rightarrow_\beta Q'$.
            \begin{solution}
                  Кажется, преобразование \(| \cdot | : \Lambda_{\mathrm{ч}} \to \Lambda_{\mathrm{к}}\).

                  Нет. Пусть \(Q' = \lambda x^\tau.x, P' = \lambda x^\sigma.(\lambda x^\sigma.x)\ x\). \(|Q'| \equiv \lambda x.x, |P'| \equiv \lambda x.(\lambda x.x)\ x\), тогда \(|P'| \to_\beta \lambda x.x \equiv |Q'|\). Но \(P' \not\to_\beta Q'\), т.к. единственный возможный шаг это \(P' \to_\beta \lambda x^\sigma.x \neq_\alpha \lambda x^\tau.x\)

                  Иначе переберём как было сделано \(P \to_\beta Q\):
                  \begin{enumerate}
                        \item \(P \equiv A\ B, Q = C\ D\) и либо \(A \to_\beta C\) и \(B =_\alpha D\) либо \(A =_\alpha C\) и \(B \to_\beta D\). Тогда индукция даёт \(P' \equiv A'\ B'\)
                  \end{enumerate}
            \end{solution}
      \item Покажите, что если $A =_\alpha B$ и $\Gamma\vdash A:\tau$, то $\Gamma\vdash B:\tau$ (или, иными словами, доказательство не зависит от выбора пред-лямбда-терма).
            \begin{solution}\itemfix
                  \begin{enumerate}
                        \item \(A \equiv x, B \equiv x\). Тогда искомое очевидно.
                        \item \(A \equiv P\ Q, B \equiv R\ S, P =_\alpha R, Q =_\alpha S\) --- индукция:

                              \(\Gamma \vdash P\ Q : \tau\), тогда \(\Gamma \vdash P : \sigma \to \tau, \Gamma \vdash Q : \sigma\). По индукционному предположению будет \(\Gamma \vdash R : \sigma \to \tau, \Gamma \vdash S : \sigma\) и следовательно по второму правилу вывода искомое верно.
                        \item \(A \equiv \lambda x.P, B \equiv \lambda y.Q\) и существует \(t\) --- новая переменная, такая что \(P[x \coloneqq t] =_\alpha Q[y \coloneqq t]\) --- опять индукция:

                              \(\Gamma \vdash \lambda x.P : \tau \to \sigma\), тогда \(\Gamma, x : \tau \vdash P : \sigma\), но тогда и \(\Gamma, t : \tau \vdash P[x \coloneqq t] : \sigma\), т.к. это просто переименование и следовательно \(\Gamma, t : \tau \vdash Q[y \coloneqq t] : \sigma\) по индукционному предположению. Опять же \(\Gamma, y : \tau \vdash Q : \sigma\) и тогда по правилу вывода \(\Gamma \vdash \lambda y.Q : \tau \to \sigma\).
                  \end{enumerate}
            \end{solution}
\end{enumerate}

\end{document}
