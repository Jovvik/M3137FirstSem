\documentclass[12pt, a4paper]{article}

%<*preamble>
% Math symbols
\usepackage{amsmath, amsthm, amsfonts, amssymb}
\usepackage{accents}
\usepackage{esvect}
\usepackage{mathrsfs}
\usepackage{mathtools}
\mathtoolsset{showonlyrefs}
\usepackage{cmll}
\usepackage{stmaryrd}
\usepackage{physics}
\usepackage[normalem]{ulem}
\usepackage{ebproof}
\usepackage{extarrows}

% Page layout
\usepackage{geometry, a4wide, parskip, fancyhdr}

% Font, encoding, russian support
\usepackage[russian]{babel}
\usepackage[sb]{libertine}
\usepackage{xltxtra}

% Listings
\usepackage{listings}
\lstset{basicstyle=\ttfamily,breaklines=true}
\setmonofont{Inconsolata}

% Miscellaneous
\usepackage{array}
\usepackage{calc}
\usepackage{caption}
\usepackage{subcaption}
\captionsetup{justification=centering,margin=2cm}
\usepackage{catchfilebetweentags}
\usepackage{enumitem}
\usepackage{etoolbox}
\usepackage{float}
\usepackage{lastpage}
\usepackage{minted}
\usepackage{svg}
\usepackage{wrapfig}
\usepackage{xcolor}
\usepackage[makeroom]{cancel}

\newcolumntype{L}{>{$}l<{$}}
    \newcolumntype{C}{>{$}c<{$}}
\newcolumntype{R}{>{$}r<{$}}

% Footnotes
\usepackage[hang]{footmisc}
\setlength{\footnotemargin}{2mm}
\makeatletter
\def\blfootnote{\gdef\@thefnmark{}\@footnotetext}
\makeatother

% References
\usepackage{hyperref}
\hypersetup{
    colorlinks,
    linkcolor={blue!80!black},
    citecolor={blue!80!black},
    urlcolor={blue!80!black},
}

% tikz
\usepackage{tikz}
\usepackage{tikz-cd}
\usetikzlibrary{arrows.meta}
\usetikzlibrary{decorations.pathmorphing}
\usetikzlibrary{calc}
\usetikzlibrary{patterns}
\usepackage{pgfplots}
\pgfplotsset{width=10cm,compat=1.9}
\newcommand\irregularcircle[2]{% radius, irregularity
    \pgfextra {\pgfmathsetmacro\len{(#1)+rand*(#2)}}
    +(0:\len pt)
    \foreach \a in {10,20,...,350}{
            \pgfextra {\pgfmathsetmacro\len{(#1)+rand*(#2)}}
            -- +(\a:\len pt)
        } -- cycle
}

\providetoggle{useproofs}
\settoggle{useproofs}{false}

\pagestyle{fancy}
\lfoot{M3137y2019}
\cfoot{}
\rhead{стр. \thepage\ из \pageref*{LastPage}}

\newcommand{\R}{\mathbb{R}}
\newcommand{\Q}{\mathbb{Q}}
\newcommand{\Z}{\mathbb{Z}}
\newcommand{\B}{\mathbb{B}}
\newcommand{\N}{\mathbb{N}}
\renewcommand{\Re}{\mathfrak{R}}
\renewcommand{\Im}{\mathfrak{I}}

\newcommand{\const}{\text{const}}
\newcommand{\cond}{\text{cond}}

\newcommand{\teormin}{\textcolor{red}{!}\ }

\DeclareMathOperator*{\xor}{\oplus}
\DeclareMathOperator*{\equ}{\sim}
\DeclareMathOperator{\sign}{\text{sign}}
\DeclareMathOperator{\Sym}{\text{Sym}}
\DeclareMathOperator{\Asym}{\text{Asym}}

\DeclarePairedDelimiter{\ceil}{\lceil}{\rceil}

% godel
\newbox\gnBoxA
\newdimen\gnCornerHgt
\setbox\gnBoxA=\hbox{$\ulcorner$}
\global\gnCornerHgt=\ht\gnBoxA
\newdimen\gnArgHgt
\def\godel #1{%
    \setbox\gnBoxA=\hbox{$#1$}%
    \gnArgHgt=\ht\gnBoxA%
    \ifnum     \gnArgHgt<\gnCornerHgt \gnArgHgt=0pt%
    \else \advance \gnArgHgt by -\gnCornerHgt%
    \fi \raise\gnArgHgt\hbox{$\ulcorner$} \box\gnBoxA %
    \raise\gnArgHgt\hbox{$\urcorner$}}

% \theoremstyle{plain}

\theoremstyle{definition}
\newtheorem{theorem}{Теорема}
\newtheorem*{definition}{Определение}
\newtheorem{axiom}{Аксиома}
\newtheorem*{axiom*}{Аксиома}
\newtheorem{lemma}{Лемма}

\theoremstyle{remark}
\newtheorem*{remark}{Примечание}
\newtheorem*{exercise}{Упражнение}
\newtheorem{corollary}{Следствие}[theorem]
\newtheorem*{statement}{Утверждение}
\newtheorem*{corollary*}{Следствие}
\newtheorem*{example}{Пример}
\newtheorem{observation}{Наблюдение}
\newtheorem*{prop}{Свойства}
\newtheorem*{obozn}{Обозначение}

% subtheorem
\makeatletter
\newenvironment{subtheorem}[1]{%
    \def\subtheoremcounter{#1}%
    \refstepcounter{#1}%
    \protected@edef\theparentnumber{\csname the#1\endcsname}%
    \setcounter{parentnumber}{\value{#1}}%
    \setcounter{#1}{0}%
    \expandafter\def\csname the#1\endcsname{\theparentnumber.\Alph{#1}}%
    \ignorespaces
}{%
    \setcounter{\subtheoremcounter}{\value{parentnumber}}%
    \ignorespacesafterend
}
\makeatother
\newcounter{parentnumber}

\newtheorem{manualtheoreminner}{Теорема}
\newenvironment{manualtheorem}[1]{%
    \renewcommand\themanualtheoreminner{#1}%
    \manualtheoreminner
}{\endmanualtheoreminner}

\newcommand{\dbltilde}[1]{\accentset{\approx}{#1}}
\newcommand{\intt}{\int\!}

% magical thing that fixes paragraphs
\makeatletter
\patchcmd{\CatchFBT@Fin@l}{\endlinechar\m@ne}{}
{}{\typeout{Unsuccessful patch!}}
\makeatother

\newcommand{\get}[2]{
    \ExecuteMetaData[#1]{#2}
}

\newcommand{\getproof}[2]{
    \iftoggle{useproofs}{\ExecuteMetaData[#1]{#2proof}}{}
}

\newcommand{\getwithproof}[2]{
    \get{#1}{#2}
    \getproof{#1}{#2}
}

\newcommand{\import}[3]{
    \subsection{#1}
    \getwithproof{#2}{#3}
}

\newcommand{\given}[1]{
    Дано выше. (\ref{#1}, стр. \pageref{#1})
}

\renewcommand{\ker}{\text{Ker }}
\newcommand{\im}{\text{Im }}
\renewcommand{\grad}{\text{grad}}
\newcommand{\rg}{\text{rg}}
\newcommand{\defeq}{\stackrel{\text{def}}{=}}
\newcommand{\defeqfor}[1]{\stackrel{\text{def } #1}{=}}
\newcommand{\itemfix}{\leavevmode\makeatletter\makeatother}
\newcommand{\?}{\textcolor{red}{???}}
\renewcommand{\emptyset}{\varnothing}
\newcommand{\longarrow}[1]{\xRightarrow[#1]{\qquad}}
\DeclareMathOperator*{\esup}{\text{ess sup}}
\newcommand\smallO{
    \mathchoice
    {{\scriptstyle\mathcal{O}}}% \displaystyle
    {{\scriptstyle\mathcal{O}}}% \textstyle
    {{\scriptscriptstyle\mathcal{O}}}% \scriptstyle
    {\scalebox{.6}{$\scriptscriptstyle\mathcal{O}$}}%\scriptscriptstyle
}
\renewcommand{\div}{\text{div}\ }
\newcommand{\rot}{\text{rot}\ }
\newcommand{\cov}{\text{cov}}

\makeatletter
\newcommand{\oplabel}[1]{\refstepcounter{equation}(\theequation\ltx@label{#1})}
\makeatother

\newcommand{\symref}[2]{\stackrel{\oplabel{#1}}{#2}}
\newcommand{\symrefeq}[1]{\symref{#1}{=}}

% xrightrightarrows
\makeatletter
\newcommand*{\relrelbarsep}{.386ex}
\newcommand*{\relrelbar}{%
    \mathrel{%
        \mathpalette\@relrelbar\relrelbarsep
    }%
}
\newcommand*{\@relrelbar}[2]{%
    \raise#2\hbox to 0pt{$\m@th#1\relbar$\hss}%
    \lower#2\hbox{$\m@th#1\relbar$}%
}
\providecommand*{\rightrightarrowsfill@}{%
    \arrowfill@\relrelbar\relrelbar\rightrightarrows
}
\providecommand*{\leftleftarrowsfill@}{%
    \arrowfill@\leftleftarrows\relrelbar\relrelbar
}
\providecommand*{\xrightrightarrows}[2][]{%
    \ext@arrow 0359\rightrightarrowsfill@{#1}{#2}%
}
\providecommand*{\xleftleftarrows}[2][]{%
    \ext@arrow 3095\leftleftarrowsfill@{#1}{#2}%
}

\allowdisplaybreaks

\newcommand{\unfinished}{\textcolor{red}{Не дописано}}

% Reproducible pdf builds 
\special{pdf:trailerid [
<00112233445566778899aabbccddeeff>
<00112233445566778899aabbccddeeff>
]}
%</preamble>


\lhead{Теория типов \textit{(практика)}}
\cfoot{}
\rfoot{26.10.2021}

\begin{document}

\section*{Домашнее задание №7: <<типовая система Хиндли-Милнера>>}
\begin{enumerate}
    \item \emph{О выразительной силе HM.} Заметим, что список --- это <<параметризованные>> числа в
          аксиоматике Пеано. Число --- это длина списка, а к каждому штриху мы присоединяем какое-то значение.
          Операции добавления и удаления элемента из списка --- это операции прибавления и вычитания
          единицы к числу.

          Рассмотрим тип <<бинарного списка>> (расширение Окамля):

          \begin{minted}{OCaml}
type 'a bin_list = Nil | Zero of (('a*'a) bin_list) |
    One of 'a * (('a*'a) bin_list);;
\end{minted}

          и операцию добавления элемента к списку:

          \begin{minted}{OCaml}
let rec add elem lst = match lst wi th
    Nil -> One (elem,Nil)
  | Zero tl -> One (elem,tl)
  | One (hd,tl) -> Zero (add (elem,hd) tl)
\end{minted}

          \begin{enumerate}
              \item Какой тип имеет \verb!add! (обратите внимание на ключевое слово \verb!rec!: для
                    точного указания соответствующего лямбда-выражения и вывода типа необходимо использовать Y-комбинатор)?
                    Считайте, что семейство типов \verb!bin_list 'a! предопределено и обозначается как $\tau_\alpha$.
                    Выразим ли этот тип в системе Хиндли-Милнера?
              \item Реализуйте предложенный тип и функцию \verb!add! на Хаскеле (используйте опцию \verb!RankNTypes!).
                    Также реализуйте функцию для удаления элемента списка (головы).
              \item Предложите функцию для эффективного соединения двух списков (источник для
                    вдохновения --- сложение двух чисел в столбик).
              \item Предложите функцию для эффективного выделения $n$-го элемента из списка.
          \end{enumerate}
    \item На занятии мы рассмотрели функцию \verb!strange_pair x = (x 1, x "a")!.
          Покажите, что данную функцию невозможно типизировать в типовой системе Хиндли-Милнера.
          Указания: (а) ограничение мономорфизма отношения к делу не имеет;
          (б) ограничение на правило введения квантора всеобщности может оказаться существенным.
    \item Покажем, что алгоритм $W$ действительно находит корректный тип для лямбда-выражения
          (доказательство, что он находит наиболее общий тип, мы оставим в стороне).
          Для этого докажем по индукции, что $W(\Gamma,X)$ действительно находит такие тип $\tau$ и подстановку $S$,
          что $S\Gamma \vdash X:\tau$:
          \begin{enumerate}
              \item покажите базу индукции: $W(\Gamma,x)$;
                    \[\begin{prooftree}
                            \hypo{\Gamma, x : \forall \alpha_1 \dots \alpha_n.\tau : x \vdash \forall \alpha_1 \dots \alpha_n.\tau}
                            \infer1{\Gamma \vdash x : \forall \beta_1 \dots \beta_n.\tau}
                        \end{prooftree}\]
              \item покажите случай аппликации: $W(\Gamma,P\ Q)$;
                    \[\begin{prooftree}
                            \hypo{W(\Gamma, P) = (\tau_p, S_1)}
                            \infer1{S_1\Gamma \vdash P : \tau_p}
                            \infer1{S_{21}\Gamma \vdash P : S_2 \tau_p}
                            \infer1{S_{321}\Gamma \vdash P : S_3(S_2 \tau_p)}
                            \infer1{S_{321}\Gamma \vdash P : S_3(\tau_q \to \gamma)}
                            \infer1{S_{321}\Gamma \vdash P : (S_3 \tau_q) \to S_3 \gamma}
                            \hypo{W(S_1\Gamma, Q) = (\tau_q, S_2)}
                            \infer1{S_{21}\Gamma \vdash Q : \tau_q}
                            \infer1{S_{321}\Gamma \vdash Q : S_3 \tau_q}
                            \infer2{(S_3 \circ S_2 \circ S_1)\Gamma \vdash P\ Q : S_3 \gamma}
                        \end{prooftree}\]
              \item покажите случай лямбда-абстракции: $W(\Gamma,\lambda x.P)$;
                    \[\begin{prooftree}
                            \hypo{W(\Gamma \cup \{x : \tau_x\}, P) =  (\tau_p, S_1)}
                            \infer1{S_1(\Gamma \cup \{x : \tau_x\}) \vdash P : S_1\tau_p}
                            \infer1{S_1\Gamma, x : S_1\tau_x \vdash P : S_1\tau_p}
                            \infer1{S_1\Gamma \vdash \lambda x. P : S_1\tau_x \to S_1\tau_p}
                            \infer1{S_1\Gamma \vdash \lambda x. P : S_1(\tau_x \to \tau_p)}
                        \end{prooftree}\]
              \item покажите случай \verb!let!-выражения: $W(\Gamma,\texttt{let}\ x=P\ \texttt{in}\ Q)$.
                    \[\begin{prooftree}
                            \hypo{W(\Gamma, P) =  (\tau_p, S_1)}
                            \infer1{S_{1}\Gamma \vdash P : S_{1}\tau_p}
                            \infer1{S_{21}\Gamma \vdash P : S_{21}\tau_p}
                            \hypo{W(S_1\Gamma \cup \{x : \forall \{\alpha_i\}.\tau_p\}, Q) =  (\tau_q, S_2), \ \{\alpha_i\} \in FV(\tau_p)}
                            \infer1{S_{2}(S_1\Gamma \cup \{x : \forall \{\alpha_i\}.\tau_p\}) \vdash Q : \tau_q, \ \{\alpha_i\} \in FV(\tau_p)}
                            \infer1{\dots}
                            \infer1{S_{2}(S_1\Gamma \cup \{x : S_{1}\tau_p\}) \vdash Q : \tau_q}
                            \infer1{S_{21}\Gamma, x : S_{21}\tau_p \vdash Q : \tau_q}
                            \infer2{S_{21}\Gamma \vdash let\ x = P\ in\ Q : \tau_q}
                        \end{prooftree}\]
          \end{enumerate}
    \item Покажите, что в Хаскеле выражается $Y: \forall \alpha.(\alpha\rightarrow\alpha)\rightarrow\alpha$ и
          правило исключённого третьего $E: \alpha\vee\neg\alpha$.

          \begin{minted}{haskell}
magic :: a
magic = magic

Y :: (a -> a) -> a
Y = magic

E :: Either a (a -> Void)
E = magic
          \end{minted}
    \item Возможно ли в C++ построить выражения с типами ранга два и выше (включая конструкции с темплейтами)?
          Приведите пример, если да.
\end{enumerate}

\end{document}
